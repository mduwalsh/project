\chapter{Conclusion}
This research shows how Vose's haploid model for Genetic Algorithms
extends to the diploid case, improving the computation of infinite
population evolutionary trajectories by significantly reducing the
time and space used.  Efficiency is achieved through decoupling
haploid evolution from the evolution of infinite diploid populations
and employing Walsh transform methods to compute the effects of
mask-based crossover and mutation.  The efficient computation of
distance between finite and infinite diploid populations is achieved
by leveraging the reduction from diploid to haploid case.

Simulations are thereby made feasible which otherwise would require
excessive resources, as illustrated through computations confirming
the convergence of finite diploid population short-term behaviour to
the behaviour predicted by the diploid model. Results agree with the
expected rate of convergence for the single-step haploid case;
distance is inversely proportional to square root of population size.

Evolutionary limits predicted by Vose for infinite population were explored and analysed. Simulations showed when necessary condition in $\bm{\mu}$ and $\bm{\chi}$ distribution is met, finite population also showed oscillating behavior and converge to evolutionary limits for infinite population. In case of violation in the condition, infinite population ceased to oscillate but finite population when $\bm{\epsilon}$ introduced to $\bm{\mu}$ or $\bm{\chi}$ was not large, continued to oscillate.

In this research, we did not consider fitness factor for selection for simplicity of model. In future, we plan to extend our work accomodating fitness factor in our model and investigate convergence of short-term behavior of finite population to infinite population.


