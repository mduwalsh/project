% l = 8
\mbox{}\\[-0.75in]
\begin{figure}[!b]
\begin{center}
\subfloat{
\resizebox{8cm}{4.5cm}{\includegraphics{figures/eps/vio/mu/b8/e0.01/n00004096_fin_hap.eps}}} \hspace{-3em}%
\subfloat{
\resizebox{8cm}{4.5cm}{\includegraphics{figures/eps/vio/mu/b8/e0.01/n00004096_fin_hap_wovio.eps}}}\vspace{-1em} \hspace{-3em}%
\end{center}
\begin{center}
\subfloat{
\resizebox{8cm}{4.5cm}{\includegraphics{figures/eps/vio/mu/b8/e0.01/n00040960_fin_hap.eps}}} \hspace{-3em}%
\subfloat{
\resizebox{8cm}{4.5cm}{\includegraphics{figures/eps/vio/mu/b8/e0.01/n00040960_fin_hap_wovio.eps}}}\vspace{-1em} \hspace{-3em}%
\end{center}

\begin{center}
\subfloat{
\resizebox{8cm}{4.5cm}{\includegraphics{figures/eps/vio/mu/b8/e0.01/n00081920_fin_hap.eps}}} \hspace{-3em}%
\subfloat{
\resizebox{8cm}{4.5cm}{\includegraphics{figures/eps/vio/mu/b8/e0.01/n00081920_fin_hap_wovio.eps}}}\vspace{-1em} \hspace{-3em}%
\end{center}

\begin{center}
\subfloat{
\resizebox{8cm}{4.5cm}{\includegraphics{figures/eps/vio/mu/b8/e0.01/inf_hap.eps}}}\hspace{-3em}%
\subfloat{
\resizebox{8cm}{4.5cm}{\includegraphics{figures/eps/vio/mu/b8/e0.01/inf_hap_wovio.eps}}}\vspace{-0.5em} \hspace{-3em}%
\caption[\textbf{Infinite and finite haploid population behavior for $\bm{\mu}$ violation and $\ell = 8$ and 
$\bm{\epsilon} = 0.01$}]{\textbf{Infinite and finite haploid populations behavior for $\bm{\mu}$ violation and $\ell = 8$ and $\bm{\epsilon} = 0.01$:} 
In left column, $d'$ is distance of finite or infinite population to limit $\bm{z}^\ast$ for $g$ generations. 
In right column, $d$ is distance of finite or infinite population to limits $\bm{p}^\ast$ and $\bm{q}^\ast$.}
\label{oscillation_8h_vio_mu_0.01}
\end{center}
\end{figure}

% l = 10

\begin{figure}[h]
\begin{center}
\subfloat{
\resizebox{8cm}{4.5cm}{\includegraphics{figures/eps/vio/mu/b10/e0.01/n00004096_fin_hap.eps}}} \hspace{-3em}%
\subfloat{
\resizebox{8cm}{4.5cm}{\includegraphics{figures/eps/vio/mu/b10/e0.01/n00004096_fin_hap_wovio.eps}}}\vspace{-1em} \hspace{-3em}%
\end{center}
\begin{center}
\subfloat{
\resizebox{8cm}{4.5cm}{\includegraphics{figures/eps/vio/mu/b10/e0.01/n00040960_fin_hap.eps}}} \hspace{-3em}%
\subfloat{
\resizebox{8cm}{4.5cm}{\includegraphics{figures/eps/vio/mu/b10/e0.01/n00040960_fin_hap_wovio.eps}}}\vspace{-1em} \hspace{-3em}%
\end{center}

\begin{center}
\subfloat{
\resizebox{8cm}{4.5cm}{\includegraphics{figures/eps/vio/mu/b10/e0.01/n00081920_fin_hap.eps}}} \hspace{-3em}%
\subfloat{
\resizebox{8cm}{4.5cm}{\includegraphics{figures/eps/vio/mu/b10/e0.01/n00081920_fin_hap_wovio.eps}}}\vspace{-1em} \hspace{-3em}%
\end{center}

\begin{center}
\subfloat{
\resizebox{8cm}{4.5cm}{\includegraphics{figures/eps/vio/mu/b10/e0.01/inf_hap.eps}}}\hspace{-3em}%
\subfloat{
\resizebox{8cm}{4.5cm}{\includegraphics{figures/eps/vio/mu/b10/e0.01/inf_hap_wovio.eps}}}\vspace{-0.5em} \hspace{-3em}%
\caption[\textbf{Infinite and finite haploid population behavior for $\bm{\mu}$ violation, genome length $\ell = 10$ and $\bm{\epsilon} = 0.01$}]{\textbf{Infinite and finite haploid population behavior for $\bm{\mu}$ violation, genome length $\ell = 10$ and $\bm{\epsilon} = 0.01$:} 
  In left column, $d'$ is distance of finite or infinite population to limit $\bm{z}^\ast$ for $g$ generations. In right column, $d$ is distance of finite or infinite population to limits $\bm{p}^\ast$ and $\bm{q}^\ast$.}
\label{oscillation_10h_vio_mu_0.01}
\end{center}
\end{figure}

% l = 12

\begin{figure}[h]
\begin{center}
\subfloat{
\resizebox{8cm}{4.5cm}{\includegraphics{figures/eps/vio/mu/b12/e0.01/n00004096_fin_hap.eps}}} \hspace{-3em}%
\subfloat{
\resizebox{8cm}{4.5cm}{\includegraphics{figures/eps/vio/mu/b12/e0.01/n00004096_fin_hap_wovio.eps}}}\vspace{-1em} \hspace{-3em}%
\end{center}
\begin{center}
\subfloat{
\resizebox{8cm}{4.5cm}{\includegraphics{figures/eps/vio/mu/b12/e0.01/n00040960_fin_hap.eps}}} \hspace{-3em}%
\subfloat{
\resizebox{8cm}{4.5cm}{\includegraphics{figures/eps/vio/mu/b12/e0.01/n00040960_fin_hap_wovio.eps}}}\vspace{-1em} \hspace{-3em}%
\end{center}

\begin{center}
\subfloat{
\resizebox{8cm}{4.5cm}{\includegraphics{figures/eps/vio/mu/b12/e0.01/n00081920_fin_hap.eps}}} \hspace{-3em}%
\subfloat{
\resizebox{8cm}{4.5cm}{\includegraphics{figures/eps/vio/mu/b12/e0.01/n00081920_fin_hap_wovio.eps}}}\vspace{-1em} \hspace{-3em}%
\end{center}

\begin{center}
\subfloat{
\resizebox{8cm}{4.5cm}{\includegraphics{figures/eps/vio/mu/b12/e0.01/inf_hap.eps}}}\hspace{-3em}%
\subfloat{
\resizebox{8cm}{4.5cm}{\includegraphics{figures/eps/vio/mu/b12/e0.01/inf_hap_wovio.eps}}}\vspace{-0.5em} \hspace{-3em}%
\caption[\textbf{Infinite and finite haploid population behavior for $\bm{\mu}$ violation, genome length $\ell = 12$ and $\bm{\epsilon} = 0.01$}]{\textbf{Infinite and finite haploid population behavior for $\bm{\mu}$ violation, genome length $\ell = 12$ and $\bm{\epsilon} = 0.01$:} 
  In left column, $d'$ is distance of finite or infinite population to limit $\bm{z}^\ast$ for $g$ generations. In right column, $d$ is distance of finite or infinite population to limits $\bm{p}^\ast$ and $\bm{q}^\ast$.}
\label{oscillation_12h_vio_mu_0.01}
\end{center}
\end{figure}

% l = 14

\begin{figure}[h]
\begin{center}
\subfloat{
\resizebox{8cm}{4.5cm}{\includegraphics{figures/eps/vio/mu/b14/e0.01/n00004096_fin_hap.eps}}} \hspace{-3em}%
\subfloat{
\resizebox{8cm}{4.5cm}{\includegraphics{figures/eps/vio/mu/b14/e0.01/n00004096_fin_hap_wovio.eps}}}\vspace{-1em} \hspace{-3em}%
\end{center}
\begin{center}
\subfloat{
\resizebox{8cm}{4.5cm}{\includegraphics{figures/eps/vio/mu/b14/e0.01/n00040960_fin_hap.eps}}} \hspace{-3em}%
\subfloat{
\resizebox{8cm}{4.5cm}{\includegraphics{figures/eps/vio/mu/b14/e0.01/n00040960_fin_hap_wovio.eps}}}\vspace{-1em} \hspace{-3em}%
\end{center}

\begin{center}
\subfloat{
\resizebox{8cm}{4.5cm}{\includegraphics{figures/eps/vio/mu/b14/e0.01/n00081920_fin_hap.eps}}} \hspace{-3em}%
\subfloat{
\resizebox{8cm}{4.5cm}{\includegraphics{figures/eps/vio/mu/b14/e0.01/n00081920_fin_hap_wovio.eps}}}\vspace{-1em} \hspace{-3em}%
\end{center}

\begin{center}
\subfloat{
\resizebox{8cm}{4.5cm}{\includegraphics{figures/eps/vio/mu/b14/e0.01/inf_hap.eps}}}\hspace{-3em}%
\subfloat{
\resizebox{8cm}{4.5cm}{\includegraphics{figures/eps/vio/mu/b14/e0.01/inf_hap_wovio.eps}}}\vspace{-0.5em} \hspace{-3em}%
\caption[\textbf{Infinite and finite haploid population behavior for $\bm{\mu}$ violation, genome length $\ell = 14$ and $\bm{\epsilon} = 0.01$}]{\textbf{Infinite and finite haploid population behavior for $\bm{\mu}$ violation, genome length $\ell = 14$ and $\bm{\epsilon} = 0.01$:} 
  In left column, $d'$ is distance of finite or infinite population to limit $\bm{z}^\ast$ for $g$ generations. In right column, $d$ is distance of finite or infinite population to limits $\bm{p}^\ast$ and $\bm{q}^\ast$.}
\label{oscillation_14h_vio_mu_0.01}
\end{center}
\end{figure}

\clearpage

The right column in figures \ref{oscillation_8h_vio_mu_0.01} through \ref{oscillation_14h_vio_mu_0.01} 
shows distance of finite and infinite haploid populations to non-violation limits $\bm{p^\ast}$ and $\bm{q^\ast}$ with $\bm{\epsilon} \;=\; 0.01$. 
Those graphs indicate oscillating behavior of haploid population given violation. 
Both finite and infinite populations oscillate given violation. Since the value of $\bm{\epsilon}$ 
is small, damping of ripples is slow. The all zeros mask created in mutation distribution with $\bm{\epsilon} \;=\; 0.01$ 
is unlikely to be used during mutation, and when it is not used, behavior should be consistent with the 
behavior without violation. Moreover, $\bm{\epsilon}$ is small enough so that 
infinite population oscillation does not die out completely in 50 generations, 
even though oscillation will eventually die out completely. 
That is not the case for finite populations; if oscillation were to die out, 
it must reappear infinitelhy often because the Markov chain is regular.

The left column of figures \ref{oscillation_8h_vio_mu_0.01} through \ref{oscillation_14h_vio_mu_0.01} 
shows distance of finite and infinite haploid populations to limit $\bm{z^\ast}$ 
(limit with violation in mutation distribution $\bm{\mu}$) when $\bm{\epsilon} \;=\; 0.01$. 
The distance between finite population and limit $\bm{z}^\ast$ 
decreases as finite population size $N$ increases, 
and finite population shows behavior similar to infinite population as population size increases. 
Average distance data of haploid population for $\bm{\mu}$ violation  
with $\bm{\epsilon} \;=\; 0.01$ are tabulated in table \ref{distanceMuHapEps0.01}.

\begin{table}[h]
\caption[\textbf{Distance measured for violation in $\bm{\mu}$ with $\bm{\epsilon} \;=\; 0.01$ for haploids}]
{\textbf{Distance measured for violation in $\bm{\mu}$ with $\bm{\epsilon} \;=\; 0.01$ for haploids:} $\ell$ is genome length, 
average distance between finite and infinite population is tabulated in the last three columns, and last row is expected single step distance.}
\centering
\begin{tabularx}{0.75\textwidth}{ c *{3}{X}}
\toprule
$\ell$ & $N = 4096$ & $N = 40960$ & $N = 81920$ \\
\midrule
8 & 0.0176	& 0.0094	& 0.0093 \\
10 & 0.0168	& 0.0088 	& 0.0077 \\ 
12 & 0.0161	& 0.0064 	& 0.0053 \\
14 & 0.0157	& 0.0051 	& 0.0038 \\ 
\midrule
$1/\sqrt{N}$ & 0.0156 & 0.0049 & 0.0035 \\
\bottomrule
\end{tabularx}
\label{distanceMuHapEps0.01}
\end{table}
Table \ref{distanceMuHapEps0.01} shows that the average distance between finite and infinite population decreases with 
increasing string length, approaching the expected single step distance $1/\sqrt{N}$.

