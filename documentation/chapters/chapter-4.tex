\chapter{Conclusion}
This research shows how Vose's haploid model for Genetic Algorithms
extends to the diploid case, improving the computation of infinite
population evolutionary trajectories by significantly reducing the
time and space used.  Efficiency is achieved through decoupling
haploid evolution from the evolution of infinite diploid populations
and employing Walsh transform methods to compute the effects of
mask-based crossover and mutation.  

Simulations are thereby made feasible which otherwise would require
excessive resources, as illustrated through computations exploring 
the convergence rate of finite population short-term behavior. Results agree with the
expected rate of convergence for the single-step haploid case;
distance is inversely proportional to square root of population size.

Evolutionary limits predicted by Vose for infinite population were explored and analysed. 
Simulations showed that when the necessary condition for oscillation in infinite populations is met, 
finite populations also show oscillating behavior, and approximately converge to 
infinite population evolutionary limits in the short term. When the condition is violated, i
nfinite populations ceases to oscillate, but finite populations continue to oscillate if the violation is small.

In this research, we did not consider fitness factor for selection for simplicity of model. 
In future, we plan to extend our work accomodating fitness factor in our model and investigate 
convergence of short-term finite population behavior to infinite population limits.


