\chapter{Conclusion}
This research shows how Vose's haploid model for Genetic Algorithms
extends to the diploid case, improving the computation of infinite
population evolutionary trajectories by significantly reducing the
time and space used.  Efficiency is achieved through decoupling
haploid evolution from the evolution of infinite diploid populations
and employing Walsh transform methods to compute the effects of
mask-based crossover and mutation.  

Simulations are thereby made feasible which otherwise would require
excessive resources, as illustrated through computations exploring 
the convergence rate of finite population short-term behavior to infinite population evolutionary trajectories. 
Results confirm that distance can be inversely proportional to the square root of population size.

Simulations showed that when the necessary condition for oscillation in infinite populations is met, 
finite populations can also show oscillating behavior, approximately converging to 
infinite population evolutionary limits in the short term. When the condition is violated, 
infinite populations ceases to oscillate, but finite populations may continue to oscillate if the violation is small.

In this research, for simplicity we did not consider fitness factor for selection.  
In future, we plan to extend our work by accomodating fitness factor in our model and investigate 
the influence of that change upon the results obtained in the absence of selective pressure.



