\chapter{Violation in Crossover Distribution} \label{ch:chiviolation}
The crossover distribution $\bm{\chi}$ was modified as
\[
\bm{\chi}_i = (1-\bm{\epsilon}) \bm{\chi}_i\, ; \tabspace i = \{1, 2,.., 2^{\ell}-1\} 
\]
So that 
\[
\bm{\chi}_i + \bm{\chi}_{i+g} = 1-\bm{\epsilon} ; \tabspace g \text{ is defined in  section } \ref{Limits}
\]

Then $j$ is chosen where $\bm{\chi}_j = 0$ and set $\bm{\chi}_j = \bm{\epsilon}$. 

Violation in crossover distribution $\bm{\chi}$ is different from violation in mutation distribution $\bm{\mu}$. 
Markov chain formed by transition matrix $Q$ is regular under violation in $\bm{\mu}$ but that is not the case under violation in  $\bm{\chi}$. 
Markov chain formed under violation in $\bm{\chi}$ may not be regular unless at least there are two $1$s in $g$.

With $\bm{\chi}$ described above, non-crossover mask (all 0s) has always positive probability. 
And since we do not modify the mutation distribution $\bm{\mu}$ satisfying condition \ref{OscCond}, 
the non-mutation mask (all 0s) has always zero probability. 
However, applying same mutation mask twice with non-crossover event, 
we can achieve non mutation event in two generations. 

% Let us exlore for two cases of $g$ in \ref{OscCond}:
% 
% 1. When $g$ is all $1$s:\newline
% Any mask with a $1$ at position $k$ ($0 \leq k < \ell$) and $0$ at all other positions can mutate the $k$th bit of a string.  In next step, 
% either the string can mutate its next bit or the string can simulate non-mutation event in two generations (mutate $j$th bit twice). 
% This makes possibile for any string to mutate to 
% any other string. And any string can maintain its bits in even number of generations after aquiring its state. 
% 
% 2. When $g$ has atleast one $0$:\newline
% Any mask with a $1$ at position $k$ and $0$ at all other positions  
% will have positive probability if $g$ also is $1$ at position $k$. 
% Thus any bit where $g$ is $1$ can mutate.  
% Any mask with $1$ in just one of the positions where $g$ has $1$s and 
% also $1$ in just one of the positions where $g$ has $0$s can be used to 
% mutate a bit where $g$ is $0$ using two masks in sequence. 
% So any string can mutate to any other. 
% Also any string requires two generations to simulate non-mutation event. 

Any mask with a $1$ at position $k$ and $0$ at all other positions  
will have positive probability if $g$ also is $1$ at position $k$. 
Thus any bit where $g$ is $1$ can mutate. 

1. Mutate a bit in string where $g$ has $0$: 
Any mask with $1$ in just one of the positions where $g$ has $1$s and 
also $1$ in just one of the positions where $g$ has $0$s can be used to 
mutate a bit where $g$ is $0$ using these two masks in sequence in two steps. 

2.
Any bit in a string at position where $g$ 
is $1$ can be mutated in two steps. 

So any string can mutate to any other in even number of steps. 
Also any string requires two generations to simulate non-mutation event.
So any population can therefore mutate to any other population in even number of generations, 
and that makes the Markov chain irreducible.

In this research, experiment is designed to have finite population size as multiple of 2. 
So, there is a special set of population $P_s$ which can mutate to itself, i.e. one string in 
population can mutate to another string in population maintaining same population member strings within. 
Since any population can mutate to any other population, let us assume finite population $P_i$ 
can reach special population $P_s$ in $u$ (even) generations. Population $P_s$ can stay in its state as long as its wants 
by mutating to itself or mutate to next population state. Population $P_s$ can return to $P_i$ next $u$ generations. 
Let's say $P_s$ stays as $P_s$ for $h$ generations. Then population $P_i$ can return to its original state $P_i$ 
in total $2u+h$ generations. $h$ can be $1, \; 2, \;, 3,$ and so on. That means $P_i$ can return to its original state in 
odd or even number of generations. So, the Markov chain is also aperiodic.

Since the Markov chain formed by GA after the violation in $\bm{\chi}$ is irreducible and aperiodic, 
the Markov chain is regular (see \cite{Iosifescu1980}), and a steady state distribution 
with positive components exists for the GA (see \cite{Minc1988}). 

Simulations are repeated with violation levels $\bm{\epsilon}$ as above.
The distances of both infinite and finite populations to limit $\bm{z}^\ast$ are plotted. 
The distances of both infinite and finite populations to non-violation limits $\bm{p}^\ast$ and $\bm{q}^\ast$ are also plotted.

% figures for chi violation
\subsection{Haploid Population $\mathtt{\sim}$ $\epsilon: 0.01$}

% l = 8

\begin{figure}[H]
\begin{center}
\subfloat{
\resizebox{8cm}{5cm}{\includegraphics{figures/eps/vio/chi/b8/e0.01/n00004096_fin_hap.eps}}} \hspace{-3em}%
\subfloat{
\resizebox{8cm}{5cm}{\includegraphics{figures/eps/vio/chi/b8/e0.01/n00004096_fin_hap_wovio.eps}}}\vspace{-1em} \hspace{-3em}%
\end{center}
\begin{center}
\subfloat{
\resizebox{8cm}{5cm}{\includegraphics{figures/eps/vio/chi/b8/e0.01/n00040960_fin_hap.eps}}} \hspace{-3em}%
\subfloat{
\resizebox{8cm}{5cm}{\includegraphics{figures/eps/vio/chi/b8/e0.01/n00040960_fin_hap_wovio.eps}}}\vspace{-1em} \hspace{-3em}%
\end{center}

\begin{center}
\subfloat{
\resizebox{8cm}{5cm}{\includegraphics{figures/eps/vio/chi/b8/e0.01/n00081920_fin_hap.eps}}} \hspace{-3em}%
\subfloat{
\resizebox{8cm}{5cm}{\includegraphics{figures/eps/vio/chi/b8/e0.01/n00081920_fin_hap_wovio.eps}}}\vspace{-1em} \hspace{-3em}%
\end{center}

\begin{center}
\subfloat{
\resizebox{8cm}{5cm}{\includegraphics{figures/eps/vio/chi/b8/e0.01/inf_hap.eps}}}\hspace{-3em}%
\subfloat{
\resizebox{8cm}{5cm}{\includegraphics{figures/eps/vio/chi/b8/e0.01/inf_hap_wovio.eps}}}\vspace{-0.5em} \hspace{-3em}%


\caption{\textbf{Infinite and finite haploid population oscillation behavior in case of violation in $\bm{\chi}$ for genome length $\ell = 8$ and $\bm{\epsilon} = 0.01$:} 
  In left column, $d'$ is distance of finite population of size $n$ or infinite population to limit $\bm{z}^\ast$ for $g$ generations. In right column, $d$ is distance of finite population of size $N$ or infinite population to limits without violation.}
\label{oscillation_8h_vio_chi_0.01}
\end{center}
\end{figure}


% l = 10

\begin{figure}[H]
\begin{center}
\subfloat{
\resizebox{8cm}{5cm}{\includegraphics{figures/eps/vio/chi/b10/e0.01/n00004096_fin_hap.eps}}} \hspace{-3em}%
\subfloat{
\resizebox{8cm}{5cm}{\includegraphics{figures/eps/vio/chi/b10/e0.01/n00004096_fin_hap_wovio.eps}}}\vspace{-1em} \hspace{-3em}%
\end{center}
\begin{center}
\subfloat{
\resizebox{8cm}{5cm}{\includegraphics{figures/eps/vio/chi/b10/e0.01/n00040960_fin_hap.eps}}} \hspace{-3em}%
\subfloat{
\resizebox{8cm}{5cm}{\includegraphics{figures/eps/vio/chi/b10/e0.01/n00040960_fin_hap_wovio.eps}}}\vspace{-1em} \hspace{-3em}%
\end{center}

\begin{center}
\subfloat{
\resizebox{8cm}{5cm}{\includegraphics{figures/eps/vio/chi/b10/e0.01/n00081920_fin_hap.eps}}} \hspace{-3em}%
\subfloat{
\resizebox{8cm}{5cm}{\includegraphics{figures/eps/vio/chi/b10/e0.01/n00081920_fin_hap_wovio.eps}}}\vspace{-1em} \hspace{-3em}%
\end{center}

\begin{center}
\subfloat{
\resizebox{8cm}{5cm}{\includegraphics{figures/eps/vio/chi/b10/e0.01/inf_hap.eps}}}\hspace{-3em}%
\subfloat{
\resizebox{8cm}{5cm}{\includegraphics{figures/eps/vio/chi/b10/e0.01/inf_hap_wovio.eps}}}\vspace{-0.5em} \hspace{-3em}%

\caption{\textbf{Infinite and finite haploid population oscillation behavior in case of violation in $\bm{\chi}$ for genome length $\ell = 10$ and $\bm{\epsilon} = 0.01$:} 
  In left column, $d'$ is distance of finite population of size $n$ or infinite population to limit $\bm{z}^\ast$ for $g$ generations. In right column, $d$ is distance of finite population of size $N$ or infinite population to limits without violation.}
\label{oscillation_10h_vio_chi_0.01}
\end{center}
\end{figure}


% l = 12
\begin{figure}[H]
\begin{center}
\subfloat{
\resizebox{8cm}{5cm}{\includegraphics{figures/eps/vio/chi/b12/e0.01/n00004096_fin_hap.eps}}} \hspace{-3em}%
\subfloat{
\resizebox{8cm}{5cm}{\includegraphics{figures/eps/vio/chi/b12/e0.01/n00004096_fin_hap_wovio.eps}}}\vspace{-1em} \hspace{-3em}%
\end{center}
\begin{center}
\subfloat{
\resizebox{8cm}{5cm}{\includegraphics{figures/eps/vio/chi/b12/e0.01/n00040960_fin_hap.eps}}} \hspace{-3em}%
\subfloat{
\resizebox{8cm}{5cm}{\includegraphics{figures/eps/vio/chi/b12/e0.01/n00040960_fin_hap_wovio.eps}}}\vspace{-1em} \hspace{-3em}%
\end{center}

\begin{center}
\subfloat{
\resizebox{8cm}{5cm}{\includegraphics{figures/eps/vio/chi/b12/e0.01/n00081920_fin_hap.eps}}} \hspace{-3em}%
\subfloat{
\resizebox{8cm}{5cm}{\includegraphics{figures/eps/vio/chi/b12/e0.01/n00081920_fin_hap_wovio.eps}}}\vspace{-1em} \hspace{-3em}%
\end{center}

\begin{center}
\subfloat{
\resizebox{8cm}{5cm}{\includegraphics{figures/eps/vio/chi/b12/e0.01/inf_hap.eps}}}\hspace{-3em}%
\subfloat{
\resizebox{8cm}{5cm}{\includegraphics{figures/eps/vio/chi/b12/e0.01/inf_hap_wovio.eps}}}\vspace{-0.5em} \hspace{-3em}%


\caption{\textbf{Infinite and finite haploid population oscillation behavior in case of violation in $\bm{\chi}$ for genome length $\ell = 12$ and $\bm{\epsilon} = 0.01$:} 
  In left column, $d'$ is distance of finite population of size $n$ or infinite population to limit $\bm{z}^\ast$ for $g$ generations. In right column, $d$ is distance of finite population of size $N$ or infinite population to limits without violation.}
\label{oscillation_12h_vio_chi_0.01}
\end{center}
\end{figure}

% l = 14

\begin{figure}[H]
\begin{center}
\subfloat{
\resizebox{8cm}{5cm}{\includegraphics{figures/eps/vio/chi/b14/e0.01/n00004096_fin_hap.eps}}} \hspace{-3em}%
\subfloat{
\resizebox{8cm}{5cm}{\includegraphics{figures/eps/vio/chi/b14/e0.01/n00004096_fin_hap_wovio.eps}}}\vspace{-1em} \hspace{-3em}%
\end{center}
\begin{center}
\subfloat{
\resizebox{8cm}{5cm}{\includegraphics{figures/eps/vio/chi/b14/e0.01/n00040960_fin_hap.eps}}} \hspace{-3em}%
\subfloat{
\resizebox{8cm}{5cm}{\includegraphics{figures/eps/vio/chi/b14/e0.01/n00040960_fin_hap_wovio.eps}}}\vspace{-1em} \hspace{-3em}%
\end{center}

\begin{center}
\subfloat{
\resizebox{8cm}{5cm}{\includegraphics{figures/eps/vio/chi/b14/e0.01/n00081920_fin_hap.eps}}} \hspace{-3em}%
\subfloat{
\resizebox{8cm}{5cm}{\includegraphics{figures/eps/vio/chi/b14/e0.01/n00081920_fin_hap_wovio.eps}}}\vspace{-1em} \hspace{-3em}%
\end{center}

\begin{center}
\subfloat{
\resizebox{8cm}{5cm}{\includegraphics{figures/eps/vio/chi/b14/e0.01/inf_hap.eps}}}\hspace{-3em}%
\subfloat{
\resizebox{8cm}{5cm}{\includegraphics{figures/eps/vio/chi/b14/e0.01/inf_hap_wovio.eps}}}\vspace{-0.5em} \hspace{-3em}%

\caption{\textbf{Infinite and finite haploid population oscillation behavior in case of violation in $\bm{\chi}$ for genome length $\ell = 14$ and $\bm{\epsilon} = 0.01$:} 
  In left column, $d'$ is distance of finite population of size $n$ or infinite population to limit $\bm{z}^\ast$ for $g$ generations. In right column, $d$ is distance of finite population of size $N$ or infinite population to limits without violation.}
\label{oscillation_14h_vio_chi_0.01}
\end{center}
\end{figure}

The right column in figures \ref{oscillation_8h_vio_chi_0.01} through \ref{oscillation_14h_vio_chi_0.01} 
shows distance of finite and infinite haploid populations to non-violation limits $\bm{p^\ast}$ and $\bm{q^\ast}$ with $\bm{\epsilon} \;=\; 0.01$. 
Graphs in the right column give picture of oscillating behavior of haploid population given violation. 
Both finite and infinite populations oscillate given violation. Since value of $\bm{\epsilon}$ 
is very small, damping of ripples is slow. New masks created in crossover distribution with $\bm{\epsilon} \;=\; 0.01$ have very small 
probability of being used during crossover. The probability of the new masks being used is so low for $\bm{\epsilon} \;=\; 0.01$ that even 
infinite population oscillation did not die out completely in 50 generations. 

The left column of figures \ref{oscillation_8h_vio_chi_0.01} through \ref{oscillation_14h_vio_chi_0.01} 
shows distance of finite and infinite haploid populations to limit $\bm{z^\ast}$ 
(limit with violation in crossover distribution $\bm{\chi}$) when $\bm{\epsilon} \;=\; 0.01$. 
The distance between finite population and limit $\bm{z}^\ast$ (limit with violation in $\bm{\chi}$ distribution) 
decreases as finite population size increases, 
and finite population shows behavior similar to infinite population behavior as finite population reach large number. 
The distance data for haploid population in case of violation in $\bm{\chi}$ distribution 
with $\bm{\epsilon} \;=\; 0.01$ for different finite population size $N$ are tabulated in table \ref{distanceChiHapEps0.01}.

\begin{table}[ht]
\caption{\textbf{Distance measured for violation in $\bm{\chi}$ with $\bm{\epsilon} \;=\; 0.01$  for haploids:} $\ell$ is genome length, 
and average distance between finite and 
infinite populations is tabulated in last three columns.}
\centering
\begin{tabularx}{0.75\textwidth}{ c *{3}{X}}
\toprule
$\ell$ & $N = 4096$ & $N = 40960$ & $N = 81920$  \\
\midrule
8 & 0.0186	&  0.0150 	& 0.0115 \\
10 & 0.0158	&  0.0062 	& 0.0051 \\ 
12 & 0.0158	&  0.0056	& 0.0045 \\
14 & 0.0156	&  0.0050	& 0.0036 \\ 
\bottomrule
\end{tabularx}
\label{distanceChiHapEps0.01}
\end{table} 

From table \ref{distanceChiHapEps0.01}, average distance calculated for finite population size $4096$ is $0.0164$, 
for size $40960$ is $0.0079$ and for size $81920$ is $0.0062$. These results show average distance 
between finite population and limit $\bm{z^\ast}$ closely follows expected single step distance 
between finite and infinite population given in \ref{tableExpectedDistance}. The distance decreased as $1/\sqrt{N}$. 
Also, the distance decreased as genome length $\ell$ increased for all sizes of finite haploid populations 
with $\bm{\epsilon} \;=\; 0.01$.

\subsection{Haploid Population $\mathtt{\sim}$ $\epsilon: 0.1$}

% l = 8

\begin{figure}[h]
\begin{center}
\subfloat{
\resizebox{8cm}{5cm}{\includegraphics{figures/eps/vio/chi/b8/e0.1/n00004096_fin_hap.eps}}} \hspace{-3em}%
\subfloat{
\resizebox{8cm}{5cm}{\includegraphics{figures/eps/vio/chi/b8/e0.1/n00004096_fin_hap_wovio.eps}}}\vspace{-1em} \hspace{-3em}%
\end{center}
\begin{center}
\subfloat{
\resizebox{8cm}{5cm}{\includegraphics{figures/eps/vio/chi/b8/e0.1/n00040960_fin_hap.eps}}} \hspace{-3em}%
\subfloat{
\resizebox{8cm}{5cm}{\includegraphics{figures/eps/vio/chi/b8/e0.1/n00040960_fin_hap_wovio.eps}}}\vspace{-1em} \hspace{-3em}%
\end{center}

\begin{center}
\subfloat{
\resizebox{8cm}{5cm}{\includegraphics{figures/eps/vio/chi/b8/e0.1/n00081920_fin_hap.eps}}} \hspace{-3em}%
\subfloat{
\resizebox{8cm}{5cm}{\includegraphics{figures/eps/vio/chi/b8/e0.1/n00081920_fin_hap_wovio.eps}}}\vspace{-1em} \hspace{-3em}%
\end{center}

\begin{center}
\subfloat{
\resizebox{8cm}{5cm}{\includegraphics{figures/eps/vio/chi/b8/e0.1/inf_hap.eps}}}\hspace{-3em}%
\subfloat{
\resizebox{8cm}{5cm}{\includegraphics{figures/eps/vio/chi/b8/e0.1/inf_hap_wovio.eps}}}\vspace{-0.5em} \hspace{-3em}%


\caption{\textbf{Infinite and finite haploid population oscillation behavior in case of violation in $\bm{\chi}$ for genome length $\ell = 8$ and $\bm{\epsilon} = 0.1$:} 
  In left column, $d'$ is distance of finite population of size $n$ or infinite population to limit $\bm{z}^\ast$ for $g$ generations. In right column, $d$ is distance of finite population or infinite population to limits $\bm{p}^\ast$ and $\bm{q}^\ast$ without violation.}
\label{oscillation_8h_vio_chi_0.1}
\end{center}
\end{figure}

% l = 10

\begin{figure}[h]
\begin{center}
\subfloat{
\resizebox{8cm}{5cm}{\includegraphics{figures/eps/vio/chi/b10/e0.1/n00004096_fin_hap.eps}}} \hspace{-3em}%
\subfloat{
\resizebox{8cm}{5cm}{\includegraphics{figures/eps/vio/chi/b10/e0.1/n00004096_fin_hap_wovio.eps}}}\vspace{-1em} \hspace{-3em}%
\end{center}
\begin{center}
\subfloat{
\resizebox{8cm}{5cm}{\includegraphics{figures/eps/vio/chi/b10/e0.1/n00040960_fin_hap.eps}}} \hspace{-3em}%
\subfloat{
\resizebox{8cm}{5cm}{\includegraphics{figures/eps/vio/chi/b10/e0.1/n00040960_fin_hap_wovio.eps}}}\vspace{-1em} \hspace{-3em}%
\end{center}

\begin{center}
\subfloat{
\resizebox{8cm}{5cm}{\includegraphics{figures/eps/vio/chi/b10/e0.1/n00081920_fin_hap.eps}}} \hspace{-3em}%
\subfloat{
\resizebox{8cm}{5cm}{\includegraphics{figures/eps/vio/chi/b10/e0.1/n00081920_fin_hap_wovio.eps}}}\vspace{-1em} \hspace{-3em}%
\end{center}

\begin{center}
\subfloat{
\resizebox{8cm}{5cm}{\includegraphics{figures/eps/vio/chi/b10/e0.1/inf_hap.eps}}}\hspace{-3em}%
\subfloat{
\resizebox{8cm}{5cm}{\includegraphics{figures/eps/vio/chi/b10/e0.1/inf_hap_wovio.eps}}}\vspace{-0.5em} \hspace{-3em}%

\caption{\textbf{Infinite and finite haploid population oscillation behavior in case of violation in $\bm{\chi}$ for genome length $\ell = 10$ and $\bm{\epsilon} = 0.1$:} 
  In left column, $d'$ is distance of finite population of size $n$ or infinite population to limit $\bm{z}^\ast$ for $g$ generations. In right column, $d$ is distance of finite population or infinite population to limits $\bm{p}^\ast$ and $\bm{q}^\ast$ without violation.}
\label{oscillation_10h_vio_chi_0.1}
\end{center}
\end{figure}


% l = 12
\begin{figure}[h]
\begin{center}
\subfloat{
\resizebox{8cm}{5cm}{\includegraphics{figures/eps/vio/chi/b12/e0.1/n00004096_fin_hap.eps}}} \hspace{-3em}%
\subfloat{
\resizebox{8cm}{5cm}{\includegraphics{figures/eps/vio/chi/b12/e0.1/n00004096_fin_hap_wovio.eps}}}\vspace{-1em} \hspace{-3em}%
\end{center}
\begin{center}
\subfloat{
\resizebox{8cm}{5cm}{\includegraphics{figures/eps/vio/chi/b12/e0.1/n00040960_fin_hap.eps}}} \hspace{-3em}%
\subfloat{
\resizebox{8cm}{5cm}{\includegraphics{figures/eps/vio/chi/b12/e0.1/n00040960_fin_hap_wovio.eps}}}\vspace{-1em} \hspace{-3em}%
\end{center}

\begin{center}
\subfloat{
\resizebox{8cm}{5cm}{\includegraphics{figures/eps/vio/chi/b12/e0.1/n00081920_fin_hap.eps}}} \hspace{-3em}%
\subfloat{
\resizebox{8cm}{5cm}{\includegraphics{figures/eps/vio/chi/b12/e0.1/n00081920_fin_hap_wovio.eps}}}\vspace{-1em} \hspace{-3em}%
\end{center}

\begin{center}
\subfloat{
\resizebox{8cm}{5cm}{\includegraphics{figures/eps/vio/chi/b12/e0.1/inf_hap.eps}}}\hspace{-3em}%
\subfloat{
\resizebox{8cm}{5cm}{\includegraphics{figures/eps/vio/chi/b12/e0.1/inf_hap_wovio.eps}}}\vspace{-0.5em} \hspace{-3em}%


\caption{\textbf{Infinite and finite haploid population oscillation behavior in case of violation in $\bm{\chi}$ for genome length $\ell = 12$ and $\bm{\epsilon} = 0.1$:} 
  In left column, $d'$ is distance of finite population of size $n$ or infinite population to limit $\bm{z}^\ast$ for $g$ generations. In right column, $d$ is distance of finite population or infinite population to limits $\bm{p}^\ast$ and $\bm{q}^\ast$ without violation.}
\label{oscillation_12h_vio_chi_0.1}
\end{center}
\end{figure}

% l = 14

\begin{figure}[h]
\begin{center}
\subfloat{
\resizebox{8cm}{5cm}{\includegraphics{figures/eps/vio/chi/b14/e0.1/n00004096_fin_hap.eps}}} \hspace{-3em}%
\subfloat{
\resizebox{8cm}{5cm}{\includegraphics{figures/eps/vio/chi/b14/e0.1/n00004096_fin_hap_wovio.eps}}}\vspace{-1em} \hspace{-3em}%
\end{center}
\begin{center}
\subfloat{
\resizebox{8cm}{5cm}{\includegraphics{figures/eps/vio/chi/b14/e0.1/n00040960_fin_hap.eps}}} \hspace{-3em}%
\subfloat{
\resizebox{8cm}{5cm}{\includegraphics{figures/eps/vio/chi/b14/e0.1/n00040960_fin_hap_wovio.eps}}}\vspace{-1em} \hspace{-3em}%
\end{center}

\begin{center}
\subfloat{
\resizebox{8cm}{5cm}{\includegraphics{figures/eps/vio/chi/b14/e0.1/n00081920_fin_hap.eps}}} \hspace{-3em}%
\subfloat{
\resizebox{8cm}{5cm}{\includegraphics{figures/eps/vio/chi/b14/e0.1/n00081920_fin_hap_wovio.eps}}}\vspace{-1em} \hspace{-3em}%
\end{center}

\begin{center}
\subfloat{
\resizebox{8cm}{5cm}{\includegraphics{figures/eps/vio/chi/b14/e0.1/inf_hap.eps}}}\hspace{-3em}%
\subfloat{
\resizebox{8cm}{5cm}{\includegraphics{figures/eps/vio/chi/b14/e0.1/inf_hap_wovio.eps}}}\vspace{-0.5em} \hspace{-3em}%

\caption{\textbf{Infinite and finite haploid population oscillation behavior in case of violation in $\bm{\chi}$ for genome length $\ell = 14$ and $\bm{\epsilon} = 0.1$:} 
  In left column, $d'$ is distance of finite population of size $n$ or infinite population to limit $\bm{z}^\ast$ for $g$ generations. In right column, $d$ is distance of finite population or infinite population to limits $\bm{p}^\ast$ and $\bm{q}^\ast$ without violation.}
\label{oscillation_14h_vio_chi_0.1}
\end{center}
\end{figure}

\clearpage

The right column in figures \ref{oscillation_8h_vio_chi_0.1} through \ref{oscillation_14h_vio_chi_0.1} 
shows distance of finite and infinite haploid populations to non-violation limits $\bm{p^\ast}$ and $\bm{q^\ast}$ with $\bm{\epsilon} \;=\; 0.1$. 
Those graphs indicate oscillating behavior of haploid population given violation. 
Both finite and infinite populations oscillate given violation, and oscillations amplitudes decreases with time. 
However, for $\bm{\epsilon} \;=\; 0.1$, oscillations in infinite populations die out quickly, 
but oscillations in finite populations does not die out completely. Rate of damping of ripples with $\bm{\epsilon} \;=\; 0.1$ is  
larger than with $\bm{\epsilon} \;=\; 0.01$. New masks created in crossover distribution with $\bm{\epsilon} \;=\; 0.1$ have small  
probability of being used during crossover for oscillation in finite populations to die out completely. 

The left column of figures \ref{oscillation_8h_vio_chi_0.1} through \ref{oscillation_14h_vio_chi_0.1} 
shows distance of finite and infinite haploid populations to limit $\bm{z^\ast}$ 
(limit with violation in crossover distribution $\bm{\chi}$) when $\bm{\epsilon} \;=\; 0.1$. 
The distance between finite population and limit $\bm{z}^\ast$ (limit with violation in $\bm{\chi}$ distribution) 
decreases as finite population size increases, 
and finite population shows behavior similar to infinite population behavior as finite population size grows. 
Average distance data for haploid population in case of violation in $\bm{\chi}$ distribution 
with $\bm{\epsilon} \;=\; 0.1$ for different finite population size $N$ are tabulated in table \ref{distanceChiHapEps0.1}.

\begin{table}[ht]
\caption{\textbf{Distance measured for violation in $\bm{\chi}$ with $\bm{\epsilon} \;=\; 0.1$  for haploids:} $\ell$ is genome length, 
average distance between finite and infinite population is tabulated in the last three columns, and last row is expected single step distance.}
\centering
\begin{tabularx}{0.75\textwidth}{ c *{3}{X}}
\toprule
$\ell$ & $N = 4096$ & $N = 40960$ & $N = 81920$  \\
\midrule
8 & 0.0163	& 0.0061 	& 0.0051 \\
10 & 0.0157	&  0.0051	& 0.0037 \\	
12 & 0.0157	&  0.0051	& 0.0037 \\	
14 & 0.0156	&  0.0049	& 0.0035 \\
\midrule
$1/\sqrt{N}$ & 0.0156 & 0.0049 & 0.0035 \\
\bottomrule
\end{tabularx}
\label{distanceChiHapEps0.1}
\end{table} 

From , average distance calculated for finite population size $4096$ is $0.0158$, 
for size $40960$ is $0.0053$ and for size $81920$ is $0.0038$. The results from Table \ref{distanceChiHapEps0.1} show average distance 
between finite population and limit $\bm{z^\ast}$ approaches the expected single step distance 
between finite and infinite population. The distance decreased as $1/\sqrt{N}$. 
Also, the distance decreased as genome length $\ell$ increased for all sizes of finite haploid populations 
with $\bm{\epsilon} \;=\; 0.1$.
 

\subsection{Haploid Population $\mathtt{\sim}$ $\epsilon: 0.5$}

% l = 8
\mbox{}\\[-0.75in]
\begin{figure}[!b]
\begin{center}
\subfloat{
\resizebox{8cm}{4.5cm}{\includegraphics{figures/eps/vio/chi/b8/e0.5/n00004096_fin_hap.eps}}} \hspace{-3em}%
\subfloat{
\resizebox{8cm}{4.5cm}{\includegraphics{figures/eps/vio/chi/b8/e0.5/n00004096_fin_hap_wovio.eps}}}\vspace{-1em} \hspace{-3em}%
\end{center}
\begin{center}
\subfloat{
\resizebox{8cm}{4.5cm}{\includegraphics{figures/eps/vio/chi/b8/e0.5/n00040960_fin_hap.eps}}} \hspace{-3em}%
\subfloat{
\resizebox{8cm}{4.5cm}{\includegraphics{figures/eps/vio/chi/b8/e0.5/n00040960_fin_hap_wovio.eps}}}\vspace{-1em} \hspace{-3em}%
\end{center}

\begin{center}
\subfloat{
\resizebox{8cm}{4.5cm}{\includegraphics{figures/eps/vio/chi/b8/e0.5/n00081920_fin_hap.eps}}} \hspace{-3em}%
\subfloat{
\resizebox{8cm}{4.5cm}{\includegraphics{figures/eps/vio/chi/b8/e0.5/n00081920_fin_hap_wovio.eps}}}\vspace{-1em} \hspace{-3em}%
\end{center}

\begin{center}
\subfloat{
\resizebox{8cm}{4.5cm}{\includegraphics{figures/eps/vio/chi/b8/e0.5/inf_hap.eps}}}\hspace{-3em}%
\subfloat{
\resizebox{8cm}{4.5cm}{\includegraphics{figures/eps/vio/chi/b8/e0.5/inf_hap_wovio.eps}}}\vspace{-0.5em} \hspace{-3em}%


\caption[\textbf{Infinite and finite haploid population behavior for $\bm{\chi}$ violation, $\ell = 8$ and $\bm{\epsilon} = 0.5$}]
{\textbf{Infinite and finite haploid population behavior for $\bm{\chi}$ violation, $\ell = 8$ and $\bm{\epsilon} = 0.5$:} 
  In left column, $d'$ is distance of finite or infinite population to limit $\bm{z}^\ast$ for $g$ generations. 
  In right column, $d$ is distance of finite or infinite population to limits $\bm{p}^\ast$ and $\bm{q}^\ast$.}
\label{oscillation_8h_vio_chi_0.5}
\end{center}
\end{figure}


% l = 10

\begin{figure}[h]
\begin{center}
\subfloat{
\resizebox{8cm}{4.5cm}{\includegraphics{figures/eps/vio/chi/b10/e0.5/n00004096_fin_hap.eps}}} \hspace{-3em}%
\subfloat{
\resizebox{8cm}{4.5cm}{\includegraphics{figures/eps/vio/chi/b10/e0.5/n00004096_fin_hap_wovio.eps}}}\vspace{-1em} \hspace{-3em}%
\end{center}
\begin{center}
\subfloat{
\resizebox{8cm}{4.5cm}{\includegraphics{figures/eps/vio/chi/b10/e0.5/n00040960_fin_hap.eps}}} \hspace{-3em}%
\subfloat{
\resizebox{8cm}{4.5cm}{\includegraphics{figures/eps/vio/chi/b10/e0.5/n00040960_fin_hap_wovio.eps}}}\vspace{-1em} \hspace{-3em}%
\end{center}

\begin{center}
\subfloat{
\resizebox{8cm}{4.5cm}{\includegraphics{figures/eps/vio/chi/b10/e0.5/n00081920_fin_hap.eps}}} \hspace{-3em}%
\subfloat{
\resizebox{8cm}{4.5cm}{\includegraphics{figures/eps/vio/chi/b10/e0.5/n00081920_fin_hap_wovio.eps}}}\vspace{-1em} \hspace{-3em}%
\end{center}

\begin{center}
\subfloat{
\resizebox{8cm}{4.5cm}{\includegraphics{figures/eps/vio/chi/b10/e0.5/inf_hap.eps}}}\hspace{-3em}%
\subfloat{
\resizebox{8cm}{4.5cm}{\includegraphics{figures/eps/vio/chi/b10/e0.5/inf_hap_wovio.eps}}}\vspace{-0.5em} \hspace{-3em}%

\caption[\textbf{Infinite and finite haploid population behavior for $\bm{\chi}$ violation, genome length $\ell = 10$ and $\bm{\epsilon} = 0.5$}]{\textbf{Infinite and finite haploid population behavior for $\bm{\chi}$ violation, genome length $\ell = 10$ and $\bm{\epsilon} = 0.5$:} 
  In left column, $d'$ is distance of finite or infinite population to limit $\bm{z}^\ast$ for $g$ generations. In right column, $d$ is distance of finite or infinite population to limits $\bm{p}^\ast$ and $\bm{q}^\ast$.}
\label{oscillation_10h_vio_chi_0.5}
\end{center}
\end{figure}

% l = 12
\begin{figure}[h]
\begin{center}
\subfloat{
\resizebox{8cm}{4.5cm}{\includegraphics{figures/eps/vio/chi/b12/e0.5/n00004096_fin_hap.eps}}} \hspace{-3em}%
\subfloat{
\resizebox{8cm}{4.5cm}{\includegraphics{figures/eps/vio/chi/b12/e0.5/n00004096_fin_hap_wovio.eps}}}\vspace{-1em} \hspace{-3em}%
\end{center}
\begin{center}
\subfloat{
\resizebox{8cm}{4.5cm}{\includegraphics{figures/eps/vio/chi/b12/e0.5/n00040960_fin_hap.eps}}} \hspace{-3em}%
\subfloat{
\resizebox{8cm}{4.5cm}{\includegraphics{figures/eps/vio/chi/b12/e0.5/n00040960_fin_hap_wovio.eps}}}\vspace{-1em} \hspace{-3em}%
\end{center}

\begin{center}
\subfloat{
\resizebox{8cm}{4.5cm}{\includegraphics{figures/eps/vio/chi/b12/e0.5/n00081920_fin_hap.eps}}} \hspace{-3em}%
\subfloat{
\resizebox{8cm}{4.5cm}{\includegraphics{figures/eps/vio/chi/b12/e0.5/n00081920_fin_hap_wovio.eps}}}\vspace{-1em} \hspace{-3em}%
\end{center}

\begin{center}
\subfloat{
\resizebox{8cm}{4.5cm}{\includegraphics{figures/eps/vio/chi/b12/e0.5/inf_hap.eps}}}\hspace{-3em}%
\subfloat{
\resizebox{8cm}{4.5cm}{\includegraphics{figures/eps/vio/chi/b12/e0.5/inf_hap_wovio.eps}}}\vspace{-0.5em} \hspace{-3em}%


\caption[\textbf{Infinite and finite haploid population behavior for $\bm{\chi}$ violation, genome length $\ell = 12$ and $\bm{\epsilon} = 0.5$}]{\textbf{Infinite and finite haploid population behavior for $\bm{\chi}$ violation, genome length $\ell = 12$ and $\bm{\epsilon} = 0.5$:} 
  In left column, $d'$ is distance of finite or infinite population to limit $\bm{z}^\ast$ for $g$ generations. In right column, $d$ is distance of finite or infinite population to limits $\bm{p}^\ast$ and $\bm{q}^\ast$.}
\label{oscillation_12h_vio_chi_0.5}
\end{center}
\end{figure}

% l = 14

\begin{figure}[h]
\begin{center}
\subfloat{
\resizebox{8cm}{4.5cm}{\includegraphics{figures/eps/vio/chi/b14/e0.5/n00004096_fin_hap.eps}}} \hspace{-3em}%
\subfloat{
\resizebox{8cm}{4.5cm}{\includegraphics{figures/eps/vio/chi/b14/e0.5/n00004096_fin_hap_wovio.eps}}}\vspace{-1em} \hspace{-3em}%
\end{center}
\begin{center}
\subfloat{
\resizebox{8cm}{4.5cm}{\includegraphics{figures/eps/vio/chi/b14/e0.5/n00040960_fin_hap.eps}}} \hspace{-3em}%
\subfloat{
\resizebox{8cm}{4.5cm}{\includegraphics{figures/eps/vio/chi/b14/e0.5/n00040960_fin_hap_wovio.eps}}}\vspace{-1em} \hspace{-3em}%
\end{center}

\begin{center}
\subfloat{
\resizebox{8cm}{4.5cm}{\includegraphics{figures/eps/vio/chi/b14/e0.5/n00081920_fin_hap.eps}}} \hspace{-3em}%
\subfloat{
\resizebox{8cm}{4.5cm}{\includegraphics{figures/eps/vio/chi/b14/e0.5/n00081920_fin_hap_wovio.eps}}}\vspace{-1em} \hspace{-3em}%
\end{center}

\begin{center}
\subfloat{
\resizebox{8cm}{4.5cm}{\includegraphics{figures/eps/vio/chi/b14/e0.5/inf_hap.eps}}}\hspace{-3em}%
\subfloat{
\resizebox{8cm}{4.5cm}{\includegraphics{figures/eps/vio/chi/b14/e0.5/inf_hap_wovio.eps}}}\vspace{-0.5em} \hspace{-3em}%

\caption[\textbf{Infinite and finite haploid population behavior for $\bm{\chi}$ violation, genome length $\ell = 14$ and $\bm{\epsilon} = 0.5$}]{\textbf{Infinite and finite haploid population behavior for $\bm{\chi}$ violation, genome length $\ell = 14$ and $\bm{\epsilon} = 0.5$:} 
  In left column, $d'$ is distance of finite or infinite population to limit $\bm{z}^\ast$ for $g$ generations. In right column, $d$ is distance of finite or infinite population to limits $\bm{p}^\ast$ and $\bm{q}^\ast$.}
\label{oscillation_14h_vio_chi_0.5}
\end{center}
\end{figure}

\clearpage

The right column in figures \ref{oscillation_8h_vio_chi_0.5} through \ref{oscillation_14h_vio_chi_0.5} 
shows distance of finite and infinite haploid populations to non-violation limits $\bm{p^\ast}$ and $\bm{q^\ast}$ with $\bm{\epsilon} \;=\; 0.5$. 
The graphs indicate oscillating behavior. 
Unlike mutation with violation $\bm{\epsilon} \;=\; 0.5$, oscillation is observed for longer length of generations. 
Finite populations still show some though not very clear oscillations, and then show randomness in behavior as generation progresses. 
Infinite population also oscillates but the oscillation dies out quickly. Randomness in finite population behavior increases 
more than for smaller values of $\bm{\epsilon}$, especially as $\ell$ increases.

The left column of figures \ref{oscillation_8h_vio_chi_0.5} through \ref{oscillation_14h_vio_chi_0.5} 
shows distance of finite and infinite haploid populations to limit $\bm{z^\ast}$ 
(limit with violation in crossover distribution $\bm{\chi}$) when $\bm{\epsilon} \;=\; 0.5$. 
The distance decreases as population size increases, 
and finite population shows behavior similar to infinite population behavior as finite population size grows. 
Average distance data for haploid population in case of violation in $\bm{\chi}$ distribution 
with $\bm{\epsilon} \;=\; 0.5$ for different finite population size $N$ are tabulated in table \ref{distanceChiHapEps0.5}.

\begin{table}[h]
\caption[\textbf{Distance measured for violation in $\bm{\chi}$ with $\bm{\epsilon} \;=\; 0.5$  for haploids}]{\textbf{Distance measured for violation in $\bm{\chi}$ with $\bm{\epsilon} \;=\; 0.5$  for haploids:} $\ell$ is genome length, 
average distance between finite and infinite population is tabulated in the last three columns, and last row is expected single step distance.}
\centering
\begin{tabularx}{0.75\textwidth}{ c *{3}{X}}
\toprule
$\ell$ & $N = 4096$ & $N = 40960$ & $N = 81920$  \\
\midrule
8 & 0.0156	&  0.0051	& 0.0036 \\
10 & 0.0155	&  0.0049	& 0.0035 \\
12 & 0.0157	&  0.0050	& 0.0035 \\
14 & 0.0156	&  0.0049	& 0.0035 \\      
\midrule
$1/\sqrt{N}$ & 0.0156 & 0.0049 & 0.0035 \\
\bottomrule
\end{tabularx}
\label{distanceChiHapEps0.5}
\end{table} 

Table \ref{distanceChiHapEps0.5} shows that the average distance 
between finite and infinite populations approaches the expected single step distance $1/\sqrt{N}$. 










\subsection{Diploid Population $\mathtt{\sim}$ $\epsilon: 0.01$}
% l = 8
\begin{figure}[h]
\begin{center}
\subfloat{
\resizebox{8cm}{5cm}{\includegraphics{figures/eps/vio/chi/b8/e0.01/n00004096_fin_dip.eps}}}\hspace{-3em}%
\subfloat{
\resizebox{8cm}{5cm}{\includegraphics{figures/eps/vio/chi/b8/e0.01/n00004096_fin_dip_wovio.eps}}}\vspace{-1em}  \hspace{-3em}%
\end{center}
\begin{center}
\subfloat{
\resizebox{8cm}{5cm}{\includegraphics{figures/eps/vio/chi/b8/e0.01/n00040960_fin_dip.eps}}}\hspace{-3em}%
\subfloat{
\resizebox{8cm}{5cm}{\includegraphics{figures/eps/vio/chi/b8/e0.01/n00040960_fin_dip_wovio.eps}}}\vspace{-1em}  \hspace{-3em}%
\end{center}


\begin{center}
\subfloat{
\resizebox{8cm}{5cm}{\includegraphics{figures/eps/vio/chi/b8/e0.01/n00081920_fin_dip.eps}}}\hspace{-3em}%
\subfloat{
\resizebox{8cm}{5cm}{\includegraphics{figures/eps/vio/chi/b8/e0.01/n00081920_fin_dip_wovio.eps}}}\vspace{-1em}  \hspace{-3em}%
\end{center}

\begin{center}
\subfloat{
\resizebox{8cm}{5cm}{\includegraphics{figures/eps/vio/chi/b8/e0.01/inf_dip.eps}}}\hspace{-3em}%
\subfloat{
\resizebox{8cm}{5cm}{\includegraphics{figures/eps/vio/chi/b8/e0.01/inf_dip_wovio.eps}}}\vspace{-0.5em}  \hspace{-3em}%


\caption[\textbf{Infinite and finite diploid population oscillation behavior in case of violation in $\bm{\chi}$ for genome length $\ell = 8$ and $\bm{\epsilon} = 0.01$}]{\textbf{Infinite and finite diploid population oscillation behavior in case of violation in $\bm{\chi}$ for genome length $\ell = 8$ and $\bm{\epsilon} = 0.01$:} 
  In left column, $d'$ is distance of finite population of size $n$ or infinite population to limit $\bm{z}^\ast$ for $g$ generations. In right column, $d$ is distance of finite population or infinite population to limits $\bm{p}^\ast$ and $\bm{q}^\ast$ without violation.}
\label{oscillation_8d_vio_chi_0.01}
\end{center}
\end{figure}

% l = 10

\begin{figure}[h]
\begin{center}
\subfloat{
\resizebox{8cm}{5cm}{\includegraphics{figures/eps/vio/chi/b10/e0.01/n00004096_fin_dip.eps}}}\hspace{-3em}%
\subfloat{
\resizebox{8cm}{5cm}{\includegraphics{figures/eps/vio/chi/b10/e0.01/n00004096_fin_dip_wovio.eps}}}\vspace{-1em}  \hspace{-3em}%
\end{center}
\begin{center}
\subfloat{
\resizebox{8cm}{5cm}{\includegraphics{figures/eps/vio/chi/b10/e0.01/n00040960_fin_dip.eps}}}\hspace{-3em}%
\subfloat{
\resizebox{8cm}{5cm}{\includegraphics{figures/eps/vio/chi/b10/e0.01/n00040960_fin_dip_wovio.eps}}}\vspace{-1em}  \hspace{-3em}%
\end{center}


\begin{center}
\subfloat{
\resizebox{8cm}{5cm}{\includegraphics{figures/eps/vio/chi/b10/e0.01/n00081920_fin_dip.eps}}}\hspace{-3em}%
\subfloat{
\resizebox{8cm}{5cm}{\includegraphics{figures/eps/vio/chi/b10/e0.01/n00081920_fin_dip_wovio.eps}}}\vspace{-1em}  \hspace{-3em}%
\end{center}

\begin{center}
\subfloat{
\resizebox{8cm}{5cm}{\includegraphics{figures/eps/vio/chi/b10/e0.01/inf_dip.eps}}}\hspace{-3em}%
\subfloat{
\resizebox{8cm}{5cm}{\includegraphics{figures/eps/vio/chi/b10/e0.01/inf_dip_wovio.eps}}}\vspace{-0.5em}  \hspace{-3em}%


\caption[\textbf{Infinite and finite diploid population oscillation behavior in case of violation in $\bm{\chi}$ for genome length $\ell = 10$ and $\bm{\epsilon} = 0.01$}]{\textbf{Infinite and finite diploid population oscillation behavior in case of violation in $\bm{\chi}$ for genome length $\ell = 10$ and $\bm{\epsilon} = 0.01$:} 
  In left column, $d'$ is distance of finite population of size $n$ or infinite population to limit $\bm{z}^\ast$ for $g$ generations. In right column, $d$ is distance of finite population or infinite population to limits $\bm{p}^\ast$ and $\bm{q}^\ast$ without violation.}
\label{oscillation_10d_vio_chi_0.01}
\end{center}
\end{figure}

% l = 12

\begin{figure}[h]
\begin{center}
\subfloat{
\resizebox{8cm}{5cm}{\includegraphics{figures/eps/vio/chi/b12/e0.01/n00004096_fin_dip.eps}}}\hspace{-3em}%
\subfloat{
\resizebox{8cm}{5cm}{\includegraphics{figures/eps/vio/chi/b12/e0.01/n00004096_fin_dip_wovio.eps}}}\vspace{-1em}  \hspace{-3em}%
\end{center}
\begin{center}
\subfloat{
\resizebox{8cm}{5cm}{\includegraphics{figures/eps/vio/chi/b12/e0.01/n00040960_fin_dip.eps}}}\hspace{-3em}%
\subfloat{
\resizebox{8cm}{5cm}{\includegraphics{figures/eps/vio/chi/b12/e0.01/n00040960_fin_dip_wovio.eps}}}\vspace{-1em}  \hspace{-3em}%
\end{center}


\begin{center}
\subfloat{
\resizebox{8cm}{5cm}{\includegraphics{figures/eps/vio/chi/b12/e0.01/n00081920_fin_dip.eps}}}\hspace{-3em}%
\subfloat{
\resizebox{8cm}{5cm}{\includegraphics{figures/eps/vio/chi/b12/e0.01/n00081920_fin_dip_wovio.eps}}}\vspace{-1em}  \hspace{-3em}%
\end{center}

\begin{center}
\subfloat{
\resizebox{8cm}{5cm}{\includegraphics{figures/eps/vio/chi/b12/e0.01/inf_dip.eps}}}\hspace{-3em}%
\subfloat{
\resizebox{8cm}{5cm}{\includegraphics{figures/eps/vio/chi/b12/e0.01/inf_dip_wovio.eps}}}\vspace{-0.5em}  \hspace{-3em}%


\caption[\textbf{Infinite and finite diploid population oscillation behavior in case of violation in $\bm{\chi}$ for genome length $\ell = 12$ and $\bm{\epsilon} = 0.01$}]{\textbf{Infinite and finite diploid population oscillation behavior in case of violation in $\bm{\chi}$ for genome length $\ell = 12$ and $\bm{\epsilon} = 0.01$:} 
  In left column, $d'$ is distance of finite population of size $n$ or infinite population to limit $\bm{z}^\ast$ for $g$ generations. In right column, $d$ is distance of finite population or infinite population to limits $\bm{p}^\ast$ and $\bm{q}^\ast$ without violation.}
\label{oscillation_12d_vio_chi_0.01}
\end{center}
\end{figure}

% l = 14

\begin{figure}[h]
\begin{center}
\subfloat{
\resizebox{8cm}{5cm}{\includegraphics{figures/eps/vio/chi/b14/e0.01/n00004096_fin_dip.eps}}}\hspace{-3em}%
\subfloat{
\resizebox{8cm}{5cm}{\includegraphics{figures/eps/vio/chi/b14/e0.01/n00004096_fin_dip_wovio.eps}}}\vspace{-1em}  \hspace{-3em}%
\end{center}
\begin{center}
\subfloat{
\resizebox{8cm}{5cm}{\includegraphics{figures/eps/vio/chi/b14/e0.01/n00040960_fin_dip.eps}}}\hspace{-3em}%
\subfloat{
\resizebox{8cm}{5cm}{\includegraphics{figures/eps/vio/chi/b14/e0.01/n00040960_fin_dip_wovio.eps}}}\vspace{-1em}  \hspace{-3em}%
\end{center}


\begin{center}
\subfloat{
\resizebox{8cm}{5cm}{\includegraphics{figures/eps/vio/chi/b14/e0.01/n00081920_fin_dip.eps}}}\hspace{-3em}%
\subfloat{
\resizebox{8cm}{5cm}{\includegraphics{figures/eps/vio/chi/b14/e0.01/n00081920_fin_dip_wovio.eps}}}\vspace{-1em}  \hspace{-3em}%
\end{center}

\begin{center}
\subfloat{
\resizebox{8cm}{5cm}{\includegraphics{figures/eps/vio/chi/b14/e0.01/inf_dip.eps}}}\hspace{-3em}%
\subfloat{
\resizebox{8cm}{5cm}{\includegraphics{figures/eps/vio/chi/b14/e0.01/inf_dip_wovio.eps}}}\vspace{-0.5em}  \hspace{-3em}%


\caption[\textbf{Infinite and finite diploid population oscillation behavior in case of violation in $\bm{\chi}$ for genome length $\ell = 14$ and $\bm{\epsilon} = 0.01$}]{\textbf{Infinite and finite diploid population oscillation behavior in case of violation in $\bm{\chi}$ for genome length $\ell = 14$ and $\bm{\epsilon} = 0.01$:} 
  In left column, $d'$ is distance of finite population of size $n$ or infinite population to limit $\bm{z}^\ast$ for $g$ generations. In right column, $d$ is distance of finite population or infinite population to limits $\bm{p}^\ast$ and $\bm{q}^\ast$ without violation.}
\label{oscillation_14d_vio_chi_0.01}
\end{center}
\end{figure}

\clearpage

The right column in figures \ref{oscillation_8d_vio_chi_0.01} through \ref{oscillation_14d_vio_chi_0.01} 
shows distance of finite and infinite diploid populations to non-violation limits $\bm{p^\ast}$ and $\bm{q^\ast}$ with $\bm{\epsilon} \;=\; 0.01$. 
Those graphs indicate oscillating behavior of diploid population given violation. 
Both finite and infinite populations oscillate given violation. Like in haploid case, oscillations are sharper. Since value of $\bm{\epsilon}$ 
is small, damping of ripples is slow. A new mask introduced in crossover distribution with $\bm{\epsilon} \;=\; 0.01$ has small 
probability of being used during crossover. Infinite population oscillation does not die out in 50 generations. As value of $\ell$ 
increases, random drifting and wiggling of finite population occurs for small population size, and oscillation gets better with increase in population size. 
These are noticed more clearly in figures \ref{oscillation_12d_vio_chi_0.01} and \ref{oscillation_14d_vio_chi_0.01}.

The left column of figures \ref{oscillation_8d_vio_chi_0.01} through \ref{oscillation_14d_vio_chi_0.01} 
shows distance of finite and infinite diploid populations to limit $\bm{z^\ast}$ 
(limit with violation in crossover distribution $\bm{\chi}$) when $\bm{\epsilon} \;=\; 0.01$. 
The distance between finite population and limit $\bm{z}^\ast$ (limit with violation in $\bm{\chi}$ distribution) 
decreases as finite population size increases. 
Average distance data for diploid population in case of violation in $\bm{\chi}$ distribution 
with $\bm{\epsilon} \;=\; 0.01$ for different finite population size $N$ are tabulated in table \ref{distanceChiDipEps0.01}.


\begin{table}[ht]
\caption[\textbf{Distance measured for violation in $\bm{\chi}$ with $\bm{\epsilon} \;=\; 0.01$ diploids}]{\textbf{Distance measured for violation in $\bm{\chi}$ with $\bm{\epsilon} \;=\; 0.01$ diploids:} $\ell$ is genome length, 
average distance between finite and infinite population is tabulated in the last three columns, and last row is expected single step distance.}
\centering
\begin{tabularx}{0.75\textwidth}{ c *{3}{X}}
\toprule
$\ell$ & $N = 4096$ & $N = 40960$ & $N = 81920$  \\
\midrule
8 & 0.0156	&  0.0051	& 0.0036 \\
10 & 0.0156	&  0.0049	& 0.0035 \\
12 & 0.0156	&  0.0049	& 0.0035 \\
14 & 0.0156	&  0.0049	& 0.0035 \\
\midrule
$1/\sqrt{N}$ & 0.0156 & 0.0049 & 0.0035 \\
\bottomrule
\end{tabularx}
\label{distanceChiDipEps0.01}
\end{table} 

The results from Table \ref{distanceChiDipEps0.01} show average distance 
between finite population and limit $\bm{z^\ast}$ approaches the expected single step distance 
between finite and infinite population. The distance decreased as $1/\sqrt{N}$. 
Also, the distance is smaller in diploid populations than in haploid populations with $\bm{\epsilon} \;=\; 0.01$.



\subsection{Diploid Population $\mathtt{\sim}$ $\epsilon: 0.1$}
% l = 8
\begin{figure}[H]
\begin{center}
\subfloat{
\resizebox{8cm}{5cm}{\includegraphics{figures/eps/vio/chi/b8/e0.1/n00004096_fin_dip.eps}}}\hspace{-3em}%
\subfloat{
\resizebox{8cm}{5cm}{\includegraphics{figures/eps/vio/chi/b8/e0.1/n00004096_fin_dip_wovio.eps}}}\vspace{-1em}  \hspace{-3em}%
\end{center}
\begin{center}
\subfloat{
\resizebox{8cm}{5cm}{\includegraphics{figures/eps/vio/chi/b8/e0.1/n00040960_fin_dip.eps}}}\hspace{-3em}%
\subfloat{
\resizebox{8cm}{5cm}{\includegraphics{figures/eps/vio/chi/b8/e0.1/n00040960_fin_dip_wovio.eps}}}\vspace{-1em}  \hspace{-3em}%
\end{center}


\begin{center}
\subfloat{
\resizebox{8cm}{5cm}{\includegraphics{figures/eps/vio/chi/b8/e0.1/n00081920_fin_dip.eps}}}\hspace{-3em}%
\subfloat{
\resizebox{8cm}{5cm}{\includegraphics{figures/eps/vio/chi/b8/e0.1/n00081920_fin_dip_wovio.eps}}}\vspace{-1em}  \hspace{-3em}%
\end{center}

\begin{center}
\subfloat{
\resizebox{8cm}{5cm}{\includegraphics{figures/eps/vio/chi/b8/e0.1/inf_dip.eps}}}\hspace{-3em}%
\subfloat{
\resizebox{8cm}{5cm}{\includegraphics{figures/eps/vio/chi/b8/e0.1/inf_dip_wovio.eps}}}\vspace{-0.5em}  \hspace{-3em}%


\caption{\textbf{Infinite and finite diploid population oscillation behavior in case of violation in $\bm{\chi}$ for genome length $\ell = 8$ and $\bm{\epsilon} = 0.1$:} 
  In left column, $d'$ is distance of finite population of size $n$ or infinite population to limit $\bm{z}^\ast$ for $g$ generations. In right column, $d$ is distance of finite population of size $N$ or infinite population to limits without violation.}
\label{oscillation_8d_vio_chi_0.1}
\end{center}
\end{figure}

% l = 10

\begin{figure}[H]
\begin{center}
\subfloat{
\resizebox{8cm}{5cm}{\includegraphics{figures/eps/vio/chi/b10/e0.1/n00004096_fin_dip.eps}}}\hspace{-3em}%
\subfloat{
\resizebox{8cm}{5cm}{\includegraphics{figures/eps/vio/chi/b10/e0.1/n00004096_fin_dip_wovio.eps}}}\vspace{-1em}  \hspace{-3em}%
\end{center}
\begin{center}
\subfloat{
\resizebox{8cm}{5cm}{\includegraphics{figures/eps/vio/chi/b10/e0.1/n00040960_fin_dip.eps}}}\hspace{-3em}%
\subfloat{
\resizebox{8cm}{5cm}{\includegraphics{figures/eps/vio/chi/b10/e0.1/n00040960_fin_dip_wovio.eps}}}\vspace{-1em}  \hspace{-3em}%
\end{center}


\begin{center}
\subfloat{
\resizebox{8cm}{5cm}{\includegraphics{figures/eps/vio/chi/b10/e0.1/n00081920_fin_dip.eps}}}\hspace{-3em}%
\subfloat{
\resizebox{8cm}{5cm}{\includegraphics{figures/eps/vio/chi/b10/e0.1/n00081920_fin_dip_wovio.eps}}}\vspace{-1em}  \hspace{-3em}%
\end{center}

\begin{center}
\subfloat{
\resizebox{8cm}{5cm}{\includegraphics{figures/eps/vio/chi/b10/e0.1/inf_dip.eps}}}\hspace{-3em}%
\subfloat{
\resizebox{8cm}{5cm}{\includegraphics{figures/eps/vio/chi/b10/e0.1/inf_dip_wovio.eps}}}\vspace{-0.5em}  \hspace{-3em}%


\caption{\textbf{Infinite and finite diploid population oscillation behavior in case of violation in $\bm{\chi}$ for genome length $\ell = 10$ and $\bm{\epsilon} = 0.1$:} 
  In left column, $d'$ is distance of finite population of size $n$ or infinite population to limit $\bm{z}^\ast$ for $g$ generations. In right column, $d$ is distance of finite population of size $N$ or infinite population to limits without violation.}
\label{oscillation_10d_vio_chi_0.1}
\end{center}
\end{figure}

% l = 12

\begin{figure}[H]
\begin{center}
\subfloat{
\resizebox{8cm}{5cm}{\includegraphics{figures/eps/vio/chi/b12/e0.1/n00004096_fin_dip.eps}}}\hspace{-3em}%
\subfloat{
\resizebox{8cm}{5cm}{\includegraphics{figures/eps/vio/chi/b12/e0.1/n00004096_fin_dip_wovio.eps}}}\vspace{-1em}  \hspace{-3em}%
\end{center}
\begin{center}
\subfloat{
\resizebox{8cm}{5cm}{\includegraphics{figures/eps/vio/chi/b12/e0.1/n00040960_fin_dip.eps}}}\hspace{-3em}%
\subfloat{
\resizebox{8cm}{5cm}{\includegraphics{figures/eps/vio/chi/b12/e0.1/n00040960_fin_dip_wovio.eps}}}\vspace{-1em}  \hspace{-3em}%
\end{center}


\begin{center}
\subfloat{
\resizebox{8cm}{5cm}{\includegraphics{figures/eps/vio/chi/b12/e0.1/n00081920_fin_dip.eps}}}\hspace{-3em}%
\subfloat{
\resizebox{8cm}{5cm}{\includegraphics{figures/eps/vio/chi/b12/e0.1/n00081920_fin_dip_wovio.eps}}}\vspace{-1em}  \hspace{-3em}%
\end{center}

\begin{center}
\subfloat{
\resizebox{8cm}{5cm}{\includegraphics{figures/eps/vio/chi/b12/e0.1/inf_dip.eps}}}\hspace{-3em}%
\subfloat{
\resizebox{8cm}{5cm}{\includegraphics{figures/eps/vio/chi/b12/e0.1/inf_dip_wovio.eps}}}\vspace{-0.5em}  \hspace{-3em}%


\caption{\textbf{Infinite and finite diploid population oscillation behavior in case of violation in $\bm{\chi}$ for genome length $\ell = 12$ and $\bm{\epsilon} = 0.1$:} 
  In left column, $d'$ is distance of finite population of size $n$ or infinite population to limit $\bm{z}^\ast$ for $g$ generations. In right column, $d$ is distance of finite population of size $N$ or infinite population to limits without violation.}
\label{oscillation_12d_vio_chi_0.1}
\end{center}
\end{figure}

% l = 14

\begin{figure}[H]
\begin{center}
\subfloat{
\resizebox{8cm}{5cm}{\includegraphics{figures/eps/vio/chi/b14/e0.1/n00004096_fin_dip.eps}}}\hspace{-3em}%
\subfloat{
\resizebox{8cm}{5cm}{\includegraphics{figures/eps/vio/chi/b14/e0.1/n00004096_fin_dip_wovio.eps}}}\vspace{-1em}  \hspace{-3em}%
\end{center}
\begin{center}
\subfloat{
\resizebox{8cm}{5cm}{\includegraphics{figures/eps/vio/chi/b14/e0.1/n00040960_fin_dip.eps}}}\hspace{-3em}%
\subfloat{
\resizebox{8cm}{5cm}{\includegraphics{figures/eps/vio/chi/b14/e0.1/n00040960_fin_dip_wovio.eps}}}\vspace{-1em}  \hspace{-3em}%
\end{center}


\begin{center}
\subfloat{
\resizebox{8cm}{5cm}{\includegraphics{figures/eps/vio/chi/b14/e0.1/n00081920_fin_dip.eps}}}\hspace{-3em}%
\subfloat{
\resizebox{8cm}{5cm}{\includegraphics{figures/eps/vio/chi/b14/e0.1/n00081920_fin_dip_wovio.eps}}}\vspace{-1em}  \hspace{-3em}%
\end{center}

\begin{center}
\subfloat{
\resizebox{8cm}{5cm}{\includegraphics{figures/eps/vio/chi/b14/e0.1/inf_dip.eps}}}\hspace{-3em}%
\subfloat{
\resizebox{8cm}{5cm}{\includegraphics{figures/eps/vio/chi/b14/e0.1/inf_dip_wovio.eps}}}\vspace{-0.5em}  \hspace{-3em}%


\caption{\textbf{Infinite and finite diploid population oscillation behavior in case of violation in $\bm{\chi}$ for genome length $\ell = 14$ and $\bm{\epsilon} = 0.1$:} 
  In left column, $d'$ is distance of finite population of size $n$ or infinite population to limit $\bm{z}^\ast$ for $g$ generations. In right column, $d$ is distance of finite population of size $N$ or infinite population to limits without violation.}
\label{oscillation_14d_vio_chi_0.1}
\end{center}
\end{figure}









\begin{table}[ht]
\caption{\textbf{Distance measured for violation in $\bm{\chi}$ with $\bm{\epsilon} \;=\; 0.1$ for diploids:} $\ell$ is genome length, 
and average distance between finite and 
infinite populations is tabulated in last three columns.}
\centering
\begin{tabularx}{0.75\textwidth}{ c *{3}{X}}
\toprule
$\ell$ & $N = 4096$ & $N = 40960$ & $N = 81920$  \\
\midrule
8 & 0.0156	&  0.0050	& 0.0035 \\
10 & 0.0156	&  0.0049	& 0.0035 \\
12 & 0.0156	&  0.0049	& 0.0035 \\
14 & 0.0156	&  0.0049	& 0.0035 \\
\bottomrule
\end{tabularx}
\label{distanceChiDipEps0.1}
\end{table} 



 

\subsection{Diploid Population $\mathtt{\sim}$ $\epsilon: 0.5$}
% l = 8
\begin{figure}[h]
\begin{center}
\subfloat{
\resizebox{8cm}{5cm}{\includegraphics{figures/eps/vio/chi/b8/e0.5/n00004096_fin_dip.eps}}}\hspace{-3em}%
\subfloat{
\resizebox{8cm}{5cm}{\includegraphics{figures/eps/vio/chi/b8/e0.5/n00004096_fin_dip_wovio.eps}}}\vspace{-1em}  \hspace{-3em}%
\end{center}
\begin{center}
\subfloat{
\resizebox{8cm}{5cm}{\includegraphics{figures/eps/vio/chi/b8/e0.5/n00040960_fin_dip.eps}}}\hspace{-3em}%
\subfloat{
\resizebox{8cm}{5cm}{\includegraphics{figures/eps/vio/chi/b8/e0.5/n00040960_fin_dip_wovio.eps}}}\vspace{-1em}  \hspace{-3em}%
\end{center}


\begin{center}
\subfloat{
\resizebox{8cm}{5cm}{\includegraphics{figures/eps/vio/chi/b8/e0.5/n00081920_fin_dip.eps}}}\hspace{-3em}%
\subfloat{
\resizebox{8cm}{5cm}{\includegraphics{figures/eps/vio/chi/b8/e0.5/n00081920_fin_dip_wovio.eps}}}\vspace{-1em}  \hspace{-3em}%
\end{center}

\begin{center}
\subfloat{
\resizebox{8cm}{5cm}{\includegraphics{figures/eps/vio/chi/b8/e0.5/inf_dip.eps}}}\hspace{-3em}%
\subfloat{
\resizebox{8cm}{5cm}{\includegraphics{figures/eps/vio/chi/b8/e0.5/inf_dip_wovio.eps}}}\vspace{-0.5em}  \hspace{-3em}%


\caption{\textbf{Infinite and finite diploid population oscillation behavior in case of violation in $\bm{\chi}$ for genome length $\ell = 8$ and $\bm{\epsilon} = 0.5$:} 
  In left column, $d'$ is distance of finite population of size $n$ or infinite population to limit $\bm{z}^\ast$ for $g$ generations. In right column, $d$ is distance of finite population or infinite population to limits $\bm{p}^\ast$ and $\bm{q}^\ast$ without violation.}
\label{oscillation_8d_vio_chi_0.5}
\end{center}
\end{figure}

% l = 10

\begin{figure}[h]
\begin{center}
\subfloat{
\resizebox{8cm}{5cm}{\includegraphics{figures/eps/vio/chi/b10/e0.5/n00004096_fin_dip.eps}}}\hspace{-3em}%
\subfloat{
\resizebox{8cm}{5cm}{\includegraphics{figures/eps/vio/chi/b10/e0.5/n00004096_fin_dip_wovio.eps}}}\vspace{-1em}  \hspace{-3em}%
\end{center}
\begin{center}
\subfloat{
\resizebox{8cm}{5cm}{\includegraphics{figures/eps/vio/chi/b10/e0.5/n00040960_fin_dip.eps}}}\hspace{-3em}%
\subfloat{
\resizebox{8cm}{5cm}{\includegraphics{figures/eps/vio/chi/b10/e0.5/n00040960_fin_dip_wovio.eps}}}\vspace{-1em}  \hspace{-3em}%
\end{center}


\begin{center}
\subfloat{
\resizebox{8cm}{5cm}{\includegraphics{figures/eps/vio/chi/b10/e0.5/n00081920_fin_dip.eps}}}\hspace{-3em}%
\subfloat{
\resizebox{8cm}{5cm}{\includegraphics{figures/eps/vio/chi/b10/e0.5/n00081920_fin_dip_wovio.eps}}}\vspace{-1em}  \hspace{-3em}%
\end{center}

\begin{center}
\subfloat{
\resizebox{8cm}{5cm}{\includegraphics{figures/eps/vio/chi/b10/e0.5/inf_dip.eps}}}\hspace{-3em}%
\subfloat{
\resizebox{8cm}{5cm}{\includegraphics{figures/eps/vio/chi/b10/e0.5/inf_dip_wovio.eps}}}\vspace{-0.5em}  \hspace{-3em}%


\caption{\textbf{Infinite and finite diploid population oscillation behavior in case of violation in $\bm{\chi}$ for genome length $\ell = 10$ and $\bm{\epsilon} = 0.5$:} 
  In left column, $d'$ is distance of finite population of size $n$ or infinite population to limit $\bm{z}^\ast$ for $g$ generations. In right column, $d$ is distance of finite population or infinite population to limits $\bm{p}^\ast$ and $\bm{q}^\ast$ without violation.}
\label{oscillation_10d_vio_chi_0.5}
\end{center}
\end{figure}

% l = 12

\begin{figure}[h]
\begin{center}
\subfloat{
\resizebox{8cm}{5cm}{\includegraphics{figures/eps/vio/chi/b12/e0.5/n00004096_fin_dip.eps}}}\hspace{-3em}%
\subfloat{
\resizebox{8cm}{5cm}{\includegraphics{figures/eps/vio/chi/b12/e0.5/n00004096_fin_dip_wovio.eps}}}\vspace{-1em}  \hspace{-3em}%
\end{center}
\begin{center}
\subfloat{
\resizebox{8cm}{5cm}{\includegraphics{figures/eps/vio/chi/b12/e0.5/n00040960_fin_dip.eps}}}\hspace{-3em}%
\subfloat{
\resizebox{8cm}{5cm}{\includegraphics{figures/eps/vio/chi/b12/e0.5/n00040960_fin_dip_wovio.eps}}}\vspace{-1em}  \hspace{-3em}%
\end{center}


\begin{center}
\subfloat{
\resizebox{8cm}{5cm}{\includegraphics{figures/eps/vio/chi/b12/e0.5/n00081920_fin_dip.eps}}}\hspace{-3em}%
\subfloat{
\resizebox{8cm}{5cm}{\includegraphics{figures/eps/vio/chi/b12/e0.5/n00081920_fin_dip_wovio.eps}}}\vspace{-1em}  \hspace{-3em}%
\end{center}

\begin{center}
\subfloat{
\resizebox{8cm}{5cm}{\includegraphics{figures/eps/vio/chi/b12/e0.5/inf_dip.eps}}}\hspace{-3em}%
\subfloat{
\resizebox{8cm}{5cm}{\includegraphics{figures/eps/vio/chi/b12/e0.5/inf_dip_wovio.eps}}}\vspace{-0.5em}  \hspace{-3em}%


\caption{\textbf{Infinite and finite diploid population oscillation behavior in case of violation in $\bm{\chi}$ for genome length $\ell = 12$ and $\bm{\epsilon} = 0.5$:} 
  In left column, $d'$ is distance of finite population of size $n$ or infinite population to limit $\bm{z}^\ast$ for $g$ generations. In right column, $d$ is distance of finite population or infinite population to limits $\bm{p}^\ast$ and $\bm{q}^\ast$ without violation.}
\label{oscillation_12d_vio_chi_0.5}
\end{center}
\end{figure}

% l = 14

\begin{figure}[h]
\begin{center}
\subfloat{
\resizebox{8cm}{5cm}{\includegraphics{figures/eps/vio/chi/b14/e0.5/n00004096_fin_dip.eps}}}\hspace{-3em}%
\subfloat{
\resizebox{8cm}{5cm}{\includegraphics{figures/eps/vio/chi/b14/e0.5/n00004096_fin_dip_wovio.eps}}}\vspace{-1em}  \hspace{-3em}%
\end{center}
\begin{center}
\subfloat{
\resizebox{8cm}{5cm}{\includegraphics{figures/eps/vio/chi/b14/e0.5/n00040960_fin_dip.eps}}}\hspace{-3em}%
\subfloat{
\resizebox{8cm}{5cm}{\includegraphics{figures/eps/vio/chi/b14/e0.5/n00040960_fin_dip_wovio.eps}}}\vspace{-1em}  \hspace{-3em}%
\end{center}


\begin{center}
\subfloat{
\resizebox{8cm}{5cm}{\includegraphics{figures/eps/vio/chi/b14/e0.5/n00081920_fin_dip.eps}}}\hspace{-3em}%
\subfloat{
\resizebox{8cm}{5cm}{\includegraphics{figures/eps/vio/chi/b14/e0.5/n00081920_fin_dip_wovio.eps}}}\vspace{-1em}  \hspace{-3em}%
\end{center}

\begin{center}
\subfloat{
\resizebox{8cm}{5cm}{\includegraphics{figures/eps/vio/chi/b14/e0.5/inf_dip.eps}}}\hspace{-3em}%
\subfloat{
\resizebox{8cm}{5cm}{\includegraphics{figures/eps/vio/chi/b14/e0.5/inf_dip_wovio.eps}}}\vspace{-0.5em}  \hspace{-3em}%


\caption{\textbf{Infinite and finite diploid population oscillation behavior in case of violation in $\bm{\chi}$ for genome length $\ell = 14$ and $\bm{\epsilon} = 0.5$:} 
  In left column, $d'$ is distance of finite population of size $n$ or infinite population to limit $\bm{z}^\ast$ for $g$ generations. In right column, $d$ is distance of finite population or infinite population to limits $\bm{p}^\ast$ and $\bm{q}^\ast$ without violation.}
\label{oscillation_14d_vio_chi_0.5}
\end{center}
\end{figure}

\clearpage

The right column in figures \ref{oscillation_8d_vio_chi_0.5} through \ref{oscillation_14d_vio_chi_0.5} 
shows distance of finite and infinite diploid populations to non-violation limits $\bm{p^\ast}$ and $\bm{q^\ast}$ with $\bm{\epsilon} \;=\; 0.5$. 
Those graphs indicate oscillating behavior of diploid population given violation. 
Infinite populations oscillate for some generations, and then cease to oscillate given violation. 
Finite populations show some oscillations when $\ell \;=\; 8$ for higher population size as in figure \ref{oscillation_8d_vio_chi_0.5}, but 
for larger $\ell$, finite populations show only random wiggles. 

The left column of figures \ref{oscillation_8d_vio_chi_0.5} through \ref{oscillation_14d_vio_chi_0.5} 
shows distance of finite and infinite diploid populations to limit $\bm{z^\ast}$ 
(limit with violation in crossover distribution $\bm{\chi}$) when $\bm{\epsilon} \;=\; 0.5$. 
The distance between finite population and limit $\bm{z}^\ast$ (limit with violation in $\bm{\chi}$ distribution) 
decreases as finite population size increases.
The distance data for diploid population in case of violation in $\bm{\chi}$ distribution 
with $\bm{\epsilon} \;=\; 0.5$ for different finite population size $N$ are tabulated in table \ref{distanceChiDipEps0.5}.


\begin{table}[ht]
\caption{\textbf{Distance measured for violation in $\bm{\chi}$ with $\bm{\epsilon} \;=\; 0.5$ for diploids:} $\ell$ is genome length, 
average distance between finite and infinite population is tabulated in the last three columns, and last row is expected single step distance.}
\centering
\begin{tabularx}{0.75\textwidth}{ c *{3}{X}}
\toprule
$\ell$ & $N = 4096$ & $N = 40960$ & $N = 81920$  \\
\midrule
8 & 0.0156	&  0.0049	& 0.0035 \\	
10 & 0.0156	&  0.0049	& 0.0035 \\
12 & 0.0156	&  0.0049	& 0.0035 \\
14 & 0.0156	&  0.0049	& 0.0035 \\
\midrule
$1/\sqrt{N}$ & 0.0156 & 0.0049 & 0.0035 \\
\bottomrule
\end{tabularx}
\label{distanceChiDipEps0.5}
\end{table} 

From table \ref{distanceChiDipEps0.5}, average distance calculated for finite population size $4096$ is $0.0156$, 
for size $40960$ is $0.0049$ and for size $81920$ is $0.0035$. These results show average distance 
between finite population and limit $\bm{z^\ast}$ closely follows expected single step distance 
between finite and infinite population. The distance decreased as $1/\sqrt{N}$. 
Also, the distance is smaller in diploid populations than in haploid populations with $\bm{\epsilon} \;=\; 0.5$.
 


\section{Discussion}

In oscillation of population in presence of violation (in either case $\bm{\mu}$ or $\bm{\chi}$), 
as value of $\ell$ increases, amplitude of oscillation decreases. 
Populations with larger population size, show better oscillations. 
Since diploid population consists of string length twice the size of string length of hapliod population, 
diploid population needs larger population size than haploid population to exhibit good oscillations. 
In both cases of violation in $\bm{\mu}$ and $\bm{\chi}$, for diploid population, increasing string length $\ell$ 
degrades convergence (as finite population size increases) to infinite population behavior. The behavior is noticable in figures 
\ref{oscillation_8d_vio_mu_0.01} through \ref{oscillation_14d_vio_mu_0.1} for violation in $\bm{\mu}$, 
and in figures \ref{oscillation_8d_vio_chi_0.01} through \ref{oscillation_14d_vio_chi_0.1} for violation in $\bm{\chi}$. 
The behavior is less noticable in haploid population.

With increase in value of $\bm{\epsilon}$, 
oscillation in population diminishes, and oscillation completely ceases after certain threshold value for $\bm{\epsilon}$. 
Comparing oscillation with violation in $\bm{\mu}$ and $\bm{\chi}$, rate of dampening of oscillation with violation 
in $\bm{\chi}$ looks to be less than with violation in ${\bm{\mu}}$, 
and so we see some oscillation even for $\bm{\epsilon} = 0.5$ for haploid population in this case. 
Diploid populations jumping to other levels 
were observed for string length $\ell$ of values 12 and 14 for population size of 4096 in 
figures \ref{oscillation_12d_vio_chi_0.01}, \ref{oscillation_14d_vio_chi_0.01}, \ref{oscillation_12d_vio_chi_0.1}, 
\ref{oscillation_14d_vio_chi_0.1}, \ref{oscillation_12d_vio_chi_0.5} and \ref{oscillation_14d_vio_chi_0.5}, 
but unlike in case of violation in $\bm{\mu}$, the behavior is not completely absent when population size was larger. 
Random wiggles are seen in populations with larger population size also for $\ell$ = 12 and 14. 
Jumping to other levels are observed for string length $\ell = 10$ when population size is 4096 in figures \ref{oscillation_10d_vio_chi_0.01} 
and \ref{oscillation_10d_vio_chi_0.01}.
For $\ell = 10$, random wiggles are 
noticeable for larger population also when $\bm{\epsilon}$ value is increased to $0.5$ (see figure \ref{oscillation_10d_vio_chi_0.5}).


\begin{figure}[h]
\begin{center}
\subfloat{
\resizebox{16cm}{10cm}{\includegraphics{figures/eps/vio/dist_mu.eps}}}\hspace{-3em}%
\caption[\textbf{Distance of finite population to infinite population in case of violation in $\bm{\mu}$}]{\textbf{Distance of finite population to infinite population in case of violation in $\bm{\mu}$:}  
  $d$ is distance; $N$ is finite population size; $\bm{\epsilon}$ is level of violation;
  red line represents distance for $\ell = 8$, green line for $\ell = 10$, blue line for $\ell = 12$, pink line for $\ell = 14$ 
  and black dotted line for expected single step distance.}
\label{vio_mu_dist}
\end{center}
\end{figure}
\begin{figure}[h]
\begin{center}
\subfloat{
\resizebox{16cm}{10cm}{\includegraphics{figures/eps/vio/dist_chi.eps}}}\hspace{-3em}%
\caption[\textbf{Distance of finite population to infinite population in case of violation in $\bm{\chi}$}]{\textbf{Distance of finite population to infinite population in case of violation in $\bm{\chi}$:}  
  $d$ is distance; $N$ is finite population size; $\bm{\epsilon}$ is level of violation; 
  red line represents distance for $\ell = 8$, green line for $\ell = 10$, blue line for $\ell = 12$, pink line for $\ell = 14$ 
  and black dotted line for expected single step distance.}
\label{vio_chi_dist}
\end{center}
\end{figure}

Figures \ref{vio_mu_dist} and \ref{vio_chi_dist} summarizes the distance data from tables \ref{distanceMuHapEps0.01} $\cdots$ 
\ref{distanceChiDipEps0.5}. Distance of infinite population from finite population of 
three finite population sizes {4096, 40960, 81920} are plotted for different $\ell$. 
Plots for different violation levels $\bm{\epsilon}$ are arranged in columns. 
Plots for haploid and diploid populations are arranged in two rows. With increase in $\ell$, 
distance moves closer to the single step distance. So, since diploid population 
string length is twice of haploid population, 
distance in diploid case moves closer to the single step distance than in haploid case. 
It is also noticable in haploid population case that as $\bm{\epsilon}$ increases, 
the distance moves closer to the single step distance.

\clearpage
\section{Summary}
In this chapter, we violated the condition \ref{OscCond} through violation in mutation and crossover distribution, and 
studied infinite and finite populations oscillation behavior with the violation through experiments. 
Infinite population ceases to oscillate when the condition \ref{OscCond} for convergence to 
periodic orbits is violated, but finite population continued to approximately oscillate for small values of $\bm{\epsilon}$. 
For smaller values of $\bm{\epsilon}$, finite population does not get aware of violation because the probability of using 
new mask created in the mutation distribution $\bm{\mu}$ and the crossover distribution $\bm{\mu}$ due to violation is very low, and 
if no new mask is used, finite population follows behavior of infinite population without violation in the condition for convergence to 
periodic orbits.
