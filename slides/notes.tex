\documentclass{article}
\renewcommand{\baselinestretch}{1.5}     % line Spacing
%%%%%%%%%%%%%%%%%%%%%%%%%%%%%%%%%%%%%%%%%%%%%%%%%%%%%%%%%%%%%%%%%%%%%%%%%%%%%%%%%%%%%%%%%%%%%%%%%%%%%
% LOAD SOME USEFUL PACKAGES
%%%%%%%%%%%%%%%%%%%%%%%%%%%%%%%%%%%%%%%%%%%%%%%%%%%%%%%%%%%%%%%%%%%%%%%%%%%%%%%%%%%%%%%%%%%%%%%%%%%%%
\usepackage{nomencl}                    % produces a nomenclature
\usepackage{float}                      % figure floats
\usepackage{natbib}                     % this package allows you to link your references
\usepackage{graphicx}                                   % graphics package
\graphicspath{ {figures/}{figures/eps/}{figures/pdf/} }% specify the path where figures are located
\usepackage{fancyhdr}                   % fancy headers and footers
\usepackage{url}                        % nicely format url breaks
\usepackage[inactive]{srcltx}                   % necessary to use forward and inverse searching in DVI
\usepackage{relsize}                    % font sizing hierarchy
\usepackage{booktabs}                   % professional looking tables
%\usepackage{subfigure}% Support for small, `sub' figures and tables
\usepackage[config, labelfont={bf}]{caption,subfig} % nice sub figures
\usepackage{mathrsfs}                   % additional math scripts
\usepackage{bm}
\usepackage{amsfonts}
\usepackage{multirow}
\usepackage{tabularx, booktabs}

\newcommand{\nudge}{\hspace{0.01in}}
%%% PACKAGES THAT ARE PRELOADED WITH THE CLASS ARE: amsmath,amsthm,amssymb,setspace,geometry,hyperref,and color
%%%%%%%%%%%%%%%%%%%%%%%%%%%%%%%%%%%%%%%%%%%%%%%%%%%%%%%%%%%%%%%%%%%%%%%%%%%%%%%%%%%%%%%%%%%%%%%%%%%%%
\begin{document}
\pagenumbering{arabic}
\setcounter{page}{1}    
\newlength{\mywidth}
\setlength{\mywidth}{0.9\linewidth}
\newlength{\myheight}
\setlength{\myheight}{0.5in}
    
\begin{enumerate}
\item
  \begin{itemize}
  \item I will talk about my thesis: \hfill\mbox{ }\linebreak
      ``Simulation of Simple Evolutionary System'' 
  \end{itemize}
    
\item
  \begin{itemize}
  \item
      I will give background, then address four questions, and make
      concluding remarks.    
  \end{itemize}

\item
  \begin{itemize}
  \item Population is a collection of length l binary strings 
  \item A population can be represented by a vector; the $j$th component
      is the proportion of string j in the population.
    \item $\mathcal{R}$ is the set of length $\ell$ binary strings and operations under $\mathcal{R}$ ar bitwise modulo 2 operations.
  \end{itemize}
    
\item
  \begin{itemize}
  \item Crossover and mutation operators are defined using the
    bitwise operations in $\mathcal{R}$.
  \item Given parents $u$ and $v$, crossover produces children $u'$ and $v'$.
  \item $x$ mutates to $x + m$.
  \end{itemize}
    
\item
  \begin{itemize}
  \item This flowchart illustrates finite population genetic algorithm
  \item  We start from random initial population P.
  \item Randomly select parents u and v 
  \item Crossover u and v to produce u' and v'
  \item Keep one of u', v' and mutate
  \item Repeat to form next generation and when next generation completes, replace it with new generation
  \item The process is repeated until some threshold is meet
  \end{itemize}
    
\item
  \begin{itemize}
  \item In infinite population model, population is modeled as by a vector $\bm{p}$
        
  \item $\mathcal{G}$ is a function that maps $\bm{p}$ to the next
    generation where $\mathcal{G}(p)_j$ is probability that string $j$
    occurs in the next generation.
  \item The sequence shows evolution of p under infinite population
    model.
  \end{itemize}
  
\item
  \begin{itemize}
  \item Random Heuristic Search generalizes simple genetic algorithm and was presented in his book by Vose in 1999. Our work is
    largely based on this model
  \item Given population p, $\tau$ is transition rule that maps p to
    the next generation p'.
  \item The sequence shows finite population evolution which forms
    Markov chain
  \item However, $\tau$ is stochastic function of crossover and
    mutation, and hence, $\tau(p)$ can not be predicted with
    certainty.
  \item $\mathcal{G}(p)$ gives the expected next generation
  \item The variance between finite population and infinite population
    is given by expression where r is population size
  \end{itemize}

\item
  \begin{itemize}
  \item Our first question concerns about distance between finite
    population and infinite population
  \item Chebyshev's inequality, Jensen's inequality and geometric point of view, all, suggests the distance might decrease
    as $1/\sqrt{ r}$
  \item But all of 3 implications are from inequalities. So the distance
    might decrease much smaller than $1/\sqrt{ r}$.
  \item We investigate what happens in practice. For this we set up
    simple diploid model.
  \end{itemize}
    
\item
  \begin{itemize}
  \item It is infinite population model.
  \item We consider diploid genome $\alpha$ with genome length l.
  \item $q^n$ is population at generation n
  \item $q^n_\alpha$ is prevalence of diploid $\alpha$ at generation n
  \item $t_{\alpha}(g)$ is transmission function which is probability
    of gamete g being produced from parent $\alpha$
  \item $q_\gamma^{n+1} \; = \; \sum_{\alpha} \, q_\alpha^n \,
    t_\alpha(\gamma_0) \sum_{\beta} \,q_\beta^n \,
    t_\beta(\gamma_1)\\[-.05in]$ gives us next generation population.
  \end{itemize}
    
\item
  \begin{itemize}
  \item Diploids are determined in terms of haploid
    distributions.
  \item Haploids are determined in terms of diploid
    distributions.
  \item Evolution equations can be expressed in terms of haploid
    distributions as
  \item In the matrix form, evolution equations can be expressed as
  \item Where M(g) is mixing matrix
  \end{itemize}
    
\item
  \begin{itemize}
  \item This slide shows computation of crossover and mutation
    distributions, and transmission function  
  \item transmission function is computed with this expression
  \item Instead of using this transmission function, we want to
    compute mixing matrix in walsh basis.
  \end{itemize}
    
\item
  \begin{itemize}
  \item W is walsh matrix
  \item (hat) represents walsh transform and for matrix A, $\hat(A)$
    is given by the expression $\widehat{A} = WAW$, and for vector w, $\hat{w}$ is 
    by the expression $\widehat{w} = Ww$.
  \item Mixing matrix is given by the expression
    \[
	  \widehat{M}_{u,v} \; = \; 2^{\,\ell-1} \,[\nudge u \nudge v = {\bf
	  0}\nudge]\, \widehat{\bm{\mu}}_u \nudge \widehat{\bm{\mu}}_v \!  \sum_{k
	\nudge \in \nudge \overline{u+v} \nudge \mathcal{R}} \bm{\chi}_{k + u} +
	\bm{\chi}_{k + v}
	\]
  \item And evolution eqn in Walsh basis is 
  \[
	  \widehat{\bm{p}}_g^{\,\,\prime} \; = \; 2^{\,\ell/2} \sum_{i \nudge \in \nudge g \mathcal{R}}
	  \widehat{\bm{p}}_i \, \nudge \widehat{\bm{p}}_{i+g} \,\widehat{M}_{i,i+g}
	\]
  \end{itemize}
    
\item
  \begin{itemize}
  \item Reduction to haploid model and computing in Walsh basis
    simplified computations and made it efficient
  \item 3 sum terms in transmission function is reduced to only one sum terms for computing mixing matrix in walsh basis 
  \item We only need one mixing matrix as opposed to $2^l$
  \item Consider l = 14, $2^14$ mixing matrices would require 32 TB of
    memory, while one mixing matrix would require only 2 GB of memory
  \item We think that is great optimization in resource
  \end{itemize}
    
\item
  \begin{itemize}
  \item With naive computation as shown in equation (1), there would
    be $2^l \cdot 2^l$ multiplication terms to be computed
  \item where f is finite population and q is infinite population
  \item We simplify it as shown in eqn (2) where it needs only $2^l$ +
    cardinal value of $S_f$ terms
  \item $S_f$ is set of diploids that are in finite populations
  \end{itemize}
  
\item
  \begin{itemize}
  \item We ran simulations with $\chi$ = 0.1 and $\mu$ = 0.001 for
    different values of l
  \item These graphs show the results.
  \item d is the distance, N is population size, n is number of
    generations
  \item Graphs show as population size increases, distance decreases,
    converge to infinite population
  \item Graphs also show smoothing as genome length increases
  \item Data shows a near linear dependence of log d on log N and that
    is what we are particularly interested in
  \end{itemize}
    
\item
  \begin{itemize}
  \item So we tried to fit data in the linear equation
  \item m is slope and b is y intercept
  \item Regression shows m nearly equal to -0.5
  \item This proves distance does decrease as $1/\sqrt{N}$
  \item Also graph for y intercept b shows, $k = e^b$ decreases
    monotonically with genome length l, and increases monotonically
    with generation n
  \item Increase in k for larger n seems to be a manifestation of the
    growing nonlinearity likely that the nonlinearity results partly
    from genetic drift experienced by finite populations
  \item and partly because predicted distance is for single step case
    as generation increases difference between finite population and
    infinite population accumulates non linearly
  \end{itemize}

\item
  \begin{itemize}
  \item Conclusion from this simulation is that
  \item Vose's infinite population model makes computation in diploid
     case efficient by reducing to the haploid case
  \item Distance between finite diploid population and infinite
    diploid population can decrease like 1/√N
  \end{itemize}

    
\item 
  \begin{itemize}
  \item We move on to our 2nd question which concerns oscillation in
    finite populations.
  \end{itemize}
    
\item
  \begin{itemize}
  \item First I want to introduce term limit; we also call it fixed
    point interchangeably.
  \item The infinite population sequence  $\bm{p}, \, \mathcal{G}(\bm{p}), \, {\mathcal{G}}^2(\bm{p}), \, \cdots$ may
    converge to a point, that we call a fixed point.
  \item But under certain conditions, the sequence converges to a
    periodic orbit between two fixed points, say p* and q*
\end{itemize}

\item
  \begin{itemize}
  \item And the conditions are: For some $g$
  \begin{eqnarray*}
      -1 &=& \sum \limits_{j} (-1)^{g^T j} \bm{\mu}_j \\
      1 &=& \sum \limits_{k \in \bar{g}\mathcal{R}} \bm{\chi}_{k+g} + \bm{\chi}_k 
      \end{eqnarray*}
  \item So our 2nd question is do finite populations also exhibit
     oscillations from random intial populations?
  \end{itemize}
    
\item
  \begin{itemize}
  \item Other people in the past also have been interested in studying
    oscillations in populations.
   \item Akin (1982) proved existence of cycling for continuous-time
     2-bit diploid model
  \item Hasting (1981) studied cycling in populations with infinite
    2-bit diploid population model
  \item Wright and Bidwell (1997) provided examples when cycles in an
    infinite haploid population model occur with crossover and
    mutation for 3 bit and 4 bit populations
  \item Wright and Agapie (2001) described cyclings in infinite
    populations for up to 4 bits, and also presented data for cyclings
    in finite populations
  \end{itemize}

    
\item
  \begin{itemize}
  \item Akin considered - continuous time model, we consider -
    discrete time model
   \item Hastings' study - limited to two bits length, includes only
     crossover, not mutation
  \item Wright and Bidwell compute a specific fitness function and a
    specific initial population for randomly generated mutation and
    crossover distributions in an attempt to find cyclic behavior
  \item Wright and Agapie used dynamic mutation that depends upon
    where population is in the population space while we use static
    mutation
  \item We study oscillation for
  \item fixed fitness function and random: initial population,
    mutation and crossover distribution
  \item higher bit length (up to 14), and actually oscillation
    conditions for infinite populations are independent of string
    length
  \item both haploid and diploid populations, and for both finite and
    infinite populations
  \item We also visualize oscillation
  \end{itemize}

\item
  \begin{itemize}
  \item Simulations were run for both haploid and diploid populations
  \item for different string lengths l
  \item and for population size {4096, 40960, 81920}
  \item To visualize oscillations, distances between fixed points (p*
    , q*) and population are plotted
  \end{itemize}
    
\item
  \begin{itemize}
  \item Graphs show the results for haploid and diploid population of
    length 10
   \item top – haploid, bottom - diploids
  \item and for population size {4096, 40960, 81920}
  \item last column graphs show results for infinite population
  \item As population size increases, oscillation approaches the
    behavior exhibited by infinite population
  \item We observe more randomness in diploids than in haploids for
    same string length and population size.
  \end{itemize}
    
\item
  \begin{itemize}
  \item Graphs show the results for haploid and diploid population of
    length 12
   \item We observe that increase in l degrades oscillation, and it is
     particularly noticeable in diploid case
  \item We see an interesting behavior in this bottom left graph,
  \end{itemize}
    
\item
  \begin{itemize}
  \item Here is zoomed in picture
   \item We see oscillation between different levels
  \item we observe this kind of behavior only for higher values of l
    in diploids and for small population size
  \item It might be because there could be fixed points for other
    distributions in the vicinity of finite population trajectory and
    get attracted to those fixed points
  \item And smaller populations do not follow infinite population path
    as closely as larger populations do; so as for larger populations,
    these behaviors were not observed
  \end{itemize}
    
\item
  \begin{itemize}
  \item We also plotted average amplitudes of oscillation in haploids
    and diploids
   \item Graphs show
   \item Oscillation amplitude increases with increase in population
     size
   \item Amplitude of oscillation decreases with increase in l
  \end{itemize}
    
\item
  \begin{itemize}
  \item Our conclusions from this simulation are
   \item Finite populations exhibit approximate oscillation when
     infinite populations oscillate
  \item As l increases, oscillation amplitude decreases
  \item As population size increases, oscillation amplitude increases
    and randomness decreases
  \item Finite population can also oscillate between different pairs of
    fixed points for diploid population of smaller size and larger l
  \end{itemize}
  
\item
  \begin{itemize}
  \item Question 3 concerns the robustness of finite population oscillation under violation in mutation
  
  \end{itemize}

\item
  \begin{itemize}
  \item If Markov chain representing finite population is regular, positive steady state distribution exists 
  \item Which means no periodic orbit exists for finite population
  \item We study if finite population can exhibit approximate oscillations in that case  
  
  \end{itemize}
  
\item
  \begin{itemize}
  \item We introduce violation $\epsilon$ as following
  \item The modification in $\bm{\mu}$ makes the Markov chain regular
  \item This means no periodic orbits for finite population and no periodic orbits for infinite population
  
  \end{itemize}
  
\item
  \begin{itemize}
  \item Simulations were run for different values of $\epsilon,\, \ell \,and\, N$
  \item Distances of population to limits p∗ and q∗ without violation are plotted  
  \end{itemize}
  
\item
  \begin{itemize}
  \item Graphs show results for haploid population behavior of string length 8
  \item Graphs are arranged in rows for increasing $\epsilon$ and columns for increasing population size
  \item Graphs show convergence of finite population behavior to infinite population behavior as population increases
  \item Results show oscillating behavior for smaller values of epsilon that diminishes with time.
  \item As value of epsilon grows, oscillation degrades
  
  \end{itemize}
  
\item
  \begin{itemize}
  \item We observed oscillation also degrades as string length increases, that can be seen in results for diploids.
  
  \end{itemize}
  
\item
  \begin{itemize}
  \item Finite populations exhibit approximate oscillation even if Markov chain is regular when violation is small
  \item If violation becomes larger, finite population oscillation decreases
  \item As string length increases, oscillation degrades
  \end{itemize}
  
  %% violation in crossover
\item
  \begin{itemize}
  \item Question 4 also concerns the robustness of finite population oscillation, but under violation in crossover
  
  \end{itemize}

\item
  \begin{itemize}
  \item Violation in crossover means no periodic orbit exists for infinite population
  \item But we don't know if Markov chain is regular in this case
  \item We investigate if finite population can exhibit approximate oscillations in this case  
   
  \end{itemize}
  
\item
  \begin{itemize}
  \item We introduce violation $\epsilon$ as following  
  \end{itemize}
  
\item
  \begin{itemize}
  \item Simulations were run for different values of $\epsilon, \ell and N$
  \item Distances of population to limits p∗ and q∗ without violation are plotted  
  \end{itemize}
  
\item
  \begin{itemize}
  \item Graphs show results for haploid population behavior of string length 8
  \item Results show similar behavior to that of violation in mutation case
  \item However, we noticed that rate of damping of amplitudes of oscillation is slower than in mutation violation  
  \item And, we see more randomness in population behavior than in mutation violation, especially for diploid case
  \end{itemize}
  
\item
  \begin{itemize}
  \item Conclusions from experiment on violation in crossover are 
  \item Finite populations exhibit approximate oscillation if violation is small
  \item If violation becomes larger, finite population oscillation decreases
  \item As string length increases, oscillation degrades
  \end{itemize}
  
\item
  \begin{itemize}
  \item Overall conclusions from this research are
  \item Vose's haploid model makes computation efficient in diploid case by reducing to haploid case 
  \item Distance between finite population and infinite population can decrease like 1/√N
  \item When infinite populations oscillate, finite populations exhibit approximate oscillation
  \item When Markov chain is regular, finite population exhibits approximate oscillation for small mutation violation
  \item Finite populations exhibit approximate oscillation for small crossover violation      
  
  \end{itemize}
  
%   \item
%   \begin{itemize}
%   \item Black dots represent finite populations;\hfill\mbox{
%   }\linebreak arrows connect one generation to the next.
%   \item White dots represent infinite populations;\hfill\mbox{ }\linebreak
%       arrows connect one generation to the next.
%     \item We are concerned with the distance between the finite and infinite populations.
%   \end{itemize}
% 
% \item
%   \begin{itemize}
%   \item Dots represent populations;\hfill\mbox{ }\linebreak
%     arrows connect one generation to the next.
%   \item We are interested in oscillation between a pair of populations.
%   \end{itemize}
%   
% \item
%   \begin{itemize}
%   \item Lot of people have contributed in evolution models. Here I
%     have listed few.
%   \end{itemize}

  % \item
%   \begin{itemize}
%   \item Our first question concerns about distance between finite
%     population and infinite population
%   \end{itemize}
% 
% \item
%   \begin{itemize}
%   \item Chebyshev's inequality suggests distance might decrease as
%     $1/\sqrt{ r}$
%   \end{itemize}
% 
% \item
%   \begin{itemize}
%   \item This tetrahedron represents population space.
%   \item Dots represent finite population points; larger dot, less
%     likey point; smaller dot, more likely point
%   \item But infinite population can be anywhere in the space
%   \item The distance between finite population point and infinite
%     population is $O(1/\sqrt{r})$
%   \end{itemize}
%     
% \item
%   \begin{itemize}
%   \item Jensen's inequality also suggests the distance might decrease
%     as $1/\sqrt{ r}$
%   \item All of 3 implications are from inequalities. So the distance
%     might decrease much smaller than $1/\sqrt{ r}$.
%   \item We investigate what happens in practice. For this we set up
%     simple diploid model.
%   \end{itemize}
\end{enumerate}

\end{document}


