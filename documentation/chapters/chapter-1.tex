\chapter{Introduction} \label{ch:introduction}

\section{Introduction}
\section{Literature}
\section{Random Heuristic Search}
Vose (see \cite{Vose1999}) introduced abstract model, a generalized heuristic search method referred to as {\em Random Heuristic Search (RHS)} which is defined upon the central concept of state and transition between states. An instance of {\em RHS} can be thought of as an initial collection of elements $P_0$ chosen from some search space $\Omega$ , together with a stochastic transition rule $\tau$ , which from $P_i$ will produce another collection $P_{i+1}$. In other words, $\tau$ will be iterated to produce a sequence of generations.

The beginning collection $P_0$ is referred to as the {\em initial population}. Let $n$ be the cardinality of $\Omega$ and ${\bf1}$ denotes column vector of all 1s. The {\em simplex} is defined to be the set of population descriptors:
\footnotetext{$\langle .. \rangle$ represents a column vector.}
\[
\Lambda = \{x = \langle x_0,...,x_{n-1} \rangle: {\bf1}^T x=1, x_j \geq 0 \}
\]

An element $p$ of $\Lambda$ corresponds to a population according to the rule:
$p_j$ = the proportion in the population of the $j$th element of $\Omega$

The cardinality of each population is a constant $r$, called the population size. Given $r$, a population descriptor $p$ unambiguously determines a population.

Given the current population vector $p$, the next population vector $\tau(p)$ cannot be predicted with certainty because $\tau$ is stochastic 
and results from $r$ independent, identically distributed random choices. Let $\mathcal{G}:\Lambda \rightarrow \Lambda$ be a function that given the current population vector $p$ produces a new vector whose $i$th component is the probability that $i$th element of $\Omega$ is chosen. Thus, $\mathcal{G}(p)$ is the probability vector that specifies the distribution from which the aggregate of $r$ choices forms the subsequent generation.
Probability that population $q$ is the next population vector given current population vector $p$ can be computed as (see \cite{Vose1999}) 
\[
r! \prod \frac{(\mathcal{G}(p)_j)^{rq_j}}{(rq_j)!}
\]
\[
 = \exp\{-r \sum q_j \log \frac{q_j}{\mathcal{G}(p)_j} - \sum (\log \sqrt{2 \pi rq_j} + \frac{1}{12rq_j + \theta (rq_j)}) + O(\log r)\}
\]
where summation is restricted to indices for which $q_j > 0$.

Each random vector in the sequence $p, \tau(p), \tau^2(p),...$ depends only on the value of the preceding one, which is a special situation, and such a sequence forms a Markov chain with
transition matrix
\[
Q_{p,q} = r! \prod \frac{(\mathcal{G}(p)_j)^{rq_j}}{(rq_j)!}
\]
So the conceptualization of RHS can be replaced by Markov chain model abstraction which makes no reference to sampling $\Omega$. That is from current population $p$, produce $q = \tau (p)$ with probability $Q_{p,q}$. With transition matrix defined for Markov chain model, Vose (see \cite{Vose1999}) says the expected next generation $\mathcal{E}(\tau (p))$ is $\mathcal{G}(p)$ and the expression in transition matrix
\[
\sum q_j \log \frac{q_j}{\mathcal{G}(p)!}
\]
gives the qualitative information regarding probable next generation which is the {\em discrepancy} of $q$ with respect to $\mathcal{G}(p)$. It is a measure of how far $q$ is from the expected next population $\mathcal{G}(p)$. Discrepancy is nonnegative and is zero only when $q$ is the expected next population. Hence the factor 
\[
<<<<<<< HEAD
\exp\{-r \sum q_j \ln \frac{q_j}{\mathcal{G}(p)!}\}
=======
\exp\{-r \sum q_j \log \frac{q_j}{\mathcal{G}(p)_j}\}
>>>>>>> f3bf8bb1ca6e00c482d1b1f3863823e9ea28d462
\]
in the expression of transition matrix indicates the probability that $q$ is the next generation
decays exponentially, with constant $r$ , as the discrepancy between $q$ and the
expected next population increases.
The expression 
\[
\sum (\log \sqrt{2 \pi rq_j} + \frac{1}{12rq_j + \theta (rq_j)})
\]
measures the {\em dispersion} of the population vector $q$ and the factor
\[
<<<<<<< HEAD
\exp\{- \sum (\ln \sqrt{2 \pi rq_j} + \frac{1}{12rq_j + \theta (rq_j)})\}
=======
\exp\{- \sum (\log \sqrt{2 \pi rq_j} + \frac{1}{12rq_j + \theta (rq_j)})\}
>>>>>>> f3bf8bb1ca6e00c482d1b1f3863823e9ea28d462
\]
indicates the probability that $q$ is the next generation decays exponentially with increasing dispersion and 
$\theta  =  \sum \log(e^{x_j}x_j!/x_j^{x_j})$.

Vose (see \cite{Vose1999}) calculated variance of next generation population with respect to expected population as 
\[
\mathcal{E}(\| \tau (p) - \mathcal{G}(p) \|^2) = (1 - \|\mathcal{G}(p)\|^2) / r
\]
and mentioned $\tau (p)$ converges in probability to $\mathcal{G}(p)$ as the population size increases. Therefore, $\tau$ corresponds to $\mathcal{G}$ in the infinite
population case.





