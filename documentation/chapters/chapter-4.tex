\chapter{Violation} \label{ch:evolutionary limits}
The results from chapter \ref{ch:oscillation} show that oscillation occurs
when the crossover distribution $\bm{\chi}$, and the mutation distribution $\bm{\mu}$ 
satisfy condition \ref{OscCond}. This chapter explores the robustness of finite population oscillation. 
If population forms ergodic Markov chain, then oscillation should not occur. Does oscillation 
in finite populations cease in practice if condition \ref{OscCond} is violated? 
This chapter tries to answer this and our  
third research question, whether finite populations exhibit oscillation when the Markov chain 
is ergodic.

Error $\bm{\epsilon}$ is introduced to $\bm{\mu}$ and $\bm{\chi}$ distributions to 
violate condition \ref{OscCond}. Consequently, $\bm{p}^\ast \;=\; \bm{q}^\ast \;=\; \bm{z}^\ast$. 
The violation in the distributions are intended to make Markov chain ergodic. Initial population is 
computed using same procedure as described in section \ref{InitPopOsc}. To understand effects of weight 
of violation in $\bm{\mu}$ and $\bm{\chi}$, different values of $\bm{\epsilon}$ are used in experiments. 
String length $\ell \;\in\; \{8, 10, 12, 14\}$ are considered for simulation.
Going forward, we use `limit $\bm{z}^\ast$' to denote evolutionary limit when crossover distribution 
$\bm{\chi}$ or mutation distribution $\bm{\mu}$ is violated, and 
`non-violation limits $\bm{p}^\ast$ and $\bm{q}^\ast$' to denote limits without violation.

\section{Violation in Mutation Distribution}
The mutation distribution $\bm{\mu}$ is modified as follows
\[
\bm{\mu}_i = (1-\bm{\epsilon}) \bm{\mu}_i \nudge; \tabspace i = \{0, 1, 2,.., 2^{\ell}-1\}.
\]
So that sum of $\bm{\mu}$ distribution becomes, 
\[
1-\bm{\epsilon} = \sum \limits_{i=0}^{2^{\ell}-1} \bm{\mu}_i
\]
Then set
\[
\bm{\mu}_0 = \bm{\epsilon}
\]
% \[
% \bm{\mu}_0 = (1-\bm{\epsilon})\bm{\epsilon}
% \]
% $c$ is total number components in $\bm{\mu}$ satisfying condition $\bm{\mu}_i = 0$ and set those components value as
% \[
% \bm{\mu}_i = \bm{\epsilon}^2/c \; ; \; where \; \bm{\mu}_i = 0
% \]
The mutation distribution $\bm{\mu}$ is normalized such that  $\sum\limits_{i=0}^{2^{\ell}-1}\bm{\mu}_i \;=\; 1$.
This modification makes possible for any population member to mutate to any other possible population member via mutation. 
Let us exlore for two cases of $g$ in \ref{OscCond}:

1. When $g$ is all $1$s:\newline
Any mask $j$ with only one $1$ at any position $k$ ($0 \leq k < \ell$) can mutate $k$th bit in string, and since mask 
all $0$s in has positive probability, strings have option to not mutate. This gives possibility for any string to mutate to 
any possible string in population. Let us take an example with $\ell \;=\; 8$. Let $g \;=\; 11111111$. Then, mask 
$i \;=\; 00000100$ will have positive probability according to condition \ref{OscCond} implementation. 
Mask $i$ can be used to mutate bit of a population string at position $k \;=\; 5$ from left.

2. When $g$ has atleast one $0$:\newline
Any mask $j$ with only one $1$ at any position $k$ ($0 \leq k < \ell$) where $k$th position in $g$ also has $1$ in it 
will have positive probability. So any string can mutate that particular $k$th bit using those masks. And any mask 
with $1$ in any one of the position where $g$ has one in it and $1$ in any position where $g$ has $0$s in it can be used to 
mutate bits in string where $g$ has $0$s in it. Let us take an example with $\ell \;=\; 8$. Let $g \;=\; 11001111$. Then, 
mask $i \;=\; 00000100$ will have positive probability according to condition \ref{OscCond} implementation. Then, also mask 
$j \;=\; 00010100$ will have positive probability. Mask $i$ can be used to mutate string at position $k \;=\; 5$. 
To mutate bit of any string at position $k \;=\; 3$, string can be mutated using mask $i$ and mask $j$ in sequence.

Population in state $P$ can mutate to another state $P^\prime$ in next generation. Then it can mutate back to original state $P$ or 
stay in its state $P^\prime$ or move to another state $P^{\prime\prime}$ in next generation. Since population state can mutate to 
any state, and any state can be returned to in any period of time, the Markov chain formed by GA is irreducible and aperiodic and hence, 
ergodic (see \cite{MarkovChain}). Because of ergodic Markov chain, steady state distribution with positive value components exists for the GA.   

Simulations were repeated with the violations in (\ref{OscCond}) described above for mutation distribution.
The distances of both infinite and finite populations to limit $\bm{z}^\ast$ were plotted. 
The distances of both infinite and finite populations to non-violation limits $\bm{p}^\ast$ and $\bm{q}^\ast$ were also plotted.

% figures for mu violation
\subsection{Haploid Population}
% \subsubsection{$\bm{\epsilon: 0.01}$}
% l = 8
\begin{figure}[h]
\begin{center}
\subfloat{
\resizebox{8cm}{5cm}{\includegraphics{figures/eps/vio/mu/b8/e0.01/n00004096_fin_hap.eps}}} \hspace{-3em}%
\subfloat{
\resizebox{8cm}{5cm}{\includegraphics{figures/eps/vio/mu/b8/e0.01/n00004096_fin_hap_wovio.eps}}}\vspace{-1em} \hspace{-3em}%
\end{center}
\begin{center}
\subfloat{
\resizebox{8cm}{5cm}{\includegraphics{figures/eps/vio/mu/b8/e0.01/n00040960_fin_hap.eps}}} \hspace{-3em}%
\subfloat{
\resizebox{8cm}{5cm}{\includegraphics{figures/eps/vio/mu/b8/e0.01/n00040960_fin_hap_wovio.eps}}}\vspace{-1em} \hspace{-3em}%
\end{center}

\begin{center}
\subfloat{
\resizebox{8cm}{5cm}{\includegraphics{figures/eps/vio/mu/b8/e0.01/n00081920_fin_hap.eps}}} \hspace{-3em}%
\subfloat{
\resizebox{8cm}{5cm}{\includegraphics{figures/eps/vio/mu/b8/e0.01/n00081920_fin_hap_wovio.eps}}}\vspace{-1em} \hspace{-3em}%
\end{center}

\begin{center}
\subfloat{
\resizebox{8cm}{5cm}{\includegraphics{figures/eps/vio/mu/b8/e0.01/inf_hap.eps}}}\hspace{-3em}%
\subfloat{
\resizebox{8cm}{5cm}{\includegraphics{figures/eps/vio/mu/b8/e0.01/inf_hap_wovio.eps}}}\vspace{-0.5em} \hspace{-3em}%
\caption{\textbf{Infinite and finite haploid population oscillation behavior in case of violation in $\bm{\mu}$ for genome length $\ell = 8$ and $\bm{\epsilon} = 0.01$:} 
  In left column, $d'$ is distance of finite population of size $n$ or infinite population to limit $\bm{z}^\ast$ for $g$ generations. In right column, $d$ is distance of finite population or infinite population to limits $\bm{p}^\ast$ and $\bm{q}^\ast$ without violation.}
\label{oscillation_8h_vio_mu_0.01}
\end{center}
\end{figure}

% l = 10

\begin{figure}[h]
\begin{center}
\subfloat{
\resizebox{8cm}{5cm}{\includegraphics{figures/eps/vio/mu/b10/e0.01/n00004096_fin_hap.eps}}} \hspace{-3em}%
\subfloat{
\resizebox{8cm}{5cm}{\includegraphics{figures/eps/vio/mu/b10/e0.01/n00004096_fin_hap_wovio.eps}}}\vspace{-1em} \hspace{-3em}%
\end{center}
\begin{center}
\subfloat{
\resizebox{8cm}{5cm}{\includegraphics{figures/eps/vio/mu/b10/e0.01/n00040960_fin_hap.eps}}} \hspace{-3em}%
\subfloat{
\resizebox{8cm}{5cm}{\includegraphics{figures/eps/vio/mu/b10/e0.01/n00040960_fin_hap_wovio.eps}}}\vspace{-1em} \hspace{-3em}%
\end{center}

\begin{center}
\subfloat{
\resizebox{8cm}{5cm}{\includegraphics{figures/eps/vio/mu/b10/e0.01/n00081920_fin_hap.eps}}} \hspace{-3em}%
\subfloat{
\resizebox{8cm}{5cm}{\includegraphics{figures/eps/vio/mu/b10/e0.01/n00081920_fin_hap_wovio.eps}}}\vspace{-1em} \hspace{-3em}%
\end{center}

\begin{center}
\subfloat{
\resizebox{8cm}{5cm}{\includegraphics{figures/eps/vio/mu/b10/e0.01/inf_hap.eps}}}\hspace{-3em}%
\subfloat{
\resizebox{8cm}{5cm}{\includegraphics{figures/eps/vio/mu/b10/e0.01/inf_hap_wovio.eps}}}\vspace{-0.5em} \hspace{-3em}%
\caption{\textbf{Infinite and finite haploid population oscillation behavior in case of violation in $\bm{\mu}$ for genome length $\ell = 10$ and $\bm{\epsilon} = 0.01$:} 
  In left column, $d'$ is distance of finite population of size $n$ or infinite population to limit $\bm{z}^\ast$ for $g$ generations. In right column, $d$ is distance of finite population or infinite population to limits $\bm{p}^\ast$ and $\bm{q}^\ast$ without violation.}
\label{oscillation_10h_vio_mu_0.01}
\end{center}
\end{figure}

% l = 12

\begin{figure}[h]
\begin{center}
\subfloat{
\resizebox{8cm}{5cm}{\includegraphics{figures/eps/vio/mu/b12/e0.01/n00004096_fin_hap.eps}}} \hspace{-3em}%
\subfloat{
\resizebox{8cm}{5cm}{\includegraphics{figures/eps/vio/mu/b12/e0.01/n00004096_fin_hap_wovio.eps}}}\vspace{-1em} \hspace{-3em}%
\end{center}
\begin{center}
\subfloat{
\resizebox{8cm}{5cm}{\includegraphics{figures/eps/vio/mu/b12/e0.01/n00040960_fin_hap.eps}}} \hspace{-3em}%
\subfloat{
\resizebox{8cm}{5cm}{\includegraphics{figures/eps/vio/mu/b12/e0.01/n00040960_fin_hap_wovio.eps}}}\vspace{-1em} \hspace{-3em}%
\end{center}

\begin{center}
\subfloat{
\resizebox{8cm}{5cm}{\includegraphics{figures/eps/vio/mu/b12/e0.01/n00081920_fin_hap.eps}}} \hspace{-3em}%
\subfloat{
\resizebox{8cm}{5cm}{\includegraphics{figures/eps/vio/mu/b12/e0.01/n00081920_fin_hap_wovio.eps}}}\vspace{-1em} \hspace{-3em}%
\end{center}

\begin{center}
\subfloat{
\resizebox{8cm}{5cm}{\includegraphics{figures/eps/vio/mu/b12/e0.01/inf_hap.eps}}}\hspace{-3em}%
\subfloat{
\resizebox{8cm}{5cm}{\includegraphics{figures/eps/vio/mu/b12/e0.01/inf_hap_wovio.eps}}}\vspace{-0.5em} \hspace{-3em}%
\caption{\textbf{Infinite and finite haploid population oscillation behavior in case of violation in $\bm{\mu}$ for genome length $\ell = 12$ and $\bm{\epsilon} = 0.01$:} 
  In left column, $d'$ is distance of finite population of size $n$ or infinite population to limit $\bm{z}^\ast$ for $g$ generations. In right column, $d$ is distance of finite population or infinite population to limits $\bm{p}^\ast$ and $\bm{q}^\ast$ without violation.}
\label{oscillation_12h_vio_mu_0.01}
\end{center}
\end{figure}

% l = 14

\begin{figure}[h]
\begin{center}
\subfloat{
\resizebox{8cm}{5cm}{\includegraphics{figures/eps/vio/mu/b14/e0.01/n00004096_fin_hap.eps}}} \hspace{-3em}%
\subfloat{
\resizebox{8cm}{5cm}{\includegraphics{figures/eps/vio/mu/b14/e0.01/n00004096_fin_hap_wovio.eps}}}\vspace{-1em} \hspace{-3em}%
\end{center}
\begin{center}
\subfloat{
\resizebox{8cm}{5cm}{\includegraphics{figures/eps/vio/mu/b14/e0.01/n00040960_fin_hap.eps}}} \hspace{-3em}%
\subfloat{
\resizebox{8cm}{5cm}{\includegraphics{figures/eps/vio/mu/b14/e0.01/n00040960_fin_hap_wovio.eps}}}\vspace{-1em} \hspace{-3em}%
\end{center}

\begin{center}
\subfloat{
\resizebox{8cm}{5cm}{\includegraphics{figures/eps/vio/mu/b14/e0.01/n00081920_fin_hap.eps}}} \hspace{-3em}%
\subfloat{
\resizebox{8cm}{5cm}{\includegraphics{figures/eps/vio/mu/b14/e0.01/n00081920_fin_hap_wovio.eps}}}\vspace{-1em} \hspace{-3em}%
\end{center}

\begin{center}
\subfloat{
\resizebox{8cm}{5cm}{\includegraphics{figures/eps/vio/mu/b14/e0.01/inf_hap.eps}}}\hspace{-3em}%
\subfloat{
\resizebox{8cm}{5cm}{\includegraphics{figures/eps/vio/mu/b14/e0.01/inf_hap_wovio.eps}}}\vspace{-0.5em} \hspace{-3em}%
\caption{\textbf{Infinite and finite haploid population oscillation behavior in case of violation in $\bm{\mu}$ for genome length $\ell = 14$ and $\bm{\epsilon} = 0.01$:} 
  In left column, $d'$ is distance of finite population of size $n$ or infinite population to limit $\bm{z}^\ast$ for $g$ generations. In right column, $d$ is distance of finite population or infinite population to limits $\bm{p}^\ast$ and $\bm{q}^\ast$ without violation.}
\label{oscillation_14h_vio_mu_0.01}
\end{center}
\end{figure}

\clearpage

The right column in figures \ref{oscillation_8h_vio_mu_0.01} through \ref{oscillation_14h_vio_mu_0.01} 
shows distance of finite and infinite haploid populations to non-violation limits $\bm{p^\ast}$ and $\bm{q^\ast}$ with $\bm{\epsilon} \;=\; 0.01$. 
Those graphs indicate oscillating behavior of haploid population given violation. 
Both finite and infinite populations oscillate given violation. Oscillations are sharper. Since the value of $\bm{\epsilon}$ 
is small, damping of ripples is slow. New masks created in mutation distribution with $\bm{\epsilon} \;=\; 0.01$ have small 
probability of being used during mutation, and when they are not used, behavior should be consistent with the 
behavior without violation. Moreover, $\bm{\epsilon}$ is so small that 
infinite population oscillation does not die out completely in 50 generations.

The left column of figures \ref{oscillation_8h_vio_mu_0.01} through \ref{oscillation_14h_vio_mu_0.01} 
shows distance of finite and infinite haploid populations to limit $\bm{z^\ast}$ 
(limit with violation in mutation distribution $\bm{\mu}$) when $\bm{\epsilon} \;=\; 0.01$. 
The distance between finite population and limit $\bm{z}^\ast$ (limit with violation in $\bm{\mu}$ distribution) 
decreases as finite population size increases, 
and finite population shows behavior similar to infinite population behavior as finite population reach large number. 
The distance data for haploid population in case of violation in $\bm{\mu}$ distribution 
with $\bm{\epsilon} \;=\; 0.01$ for different finite population size $N$ are tabulated in table \ref{distanceMuHapEps0.01}.

\begin{table}[ht]
\caption{\textbf{Distance measured for violation in $\bm{\mu}$ with $\bm{\epsilon} \;=\; 0.01$ for haploids:} $\ell$ is genome length, 
average distance between finite and infinite population is tabulated in the last three columns, and last row is expected single step distance.}
\centering
\begin{tabularx}{0.75\textwidth}{ c *{3}{X}}
\toprule
$\ell$ & $N = 4096$ & $N = 40960$ & $N = 81920$ \\
\midrule
8 & 0.0176	& 0.0094	& 0.0093 \\
10 & 0.0168	& 0.0088 	& 0.0077 \\ 
12 & 0.0161	& 0.0064 	& 0.0053 \\
14 & 0.0157	& 0.0051 	& 0.0038 \\ 
\midrule
$1/\sqrt{N}$ & 0.0156 & 0.0049 & 0.0035 \\
\bottomrule
\end{tabularx}
\label{distanceMuHapEps0.01}
\end{table}

From table \ref{distanceMuHapEps0.01}, average distance calculated for finite population size $4096$ is $0.0165$, 
for size $40960$ is $0.0074$ and for size $81920$ is $0.0065$. These results show average distance 
between finite population and limit $\bm{z^\ast}$ closely follows expected single step distance 
between finite and infinite population. The distance decreased as $1/\sqrt{N}$. 
Also, the distance decreased as genome length $\ell$ increased for all sizes of finite haploid populations 
with $\bm{\epsilon} \;=\; 0.01$.


% \subsubsection{$\bm{\epsilon: 0.1}$}
% l = 8
% \mbox{}\\[-0.75in]
\begin{figure}[!b]
\begin{center}
\subfloat{
\resizebox{8cm}{4.5cm}{\includegraphics{figures/eps/vio/mu/b8/e0.1/n00004096_fin_hap.eps}}} \hspace{-3em}%
\subfloat{
\resizebox{8cm}{4.5cm}{\includegraphics{figures/eps/vio/mu/b8/e0.1/n00004096_fin_hap_wovio.eps}}}\vspace{-1em} \hspace{-3em}%
\end{center}
\begin{center}
\subfloat{
\resizebox{8cm}{4.5cm}{\includegraphics{figures/eps/vio/mu/b8/e0.1/n00040960_fin_hap.eps}}} \hspace{-3em}%
\subfloat{
\resizebox{8cm}{4.5cm}{\includegraphics{figures/eps/vio/mu/b8/e0.1/n00040960_fin_hap_wovio.eps}}}\vspace{-1em} \hspace{-3em}%
\end{center}

\begin{center}
\subfloat{
\resizebox{8cm}{4.5cm}{\includegraphics{figures/eps/vio/mu/b8/e0.1/n00081920_fin_hap.eps}}} \hspace{-3em}%
\subfloat{
\resizebox{8cm}{4.5cm}{\includegraphics{figures/eps/vio/mu/b8/e0.1/n00081920_fin_hap_wovio.eps}}}\vspace{-1em} \hspace{-3em}%
\end{center}

\begin{center}
\subfloat{
\resizebox{8cm}{4.5cm}{\includegraphics{figures/eps/vio/mu/b8/e0.1/inf_hap.eps}}}\hspace{-3em}%
\subfloat{
\resizebox{8cm}{4.5cm}{\includegraphics{figures/eps/vio/mu/b8/e0.1/inf_hap_wovio.eps}}}\vspace{-0.5em} \hspace{-3em}%

\caption[\textbf{Infinite and finite haploid population behavior for $\bm{\mu}$ violation, genome length $\ell = 8$ and $\bm{\epsilon} = 0.1$}]
{\textbf{Infinite and finite haploid population behavior for $\bm{\mu}$ violation and $\ell = 8$ and $\bm{\epsilon} = 0.1$:} 
  In left column, $d'$ is distance of finite or infinite population to limit $\bm{z}^\ast$ for $g$ generations. 
  In right column, $d$ is distance of finite or infinite population to limits $\bm{p}^\ast$ and $\bm{q}^\ast$.}
\label{oscillation_8h_vio_mu_0.1}
\end{center}
\end{figure}

% l = 10

\begin{figure}[h]
\begin{center}
\subfloat{
\resizebox{8cm}{4.5cm}{\includegraphics{figures/eps/vio/mu/b10/e0.1/n00004096_fin_hap.eps}}} \hspace{-3em}%
\subfloat{
\resizebox{8cm}{4.5cm}{\includegraphics{figures/eps/vio/mu/b10/e0.1/n00004096_fin_hap_wovio.eps}}}\vspace{-1em} \hspace{-3em}%
\end{center}
\begin{center}
\subfloat{
\resizebox{8cm}{4.5cm}{\includegraphics{figures/eps/vio/mu/b10/e0.1/n00040960_fin_hap.eps}}} \hspace{-3em}%
\subfloat{
\resizebox{8cm}{4.5cm}{\includegraphics{figures/eps/vio/mu/b10/e0.1/n00040960_fin_hap_wovio.eps}}}\vspace{-1em} \hspace{-3em}%
\end{center}

\begin{center}
\subfloat{
\resizebox{8cm}{4.5cm}{\includegraphics{figures/eps/vio/mu/b10/e0.1/n00081920_fin_hap.eps}}} \hspace{-3em}%
\subfloat{
\resizebox{8cm}{4.5cm}{\includegraphics{figures/eps/vio/mu/b10/e0.1/n00081920_fin_hap_wovio.eps}}}\vspace{-1em} \hspace{-3em}%
\end{center}

\begin{center}
\subfloat{
\resizebox{8cm}{4.5cm}{\includegraphics{figures/eps/vio/mu/b10/e0.1/inf_hap.eps}}}\hspace{-3em}%
\subfloat{
\resizebox{8cm}{4.5cm}{\includegraphics{figures/eps/vio/mu/b10/e0.1/inf_hap_wovio.eps}}}\vspace{-0.5em} \hspace{-3em}%


\caption[\textbf{Infinite and finite haploid population behavior for $\bm{\mu}$ violation, genome length $\ell = 10$ and $\bm{\epsilon} = 0.1$}]{\textbf{Infinite and finite haploid population behavior for $\bm{\mu}$ violation, genome length $\ell = 10$ and $\bm{\epsilon} = 0.1$:} 
  In left column, $d'$ is distance of finite or infinite population to limit $\bm{z}^\ast$ for $g$ generations. In right column, $d$ is distance of finite or infinite population to limits $\bm{p}^\ast$ and $\bm{q}^\ast$.}
\label{oscillation_10h_vio_mu_0.1}
\end{center}
\end{figure}

% l = 12

\begin{figure}[h]
\begin{center}
\subfloat{
\resizebox{8cm}{4.5cm}{\includegraphics{figures/eps/vio/mu/b12/e0.1/n00004096_fin_hap.eps}}} \hspace{-3em}%
\subfloat{
\resizebox{8cm}{4.5cm}{\includegraphics{figures/eps/vio/mu/b12/e0.1/n00004096_fin_hap_wovio.eps}}}\vspace{-1em} \hspace{-3em}%
\end{center}
\begin{center}
\subfloat{
\resizebox{8cm}{4.5cm}{\includegraphics{figures/eps/vio/mu/b12/e0.1/n00040960_fin_hap.eps}}} \hspace{-3em}%
\subfloat{
\resizebox{8cm}{4.5cm}{\includegraphics{figures/eps/vio/mu/b12/e0.1/n00040960_fin_hap_wovio.eps}}}\vspace{-1em} \hspace{-3em}%
\end{center}

\begin{center}
\subfloat{
\resizebox{8cm}{4.5cm}{\includegraphics{figures/eps/vio/mu/b12/e0.1/n00081920_fin_hap.eps}}} \hspace{-3em}%
\subfloat{
\resizebox{8cm}{4.5cm}{\includegraphics{figures/eps/vio/mu/b12/e0.1/n00081920_fin_hap_wovio.eps}}}\vspace{-1em} \hspace{-3em}%
\end{center}

\begin{center}
\subfloat{
\resizebox{8cm}{4.5cm}{\includegraphics{figures/eps/vio/mu/b12/e0.1/inf_hap.eps}}}\hspace{-3em}%
\subfloat{
\resizebox{8cm}{4.5cm}{\includegraphics{figures/eps/vio/mu/b12/e0.1/inf_hap_wovio.eps}}}\vspace{-0.5em} \hspace{-3em}%


\caption[\textbf{Infinite and finite haploid population behavior for $\bm{\mu}$ violation, genome length $\ell = 12$ and $\bm{\epsilon} = 0.1$}]{\textbf{Infinite and finite haploid population behavior for $\bm{\mu}$ violation, genome length $\ell = 12$ and $\bm{\epsilon} = 0.1$:} 
  In left column, $d'$ is distance of finite or infinite population to limit $\bm{z}^\ast$ for $g$ generations. In right column, $d$ is distance of finite or infinite population to limits $\bm{p}^\ast$ and $\bm{q}^\ast$.}
\label{oscillation_12h_vio_mu_0.1}
\end{center}
\end{figure}

% l = 14

\begin{figure}[h]
\begin{center}
\subfloat{
\resizebox{8cm}{4.5cm}{\includegraphics{figures/eps/vio/mu/b14/e0.1/n00004096_fin_hap.eps}}} \hspace{-3em}%
\subfloat{
\resizebox{8cm}{4.5cm}{\includegraphics{figures/eps/vio/mu/b14/e0.1/n00004096_fin_hap_wovio.eps}}}\vspace{-1em} \hspace{-3em}%
\end{center}
\begin{center}
\subfloat{
\resizebox{8cm}{4.5cm}{\includegraphics{figures/eps/vio/mu/b14/e0.1/n00040960_fin_hap.eps}}} \hspace{-3em}%
\subfloat{
\resizebox{8cm}{4.5cm}{\includegraphics{figures/eps/vio/mu/b14/e0.1/n00040960_fin_hap_wovio.eps}}}\vspace{-1em} \hspace{-3em}%
\end{center}

\begin{center}
\subfloat{
\resizebox{8cm}{4.5cm}{\includegraphics{figures/eps/vio/mu/b14/e0.1/n00081920_fin_hap.eps}}} \hspace{-3em}%
\subfloat{
\resizebox{8cm}{4.5cm}{\includegraphics{figures/eps/vio/mu/b14/e0.1/n00081920_fin_hap_wovio.eps}}}\vspace{-1em} \hspace{-3em}%
\end{center}

\begin{center}
\subfloat{
\resizebox{8cm}{4.5cm}{\includegraphics{figures/eps/vio/mu/b14/e0.1/inf_hap.eps}}}\hspace{-3em}%
\subfloat{
\resizebox{8cm}{4.5cm}{\includegraphics{figures/eps/vio/mu/b14/e0.1/inf_hap_wovio.eps}}}\vspace{-0.5em} \hspace{-3em}%


\caption[\textbf{Infinite and finite haploid population behavior for $\bm{\mu}$ violation, genome length $\ell = 14$ and $\bm{\epsilon} = 0.1$}]{\textbf{Infinite and finite haploid population behavior for $\bm{\mu}$ violation, genome length $\ell = 14$ and $\bm{\epsilon} = 0.1$:} 
  In left column, $d'$ is distance of finite or infinite population to limit $\bm{z}^\ast$ for $g$ generations. In right column, $d$ is distance of finite or infinite population to limits $\bm{p}^\ast$ and $\bm{q}^\ast$.}
\label{oscillation_14h_vio_mu_0.1}
\end{center}
\end{figure}

% \clearpage
The right column in figures \ref{oscillation_8h_vio_mu_0.1} through \ref{oscillation_14h_vio_mu_0.1} 
shows distance of finite and infinite haploid populations to non-violation limits $\bm{p^\ast}$ and $\bm{q^\ast}$ with $\bm{\epsilon} \;=\; 0.1$. 
Those graphs indicate oscillating behavior of haploid population given violation. 
Both finite and infinite populations oscillate given violation, and oscillation amplitude decreases with generation. 
However, for $\bm{\epsilon} \;=\; 0.1$, oscillations in infinite populations die out quickly, 
but oscillations in finite populations do not die out completely. Rate of damping of ripples with $\bm{\epsilon} \;=\; 0.1$ is  
larger than with $\bm{\epsilon} \;=\; 0.01$. If finite population oscillation were to die out, 
it must reappear infinitely often because the Markov chain is regular.

The left column of figures \ref{oscillation_8h_vio_mu_0.1} through \ref{oscillation_14h_vio_mu_0.1} 
shows distance of finite and infinite haploid populations to limit $\bm{z^\ast}$ 
(limit with violation in mutation distribution $\bm{\mu}$) when $\bm{\epsilon} \;=\; 0.1$. 
The distance decreases as finite population size increases, 
and finite population shows behavior similar to infinite population behavior as population size grows. 
Average distance data for haploid population in case of violation in $\bm{\mu}$ distribution 
with $\bm{\epsilon} \;=\; 0.1$ for different finite population size $N$ is tabulated in table \ref{distanceMuHapEps0.1}.

\clearpage
\begin{table}[ht]
\caption[\textbf{Distance measured for violation in $\bm{\mu}$ with $\bm{\epsilon} \;=\; 0.1$ for haploids}]{\textbf{Distance measured for violation in $\bm{\mu}$ with $\bm{\epsilon} \;=\; 0.1$ for haploids:} $\ell$ is genome length, 
average distance between finite and infinite population is tabulated in the last three columns, and last row is expected single step distance.}
\centering
\begin{tabularx}{0.75\textwidth}{ c *{3}{X}}
\toprule
$\ell$ & $N = 4096$ & $N = 40960$ & $N = 81920$ \\
\midrule
8 & 0.0158	& 0.0054 	& 0.0041 \\
10 & 0.0158	& 0.0053 	& 0.0039 \\	
12 & 0.0157	& 0.0051 	& 0.0036 \\	
14 & 0.0156	& 0.0050 	& 0.0035 \\
\midrule
$1/\sqrt{N}$ & 0.0156 & 0.0049 & 0.0035 \\
\bottomrule
\end{tabularx}
\label{distanceMuHapEps0.1}
\end{table}

Table \ref{distanceMuHapEps0.1} show average distance 
between finite population and infinite population decreases with increasing string length, 
approaching the expected single step distance $1/\sqrt{N}$. 

% \subsubsection{$\bm{\epsilon: 0.5}$}
% l = 8
% \mbox{}\\[-0.75in]
\begin{figure}[!b]
\begin{center}
\subfloat{
\resizebox{8cm}{4.5cm}{\includegraphics{figures/eps/vio/mu/b8/e0.5/n00004096_fin_hap.eps}}} \hspace{-3em}%
\subfloat{
\resizebox{8cm}{4.5cm}{\includegraphics{figures/eps/vio/mu/b8/e0.5/n00004096_fin_hap_wovio.eps}}}\vspace{-1em} \hspace{-3em}%
\end{center}
\begin{center}
\subfloat{
\resizebox{8cm}{4.5cm}{\includegraphics{figures/eps/vio/mu/b8/e0.5/n00040960_fin_hap.eps}}} \hspace{-3em}%
\subfloat{
\resizebox{8cm}{4.5cm}{\includegraphics{figures/eps/vio/mu/b8/e0.5/n00040960_fin_hap_wovio.eps}}}\vspace{-1em} \hspace{-3em}%
\end{center}

\begin{center}
\subfloat{
\resizebox{8cm}{4.5cm}{\includegraphics{figures/eps/vio/mu/b8/e0.5/n00081920_fin_hap.eps}}} \hspace{-3em}%
\subfloat{
\resizebox{8cm}{4.5cm}{\includegraphics{figures/eps/vio/mu/b8/e0.5/n00081920_fin_hap_wovio.eps}}}\vspace{-1em} \hspace{-3em}%
\end{center}

\begin{center}
\subfloat{
\resizebox{8cm}{4.5cm}{\includegraphics{figures/eps/vio/mu/b8/e0.5/inf_hap.eps}}}\hspace{-3em}%
\subfloat{
\resizebox{8cm}{4.5cm}{\includegraphics{figures/eps/vio/mu/b8/e0.5/inf_hap_wovio.eps}}}\vspace{-0.5em} \hspace{-3em}%


\caption[\textbf{Infinite and finite haploid population behavior in case of violation in $\bm{\mu}$ for genome length $\ell = 8$ and $\bm{\epsilon} = 0.5$}]
{\textbf{Infinite and finite haploid population behavior for $\bm{\mu}$ violation and $\ell = 8$ and $\bm{\epsilon} = 0.5$:} 
  In left column, $d'$ is distance of finite or infinite population to limit $\bm{z}^\ast$ for $g$ generations. 
  In right column, $d$ is distance of finite or infinite population to limits $\bm{p}^\ast$ and $\bm{q}^\ast$.}
  \label{oscillation_8h_vio_mu_0.5}
\end{center}
\end{figure}

% l = 10

\begin{figure}[h]
\begin{center}
\subfloat{
\resizebox{8cm}{4.5cm}{\includegraphics{figures/eps/vio/mu/b10/e0.5/n00004096_fin_hap.eps}}} \hspace{-3em}%
\subfloat{
\resizebox{8cm}{4.5cm}{\includegraphics{figures/eps/vio/mu/b10/e0.5/n00004096_fin_hap_wovio.eps}}}\vspace{-1em} \hspace{-3em}%
\end{center}
\begin{center}
\subfloat{
\resizebox{8cm}{4.5cm}{\includegraphics{figures/eps/vio/mu/b10/e0.5/n00040960_fin_hap.eps}}} \hspace{-3em}%
\subfloat{
\resizebox{8cm}{4.5cm}{\includegraphics{figures/eps/vio/mu/b10/e0.5/n00040960_fin_hap_wovio.eps}}}\vspace{-1em} \hspace{-3em}%
\end{center}

\begin{center}
\subfloat{
\resizebox{8cm}{4.5cm}{\includegraphics{figures/eps/vio/mu/b10/e0.5/n00081920_fin_hap.eps}}} \hspace{-3em}%
\subfloat{
\resizebox{8cm}{4.5cm}{\includegraphics{figures/eps/vio/mu/b10/e0.5/n00081920_fin_hap_wovio.eps}}}\vspace{-1em} \hspace{-3em}%
\end{center}

\begin{center}
\subfloat{
\resizebox{8cm}{4.5cm}{\includegraphics{figures/eps/vio/mu/b10/e0.5/inf_hap.eps}}}\hspace{-3em}%
\subfloat{
\resizebox{8cm}{4.5cm}{\includegraphics{figures/eps/vio/mu/b10/e0.5/inf_hap_wovio.eps}}}\vspace{-0.5em} \hspace{-3em}%


\caption[\textbf{Infinite and finite haploid population behavior $\bm{\mu}$ for violation, genome length $\ell = 10$ and $\bm{\epsilon} = 0.5$}]{\textbf{Infinite and finite haploid population behavior $\bm{\mu}$ for violation, genome length $\ell = 10$ and $\bm{\epsilon} = 0.5$:} 
  In left column, $d'$ is distance of finite or infinite population to limit $\bm{z}^\ast$ for $g$ generations. In right column, $d$ is distance of finite or infinite population to limits $\bm{p}^\ast$ and $\bm{q}^\ast$.}
\label{oscillation_10h_vio_mu_0.5}
\end{center}
\end{figure}

% l = 12

\begin{figure}[h]
\begin{center}
\subfloat{
\resizebox{8cm}{4.5cm}{\includegraphics{figures/eps/vio/mu/b12/e0.5/n00004096_fin_hap.eps}}} \hspace{-3em}%
\subfloat{
\resizebox{8cm}{4.5cm}{\includegraphics{figures/eps/vio/mu/b12/e0.5/n00004096_fin_hap_wovio.eps}}}\vspace{-1em} \hspace{-3em}%
\end{center}
\begin{center}
\subfloat{
\resizebox{8cm}{4.5cm}{\includegraphics{figures/eps/vio/mu/b12/e0.5/n00040960_fin_hap.eps}}} \hspace{-3em}%
\subfloat{
\resizebox{8cm}{4.5cm}{\includegraphics{figures/eps/vio/mu/b12/e0.5/n00040960_fin_hap_wovio.eps}}}\vspace{-1em} \hspace{-3em}%
\end{center}

\begin{center}
\subfloat{
\resizebox{8cm}{4.5cm}{\includegraphics{figures/eps/vio/mu/b12/e0.5/n00081920_fin_hap.eps}}} \hspace{-3em}%
\subfloat{
\resizebox{8cm}{4.5cm}{\includegraphics{figures/eps/vio/mu/b12/e0.5/n00081920_fin_hap_wovio.eps}}}\vspace{-1em} \hspace{-3em}%
\end{center}

\begin{center}
\subfloat{
\resizebox{8cm}{4.5cm}{\includegraphics{figures/eps/vio/mu/b12/e0.5/inf_hap.eps}}}\hspace{-3em}%
\subfloat{
\resizebox{8cm}{4.5cm}{\includegraphics{figures/eps/vio/mu/b12/e0.5/inf_hap_wovio.eps}}}\vspace{-0.5em} \hspace{-3em}%


\caption[\textbf{Infinite and finite haploid population behavior $\bm{\mu}$ for violation, genome length $\ell = 12$ and $\bm{\epsilon} = 0.5$}]{\textbf{Infinite and finite haploid population behavior $\bm{\mu}$ for violation, genome length $\ell = 12$ and $\bm{\epsilon} = 0.5$:} 
  In left column, $d'$ is distance of finite or infinite population to limit $\bm{z}^\ast$ for $g$ generations. In right column, $d$ is distance of finite or infinite population to limits $\bm{p}^\ast$ and $\bm{q}^\ast$.}
\label{oscillation_12h_vio_mu_0.5}
\end{center}
\end{figure}

% l = 14

\begin{figure}[h]
\begin{center}
\subfloat{
\resizebox{8cm}{4.5cm}{\includegraphics{figures/eps/vio/mu/b14/e0.5/n00004096_fin_hap.eps}}} \hspace{-3em}%
\subfloat{
\resizebox{8cm}{4.5cm}{\includegraphics{figures/eps/vio/mu/b14/e0.5/n00004096_fin_hap_wovio.eps}}}\vspace{-1em} \hspace{-3em}%
\end{center}
\begin{center}
\subfloat{
\resizebox{8cm}{4.5cm}{\includegraphics{figures/eps/vio/mu/b14/e0.5/n00040960_fin_hap.eps}}} \hspace{-3em}%
\subfloat{
\resizebox{8cm}{4.5cm}{\includegraphics{figures/eps/vio/mu/b14/e0.5/n00040960_fin_hap_wovio.eps}}}\vspace{-1em} \hspace{-3em}%
\end{center}

\begin{center}
\subfloat{
\resizebox{8cm}{4.5cm}{\includegraphics{figures/eps/vio/mu/b14/e0.5/n00081920_fin_hap.eps}}} \hspace{-3em}%
\subfloat{
\resizebox{8cm}{4.5cm}{\includegraphics{figures/eps/vio/mu/b14/e0.5/n00081920_fin_hap_wovio.eps}}}\vspace{-1em} \hspace{-3em}%
\end{center}

\begin{center}
\subfloat{
\resizebox{8cm}{4.5cm}{\includegraphics{figures/eps/vio/mu/b14/e0.5/inf_hap.eps}}}\hspace{-3em}%
\subfloat{
\resizebox{8cm}{4.5cm}{\includegraphics{figures/eps/vio/mu/b14/e0.5/inf_hap_wovio.eps}}}\vspace{-0.5em} \hspace{-3em}%

\caption[\textbf{Infinite and finite haploid population behavior $\bm{\mu}$ for violation, genome length $\ell = 14$ and $\bm{\epsilon} = 0.5$}]{\textbf{Infinite and finite haploid population behavior $\bm{\mu}$ for violation, genome length $\ell = 14$ and $\bm{\epsilon} = 0.5$:} 
  In left column, $d'$ is distance of finite or infinite population to limit $\bm{z}^\ast$ for $g$ generations. In right column, $d$ is distance of finite or infinite population to limits $\bm{p}^\ast$ and $\bm{q}^\ast$.}
\label{oscillation_14h_vio_mu_0.5}
\end{center}
\end{figure}

% \clearpage
The right column in figures \ref{oscillation_8h_vio_mu_0.5} through \ref{oscillation_14h_vio_mu_0.5} 
shows distance of finite and infinite haploid populations to non-violation limits $\bm{p^\ast}$ and $\bm{q^\ast}$ with $\bm{\epsilon} \;=\; 0.5$. 
The graphs indicate oscillating behavior of haploid population given violation. 
Neither finite nor infinite populations show noticable oscillation given violation. 
The all zeros mask created in mutation distribution with $\bm{\epsilon} \;=\; 0.5$ has a large probability 
of being used during mutation, so oscillation decreased signficantly.

The left column of figures \ref{oscillation_8h_vio_mu_0.5} through \ref{oscillation_14h_vio_mu_0.5} 
shows distance of finite and infinite haploid populations to limit $\bm{z^\ast}$ 
(limit with violation in mutation distribution $\bm{\mu}$) when $\bm{\epsilon} \;=\; 0.5$. 
The distance decreases as finite population size increases, 
and finite population shows behavior similar to infinite population behavior as finite population size grows. 
Average distance data for haploid population in case of violation in $\bm{\mu}$ distribution 
with $\bm{\epsilon} \;=\; 0.5$ for different finite population size $N$ are tabulated in table \ref{distanceMuHapEps0.5}.

\clearpage
\begin{table}[h]
\caption[\textbf{Distance measured for violation in $\bm{\mu}$ with $\bm{\epsilon} \;=\; 0.5$ for haploids}]{\textbf{Distance measured for violation in $\bm{\mu}$ with $\bm{\epsilon} \;=\; 0.5$ for haploids:} $\ell$ is genome length, 
average distance between finite and infinite population is tabulated in the last three columns, and last row is expected single step distance.}
\centering
\begin{tabularx}{0.75\textwidth}{ c *{3}{X}}
\toprule
$\ell$ & $N = 4096$ & $N = 40960$ & $N = 81920$ \\
\midrule
8 & 0.0161	&  0.0056	& 0.0042 \\
10 & 0.0161	&  0.0055	& 0.0040 \\
12 & 0.0157	&  0.0051	& 0.0036 \\
14 & 0.0157	&  0.0051	& 0.0036 \\
\midrule
$1/\sqrt{N}$ & 0.0156 & 0.0049 & 0.0035 \\
\bottomrule
\end{tabularx}
\label{distanceMuHapEps0.5}
\end{table}
 
Table \ref{distanceMuHapEps0.5} shows that the average distance 
between finite population and infinite population decreases with increasing string length, 
approaching the expected single step distance $1/\sqrt{N}$. 

\subsection{Diploid Population}
% \subsubsection{$\bm{\epsilon: 0.01}$}
% l = 8
% \mbox{}\\[-0.75in]
\begin{figure}[!b]
\begin{center}
\subfloat{
\resizebox{8cm}{4.5cm}{\includegraphics{figures/eps/vio/mu/b8/e0.01/n00004096_fin_dip.eps}}}\hspace{-3em}%
\subfloat{
\resizebox{8cm}{4.5cm}{\includegraphics{figures/eps/vio/mu/b8/e0.01/n00004096_fin_dip_wovio.eps}}}\vspace{-1em}  \hspace{-3em}%
\end{center}
\begin{center}
\subfloat{
\resizebox{8cm}{4.5cm}{\includegraphics{figures/eps/vio/mu/b8/e0.01/n00040960_fin_dip.eps}}}\hspace{-3em}%
\subfloat{
\resizebox{8cm}{4.5cm}{\includegraphics{figures/eps/vio/mu/b8/e0.01/n00040960_fin_dip_wovio.eps}}}\vspace{-1em}  \hspace{-3em}%
\end{center}

\begin{center}
\subfloat{
\resizebox{8cm}{4.5cm}{\includegraphics{figures/eps/vio/mu/b8/e0.01/n00081920_fin_dip.eps}}}\hspace{-3em}%
\subfloat{
\resizebox{8cm}{4.5cm}{\includegraphics{figures/eps/vio/mu/b8/e0.01/n00081920_fin_dip_wovio.eps}}}\vspace{-1em}  \hspace{-3em}%
\end{center}

\begin{center}
\subfloat{
\resizebox{8cm}{4.5cm}{\includegraphics{figures/eps/vio/mu/b8/e0.01/inf_dip.eps}}}\hspace{-3em}%
\subfloat{
\resizebox{8cm}{4.5cm}{\includegraphics{figures/eps/vio/mu/b8/e0.01/inf_dip_wovio.eps}}}\vspace{-0.5em}  \hspace{-3em}%


\caption[\textbf{Infinite and finite diploid population behavior for $\bm{\mu}$ violation, genome length $\ell = 8$ and $\bm{\epsilon} = 0.01$}]
{\textbf{Infinite and finite diploid population behavior for $\bm{\mu}$ violation, $\ell = 8$ and $\bm{\epsilon} = 0.01$:} 
  In left column, $d'$ is distance of finite or infinite population to limit $\bm{z}^\ast$ for $g$ generations. 
  In right column, $d$ is distance of finite or infinite population to limits $\bm{p}^\ast$ and $\bm{q}^\ast$. Green line is distance to $\bm{p}^\ast$ and red line is distance to $\bm{q}^\ast$.}
\label{oscillation_8d_vio_mu_0.01}
\end{center}
\end{figure}

% l = 10

\begin{figure}[h]
\begin{center}
\subfloat{
\resizebox{8cm}{4.5cm}{\includegraphics{figures/eps/vio/mu/b10/e0.01/n00004096_fin_dip.eps}}}\hspace{-3em}%
\subfloat{
\resizebox{8cm}{4.5cm}{\includegraphics{figures/eps/vio/mu/b10/e0.01/n00004096_fin_dip_wovio.eps}}}\vspace{-1em}  \hspace{-3em}%
\end{center}
\begin{center}
\subfloat{
\resizebox{8cm}{4.5cm}{\includegraphics{figures/eps/vio/mu/b10/e0.01/n00040960_fin_dip.eps}}}\hspace{-3em}%
\subfloat{
\resizebox{8cm}{4.5cm}{\includegraphics{figures/eps/vio/mu/b10/e0.01/n00040960_fin_dip_wovio.eps}}}\vspace{-1em}  \hspace{-3em}%
\end{center}


\begin{center}
\subfloat{
\resizebox{8cm}{4.5cm}{\includegraphics{figures/eps/vio/mu/b10/e0.01/n00081920_fin_dip.eps}}}\hspace{-3em}%
\subfloat{
\resizebox{8cm}{4.5cm}{\includegraphics{figures/eps/vio/mu/b10/e0.01/n00081920_fin_dip_wovio.eps}}}\vspace{-1em}  \hspace{-3em}%
\end{center}

\begin{center}
\subfloat{
\resizebox{8cm}{4.5cm}{\includegraphics{figures/eps/vio/mu/b10/e0.01/inf_dip.eps}}}\hspace{-3em}%
\subfloat{
\resizebox{8cm}{4.5cm}{\includegraphics{figures/eps/vio/mu/b10/e0.01/inf_dip_wovio.eps}}}\vspace{-0.5em}  \hspace{-3em}%


\caption[\textbf{Infinite and finite diploid population behavior for $\bm{\mu}$ violation, genome length $\ell = 10$ and $\bm{\epsilon} = 0.01$}]{\textbf{Infinite and finite diploid population behavior for $\bm{\mu}$ violation, genome length $\ell = 10$ and $\bm{\epsilon} = 0.01$:} 
  In left column, $d'$ is distance of finite or infinite population to limit $\bm{z}^\ast$ for $g$ generations. In right column, $d$ is distance of finite or infinite population to limits $\bm{p}^\ast$ and $\bm{q}^\ast$. Green line is distance to $\bm{p}^\ast$ and red line is distance to $\bm{q}^\ast$.}
\label{oscillation_10d_vio_mu_0.01}
\end{center}
\end{figure}

% l = 12

\begin{figure}[h]
\begin{center}
\subfloat{
\resizebox{8cm}{4.5cm}{\includegraphics{figures/eps/vio/mu/b12/e0.01/n00004096_fin_dip.eps}}}\hspace{-3em}%
\subfloat{
\resizebox{8cm}{4.5cm}{\includegraphics{figures/eps/vio/mu/b12/e0.01/n00004096_fin_dip_wovio.eps}}}\vspace{-1em}  \hspace{-3em}%
\end{center}
\begin{center}
\subfloat{
\resizebox{8cm}{4.5cm}{\includegraphics{figures/eps/vio/mu/b12/e0.01/n00040960_fin_dip.eps}}}\hspace{-3em}%
\subfloat{
\resizebox{8cm}{4.5cm}{\includegraphics{figures/eps/vio/mu/b12/e0.01/n00040960_fin_dip_wovio.eps}}}\vspace{-1em}  \hspace{-3em}%
\end{center}


\begin{center}
\subfloat{
\resizebox{8cm}{4.5cm}{\includegraphics{figures/eps/vio/mu/b12/e0.01/n00081920_fin_dip.eps}}}\hspace{-3em}%
\subfloat{
\resizebox{8cm}{4.5cm}{\includegraphics{figures/eps/vio/mu/b12/e0.01/n00081920_fin_dip_wovio.eps}}}\vspace{-1em}  \hspace{-3em}%
\end{center}

\begin{center}
\subfloat{
\resizebox{8cm}{4.5cm}{\includegraphics{figures/eps/vio/mu/b12/e0.01/inf_dip.eps}}}\hspace{-3em}%
\subfloat{
\resizebox{8cm}{4.5cm}{\includegraphics{figures/eps/vio/mu/b12/e0.01/inf_dip_wovio.eps}}}\vspace{-0.5em}  \hspace{-3em}%


\caption[\textbf{Infinite and finite diploid population behavior for $\bm{\mu}$ violation, genome length $\ell = 12$ and $\bm{\epsilon} = 0.01$}]{\textbf{Infinite and finite diploid population behavior for $\bm{\mu}$ violation, genome length $\ell = 12$ and $\bm{\epsilon} = 0.01$:} 
  In left column, $d'$ is distance of finite or infinite population to limit $\bm{z}^\ast$ for $g$ generations. In right column, $d$ is distance of finite or infinite population to limits $\bm{p}^\ast$ and $\bm{q}^\ast$. Green line is distance to $\bm{p}^\ast$ and red line is distance to $\bm{q}^\ast$.}
\label{oscillation_12d_vio_mu_0.01}
\end{center}
\end{figure}

% l = 14

\begin{figure}[h]
\begin{center}
\subfloat{
\resizebox{8cm}{4.5cm}{\includegraphics{figures/eps/vio/mu/b14/e0.01/n00004096_fin_dip.eps}}}\hspace{-3em}%
\subfloat{
\resizebox{8cm}{4.5cm}{\includegraphics{figures/eps/vio/mu/b14/e0.01/n00004096_fin_dip_wovio.eps}}}\vspace{-1em}  \hspace{-3em}%
\end{center}
\begin{center}
\subfloat{
\resizebox{8cm}{4.5cm}{\includegraphics{figures/eps/vio/mu/b14/e0.01/n00040960_fin_dip.eps}}}\hspace{-3em}%
\subfloat{
\resizebox{8cm}{4.5cm}{\includegraphics{figures/eps/vio/mu/b14/e0.01/n00040960_fin_dip_wovio.eps}}}\vspace{-1em}  \hspace{-3em}%
\end{center}


\begin{center}
\subfloat{
\resizebox{8cm}{4.5cm}{\includegraphics{figures/eps/vio/mu/b14/e0.01/n00081920_fin_dip.eps}}}\hspace{-3em}%
\subfloat{
\resizebox{8cm}{4.5cm}{\includegraphics{figures/eps/vio/mu/b14/e0.01/n00081920_fin_dip_wovio.eps}}}\vspace{-1em}  \hspace{-3em}%
\end{center}

\begin{center}
\subfloat{
\resizebox{8cm}{4.5cm}{\includegraphics{figures/eps/vio/mu/b14/e0.01/inf_dip.eps}}}\hspace{-3em}%
\subfloat{
\resizebox{8cm}{4.5cm}{\includegraphics{figures/eps/vio/mu/b14/e0.01/inf_dip_wovio.eps}}}\vspace{-0.5em}  \hspace{-3em}%


\caption[\textbf{Infinite and finite diploid population behavior for $\bm{\mu}$ violation, genome length $\ell = 14$ and $\bm{\epsilon} = 0.01$}]{\textbf{Infinite and finite diploid population behavior for $\bm{\mu}$ violation, genome length $\ell = 14$ and $\bm{\epsilon} = 0.01$:} 
  In left column, $d'$ is distance of finite or infinite population to limit $\bm{z}^\ast$ for $g$ generations. In right column, $d$ is distance of finite or infinite population to limits $\bm{p}^\ast$ and $\bm{q}^\ast$. Green line is distance to $\bm{p}^\ast$ and red line is distance to $\bm{q}^\ast$.}
\label{oscillation_14d_vio_mu_0.01}
\end{center}
\end{figure}

% \clearpage
The right column in figures \ref{oscillation_8d_vio_mu_0.01} through \ref{oscillation_14d_vio_mu_0.01} 
shows distance of finite and infinite diploid populations with $\bm{\epsilon} \;=\; 0.01$ to non-violation limits $\bm{p^\ast}$ and $\bm{q^\ast}$. 
Those graphs indicate oscillating behavior of finite diploid population given violation. 
Infinite populations initially oscillate given violation but the oscillation dies out. Since the value of $\bm{\epsilon}$ 
is very small, damping of ripples is slow. Infinite population oscillation does not die out in 50 generations but will eventually die out. 
Even though oscillation in finite population is tapering down,
it will not die out completely; because the Markov chain is regular, 
finite population oscillation will reappear infinitely often.

The left column of figures \ref{oscillation_8d_vio_mu_0.01} through \ref{oscillation_14d_vio_mu_0.01} 
shows distance of finite and infinite diploid populations to limit $\bm{z^\ast}$ 
(limit with violation in mutation distribution $\bm{\mu}$) when $\bm{\epsilon} \;=\; 0.01$. 
The distance decreases as finite population size increases, 
and finite population shows behavior similar to infinite population as population size grows. 
Average distance data for diploid population for $\bm{\mu}$ violation 
with $\bm{\epsilon} \;=\; 0.01$ are tabulated in table \ref{distanceMuDipEps0.01}.

\clearpage
\begin{table}[h]
\caption[\textbf{Distance measured for violation in $\bm{\mu}$ with $\bm{\epsilon} \;=\; 0.01$ for diploids}]
{\textbf{Distance measured for violation in $\bm{\mu}$ with $\bm{\epsilon} \;=\; 0.01$ for diploids:} $\ell$ is genome length, 
average distance between finite and infinite population is tabulated in the last three columns, and last row is expected single step distance.}
\centering
\begin{tabularx}{0.75\textwidth}{ c *{3}{X}}
\toprule
$\ell$ & $N = 4096$ & $N = 40960$ & $N = 81920$ \\
\midrule
8 & 0.0156	&  0.0050	& 0.0035 \\
10 & 0.0156	&  0.0049	& 0.0035 \\
12 & 0.0156	&  0.0049	& 0.0035 \\
14 & 0.0156	&  0.0049	& 0.0035 \\
\midrule
$1/\sqrt{N}$ & 0.0156 & 0.0049 & 0.0035 \\
\bottomrule
\end{tabularx}
\label{distanceMuDipEps0.01}
\end{table}
Table \ref{distanceMuDipEps0.01} shows that the average distance between 
finite population and infinite population decreases with increasing string length, 
approaching the expected single step distance $1/\sqrt{N}$. 




% \subsubsection{$\bm{\epsilon: 0.1}$}
% l = 8
% \mbox{}\\[-0.75in]
\begin{figure}[!b]
\begin{center}
\subfloat{
\resizebox{8cm}{4.5cm}{\includegraphics{figures/eps/vio/mu/b8/e0.1/n00004096_fin_dip.eps}}}\hspace{-3em}%
\subfloat{
\resizebox{8cm}{4.5cm}{\includegraphics{figures/eps/vio/mu/b8/e0.1/n00004096_fin_dip_wovio.eps}}}\vspace{-1em}  \hspace{-3em}%
\end{center}
\begin{center}
\subfloat{
\resizebox{8cm}{4.5cm}{\includegraphics{figures/eps/vio/mu/b8/e0.1/n00040960_fin_dip.eps}}}\hspace{-3em}%
\subfloat{
\resizebox{8cm}{4.5cm}{\includegraphics{figures/eps/vio/mu/b8/e0.1/n00040960_fin_dip_wovio.eps}}}\vspace{-1em}  \hspace{-3em}%
\end{center}

\begin{center}
\subfloat{
\resizebox{8cm}{4.5cm}{\includegraphics{figures/eps/vio/mu/b8/e0.1/n00081920_fin_dip.eps}}}\hspace{-3em}%
\subfloat{
\resizebox{8cm}{4.5cm}{\includegraphics{figures/eps/vio/mu/b8/e0.1/n00081920_fin_dip_wovio.eps}}}\vspace{-1em}  \hspace{-3em}%
\end{center}

\begin{center}
\subfloat{
\resizebox{8cm}{4.5cm}{\includegraphics{figures/eps/vio/mu/b8/e0.1/inf_dip.eps}}}\hspace{-3em}%
\subfloat{
\resizebox{8cm}{4.5cm}{\includegraphics{figures/eps/vio/mu/b8/e0.1/inf_dip_wovio.eps}}}\vspace{-0.5em}  \hspace{-3em}%


\caption[\textbf{Infinite and finite diploid population behavior for $\bm{\mu}$ violation, $\ell = 8$ and $\bm{\epsilon} = 0.1$}]
{\textbf{Infinite and finite diploid population behavior for $\bm{\mu}$ violation, $\ell = 8$ and $\bm{\epsilon} = 0.1$:} 
  In left column, $d'$ is distance of finite or infinite population to limit $\bm{z}^\ast$ for $g$ generations. 
  In right column, $d$ is distance of finite or infinite population to limits $\bm{p}^\ast$ and $\bm{q}^\ast$. Green line is distance to $\bm{p}^\ast$ and red line is distance to $\bm{q}^\ast$.}
\label{oscillation_8d_vio_mu_0.1}
\end{center}
\end{figure}

% l = 10

\begin{figure}[h]
\begin{center}
\subfloat{
\resizebox{8cm}{4.5cm}{\includegraphics{figures/eps/vio/mu/b10/e0.1/n00004096_fin_dip.eps}}}\hspace{-3em}%
\subfloat{
\resizebox{8cm}{4.5cm}{\includegraphics{figures/eps/vio/mu/b10/e0.1/n00004096_fin_dip_wovio.eps}}}\vspace{-1em}  \hspace{-3em}%
\end{center}
\begin{center}
\subfloat{
\resizebox{8cm}{4.5cm}{\includegraphics{figures/eps/vio/mu/b10/e0.1/n00040960_fin_dip.eps}}}\hspace{-3em}%
\subfloat{
\resizebox{8cm}{4.5cm}{\includegraphics{figures/eps/vio/mu/b10/e0.1/n00040960_fin_dip_wovio.eps}}}\vspace{-1em}  \hspace{-3em}%
\end{center}


\begin{center}
\subfloat{
\resizebox{8cm}{4.5cm}{\includegraphics{figures/eps/vio/mu/b10/e0.1/n00081920_fin_dip.eps}}}\hspace{-3em}%
\subfloat{
\resizebox{8cm}{4.5cm}{\includegraphics{figures/eps/vio/mu/b10/e0.1/n00081920_fin_dip_wovio.eps}}}\vspace{-1em}  \hspace{-3em}%
\end{center}

\begin{center}
\subfloat{
\resizebox{8cm}{4.5cm}{\includegraphics{figures/eps/vio/mu/b10/e0.1/inf_dip.eps}}}\hspace{-3em}%
\subfloat{
\resizebox{8cm}{4.5cm}{\includegraphics{figures/eps/vio/mu/b10/e0.1/inf_dip_wovio.eps}}}\vspace{-0.5em}  \hspace{-3em}%


\caption[\textbf{Infinite and finite diploid population behavior for $\bm{\mu}$ violation, genome length $\ell = 10$ and $\bm{\epsilon} = 0.1$}]{\textbf{Infinite and finite diploid population behavior for $\bm{\mu}$ violation, genome length $\ell = 10$ and $\bm{\epsilon} = 0.1$:} 
  In left column, $d'$ is distance of finite or infinite population to limit $\bm{z}^\ast$ for $g$ generations. In right column, $d$ is distance of finite or infinite population to limits $\bm{p}^\ast$ and $\bm{q}^\ast$. Green line is distance to $\bm{p}^\ast$ and red line is distance to $\bm{q}^\ast$.}
\label{oscillation_10d_vio_mu_0.1}
\end{center}
\end{figure}

% l = 12

\begin{figure}[h]
\begin{center}
\subfloat{
\resizebox{8cm}{4.5cm}{\includegraphics{figures/eps/vio/mu/b12/e0.1/n00004096_fin_dip.eps}}}\hspace{-3em}%
\subfloat{
\resizebox{8cm}{4.5cm}{\includegraphics{figures/eps/vio/mu/b12/e0.1/n00004096_fin_dip_wovio.eps}}}\vspace{-1em}  \hspace{-3em}%
\end{center}
\begin{center}
\subfloat{
\resizebox{8cm}{4.5cm}{\includegraphics{figures/eps/vio/mu/b12/e0.1/n00040960_fin_dip.eps}}}\hspace{-3em}%
\subfloat{
\resizebox{8cm}{4.5cm}{\includegraphics{figures/eps/vio/mu/b12/e0.1/n00040960_fin_dip_wovio.eps}}}\vspace{-1em}  \hspace{-3em}%
\end{center}


\begin{center}
\subfloat{
\resizebox{8cm}{4.5cm}{\includegraphics{figures/eps/vio/mu/b12/e0.1/n00081920_fin_dip.eps}}}\hspace{-3em}%
\subfloat{
\resizebox{8cm}{4.5cm}{\includegraphics{figures/eps/vio/mu/b12/e0.1/n00081920_fin_dip_wovio.eps}}}\vspace{-1em}  \hspace{-3em}%
\end{center}

\begin{center}
\subfloat{
\resizebox{8cm}{4.5cm}{\includegraphics{figures/eps/vio/mu/b12/e0.1/inf_dip.eps}}}\hspace{-3em}%
\subfloat{
\resizebox{8cm}{4.5cm}{\includegraphics{figures/eps/vio/mu/b12/e0.1/inf_dip_wovio.eps}}}\vspace{-0.5em}  \hspace{-3em}%


\caption[\textbf{Infinite and finite diploid population behavior for $\bm{\mu}$ violation, genome length $\ell = 12$ and $\bm{\epsilon} = 0.1$}]{\textbf{Infinite and finite diploid population behavior for $\bm{\mu}$ violation, genome length $\ell = 12$ and $\bm{\epsilon} = 0.1$:} 
  In left column, $d'$ is distance of finite or infinite population to limit $\bm{z}^\ast$ for $g$ generations. In right column, $d$ is distance of finite or infinite population to limits $\bm{p}^\ast$ and $\bm{q}^\ast$. Green line is distance to $\bm{p}^\ast$ and red line is distance to $\bm{q}^\ast$.}
\label{oscillation_12d_vio_mu_0.1}
\end{center}
\end{figure}

% l = 14

\begin{figure}[h]
\begin{center}
\subfloat{
\resizebox{8cm}{4.5cm}{\includegraphics{figures/eps/vio/mu/b14/e0.1/n00004096_fin_dip.eps}}}\hspace{-3em}%
\subfloat{
\resizebox{8cm}{4.5cm}{\includegraphics{figures/eps/vio/mu/b14/e0.1/n00004096_fin_dip_wovio.eps}}}\vspace{-1em}  \hspace{-3em}%
\end{center}
\begin{center}
\subfloat{
\resizebox{8cm}{4.5cm}{\includegraphics{figures/eps/vio/mu/b14/e0.1/n00040960_fin_dip.eps}}}\hspace{-3em}%
\subfloat{
\resizebox{8cm}{4.5cm}{\includegraphics{figures/eps/vio/mu/b14/e0.1/n00040960_fin_dip_wovio.eps}}}\vspace{-1em}  \hspace{-3em}%
\end{center}


\begin{center}
\subfloat{
\resizebox{8cm}{4.5cm}{\includegraphics{figures/eps/vio/mu/b14/e0.1/n00081920_fin_dip.eps}}}\hspace{-3em}%
\subfloat{
\resizebox{8cm}{4.5cm}{\includegraphics{figures/eps/vio/mu/b14/e0.1/n00081920_fin_dip_wovio.eps}}}\vspace{-1em}  \hspace{-3em}%
\end{center}

\begin{center}
\subfloat{
\resizebox{8cm}{4.5cm}{\includegraphics{figures/eps/vio/mu/b14/e0.1/inf_dip.eps}}}\hspace{-3em}%
\subfloat{
\resizebox{8cm}{4.5cm}{\includegraphics{figures/eps/vio/mu/b14/e0.1/inf_dip_wovio.eps}}}\vspace{-0.5em}  \hspace{-3em}%


\caption[\textbf{Infinite and finite diploid population behavior for $\bm{\mu}$ violation, genome length $\ell = 14$ and $\bm{\epsilon} = 0.1$}]
{\textbf{Infinite and finite diploid population behavior for $\bm{\mu}$ violation, genome length $\ell = 14$ and $\bm{\epsilon} = 0.1$:} 
  In left column, $d'$ is distance of finite or infinite population to limit $\bm{z}^\ast$ for $g$ generations. 
  In right column, $d$ is distance of finite or infinite population to limits $\bm{p}^\ast$ and $\bm{q}^\ast$. 
  Green line is distance to $\bm{p}^\ast$ and red line is distance to $\bm{q}^\ast$.}
\label{oscillation_14d_vio_mu_0.1}
\end{center}
\end{figure}

% \clearpage
The right column in figures \ref{oscillation_8d_vio_mu_0.1} through \ref{oscillation_14d_vio_mu_0.1} 
shows distance of finite and infinite diploid populations with $\bm{\epsilon} \;=\; 0.1$ to non-violation limits $\bm{p^\ast}$ and $\bm{q^\ast}$. 
Those graphs indicate the oscillating behavior of finite diploid populations given violation; 
oscillation amplitudes decrease with time. 
Like in the haploid case, (for $\bm{\epsilon} \;=\; 0.1$) oscillations in infinite populations die out quickly, 
but finite population oscillation does not (and will reappear infinitely often). Rate of damping of ripples with $\bm{\epsilon} \;=\; 0.1$ is  
larger than with $\bm{\epsilon} \;=\; 0.01$. The all zeros mask created in mutation distribution with $\bm{\epsilon} \;=\; 0.1$ 
has a larger probability of being used during mutation as compared with the $\bm{\epsilon} = 0.01$ case.
More random wiggling of finite population is noticed than in case of $\bm{\epsilon} = 0.01$, and as value of $\ell$ 
increases, random wiggling is more prevalent.

The left column of figures \ref{oscillation_8d_vio_mu_0.1} through \ref{oscillation_14d_vio_mu_0.1} 
shows distance of finite and infinite diploid populations to limit $\bm{z^\ast}$ 
(limit with violation in mutation distribution $\bm{\mu}$) when $\bm{\epsilon} \;=\; 0.1$. 
The distance decreases as finite population size increases, 
and finite population shows behavior similar to infinite population behavior as population size grows. 
 
 \clearpage
 Average distance data for diploid population in case of violation in $\bm{\mu}$ distribution 
with $\bm{\epsilon} \;=\; 0.1$ are tabulated in table \ref{distanceMuDipEps0.1}. 
\begin{table}[ht]
\caption[\textbf{Distance measured for violation in $\bm{\mu}$ with $\bm{\epsilon} \;=\; 0.1$ for diploids}]{\textbf{Distance measured for violation in $\bm{\mu}$ with $\bm{\epsilon} \;=\; 0.1$ for diploids:} $\ell$ is genome length, 
average distance between finite and infinite population is tabulated in the last three columns, and last row is expected single step distance.}
\centering
\begin{tabularx}{0.75\textwidth}{ c *{3}{X}}
\toprule
$\ell$ & $N = 4096$ & $N = 40960$ & $N = 81920$ \\
\midrule
8 & 0.0156	&  0.0049	& 0.0035 \\
10 & 0.0156	&  0.0049	& 0.0035 \\
12 & 0.0156	&  0.0049	& 0.0035 \\
14 & 0.0156	&  0.0049	& 0.0035 \\
\midrule
$1/\sqrt{N}$ & 0.0156 & 0.0049 & 0.0035 \\
\bottomrule
\end{tabularx}
\label{distanceMuDipEps0.1}
\end{table}

Table \ref{distanceMuDipEps0.1} shows the average distance 
between finite population and infinite population agrees with the expected single step distance 
$1/\sqrt{N}$. 
 
% \subsubsection{$\bm{\epsilon: 0.5}$}
% l = 8
% \mbox{}\\[-0.75in]
\begin{figure}[!b]
\begin{center}
\subfloat{
\resizebox{8cm}{4.5cm}{\includegraphics{figures/eps/vio/mu/b8/e0.5/n00004096_fin_dip.eps}}}\hspace{-3em}%
\subfloat{
\resizebox{8cm}{4.5cm}{\includegraphics{figures/eps/vio/mu/b8/e0.5/n00004096_fin_dip_wovio.eps}}}\vspace{-1em}  \hspace{-3em}%
\end{center}
\begin{center}
\subfloat{
\resizebox{8cm}{4.5cm}{\includegraphics{figures/eps/vio/mu/b8/e0.5/n00040960_fin_dip.eps}}}\hspace{-3em}%
\subfloat{
\resizebox{8cm}{4.5cm}{\includegraphics{figures/eps/vio/mu/b8/e0.5/n00040960_fin_dip_wovio.eps}}}\vspace{-1em}  \hspace{-3em}%
\end{center}

\begin{center}
\subfloat{
\resizebox{8cm}{4.5cm}{\includegraphics{figures/eps/vio/mu/b8/e0.5/n00081920_fin_dip.eps}}}\hspace{-3em}%
\subfloat{
\resizebox{8cm}{4.5cm}{\includegraphics{figures/eps/vio/mu/b8/e0.5/n00081920_fin_dip_wovio.eps}}}\vspace{-1em}  \hspace{-3em}%
\end{center}

\begin{center}
\subfloat{
\resizebox{8cm}{4.5cm}{\includegraphics{figures/eps/vio/mu/b8/e0.5/inf_dip.eps}}}\hspace{-3em}%
\subfloat{
\resizebox{8cm}{4.5cm}{\includegraphics{figures/eps/vio/mu/b8/e0.5/inf_dip_wovio.eps}}}\vspace{-0.5em}  \hspace{-3em}%


\caption[\textbf{Infinite and finite diploid population behavior for $\bm{\mu}$ violation, $\ell = 8$ and $\bm{\epsilon} = 0.5$}]
{\textbf{Infinite and finite diploid population behavior for $\bm{\mu}$ violation, $\ell = 8$ and $\bm{\epsilon} = 0.5$:} 
  In left column, $d'$ is distance of finite or infinite population to limit $\bm{z}^\ast$ for $g$ generations. 
  In right column, $d$ is distance of finite or infinite population to limits $\bm{p}^\ast$ and $\bm{q}^\ast$. Green line is distance to $\bm{p}^\ast$ and red line is distance to $\bm{q}^\ast$.}
\label{oscillation_8d_vio_mu_0.5}
\end{center}
\end{figure}

% l = 10

\begin{figure}[h]
\begin{center}
\subfloat{
\resizebox{8cm}{4.5cm}{\includegraphics{figures/eps/vio/mu/b10/e0.5/n00004096_fin_dip.eps}}}\hspace{-3em}%
\subfloat{
\resizebox{8cm}{4.5cm}{\includegraphics{figures/eps/vio/mu/b10/e0.5/n00004096_fin_dip_wovio.eps}}}\vspace{-1em}  \hspace{-3em}%
\end{center}
\begin{center}
\subfloat{
\resizebox{8cm}{4.5cm}{\includegraphics{figures/eps/vio/mu/b10/e0.5/n00040960_fin_dip.eps}}}\hspace{-3em}%
\subfloat{
\resizebox{8cm}{4.5cm}{\includegraphics{figures/eps/vio/mu/b10/e0.5/n00040960_fin_dip_wovio.eps}}}\vspace{-1em}  \hspace{-3em}%
\end{center}


\begin{center}
\subfloat{
\resizebox{8cm}{4.5cm}{\includegraphics{figures/eps/vio/mu/b10/e0.5/n00081920_fin_dip.eps}}}\hspace{-3em}%
\subfloat{
\resizebox{8cm}{4.5cm}{\includegraphics{figures/eps/vio/mu/b10/e0.5/n00081920_fin_dip_wovio.eps}}}\vspace{-1em}  \hspace{-3em}%
\end{center}

\begin{center}
\subfloat{
\resizebox{8cm}{4.5cm}{\includegraphics{figures/eps/vio/mu/b10/e0.5/inf_dip.eps}}}\hspace{-3em}%
\subfloat{
\resizebox{8cm}{4.5cm}{\includegraphics{figures/eps/vio/mu/b10/e0.5/inf_dip_wovio.eps}}}\vspace{-0.5em}  \hspace{-3em}%


\caption[\textbf{Infinite and finite diploid population behavior for $\bm{\mu}$ violation, genome length $\ell = 10$ and $\bm{\epsilon} = 0.5$}]{\textbf{Infinite and finite diploid population behavior for $\bm{\mu}$ violation, genome length $\ell = 10$ and $\bm{\epsilon} = 0.5$:} 
  In left column, $d'$ is distance of finite or infinite population to limit $\bm{z}^\ast$ for $g$ generations. In right column, $d$ is distance of finite or infinite population to limits $\bm{p}^\ast$ and $\bm{q}^\ast$. Green line is distance to $\bm{p}^\ast$ and red line is distance to $\bm{q}^\ast$.}
\label{oscillation_10d_vio_mu_0.5}
\end{center}
\end{figure}

% l = 12

\begin{figure}[h]
\begin{center}
\subfloat{
\resizebox{8cm}{4.5cm}{\includegraphics{figures/eps/vio/mu/b12/e0.5/n00004096_fin_dip.eps}}}\hspace{-3em}%
\subfloat{
\resizebox{8cm}{4.5cm}{\includegraphics{figures/eps/vio/mu/b12/e0.5/n00004096_fin_dip_wovio.eps}}}\vspace{-1em}  \hspace{-3em}%
\end{center}
\begin{center}
\subfloat{
\resizebox{8cm}{4.5cm}{\includegraphics{figures/eps/vio/mu/b12/e0.5/n00040960_fin_dip.eps}}}\hspace{-3em}%
\subfloat{
\resizebox{8cm}{4.5cm}{\includegraphics{figures/eps/vio/mu/b12/e0.5/n00040960_fin_dip_wovio.eps}}}\vspace{-1em}  \hspace{-3em}%
\end{center}


\begin{center}
\subfloat{
\resizebox{8cm}{4.5cm}{\includegraphics{figures/eps/vio/mu/b12/e0.5/n00081920_fin_dip.eps}}}\hspace{-3em}%
\subfloat{
\resizebox{8cm}{4.5cm}{\includegraphics{figures/eps/vio/mu/b12/e0.5/n00081920_fin_dip_wovio.eps}}}\vspace{-1em}  \hspace{-3em}%
\end{center}

\begin{center}
\subfloat{
\resizebox{8cm}{4.5cm}{\includegraphics{figures/eps/vio/mu/b12/e0.5/inf_dip.eps}}}\hspace{-3em}%
\subfloat{
\resizebox{8cm}{4.5cm}{\includegraphics{figures/eps/vio/mu/b12/e0.5/inf_dip_wovio.eps}}}\vspace{-0.5em}  \hspace{-3em}%


\caption[\textbf{Infinite and finite diploid population behavior for $\bm{\mu}$ violation, genome length $\ell = 12$ and $\bm{\epsilon} = 0.5$}]{\textbf{Infinite and finite diploid population behavior for $\bm{\mu}$ violation, genome length $\ell = 12$ and $\bm{\epsilon} = 0.5$:} 
  In left column, $d'$ is distance of finite or infinite population to limit $\bm{z}^\ast$ for $g$ generations. In right column, $d$ is distance of finite or infinite population to limits $\bm{p}^\ast$ and $\bm{q}^\ast$. Green line is distance to $\bm{p}^\ast$ and red line is distance to $\bm{q}^\ast$.}
\label{oscillation_12d_vio_mu_0.5}
\end{center}
\end{figure}

% l = 14

\begin{figure}[h]
\begin{center}
\subfloat{
\resizebox{8cm}{4.5cm}{\includegraphics{figures/eps/vio/mu/b14/e0.5/n00004096_fin_dip.eps}}}\hspace{-3em}%
\subfloat{
\resizebox{8cm}{4.5cm}{\includegraphics{figures/eps/vio/mu/b14/e0.5/n00004096_fin_dip_wovio.eps}}}\vspace{-1em}  \hspace{-3em}%
\end{center}
\begin{center}
\subfloat{
\resizebox{8cm}{4.5cm}{\includegraphics{figures/eps/vio/mu/b14/e0.5/n00040960_fin_dip.eps}}}\hspace{-3em}%
\subfloat{
\resizebox{8cm}{4.5cm}{\includegraphics{figures/eps/vio/mu/b14/e0.5/n00040960_fin_dip_wovio.eps}}}\vspace{-1em}  \hspace{-3em}%
\end{center}


\begin{center}
\subfloat{
\resizebox{8cm}{4.5cm}{\includegraphics{figures/eps/vio/mu/b14/e0.5/n00081920_fin_dip.eps}}}\hspace{-3em}%
\subfloat{
\resizebox{8cm}{4.5cm}{\includegraphics{figures/eps/vio/mu/b14/e0.5/n00081920_fin_dip_wovio.eps}}}\vspace{-1em}  \hspace{-3em}%
\end{center}

\begin{center}
\subfloat{
\resizebox{8cm}{4.5cm}{\includegraphics{figures/eps/vio/mu/b14/e0.5/inf_dip.eps}}}\hspace{-3em}%
\subfloat{
\resizebox{8cm}{4.5cm}{\includegraphics{figures/eps/vio/mu/b14/e0.5/inf_dip_wovio.eps}}}\vspace{-0.5em}  \hspace{-3em}%


\caption[\textbf{Infinite and finite diploid population behavior for $\bm{\mu}$ violation, genome length $\ell = 14$ and $\bm{\epsilon} = 0.5$}]{\textbf{Infinite and finite diploid population behavior for $\bm{\mu}$ violation, genome length $\ell = 14$ and $\bm{\epsilon} = 0.5$:} 
  In left column, $d'$ is distance of finite or infinite population to limit $\bm{z}^\ast$ for $g$ generations. In right column, $d$ is distance of finite or infinite population to limits $\bm{p}^\ast$ and $\bm{q}^\ast$. Green line is distance to $\bm{p}^\ast$ and red line is distance to $\bm{q}^\ast$.}
\label{oscillation_14d_vio_mu_0.5}
\end{center}
\end{figure}

% \clearpage

The right column in figures \ref{oscillation_8d_vio_mu_0.5} through \ref{oscillation_14d_vio_mu_0.5} 
shows distance of finite and infinite diploid populations with $\bm{\epsilon} \;=\; 0.5$ to non-violation limits $\bm{p^\ast}$ and $\bm{q^\ast}$. 
Neither finite nor infinite populations show noticeable oscillation given violation. 
The all zeros mask created in mutation distribution with $\bm{\epsilon} \;=\; 0.5$ has a large probability 
of being used during mutation, so finite population oscillation decreased significantly.

The left column of figures \ref{oscillation_8d_vio_mu_0.5} through \ref{oscillation_14d_vio_mu_0.5} 
shows distance of finite and infinite diploid populations to limit $\bm{z^\ast}$ 
(limit with violation in mutation distribution $\bm{\mu}$) when $\bm{\epsilon} \;=\; 0.5$. 
The distance decreases as finite population size increases, 
and finite population shows behavior similar to infinite population as population size grows. 
Average distance data for diploid population in case of violation in $\bm{\mu}$ distribution 
with $\bm{\epsilon} \;=\; 0.5$ are tabulated in table \ref{distanceMuDipEps0.5}.

\clearpage
\begin{table}[ht]
\caption[\textbf{Distance measured for violation in $\bm{\mu}$ with $\bm{\epsilon} \;=\; 0.5$ for diploids}]{\textbf{Distance measured for violation in $\bm{\mu}$ with $\bm{\epsilon} \;=\; 0.5$ for diploids:} $\ell$ is genome length, 
average distance between finite and infinite population is tabulated in the last three columns, and last row is expected single step distance.}
\centering
\begin{tabularx}{0.75\textwidth}{ c *{3}{X}}
\toprule
$\ell$ & $N = 4096$ & $N = 40960$ & $N = 81920$ \\
\midrule
8 & 0.0156	&  0.0049	& 0.0035 \\	
10 & 0.0156	&  0.0049 	& 0.0035 \\
12 & 0.0156	&  0.0049	& 0.0035 \\
14 & 0.0156	&  0.0049	& 0.0035 \\
\midrule
$1/\sqrt{N}$ & 0.0156 & 0.0049 & 0.0035 \\
\bottomrule
\end{tabularx}
\label{distanceMuDipEps0.5}
\end{table}

Table \ref{distanceMuDipEps0.5} shows that the average distance between 
finite population and infinite population agrees with the expected single step distance $1/\sqrt{N}$.  


As value of $\ell$ increases, amplitude of oscillation decreases. 
Also as value of $\ell$ increases, random wiggling occurs for smaller population size, 
and larger population size is needed to observe good oscillations. Random wiggling of 
population is more evident in diploid case than in haploid case because length of 
population member string in diploid is $2\ell$. With increase in value of $\bm{\epsilon}$, 
oscillation in population diminishes, and oscillation completely ceases after certain threshold value for $\bm{\epsilon}$. 

\section{Violation in Crossover Distribution}
The crossover distribution $\bm{\chi}$ was modified as
\[
\bm{\chi}_i = (1-\bm{\epsilon}) \bm{\chi} ; \tabspace i = \{1, 2,.., 2^{\ell}-1\} 
\]
So that 
\[
\bm{\chi}_i + \bm{\chi}_{i+g} = 1-\bm{\epsilon} ; \tabspace g \text{ is defined in  section } \ref{Limits}
\]

Then $j$ is chosen where $\bm{\chi}_j = 0$ and set $\bm{\chi}_j = \bm{\epsilon}$. 

Simulations were repeated with the violations in (\ref{OscCond}) described above for crossover distribution.
The distances of both infinite and finite populations to limit $\bm{z}^\ast$ were plotted. 
The distances of both infinite and finite populations to non-violation limits $\bm{p}^\ast$ and $\bm{q}^\ast$ were also plotted.

% figures for chi violation
\subsection{Haploid Population}
% \subsubsection{$\bm{\epsilon}: 0.01$}

% l = 8

\begin{figure}[H]
\begin{center}
\subfloat{
\resizebox{8cm}{5cm}{\includegraphics{figures/eps/vio/chi/b8/e0.01/n00004096_fin_hap.eps}}} \hspace{-3em}%
\subfloat{
\resizebox{8cm}{5cm}{\includegraphics{figures/eps/vio/chi/b8/e0.01/n00004096_fin_hap_wovio.eps}}}\vspace{-1em} \hspace{-3em}%
\end{center}
\begin{center}
\subfloat{
\resizebox{8cm}{5cm}{\includegraphics{figures/eps/vio/chi/b8/e0.01/n00040960_fin_hap.eps}}} \hspace{-3em}%
\subfloat{
\resizebox{8cm}{5cm}{\includegraphics{figures/eps/vio/chi/b8/e0.01/n00040960_fin_hap_wovio.eps}}}\vspace{-1em} \hspace{-3em}%
\end{center}

\begin{center}
\subfloat{
\resizebox{8cm}{5cm}{\includegraphics{figures/eps/vio/chi/b8/e0.01/n00081920_fin_hap.eps}}} \hspace{-3em}%
\subfloat{
\resizebox{8cm}{5cm}{\includegraphics{figures/eps/vio/chi/b8/e0.01/n00081920_fin_hap_wovio.eps}}}\vspace{-1em} \hspace{-3em}%
\end{center}

\begin{center}
\subfloat{
\resizebox{8cm}{5cm}{\includegraphics{figures/eps/vio/chi/b8/e0.01/inf_hap.eps}}}\hspace{-3em}%
\subfloat{
\resizebox{8cm}{5cm}{\includegraphics{figures/eps/vio/chi/b8/e0.01/inf_hap_wovio.eps}}}\vspace{-0.5em} \hspace{-3em}%


\caption{\textbf{Infinite and finite haploid population oscillation behavior in case of violation in $\bm{\chi}$ for genome length $\ell = 8$ and $\bm{\epsilon} = 0.01$:} 
  In left column, $d'$ is distance of finite population of size $n$ or infinite population to limit $\bm{z}^\ast$ for $g$ generations. In right column, $d$ is distance of finite population of size $N$ or infinite population to limits without violation.}
\label{oscillation_8h_vio_chi_0.01}
\end{center}
\end{figure}


% l = 10

\begin{figure}[H]
\begin{center}
\subfloat{
\resizebox{8cm}{5cm}{\includegraphics{figures/eps/vio/chi/b10/e0.01/n00004096_fin_hap.eps}}} \hspace{-3em}%
\subfloat{
\resizebox{8cm}{5cm}{\includegraphics{figures/eps/vio/chi/b10/e0.01/n00004096_fin_hap_wovio.eps}}}\vspace{-1em} \hspace{-3em}%
\end{center}
\begin{center}
\subfloat{
\resizebox{8cm}{5cm}{\includegraphics{figures/eps/vio/chi/b10/e0.01/n00040960_fin_hap.eps}}} \hspace{-3em}%
\subfloat{
\resizebox{8cm}{5cm}{\includegraphics{figures/eps/vio/chi/b10/e0.01/n00040960_fin_hap_wovio.eps}}}\vspace{-1em} \hspace{-3em}%
\end{center}

\begin{center}
\subfloat{
\resizebox{8cm}{5cm}{\includegraphics{figures/eps/vio/chi/b10/e0.01/n00081920_fin_hap.eps}}} \hspace{-3em}%
\subfloat{
\resizebox{8cm}{5cm}{\includegraphics{figures/eps/vio/chi/b10/e0.01/n00081920_fin_hap_wovio.eps}}}\vspace{-1em} \hspace{-3em}%
\end{center}

\begin{center}
\subfloat{
\resizebox{8cm}{5cm}{\includegraphics{figures/eps/vio/chi/b10/e0.01/inf_hap.eps}}}\hspace{-3em}%
\subfloat{
\resizebox{8cm}{5cm}{\includegraphics{figures/eps/vio/chi/b10/e0.01/inf_hap_wovio.eps}}}\vspace{-0.5em} \hspace{-3em}%

\caption{\textbf{Infinite and finite haploid population oscillation behavior in case of violation in $\bm{\chi}$ for genome length $\ell = 10$ and $\bm{\epsilon} = 0.01$:} 
  In left column, $d'$ is distance of finite population of size $n$ or infinite population to limit $\bm{z}^\ast$ for $g$ generations. In right column, $d$ is distance of finite population of size $N$ or infinite population to limits without violation.}
\label{oscillation_10h_vio_chi_0.01}
\end{center}
\end{figure}


% l = 12
\begin{figure}[H]
\begin{center}
\subfloat{
\resizebox{8cm}{5cm}{\includegraphics{figures/eps/vio/chi/b12/e0.01/n00004096_fin_hap.eps}}} \hspace{-3em}%
\subfloat{
\resizebox{8cm}{5cm}{\includegraphics{figures/eps/vio/chi/b12/e0.01/n00004096_fin_hap_wovio.eps}}}\vspace{-1em} \hspace{-3em}%
\end{center}
\begin{center}
\subfloat{
\resizebox{8cm}{5cm}{\includegraphics{figures/eps/vio/chi/b12/e0.01/n00040960_fin_hap.eps}}} \hspace{-3em}%
\subfloat{
\resizebox{8cm}{5cm}{\includegraphics{figures/eps/vio/chi/b12/e0.01/n00040960_fin_hap_wovio.eps}}}\vspace{-1em} \hspace{-3em}%
\end{center}

\begin{center}
\subfloat{
\resizebox{8cm}{5cm}{\includegraphics{figures/eps/vio/chi/b12/e0.01/n00081920_fin_hap.eps}}} \hspace{-3em}%
\subfloat{
\resizebox{8cm}{5cm}{\includegraphics{figures/eps/vio/chi/b12/e0.01/n00081920_fin_hap_wovio.eps}}}\vspace{-1em} \hspace{-3em}%
\end{center}

\begin{center}
\subfloat{
\resizebox{8cm}{5cm}{\includegraphics{figures/eps/vio/chi/b12/e0.01/inf_hap.eps}}}\hspace{-3em}%
\subfloat{
\resizebox{8cm}{5cm}{\includegraphics{figures/eps/vio/chi/b12/e0.01/inf_hap_wovio.eps}}}\vspace{-0.5em} \hspace{-3em}%


\caption{\textbf{Infinite and finite haploid population oscillation behavior in case of violation in $\bm{\chi}$ for genome length $\ell = 12$ and $\bm{\epsilon} = 0.01$:} 
  In left column, $d'$ is distance of finite population of size $n$ or infinite population to limit $\bm{z}^\ast$ for $g$ generations. In right column, $d$ is distance of finite population of size $N$ or infinite population to limits without violation.}
\label{oscillation_12h_vio_chi_0.01}
\end{center}
\end{figure}

% l = 14

\begin{figure}[H]
\begin{center}
\subfloat{
\resizebox{8cm}{5cm}{\includegraphics{figures/eps/vio/chi/b14/e0.01/n00004096_fin_hap.eps}}} \hspace{-3em}%
\subfloat{
\resizebox{8cm}{5cm}{\includegraphics{figures/eps/vio/chi/b14/e0.01/n00004096_fin_hap_wovio.eps}}}\vspace{-1em} \hspace{-3em}%
\end{center}
\begin{center}
\subfloat{
\resizebox{8cm}{5cm}{\includegraphics{figures/eps/vio/chi/b14/e0.01/n00040960_fin_hap.eps}}} \hspace{-3em}%
\subfloat{
\resizebox{8cm}{5cm}{\includegraphics{figures/eps/vio/chi/b14/e0.01/n00040960_fin_hap_wovio.eps}}}\vspace{-1em} \hspace{-3em}%
\end{center}

\begin{center}
\subfloat{
\resizebox{8cm}{5cm}{\includegraphics{figures/eps/vio/chi/b14/e0.01/n00081920_fin_hap.eps}}} \hspace{-3em}%
\subfloat{
\resizebox{8cm}{5cm}{\includegraphics{figures/eps/vio/chi/b14/e0.01/n00081920_fin_hap_wovio.eps}}}\vspace{-1em} \hspace{-3em}%
\end{center}

\begin{center}
\subfloat{
\resizebox{8cm}{5cm}{\includegraphics{figures/eps/vio/chi/b14/e0.01/inf_hap.eps}}}\hspace{-3em}%
\subfloat{
\resizebox{8cm}{5cm}{\includegraphics{figures/eps/vio/chi/b14/e0.01/inf_hap_wovio.eps}}}\vspace{-0.5em} \hspace{-3em}%

\caption{\textbf{Infinite and finite haploid population oscillation behavior in case of violation in $\bm{\chi}$ for genome length $\ell = 14$ and $\bm{\epsilon} = 0.01$:} 
  In left column, $d'$ is distance of finite population of size $n$ or infinite population to limit $\bm{z}^\ast$ for $g$ generations. In right column, $d$ is distance of finite population of size $N$ or infinite population to limits without violation.}
\label{oscillation_14h_vio_chi_0.01}
\end{center}
\end{figure}

The right column in figures \ref{oscillation_8h_vio_chi_0.01} through \ref{oscillation_14h_vio_chi_0.01} 
shows distance of finite and infinite haploid populations to non-violation limits $\bm{p^\ast}$ and $\bm{q^\ast}$ with $\bm{\epsilon} \;=\; 0.01$. 
Graphs in the right column give picture of oscillating behavior of haploid population given violation. 
Both finite and infinite populations oscillate given violation. Since value of $\bm{\epsilon}$ 
is very small, damping of ripples is slow. New masks created in crossover distribution with $\bm{\epsilon} \;=\; 0.01$ have very small 
probability of being used during crossover. The probability of the new masks being used is so low for $\bm{\epsilon} \;=\; 0.01$ that even 
infinite population oscillation did not die out completely in 50 generations. 

The left column of figures \ref{oscillation_8h_vio_chi_0.01} through \ref{oscillation_14h_vio_chi_0.01} 
shows distance of finite and infinite haploid populations to limit $\bm{z^\ast}$ 
(limit with violation in crossover distribution $\bm{\chi}$) when $\bm{\epsilon} \;=\; 0.01$. 
The distance between finite population and limit $\bm{z}^\ast$ (limit with violation in $\bm{\chi}$ distribution) 
decreases as finite population size increases, 
and finite population shows behavior similar to infinite population behavior as finite population reach large number. 
The distance data for haploid population in case of violation in $\bm{\chi}$ distribution 
with $\bm{\epsilon} \;=\; 0.01$ for different finite population size $N$ are tabulated in table \ref{distanceChiHapEps0.01}.

\begin{table}[ht]
\caption{\textbf{Distance measured for violation in $\bm{\chi}$ with $\bm{\epsilon} \;=\; 0.01$  for haploids:} $\ell$ is genome length, 
and average distance between finite and 
infinite populations is tabulated in last three columns.}
\centering
\begin{tabularx}{0.75\textwidth}{ c *{3}{X}}
\toprule
$\ell$ & $N = 4096$ & $N = 40960$ & $N = 81920$  \\
\midrule
8 & 0.0186	&  0.0150 	& 0.0115 \\
10 & 0.0158	&  0.0062 	& 0.0051 \\ 
12 & 0.0158	&  0.0056	& 0.0045 \\
14 & 0.0156	&  0.0050	& 0.0036 \\ 
\bottomrule
\end{tabularx}
\label{distanceChiHapEps0.01}
\end{table} 

From table \ref{distanceChiHapEps0.01}, average distance calculated for finite population size $4096$ is $0.0164$, 
for size $40960$ is $0.0079$ and for size $81920$ is $0.0062$. These results show average distance 
between finite population and limit $\bm{z^\ast}$ closely follows expected single step distance 
between finite and infinite population given in \ref{tableExpectedDistance}. The distance decreased as $1/\sqrt{N}$. 
Also, the distance decreased as genome length $\ell$ increased for all sizes of finite haploid populations 
with $\bm{\epsilon} \;=\; 0.01$.

% \subsubsection{$\bm{\epsilon}: 0.1$}

% l = 8

\begin{figure}[h]
\begin{center}
\subfloat{
\resizebox{8cm}{5cm}{\includegraphics{figures/eps/vio/chi/b8/e0.1/n00004096_fin_hap.eps}}} \hspace{-3em}%
\subfloat{
\resizebox{8cm}{5cm}{\includegraphics{figures/eps/vio/chi/b8/e0.1/n00004096_fin_hap_wovio.eps}}}\vspace{-1em} \hspace{-3em}%
\end{center}
\begin{center}
\subfloat{
\resizebox{8cm}{5cm}{\includegraphics{figures/eps/vio/chi/b8/e0.1/n00040960_fin_hap.eps}}} \hspace{-3em}%
\subfloat{
\resizebox{8cm}{5cm}{\includegraphics{figures/eps/vio/chi/b8/e0.1/n00040960_fin_hap_wovio.eps}}}\vspace{-1em} \hspace{-3em}%
\end{center}

\begin{center}
\subfloat{
\resizebox{8cm}{5cm}{\includegraphics{figures/eps/vio/chi/b8/e0.1/n00081920_fin_hap.eps}}} \hspace{-3em}%
\subfloat{
\resizebox{8cm}{5cm}{\includegraphics{figures/eps/vio/chi/b8/e0.1/n00081920_fin_hap_wovio.eps}}}\vspace{-1em} \hspace{-3em}%
\end{center}

\begin{center}
\subfloat{
\resizebox{8cm}{5cm}{\includegraphics{figures/eps/vio/chi/b8/e0.1/inf_hap.eps}}}\hspace{-3em}%
\subfloat{
\resizebox{8cm}{5cm}{\includegraphics{figures/eps/vio/chi/b8/e0.1/inf_hap_wovio.eps}}}\vspace{-0.5em} \hspace{-3em}%


\caption{\textbf{Infinite and finite haploid population oscillation behavior in case of violation in $\bm{\chi}$ for genome length $\ell = 8$ and $\bm{\epsilon} = 0.1$:} 
  In left column, $d'$ is distance of finite population of size $n$ or infinite population to limit $\bm{z}^\ast$ for $g$ generations. In right column, $d$ is distance of finite population or infinite population to limits $\bm{p}^\ast$ and $\bm{q}^\ast$ without violation.}
\label{oscillation_8h_vio_chi_0.1}
\end{center}
\end{figure}

% l = 10

\begin{figure}[h]
\begin{center}
\subfloat{
\resizebox{8cm}{5cm}{\includegraphics{figures/eps/vio/chi/b10/e0.1/n00004096_fin_hap.eps}}} \hspace{-3em}%
\subfloat{
\resizebox{8cm}{5cm}{\includegraphics{figures/eps/vio/chi/b10/e0.1/n00004096_fin_hap_wovio.eps}}}\vspace{-1em} \hspace{-3em}%
\end{center}
\begin{center}
\subfloat{
\resizebox{8cm}{5cm}{\includegraphics{figures/eps/vio/chi/b10/e0.1/n00040960_fin_hap.eps}}} \hspace{-3em}%
\subfloat{
\resizebox{8cm}{5cm}{\includegraphics{figures/eps/vio/chi/b10/e0.1/n00040960_fin_hap_wovio.eps}}}\vspace{-1em} \hspace{-3em}%
\end{center}

\begin{center}
\subfloat{
\resizebox{8cm}{5cm}{\includegraphics{figures/eps/vio/chi/b10/e0.1/n00081920_fin_hap.eps}}} \hspace{-3em}%
\subfloat{
\resizebox{8cm}{5cm}{\includegraphics{figures/eps/vio/chi/b10/e0.1/n00081920_fin_hap_wovio.eps}}}\vspace{-1em} \hspace{-3em}%
\end{center}

\begin{center}
\subfloat{
\resizebox{8cm}{5cm}{\includegraphics{figures/eps/vio/chi/b10/e0.1/inf_hap.eps}}}\hspace{-3em}%
\subfloat{
\resizebox{8cm}{5cm}{\includegraphics{figures/eps/vio/chi/b10/e0.1/inf_hap_wovio.eps}}}\vspace{-0.5em} \hspace{-3em}%

\caption{\textbf{Infinite and finite haploid population oscillation behavior in case of violation in $\bm{\chi}$ for genome length $\ell = 10$ and $\bm{\epsilon} = 0.1$:} 
  In left column, $d'$ is distance of finite population of size $n$ or infinite population to limit $\bm{z}^\ast$ for $g$ generations. In right column, $d$ is distance of finite population or infinite population to limits $\bm{p}^\ast$ and $\bm{q}^\ast$ without violation.}
\label{oscillation_10h_vio_chi_0.1}
\end{center}
\end{figure}


% l = 12
\begin{figure}[h]
\begin{center}
\subfloat{
\resizebox{8cm}{5cm}{\includegraphics{figures/eps/vio/chi/b12/e0.1/n00004096_fin_hap.eps}}} \hspace{-3em}%
\subfloat{
\resizebox{8cm}{5cm}{\includegraphics{figures/eps/vio/chi/b12/e0.1/n00004096_fin_hap_wovio.eps}}}\vspace{-1em} \hspace{-3em}%
\end{center}
\begin{center}
\subfloat{
\resizebox{8cm}{5cm}{\includegraphics{figures/eps/vio/chi/b12/e0.1/n00040960_fin_hap.eps}}} \hspace{-3em}%
\subfloat{
\resizebox{8cm}{5cm}{\includegraphics{figures/eps/vio/chi/b12/e0.1/n00040960_fin_hap_wovio.eps}}}\vspace{-1em} \hspace{-3em}%
\end{center}

\begin{center}
\subfloat{
\resizebox{8cm}{5cm}{\includegraphics{figures/eps/vio/chi/b12/e0.1/n00081920_fin_hap.eps}}} \hspace{-3em}%
\subfloat{
\resizebox{8cm}{5cm}{\includegraphics{figures/eps/vio/chi/b12/e0.1/n00081920_fin_hap_wovio.eps}}}\vspace{-1em} \hspace{-3em}%
\end{center}

\begin{center}
\subfloat{
\resizebox{8cm}{5cm}{\includegraphics{figures/eps/vio/chi/b12/e0.1/inf_hap.eps}}}\hspace{-3em}%
\subfloat{
\resizebox{8cm}{5cm}{\includegraphics{figures/eps/vio/chi/b12/e0.1/inf_hap_wovio.eps}}}\vspace{-0.5em} \hspace{-3em}%


\caption{\textbf{Infinite and finite haploid population oscillation behavior in case of violation in $\bm{\chi}$ for genome length $\ell = 12$ and $\bm{\epsilon} = 0.1$:} 
  In left column, $d'$ is distance of finite population of size $n$ or infinite population to limit $\bm{z}^\ast$ for $g$ generations. In right column, $d$ is distance of finite population or infinite population to limits $\bm{p}^\ast$ and $\bm{q}^\ast$ without violation.}
\label{oscillation_12h_vio_chi_0.1}
\end{center}
\end{figure}

% l = 14

\begin{figure}[h]
\begin{center}
\subfloat{
\resizebox{8cm}{5cm}{\includegraphics{figures/eps/vio/chi/b14/e0.1/n00004096_fin_hap.eps}}} \hspace{-3em}%
\subfloat{
\resizebox{8cm}{5cm}{\includegraphics{figures/eps/vio/chi/b14/e0.1/n00004096_fin_hap_wovio.eps}}}\vspace{-1em} \hspace{-3em}%
\end{center}
\begin{center}
\subfloat{
\resizebox{8cm}{5cm}{\includegraphics{figures/eps/vio/chi/b14/e0.1/n00040960_fin_hap.eps}}} \hspace{-3em}%
\subfloat{
\resizebox{8cm}{5cm}{\includegraphics{figures/eps/vio/chi/b14/e0.1/n00040960_fin_hap_wovio.eps}}}\vspace{-1em} \hspace{-3em}%
\end{center}

\begin{center}
\subfloat{
\resizebox{8cm}{5cm}{\includegraphics{figures/eps/vio/chi/b14/e0.1/n00081920_fin_hap.eps}}} \hspace{-3em}%
\subfloat{
\resizebox{8cm}{5cm}{\includegraphics{figures/eps/vio/chi/b14/e0.1/n00081920_fin_hap_wovio.eps}}}\vspace{-1em} \hspace{-3em}%
\end{center}

\begin{center}
\subfloat{
\resizebox{8cm}{5cm}{\includegraphics{figures/eps/vio/chi/b14/e0.1/inf_hap.eps}}}\hspace{-3em}%
\subfloat{
\resizebox{8cm}{5cm}{\includegraphics{figures/eps/vio/chi/b14/e0.1/inf_hap_wovio.eps}}}\vspace{-0.5em} \hspace{-3em}%

\caption{\textbf{Infinite and finite haploid population oscillation behavior in case of violation in $\bm{\chi}$ for genome length $\ell = 14$ and $\bm{\epsilon} = 0.1$:} 
  In left column, $d'$ is distance of finite population of size $n$ or infinite population to limit $\bm{z}^\ast$ for $g$ generations. In right column, $d$ is distance of finite population or infinite population to limits $\bm{p}^\ast$ and $\bm{q}^\ast$ without violation.}
\label{oscillation_14h_vio_chi_0.1}
\end{center}
\end{figure}

\clearpage

The right column in figures \ref{oscillation_8h_vio_chi_0.1} through \ref{oscillation_14h_vio_chi_0.1} 
shows distance of finite and infinite haploid populations to non-violation limits $\bm{p^\ast}$ and $\bm{q^\ast}$ with $\bm{\epsilon} \;=\; 0.1$. 
Those graphs indicate oscillating behavior of haploid population given violation. 
Both finite and infinite populations oscillate given violation, and oscillations amplitudes decreases with time. 
However, for $\bm{\epsilon} \;=\; 0.1$, oscillations in infinite populations die out quickly, 
but oscillations in finite populations does not die out completely. Rate of damping of ripples with $\bm{\epsilon} \;=\; 0.1$ is  
larger than with $\bm{\epsilon} \;=\; 0.01$. New masks created in crossover distribution with $\bm{\epsilon} \;=\; 0.1$ have small  
probability of being used during crossover for oscillation in finite populations to die out completely. 

The left column of figures \ref{oscillation_8h_vio_chi_0.1} through \ref{oscillation_14h_vio_chi_0.1} 
shows distance of finite and infinite haploid populations to limit $\bm{z^\ast}$ 
(limit with violation in crossover distribution $\bm{\chi}$) when $\bm{\epsilon} \;=\; 0.1$. 
The distance between finite population and limit $\bm{z}^\ast$ (limit with violation in $\bm{\chi}$ distribution) 
decreases as finite population size increases, 
and finite population shows behavior similar to infinite population behavior as finite population size grows. 
Average distance data for haploid population in case of violation in $\bm{\chi}$ distribution 
with $\bm{\epsilon} \;=\; 0.1$ for different finite population size $N$ are tabulated in table \ref{distanceChiHapEps0.1}.

\begin{table}[ht]
\caption{\textbf{Distance measured for violation in $\bm{\chi}$ with $\bm{\epsilon} \;=\; 0.1$  for haploids:} $\ell$ is genome length, 
average distance between finite and infinite population is tabulated in the last three columns, and last row is expected single step distance.}
\centering
\begin{tabularx}{0.75\textwidth}{ c *{3}{X}}
\toprule
$\ell$ & $N = 4096$ & $N = 40960$ & $N = 81920$  \\
\midrule
8 & 0.0163	& 0.0061 	& 0.0051 \\
10 & 0.0157	&  0.0051	& 0.0037 \\	
12 & 0.0157	&  0.0051	& 0.0037 \\	
14 & 0.0156	&  0.0049	& 0.0035 \\
\midrule
$1/\sqrt{N}$ & 0.0156 & 0.0049 & 0.0035 \\
\bottomrule
\end{tabularx}
\label{distanceChiHapEps0.1}
\end{table} 

From , average distance calculated for finite population size $4096$ is $0.0158$, 
for size $40960$ is $0.0053$ and for size $81920$ is $0.0038$. The results from Table \ref{distanceChiHapEps0.1} show average distance 
between finite population and limit $\bm{z^\ast}$ approaches the expected single step distance 
between finite and infinite population. The distance decreased as $1/\sqrt{N}$. 
Also, the distance decreased as genome length $\ell$ increased for all sizes of finite haploid populations 
with $\bm{\epsilon} \;=\; 0.1$.
 

% \subsubsection{$\bm{\epsilon}: 0.5$}

% l = 8
\mbox{}\\[-0.75in]
\begin{figure}[!b]
\begin{center}
\subfloat{
\resizebox{8cm}{4.5cm}{\includegraphics{figures/eps/vio/chi/b8/e0.5/n00004096_fin_hap.eps}}} \hspace{-3em}%
\subfloat{
\resizebox{8cm}{4.5cm}{\includegraphics{figures/eps/vio/chi/b8/e0.5/n00004096_fin_hap_wovio.eps}}}\vspace{-1em} \hspace{-3em}%
\end{center}
\begin{center}
\subfloat{
\resizebox{8cm}{4.5cm}{\includegraphics{figures/eps/vio/chi/b8/e0.5/n00040960_fin_hap.eps}}} \hspace{-3em}%
\subfloat{
\resizebox{8cm}{4.5cm}{\includegraphics{figures/eps/vio/chi/b8/e0.5/n00040960_fin_hap_wovio.eps}}}\vspace{-1em} \hspace{-3em}%
\end{center}

\begin{center}
\subfloat{
\resizebox{8cm}{4.5cm}{\includegraphics{figures/eps/vio/chi/b8/e0.5/n00081920_fin_hap.eps}}} \hspace{-3em}%
\subfloat{
\resizebox{8cm}{4.5cm}{\includegraphics{figures/eps/vio/chi/b8/e0.5/n00081920_fin_hap_wovio.eps}}}\vspace{-1em} \hspace{-3em}%
\end{center}

\begin{center}
\subfloat{
\resizebox{8cm}{4.5cm}{\includegraphics{figures/eps/vio/chi/b8/e0.5/inf_hap.eps}}}\hspace{-3em}%
\subfloat{
\resizebox{8cm}{4.5cm}{\includegraphics{figures/eps/vio/chi/b8/e0.5/inf_hap_wovio.eps}}}\vspace{-0.5em} \hspace{-3em}%


\caption[\textbf{Infinite and finite haploid population behavior for $\bm{\chi}$ violation, $\ell = 8$ and $\bm{\epsilon} = 0.5$}]
{\textbf{Infinite and finite haploid population behavior for $\bm{\chi}$ violation, $\ell = 8$ and $\bm{\epsilon} = 0.5$:} 
  In left column, $d'$ is distance of finite or infinite population to limit $\bm{z}^\ast$ for $g$ generations. 
  In right column, $d$ is distance of finite or infinite population to limits $\bm{p}^\ast$ and $\bm{q}^\ast$.}
\label{oscillation_8h_vio_chi_0.5}
\end{center}
\end{figure}


% l = 10

\begin{figure}[h]
\begin{center}
\subfloat{
\resizebox{8cm}{4.5cm}{\includegraphics{figures/eps/vio/chi/b10/e0.5/n00004096_fin_hap.eps}}} \hspace{-3em}%
\subfloat{
\resizebox{8cm}{4.5cm}{\includegraphics{figures/eps/vio/chi/b10/e0.5/n00004096_fin_hap_wovio.eps}}}\vspace{-1em} \hspace{-3em}%
\end{center}
\begin{center}
\subfloat{
\resizebox{8cm}{4.5cm}{\includegraphics{figures/eps/vio/chi/b10/e0.5/n00040960_fin_hap.eps}}} \hspace{-3em}%
\subfloat{
\resizebox{8cm}{4.5cm}{\includegraphics{figures/eps/vio/chi/b10/e0.5/n00040960_fin_hap_wovio.eps}}}\vspace{-1em} \hspace{-3em}%
\end{center}

\begin{center}
\subfloat{
\resizebox{8cm}{4.5cm}{\includegraphics{figures/eps/vio/chi/b10/e0.5/n00081920_fin_hap.eps}}} \hspace{-3em}%
\subfloat{
\resizebox{8cm}{4.5cm}{\includegraphics{figures/eps/vio/chi/b10/e0.5/n00081920_fin_hap_wovio.eps}}}\vspace{-1em} \hspace{-3em}%
\end{center}

\begin{center}
\subfloat{
\resizebox{8cm}{4.5cm}{\includegraphics{figures/eps/vio/chi/b10/e0.5/inf_hap.eps}}}\hspace{-3em}%
\subfloat{
\resizebox{8cm}{4.5cm}{\includegraphics{figures/eps/vio/chi/b10/e0.5/inf_hap_wovio.eps}}}\vspace{-0.5em} \hspace{-3em}%

\caption[\textbf{Infinite and finite haploid population behavior for $\bm{\chi}$ violation, genome length $\ell = 10$ and $\bm{\epsilon} = 0.5$}]{\textbf{Infinite and finite haploid population behavior for $\bm{\chi}$ violation, genome length $\ell = 10$ and $\bm{\epsilon} = 0.5$:} 
  In left column, $d'$ is distance of finite or infinite population to limit $\bm{z}^\ast$ for $g$ generations. In right column, $d$ is distance of finite or infinite population to limits $\bm{p}^\ast$ and $\bm{q}^\ast$.}
\label{oscillation_10h_vio_chi_0.5}
\end{center}
\end{figure}

% l = 12
\begin{figure}[h]
\begin{center}
\subfloat{
\resizebox{8cm}{4.5cm}{\includegraphics{figures/eps/vio/chi/b12/e0.5/n00004096_fin_hap.eps}}} \hspace{-3em}%
\subfloat{
\resizebox{8cm}{4.5cm}{\includegraphics{figures/eps/vio/chi/b12/e0.5/n00004096_fin_hap_wovio.eps}}}\vspace{-1em} \hspace{-3em}%
\end{center}
\begin{center}
\subfloat{
\resizebox{8cm}{4.5cm}{\includegraphics{figures/eps/vio/chi/b12/e0.5/n00040960_fin_hap.eps}}} \hspace{-3em}%
\subfloat{
\resizebox{8cm}{4.5cm}{\includegraphics{figures/eps/vio/chi/b12/e0.5/n00040960_fin_hap_wovio.eps}}}\vspace{-1em} \hspace{-3em}%
\end{center}

\begin{center}
\subfloat{
\resizebox{8cm}{4.5cm}{\includegraphics{figures/eps/vio/chi/b12/e0.5/n00081920_fin_hap.eps}}} \hspace{-3em}%
\subfloat{
\resizebox{8cm}{4.5cm}{\includegraphics{figures/eps/vio/chi/b12/e0.5/n00081920_fin_hap_wovio.eps}}}\vspace{-1em} \hspace{-3em}%
\end{center}

\begin{center}
\subfloat{
\resizebox{8cm}{4.5cm}{\includegraphics{figures/eps/vio/chi/b12/e0.5/inf_hap.eps}}}\hspace{-3em}%
\subfloat{
\resizebox{8cm}{4.5cm}{\includegraphics{figures/eps/vio/chi/b12/e0.5/inf_hap_wovio.eps}}}\vspace{-0.5em} \hspace{-3em}%


\caption[\textbf{Infinite and finite haploid population behavior for $\bm{\chi}$ violation, genome length $\ell = 12$ and $\bm{\epsilon} = 0.5$}]{\textbf{Infinite and finite haploid population behavior for $\bm{\chi}$ violation, genome length $\ell = 12$ and $\bm{\epsilon} = 0.5$:} 
  In left column, $d'$ is distance of finite or infinite population to limit $\bm{z}^\ast$ for $g$ generations. In right column, $d$ is distance of finite or infinite population to limits $\bm{p}^\ast$ and $\bm{q}^\ast$.}
\label{oscillation_12h_vio_chi_0.5}
\end{center}
\end{figure}

% l = 14

\begin{figure}[h]
\begin{center}
\subfloat{
\resizebox{8cm}{4.5cm}{\includegraphics{figures/eps/vio/chi/b14/e0.5/n00004096_fin_hap.eps}}} \hspace{-3em}%
\subfloat{
\resizebox{8cm}{4.5cm}{\includegraphics{figures/eps/vio/chi/b14/e0.5/n00004096_fin_hap_wovio.eps}}}\vspace{-1em} \hspace{-3em}%
\end{center}
\begin{center}
\subfloat{
\resizebox{8cm}{4.5cm}{\includegraphics{figures/eps/vio/chi/b14/e0.5/n00040960_fin_hap.eps}}} \hspace{-3em}%
\subfloat{
\resizebox{8cm}{4.5cm}{\includegraphics{figures/eps/vio/chi/b14/e0.5/n00040960_fin_hap_wovio.eps}}}\vspace{-1em} \hspace{-3em}%
\end{center}

\begin{center}
\subfloat{
\resizebox{8cm}{4.5cm}{\includegraphics{figures/eps/vio/chi/b14/e0.5/n00081920_fin_hap.eps}}} \hspace{-3em}%
\subfloat{
\resizebox{8cm}{4.5cm}{\includegraphics{figures/eps/vio/chi/b14/e0.5/n00081920_fin_hap_wovio.eps}}}\vspace{-1em} \hspace{-3em}%
\end{center}

\begin{center}
\subfloat{
\resizebox{8cm}{4.5cm}{\includegraphics{figures/eps/vio/chi/b14/e0.5/inf_hap.eps}}}\hspace{-3em}%
\subfloat{
\resizebox{8cm}{4.5cm}{\includegraphics{figures/eps/vio/chi/b14/e0.5/inf_hap_wovio.eps}}}\vspace{-0.5em} \hspace{-3em}%

\caption[\textbf{Infinite and finite haploid population behavior for $\bm{\chi}$ violation, genome length $\ell = 14$ and $\bm{\epsilon} = 0.5$}]{\textbf{Infinite and finite haploid population behavior for $\bm{\chi}$ violation, genome length $\ell = 14$ and $\bm{\epsilon} = 0.5$:} 
  In left column, $d'$ is distance of finite or infinite population to limit $\bm{z}^\ast$ for $g$ generations. In right column, $d$ is distance of finite or infinite population to limits $\bm{p}^\ast$ and $\bm{q}^\ast$.}
\label{oscillation_14h_vio_chi_0.5}
\end{center}
\end{figure}

\clearpage

The right column in figures \ref{oscillation_8h_vio_chi_0.5} through \ref{oscillation_14h_vio_chi_0.5} 
shows distance of finite and infinite haploid populations to non-violation limits $\bm{p^\ast}$ and $\bm{q^\ast}$ with $\bm{\epsilon} \;=\; 0.5$. 
The graphs indicate oscillating behavior. 
Unlike mutation with violation $\bm{\epsilon} \;=\; 0.5$, oscillation is observed for longer length of generations. 
Finite populations still show some though not very clear oscillations, and then show randomness in behavior as generation progresses. 
Infinite population also oscillates but the oscillation dies out quickly. Randomness in finite population behavior increases 
more than for smaller values of $\bm{\epsilon}$, especially as $\ell$ increases.

The left column of figures \ref{oscillation_8h_vio_chi_0.5} through \ref{oscillation_14h_vio_chi_0.5} 
shows distance of finite and infinite haploid populations to limit $\bm{z^\ast}$ 
(limit with violation in crossover distribution $\bm{\chi}$) when $\bm{\epsilon} \;=\; 0.5$. 
The distance decreases as population size increases, 
and finite population shows behavior similar to infinite population behavior as finite population size grows. 
Average distance data for haploid population in case of violation in $\bm{\chi}$ distribution 
with $\bm{\epsilon} \;=\; 0.5$ for different finite population size $N$ are tabulated in table \ref{distanceChiHapEps0.5}.

\begin{table}[h]
\caption[\textbf{Distance measured for violation in $\bm{\chi}$ with $\bm{\epsilon} \;=\; 0.5$  for haploids}]{\textbf{Distance measured for violation in $\bm{\chi}$ with $\bm{\epsilon} \;=\; 0.5$  for haploids:} $\ell$ is genome length, 
average distance between finite and infinite population is tabulated in the last three columns, and last row is expected single step distance.}
\centering
\begin{tabularx}{0.75\textwidth}{ c *{3}{X}}
\toprule
$\ell$ & $N = 4096$ & $N = 40960$ & $N = 81920$  \\
\midrule
8 & 0.0156	&  0.0051	& 0.0036 \\
10 & 0.0155	&  0.0049	& 0.0035 \\
12 & 0.0157	&  0.0050	& 0.0035 \\
14 & 0.0156	&  0.0049	& 0.0035 \\      
\midrule
$1/\sqrt{N}$ & 0.0156 & 0.0049 & 0.0035 \\
\bottomrule
\end{tabularx}
\label{distanceChiHapEps0.5}
\end{table} 

Table \ref{distanceChiHapEps0.5} shows that the average distance 
between finite and infinite populations approaches the expected single step distance $1/\sqrt{N}$. 









 
\subsection{Diploid Population}
% \subsubsection{$\bm{\epsilon}: 0.01$}
% l = 8
\begin{figure}[h]
\begin{center}
\subfloat{
\resizebox{8cm}{5cm}{\includegraphics{figures/eps/vio/chi/b8/e0.01/n00004096_fin_dip.eps}}}\hspace{-3em}%
\subfloat{
\resizebox{8cm}{5cm}{\includegraphics{figures/eps/vio/chi/b8/e0.01/n00004096_fin_dip_wovio.eps}}}\vspace{-1em}  \hspace{-3em}%
\end{center}
\begin{center}
\subfloat{
\resizebox{8cm}{5cm}{\includegraphics{figures/eps/vio/chi/b8/e0.01/n00040960_fin_dip.eps}}}\hspace{-3em}%
\subfloat{
\resizebox{8cm}{5cm}{\includegraphics{figures/eps/vio/chi/b8/e0.01/n00040960_fin_dip_wovio.eps}}}\vspace{-1em}  \hspace{-3em}%
\end{center}


\begin{center}
\subfloat{
\resizebox{8cm}{5cm}{\includegraphics{figures/eps/vio/chi/b8/e0.01/n00081920_fin_dip.eps}}}\hspace{-3em}%
\subfloat{
\resizebox{8cm}{5cm}{\includegraphics{figures/eps/vio/chi/b8/e0.01/n00081920_fin_dip_wovio.eps}}}\vspace{-1em}  \hspace{-3em}%
\end{center}

\begin{center}
\subfloat{
\resizebox{8cm}{5cm}{\includegraphics{figures/eps/vio/chi/b8/e0.01/inf_dip.eps}}}\hspace{-3em}%
\subfloat{
\resizebox{8cm}{5cm}{\includegraphics{figures/eps/vio/chi/b8/e0.01/inf_dip_wovio.eps}}}\vspace{-0.5em}  \hspace{-3em}%


\caption[\textbf{Infinite and finite diploid population oscillation behavior in case of violation in $\bm{\chi}$ for genome length $\ell = 8$ and $\bm{\epsilon} = 0.01$}]{\textbf{Infinite and finite diploid population oscillation behavior in case of violation in $\bm{\chi}$ for genome length $\ell = 8$ and $\bm{\epsilon} = 0.01$:} 
  In left column, $d'$ is distance of finite population of size $n$ or infinite population to limit $\bm{z}^\ast$ for $g$ generations. In right column, $d$ is distance of finite population or infinite population to limits $\bm{p}^\ast$ and $\bm{q}^\ast$ without violation.}
\label{oscillation_8d_vio_chi_0.01}
\end{center}
\end{figure}

% l = 10

\begin{figure}[h]
\begin{center}
\subfloat{
\resizebox{8cm}{5cm}{\includegraphics{figures/eps/vio/chi/b10/e0.01/n00004096_fin_dip.eps}}}\hspace{-3em}%
\subfloat{
\resizebox{8cm}{5cm}{\includegraphics{figures/eps/vio/chi/b10/e0.01/n00004096_fin_dip_wovio.eps}}}\vspace{-1em}  \hspace{-3em}%
\end{center}
\begin{center}
\subfloat{
\resizebox{8cm}{5cm}{\includegraphics{figures/eps/vio/chi/b10/e0.01/n00040960_fin_dip.eps}}}\hspace{-3em}%
\subfloat{
\resizebox{8cm}{5cm}{\includegraphics{figures/eps/vio/chi/b10/e0.01/n00040960_fin_dip_wovio.eps}}}\vspace{-1em}  \hspace{-3em}%
\end{center}


\begin{center}
\subfloat{
\resizebox{8cm}{5cm}{\includegraphics{figures/eps/vio/chi/b10/e0.01/n00081920_fin_dip.eps}}}\hspace{-3em}%
\subfloat{
\resizebox{8cm}{5cm}{\includegraphics{figures/eps/vio/chi/b10/e0.01/n00081920_fin_dip_wovio.eps}}}\vspace{-1em}  \hspace{-3em}%
\end{center}

\begin{center}
\subfloat{
\resizebox{8cm}{5cm}{\includegraphics{figures/eps/vio/chi/b10/e0.01/inf_dip.eps}}}\hspace{-3em}%
\subfloat{
\resizebox{8cm}{5cm}{\includegraphics{figures/eps/vio/chi/b10/e0.01/inf_dip_wovio.eps}}}\vspace{-0.5em}  \hspace{-3em}%


\caption[\textbf{Infinite and finite diploid population oscillation behavior in case of violation in $\bm{\chi}$ for genome length $\ell = 10$ and $\bm{\epsilon} = 0.01$}]{\textbf{Infinite and finite diploid population oscillation behavior in case of violation in $\bm{\chi}$ for genome length $\ell = 10$ and $\bm{\epsilon} = 0.01$:} 
  In left column, $d'$ is distance of finite population of size $n$ or infinite population to limit $\bm{z}^\ast$ for $g$ generations. In right column, $d$ is distance of finite population or infinite population to limits $\bm{p}^\ast$ and $\bm{q}^\ast$ without violation.}
\label{oscillation_10d_vio_chi_0.01}
\end{center}
\end{figure}

% l = 12

\begin{figure}[h]
\begin{center}
\subfloat{
\resizebox{8cm}{5cm}{\includegraphics{figures/eps/vio/chi/b12/e0.01/n00004096_fin_dip.eps}}}\hspace{-3em}%
\subfloat{
\resizebox{8cm}{5cm}{\includegraphics{figures/eps/vio/chi/b12/e0.01/n00004096_fin_dip_wovio.eps}}}\vspace{-1em}  \hspace{-3em}%
\end{center}
\begin{center}
\subfloat{
\resizebox{8cm}{5cm}{\includegraphics{figures/eps/vio/chi/b12/e0.01/n00040960_fin_dip.eps}}}\hspace{-3em}%
\subfloat{
\resizebox{8cm}{5cm}{\includegraphics{figures/eps/vio/chi/b12/e0.01/n00040960_fin_dip_wovio.eps}}}\vspace{-1em}  \hspace{-3em}%
\end{center}


\begin{center}
\subfloat{
\resizebox{8cm}{5cm}{\includegraphics{figures/eps/vio/chi/b12/e0.01/n00081920_fin_dip.eps}}}\hspace{-3em}%
\subfloat{
\resizebox{8cm}{5cm}{\includegraphics{figures/eps/vio/chi/b12/e0.01/n00081920_fin_dip_wovio.eps}}}\vspace{-1em}  \hspace{-3em}%
\end{center}

\begin{center}
\subfloat{
\resizebox{8cm}{5cm}{\includegraphics{figures/eps/vio/chi/b12/e0.01/inf_dip.eps}}}\hspace{-3em}%
\subfloat{
\resizebox{8cm}{5cm}{\includegraphics{figures/eps/vio/chi/b12/e0.01/inf_dip_wovio.eps}}}\vspace{-0.5em}  \hspace{-3em}%


\caption[\textbf{Infinite and finite diploid population oscillation behavior in case of violation in $\bm{\chi}$ for genome length $\ell = 12$ and $\bm{\epsilon} = 0.01$}]{\textbf{Infinite and finite diploid population oscillation behavior in case of violation in $\bm{\chi}$ for genome length $\ell = 12$ and $\bm{\epsilon} = 0.01$:} 
  In left column, $d'$ is distance of finite population of size $n$ or infinite population to limit $\bm{z}^\ast$ for $g$ generations. In right column, $d$ is distance of finite population or infinite population to limits $\bm{p}^\ast$ and $\bm{q}^\ast$ without violation.}
\label{oscillation_12d_vio_chi_0.01}
\end{center}
\end{figure}

% l = 14

\begin{figure}[h]
\begin{center}
\subfloat{
\resizebox{8cm}{5cm}{\includegraphics{figures/eps/vio/chi/b14/e0.01/n00004096_fin_dip.eps}}}\hspace{-3em}%
\subfloat{
\resizebox{8cm}{5cm}{\includegraphics{figures/eps/vio/chi/b14/e0.01/n00004096_fin_dip_wovio.eps}}}\vspace{-1em}  \hspace{-3em}%
\end{center}
\begin{center}
\subfloat{
\resizebox{8cm}{5cm}{\includegraphics{figures/eps/vio/chi/b14/e0.01/n00040960_fin_dip.eps}}}\hspace{-3em}%
\subfloat{
\resizebox{8cm}{5cm}{\includegraphics{figures/eps/vio/chi/b14/e0.01/n00040960_fin_dip_wovio.eps}}}\vspace{-1em}  \hspace{-3em}%
\end{center}


\begin{center}
\subfloat{
\resizebox{8cm}{5cm}{\includegraphics{figures/eps/vio/chi/b14/e0.01/n00081920_fin_dip.eps}}}\hspace{-3em}%
\subfloat{
\resizebox{8cm}{5cm}{\includegraphics{figures/eps/vio/chi/b14/e0.01/n00081920_fin_dip_wovio.eps}}}\vspace{-1em}  \hspace{-3em}%
\end{center}

\begin{center}
\subfloat{
\resizebox{8cm}{5cm}{\includegraphics{figures/eps/vio/chi/b14/e0.01/inf_dip.eps}}}\hspace{-3em}%
\subfloat{
\resizebox{8cm}{5cm}{\includegraphics{figures/eps/vio/chi/b14/e0.01/inf_dip_wovio.eps}}}\vspace{-0.5em}  \hspace{-3em}%


\caption[\textbf{Infinite and finite diploid population oscillation behavior in case of violation in $\bm{\chi}$ for genome length $\ell = 14$ and $\bm{\epsilon} = 0.01$}]{\textbf{Infinite and finite diploid population oscillation behavior in case of violation in $\bm{\chi}$ for genome length $\ell = 14$ and $\bm{\epsilon} = 0.01$:} 
  In left column, $d'$ is distance of finite population of size $n$ or infinite population to limit $\bm{z}^\ast$ for $g$ generations. In right column, $d$ is distance of finite population or infinite population to limits $\bm{p}^\ast$ and $\bm{q}^\ast$ without violation.}
\label{oscillation_14d_vio_chi_0.01}
\end{center}
\end{figure}

\clearpage

The right column in figures \ref{oscillation_8d_vio_chi_0.01} through \ref{oscillation_14d_vio_chi_0.01} 
shows distance of finite and infinite diploid populations to non-violation limits $\bm{p^\ast}$ and $\bm{q^\ast}$ with $\bm{\epsilon} \;=\; 0.01$. 
Those graphs indicate oscillating behavior of diploid population given violation. 
Both finite and infinite populations oscillate given violation. Like in haploid case, oscillations are sharper. Since value of $\bm{\epsilon}$ 
is small, damping of ripples is slow. A new mask introduced in crossover distribution with $\bm{\epsilon} \;=\; 0.01$ has small 
probability of being used during crossover. Infinite population oscillation does not die out in 50 generations. As value of $\ell$ 
increases, random drifting and wiggling of finite population occurs for small population size, and oscillation gets better with increase in population size. 
These are noticed more clearly in figures \ref{oscillation_12d_vio_chi_0.01} and \ref{oscillation_14d_vio_chi_0.01}.

The left column of figures \ref{oscillation_8d_vio_chi_0.01} through \ref{oscillation_14d_vio_chi_0.01} 
shows distance of finite and infinite diploid populations to limit $\bm{z^\ast}$ 
(limit with violation in crossover distribution $\bm{\chi}$) when $\bm{\epsilon} \;=\; 0.01$. 
The distance between finite population and limit $\bm{z}^\ast$ (limit with violation in $\bm{\chi}$ distribution) 
decreases as finite population size increases. 
Average distance data for diploid population in case of violation in $\bm{\chi}$ distribution 
with $\bm{\epsilon} \;=\; 0.01$ for different finite population size $N$ are tabulated in table \ref{distanceChiDipEps0.01}.


\begin{table}[ht]
\caption[\textbf{Distance measured for violation in $\bm{\chi}$ with $\bm{\epsilon} \;=\; 0.01$ diploids}]{\textbf{Distance measured for violation in $\bm{\chi}$ with $\bm{\epsilon} \;=\; 0.01$ diploids:} $\ell$ is genome length, 
average distance between finite and infinite population is tabulated in the last three columns, and last row is expected single step distance.}
\centering
\begin{tabularx}{0.75\textwidth}{ c *{3}{X}}
\toprule
$\ell$ & $N = 4096$ & $N = 40960$ & $N = 81920$  \\
\midrule
8 & 0.0156	&  0.0051	& 0.0036 \\
10 & 0.0156	&  0.0049	& 0.0035 \\
12 & 0.0156	&  0.0049	& 0.0035 \\
14 & 0.0156	&  0.0049	& 0.0035 \\
\midrule
$1/\sqrt{N}$ & 0.0156 & 0.0049 & 0.0035 \\
\bottomrule
\end{tabularx}
\label{distanceChiDipEps0.01}
\end{table} 

The results from Table \ref{distanceChiDipEps0.01} show average distance 
between finite population and limit $\bm{z^\ast}$ approaches the expected single step distance 
between finite and infinite population. The distance decreased as $1/\sqrt{N}$. 
Also, the distance is smaller in diploid populations than in haploid populations with $\bm{\epsilon} \;=\; 0.01$.



% \subsubsection{$\bm{\epsilon}: 0.1$}
% l = 8
\begin{figure}[H]
\begin{center}
\subfloat{
\resizebox{8cm}{5cm}{\includegraphics{figures/eps/vio/chi/b8/e0.1/n00004096_fin_dip.eps}}}\hspace{-3em}%
\subfloat{
\resizebox{8cm}{5cm}{\includegraphics{figures/eps/vio/chi/b8/e0.1/n00004096_fin_dip_wovio.eps}}}\vspace{-1em}  \hspace{-3em}%
\end{center}
\begin{center}
\subfloat{
\resizebox{8cm}{5cm}{\includegraphics{figures/eps/vio/chi/b8/e0.1/n00040960_fin_dip.eps}}}\hspace{-3em}%
\subfloat{
\resizebox{8cm}{5cm}{\includegraphics{figures/eps/vio/chi/b8/e0.1/n00040960_fin_dip_wovio.eps}}}\vspace{-1em}  \hspace{-3em}%
\end{center}


\begin{center}
\subfloat{
\resizebox{8cm}{5cm}{\includegraphics{figures/eps/vio/chi/b8/e0.1/n00081920_fin_dip.eps}}}\hspace{-3em}%
\subfloat{
\resizebox{8cm}{5cm}{\includegraphics{figures/eps/vio/chi/b8/e0.1/n00081920_fin_dip_wovio.eps}}}\vspace{-1em}  \hspace{-3em}%
\end{center}

\begin{center}
\subfloat{
\resizebox{8cm}{5cm}{\includegraphics{figures/eps/vio/chi/b8/e0.1/inf_dip.eps}}}\hspace{-3em}%
\subfloat{
\resizebox{8cm}{5cm}{\includegraphics{figures/eps/vio/chi/b8/e0.1/inf_dip_wovio.eps}}}\vspace{-0.5em}  \hspace{-3em}%


\caption{\textbf{Infinite and finite diploid population oscillation behavior in case of violation in $\bm{\chi}$ for genome length $\ell = 8$ and $\bm{\epsilon} = 0.1$:} 
  In left column, $d'$ is distance of finite population of size $n$ or infinite population to limit $\bm{z}^\ast$ for $g$ generations. In right column, $d$ is distance of finite population of size $N$ or infinite population to limits without violation.}
\label{oscillation_8d_vio_chi_0.1}
\end{center}
\end{figure}

% l = 10

\begin{figure}[H]
\begin{center}
\subfloat{
\resizebox{8cm}{5cm}{\includegraphics{figures/eps/vio/chi/b10/e0.1/n00004096_fin_dip.eps}}}\hspace{-3em}%
\subfloat{
\resizebox{8cm}{5cm}{\includegraphics{figures/eps/vio/chi/b10/e0.1/n00004096_fin_dip_wovio.eps}}}\vspace{-1em}  \hspace{-3em}%
\end{center}
\begin{center}
\subfloat{
\resizebox{8cm}{5cm}{\includegraphics{figures/eps/vio/chi/b10/e0.1/n00040960_fin_dip.eps}}}\hspace{-3em}%
\subfloat{
\resizebox{8cm}{5cm}{\includegraphics{figures/eps/vio/chi/b10/e0.1/n00040960_fin_dip_wovio.eps}}}\vspace{-1em}  \hspace{-3em}%
\end{center}


\begin{center}
\subfloat{
\resizebox{8cm}{5cm}{\includegraphics{figures/eps/vio/chi/b10/e0.1/n00081920_fin_dip.eps}}}\hspace{-3em}%
\subfloat{
\resizebox{8cm}{5cm}{\includegraphics{figures/eps/vio/chi/b10/e0.1/n00081920_fin_dip_wovio.eps}}}\vspace{-1em}  \hspace{-3em}%
\end{center}

\begin{center}
\subfloat{
\resizebox{8cm}{5cm}{\includegraphics{figures/eps/vio/chi/b10/e0.1/inf_dip.eps}}}\hspace{-3em}%
\subfloat{
\resizebox{8cm}{5cm}{\includegraphics{figures/eps/vio/chi/b10/e0.1/inf_dip_wovio.eps}}}\vspace{-0.5em}  \hspace{-3em}%


\caption{\textbf{Infinite and finite diploid population oscillation behavior in case of violation in $\bm{\chi}$ for genome length $\ell = 10$ and $\bm{\epsilon} = 0.1$:} 
  In left column, $d'$ is distance of finite population of size $n$ or infinite population to limit $\bm{z}^\ast$ for $g$ generations. In right column, $d$ is distance of finite population of size $N$ or infinite population to limits without violation.}
\label{oscillation_10d_vio_chi_0.1}
\end{center}
\end{figure}

% l = 12

\begin{figure}[H]
\begin{center}
\subfloat{
\resizebox{8cm}{5cm}{\includegraphics{figures/eps/vio/chi/b12/e0.1/n00004096_fin_dip.eps}}}\hspace{-3em}%
\subfloat{
\resizebox{8cm}{5cm}{\includegraphics{figures/eps/vio/chi/b12/e0.1/n00004096_fin_dip_wovio.eps}}}\vspace{-1em}  \hspace{-3em}%
\end{center}
\begin{center}
\subfloat{
\resizebox{8cm}{5cm}{\includegraphics{figures/eps/vio/chi/b12/e0.1/n00040960_fin_dip.eps}}}\hspace{-3em}%
\subfloat{
\resizebox{8cm}{5cm}{\includegraphics{figures/eps/vio/chi/b12/e0.1/n00040960_fin_dip_wovio.eps}}}\vspace{-1em}  \hspace{-3em}%
\end{center}


\begin{center}
\subfloat{
\resizebox{8cm}{5cm}{\includegraphics{figures/eps/vio/chi/b12/e0.1/n00081920_fin_dip.eps}}}\hspace{-3em}%
\subfloat{
\resizebox{8cm}{5cm}{\includegraphics{figures/eps/vio/chi/b12/e0.1/n00081920_fin_dip_wovio.eps}}}\vspace{-1em}  \hspace{-3em}%
\end{center}

\begin{center}
\subfloat{
\resizebox{8cm}{5cm}{\includegraphics{figures/eps/vio/chi/b12/e0.1/inf_dip.eps}}}\hspace{-3em}%
\subfloat{
\resizebox{8cm}{5cm}{\includegraphics{figures/eps/vio/chi/b12/e0.1/inf_dip_wovio.eps}}}\vspace{-0.5em}  \hspace{-3em}%


\caption{\textbf{Infinite and finite diploid population oscillation behavior in case of violation in $\bm{\chi}$ for genome length $\ell = 12$ and $\bm{\epsilon} = 0.1$:} 
  In left column, $d'$ is distance of finite population of size $n$ or infinite population to limit $\bm{z}^\ast$ for $g$ generations. In right column, $d$ is distance of finite population of size $N$ or infinite population to limits without violation.}
\label{oscillation_12d_vio_chi_0.1}
\end{center}
\end{figure}

% l = 14

\begin{figure}[H]
\begin{center}
\subfloat{
\resizebox{8cm}{5cm}{\includegraphics{figures/eps/vio/chi/b14/e0.1/n00004096_fin_dip.eps}}}\hspace{-3em}%
\subfloat{
\resizebox{8cm}{5cm}{\includegraphics{figures/eps/vio/chi/b14/e0.1/n00004096_fin_dip_wovio.eps}}}\vspace{-1em}  \hspace{-3em}%
\end{center}
\begin{center}
\subfloat{
\resizebox{8cm}{5cm}{\includegraphics{figures/eps/vio/chi/b14/e0.1/n00040960_fin_dip.eps}}}\hspace{-3em}%
\subfloat{
\resizebox{8cm}{5cm}{\includegraphics{figures/eps/vio/chi/b14/e0.1/n00040960_fin_dip_wovio.eps}}}\vspace{-1em}  \hspace{-3em}%
\end{center}


\begin{center}
\subfloat{
\resizebox{8cm}{5cm}{\includegraphics{figures/eps/vio/chi/b14/e0.1/n00081920_fin_dip.eps}}}\hspace{-3em}%
\subfloat{
\resizebox{8cm}{5cm}{\includegraphics{figures/eps/vio/chi/b14/e0.1/n00081920_fin_dip_wovio.eps}}}\vspace{-1em}  \hspace{-3em}%
\end{center}

\begin{center}
\subfloat{
\resizebox{8cm}{5cm}{\includegraphics{figures/eps/vio/chi/b14/e0.1/inf_dip.eps}}}\hspace{-3em}%
\subfloat{
\resizebox{8cm}{5cm}{\includegraphics{figures/eps/vio/chi/b14/e0.1/inf_dip_wovio.eps}}}\vspace{-0.5em}  \hspace{-3em}%


\caption{\textbf{Infinite and finite diploid population oscillation behavior in case of violation in $\bm{\chi}$ for genome length $\ell = 14$ and $\bm{\epsilon} = 0.1$:} 
  In left column, $d'$ is distance of finite population of size $n$ or infinite population to limit $\bm{z}^\ast$ for $g$ generations. In right column, $d$ is distance of finite population of size $N$ or infinite population to limits without violation.}
\label{oscillation_14d_vio_chi_0.1}
\end{center}
\end{figure}









\begin{table}[ht]
\caption{\textbf{Distance measured for violation in $\bm{\chi}$ with $\bm{\epsilon} \;=\; 0.1$ for diploids:} $\ell$ is genome length, 
and average distance between finite and 
infinite populations is tabulated in last three columns.}
\centering
\begin{tabularx}{0.75\textwidth}{ c *{3}{X}}
\toprule
$\ell$ & $N = 4096$ & $N = 40960$ & $N = 81920$  \\
\midrule
8 & 0.0156	&  0.0050	& 0.0035 \\
10 & 0.0156	&  0.0049	& 0.0035 \\
12 & 0.0156	&  0.0049	& 0.0035 \\
14 & 0.0156	&  0.0049	& 0.0035 \\
\bottomrule
\end{tabularx}
\label{distanceChiDipEps0.1}
\end{table} 



 

% \subsubsection{$\bm{\epsilon}: 0.5$}
% l = 8
\begin{figure}[h]
\begin{center}
\subfloat{
\resizebox{8cm}{5cm}{\includegraphics{figures/eps/vio/chi/b8/e0.5/n00004096_fin_dip.eps}}}\hspace{-3em}%
\subfloat{
\resizebox{8cm}{5cm}{\includegraphics{figures/eps/vio/chi/b8/e0.5/n00004096_fin_dip_wovio.eps}}}\vspace{-1em}  \hspace{-3em}%
\end{center}
\begin{center}
\subfloat{
\resizebox{8cm}{5cm}{\includegraphics{figures/eps/vio/chi/b8/e0.5/n00040960_fin_dip.eps}}}\hspace{-3em}%
\subfloat{
\resizebox{8cm}{5cm}{\includegraphics{figures/eps/vio/chi/b8/e0.5/n00040960_fin_dip_wovio.eps}}}\vspace{-1em}  \hspace{-3em}%
\end{center}


\begin{center}
\subfloat{
\resizebox{8cm}{5cm}{\includegraphics{figures/eps/vio/chi/b8/e0.5/n00081920_fin_dip.eps}}}\hspace{-3em}%
\subfloat{
\resizebox{8cm}{5cm}{\includegraphics{figures/eps/vio/chi/b8/e0.5/n00081920_fin_dip_wovio.eps}}}\vspace{-1em}  \hspace{-3em}%
\end{center}

\begin{center}
\subfloat{
\resizebox{8cm}{5cm}{\includegraphics{figures/eps/vio/chi/b8/e0.5/inf_dip.eps}}}\hspace{-3em}%
\subfloat{
\resizebox{8cm}{5cm}{\includegraphics{figures/eps/vio/chi/b8/e0.5/inf_dip_wovio.eps}}}\vspace{-0.5em}  \hspace{-3em}%


\caption{\textbf{Infinite and finite diploid population oscillation behavior in case of violation in $\bm{\chi}$ for genome length $\ell = 8$ and $\bm{\epsilon} = 0.5$:} 
  In left column, $d'$ is distance of finite population of size $n$ or infinite population to limit $\bm{z}^\ast$ for $g$ generations. In right column, $d$ is distance of finite population or infinite population to limits $\bm{p}^\ast$ and $\bm{q}^\ast$ without violation.}
\label{oscillation_8d_vio_chi_0.5}
\end{center}
\end{figure}

% l = 10

\begin{figure}[h]
\begin{center}
\subfloat{
\resizebox{8cm}{5cm}{\includegraphics{figures/eps/vio/chi/b10/e0.5/n00004096_fin_dip.eps}}}\hspace{-3em}%
\subfloat{
\resizebox{8cm}{5cm}{\includegraphics{figures/eps/vio/chi/b10/e0.5/n00004096_fin_dip_wovio.eps}}}\vspace{-1em}  \hspace{-3em}%
\end{center}
\begin{center}
\subfloat{
\resizebox{8cm}{5cm}{\includegraphics{figures/eps/vio/chi/b10/e0.5/n00040960_fin_dip.eps}}}\hspace{-3em}%
\subfloat{
\resizebox{8cm}{5cm}{\includegraphics{figures/eps/vio/chi/b10/e0.5/n00040960_fin_dip_wovio.eps}}}\vspace{-1em}  \hspace{-3em}%
\end{center}


\begin{center}
\subfloat{
\resizebox{8cm}{5cm}{\includegraphics{figures/eps/vio/chi/b10/e0.5/n00081920_fin_dip.eps}}}\hspace{-3em}%
\subfloat{
\resizebox{8cm}{5cm}{\includegraphics{figures/eps/vio/chi/b10/e0.5/n00081920_fin_dip_wovio.eps}}}\vspace{-1em}  \hspace{-3em}%
\end{center}

\begin{center}
\subfloat{
\resizebox{8cm}{5cm}{\includegraphics{figures/eps/vio/chi/b10/e0.5/inf_dip.eps}}}\hspace{-3em}%
\subfloat{
\resizebox{8cm}{5cm}{\includegraphics{figures/eps/vio/chi/b10/e0.5/inf_dip_wovio.eps}}}\vspace{-0.5em}  \hspace{-3em}%


\caption{\textbf{Infinite and finite diploid population oscillation behavior in case of violation in $\bm{\chi}$ for genome length $\ell = 10$ and $\bm{\epsilon} = 0.5$:} 
  In left column, $d'$ is distance of finite population of size $n$ or infinite population to limit $\bm{z}^\ast$ for $g$ generations. In right column, $d$ is distance of finite population or infinite population to limits $\bm{p}^\ast$ and $\bm{q}^\ast$ without violation.}
\label{oscillation_10d_vio_chi_0.5}
\end{center}
\end{figure}

% l = 12

\begin{figure}[h]
\begin{center}
\subfloat{
\resizebox{8cm}{5cm}{\includegraphics{figures/eps/vio/chi/b12/e0.5/n00004096_fin_dip.eps}}}\hspace{-3em}%
\subfloat{
\resizebox{8cm}{5cm}{\includegraphics{figures/eps/vio/chi/b12/e0.5/n00004096_fin_dip_wovio.eps}}}\vspace{-1em}  \hspace{-3em}%
\end{center}
\begin{center}
\subfloat{
\resizebox{8cm}{5cm}{\includegraphics{figures/eps/vio/chi/b12/e0.5/n00040960_fin_dip.eps}}}\hspace{-3em}%
\subfloat{
\resizebox{8cm}{5cm}{\includegraphics{figures/eps/vio/chi/b12/e0.5/n00040960_fin_dip_wovio.eps}}}\vspace{-1em}  \hspace{-3em}%
\end{center}


\begin{center}
\subfloat{
\resizebox{8cm}{5cm}{\includegraphics{figures/eps/vio/chi/b12/e0.5/n00081920_fin_dip.eps}}}\hspace{-3em}%
\subfloat{
\resizebox{8cm}{5cm}{\includegraphics{figures/eps/vio/chi/b12/e0.5/n00081920_fin_dip_wovio.eps}}}\vspace{-1em}  \hspace{-3em}%
\end{center}

\begin{center}
\subfloat{
\resizebox{8cm}{5cm}{\includegraphics{figures/eps/vio/chi/b12/e0.5/inf_dip.eps}}}\hspace{-3em}%
\subfloat{
\resizebox{8cm}{5cm}{\includegraphics{figures/eps/vio/chi/b12/e0.5/inf_dip_wovio.eps}}}\vspace{-0.5em}  \hspace{-3em}%


\caption{\textbf{Infinite and finite diploid population oscillation behavior in case of violation in $\bm{\chi}$ for genome length $\ell = 12$ and $\bm{\epsilon} = 0.5$:} 
  In left column, $d'$ is distance of finite population of size $n$ or infinite population to limit $\bm{z}^\ast$ for $g$ generations. In right column, $d$ is distance of finite population or infinite population to limits $\bm{p}^\ast$ and $\bm{q}^\ast$ without violation.}
\label{oscillation_12d_vio_chi_0.5}
\end{center}
\end{figure}

% l = 14

\begin{figure}[h]
\begin{center}
\subfloat{
\resizebox{8cm}{5cm}{\includegraphics{figures/eps/vio/chi/b14/e0.5/n00004096_fin_dip.eps}}}\hspace{-3em}%
\subfloat{
\resizebox{8cm}{5cm}{\includegraphics{figures/eps/vio/chi/b14/e0.5/n00004096_fin_dip_wovio.eps}}}\vspace{-1em}  \hspace{-3em}%
\end{center}
\begin{center}
\subfloat{
\resizebox{8cm}{5cm}{\includegraphics{figures/eps/vio/chi/b14/e0.5/n00040960_fin_dip.eps}}}\hspace{-3em}%
\subfloat{
\resizebox{8cm}{5cm}{\includegraphics{figures/eps/vio/chi/b14/e0.5/n00040960_fin_dip_wovio.eps}}}\vspace{-1em}  \hspace{-3em}%
\end{center}


\begin{center}
\subfloat{
\resizebox{8cm}{5cm}{\includegraphics{figures/eps/vio/chi/b14/e0.5/n00081920_fin_dip.eps}}}\hspace{-3em}%
\subfloat{
\resizebox{8cm}{5cm}{\includegraphics{figures/eps/vio/chi/b14/e0.5/n00081920_fin_dip_wovio.eps}}}\vspace{-1em}  \hspace{-3em}%
\end{center}

\begin{center}
\subfloat{
\resizebox{8cm}{5cm}{\includegraphics{figures/eps/vio/chi/b14/e0.5/inf_dip.eps}}}\hspace{-3em}%
\subfloat{
\resizebox{8cm}{5cm}{\includegraphics{figures/eps/vio/chi/b14/e0.5/inf_dip_wovio.eps}}}\vspace{-0.5em}  \hspace{-3em}%


\caption{\textbf{Infinite and finite diploid population oscillation behavior in case of violation in $\bm{\chi}$ for genome length $\ell = 14$ and $\bm{\epsilon} = 0.5$:} 
  In left column, $d'$ is distance of finite population of size $n$ or infinite population to limit $\bm{z}^\ast$ for $g$ generations. In right column, $d$ is distance of finite population or infinite population to limits $\bm{p}^\ast$ and $\bm{q}^\ast$ without violation.}
\label{oscillation_14d_vio_chi_0.5}
\end{center}
\end{figure}

\clearpage

The right column in figures \ref{oscillation_8d_vio_chi_0.5} through \ref{oscillation_14d_vio_chi_0.5} 
shows distance of finite and infinite diploid populations to non-violation limits $\bm{p^\ast}$ and $\bm{q^\ast}$ with $\bm{\epsilon} \;=\; 0.5$. 
Those graphs indicate oscillating behavior of diploid population given violation. 
Infinite populations oscillate for some generations, and then cease to oscillate given violation. 
Finite populations show some oscillations when $\ell \;=\; 8$ for higher population size as in figure \ref{oscillation_8d_vio_chi_0.5}, but 
for larger $\ell$, finite populations show only random wiggles. 

The left column of figures \ref{oscillation_8d_vio_chi_0.5} through \ref{oscillation_14d_vio_chi_0.5} 
shows distance of finite and infinite diploid populations to limit $\bm{z^\ast}$ 
(limit with violation in crossover distribution $\bm{\chi}$) when $\bm{\epsilon} \;=\; 0.5$. 
The distance between finite population and limit $\bm{z}^\ast$ (limit with violation in $\bm{\chi}$ distribution) 
decreases as finite population size increases.
The distance data for diploid population in case of violation in $\bm{\chi}$ distribution 
with $\bm{\epsilon} \;=\; 0.5$ for different finite population size $N$ are tabulated in table \ref{distanceChiDipEps0.5}.


\begin{table}[ht]
\caption{\textbf{Distance measured for violation in $\bm{\chi}$ with $\bm{\epsilon} \;=\; 0.5$ for diploids:} $\ell$ is genome length, 
average distance between finite and infinite population is tabulated in the last three columns, and last row is expected single step distance.}
\centering
\begin{tabularx}{0.75\textwidth}{ c *{3}{X}}
\toprule
$\ell$ & $N = 4096$ & $N = 40960$ & $N = 81920$  \\
\midrule
8 & 0.0156	&  0.0049	& 0.0035 \\	
10 & 0.0156	&  0.0049	& 0.0035 \\
12 & 0.0156	&  0.0049	& 0.0035 \\
14 & 0.0156	&  0.0049	& 0.0035 \\
\midrule
$1/\sqrt{N}$ & 0.0156 & 0.0049 & 0.0035 \\
\bottomrule
\end{tabularx}
\label{distanceChiDipEps0.5}
\end{table} 

From table \ref{distanceChiDipEps0.5}, average distance calculated for finite population size $4096$ is $0.0156$, 
for size $40960$ is $0.0049$ and for size $81920$ is $0.0035$. These results show average distance 
between finite population and limit $\bm{z^\ast}$ closely follows expected single step distance 
between finite and infinite population. The distance decreased as $1/\sqrt{N}$. 
Also, the distance is smaller in diploid populations than in haploid populations with $\bm{\epsilon} \;=\; 0.5$.
 


As value of $\ell$ increases, amplitude of oscillation decreases. 
Like in case of $\bm{\mu}$ violation, also as value of $\ell$ increases, random wiggling occurs for smaller population size, 
and larger population size is needed to observe good oscillations. Random wiggling of 
population is more evident in diploid case than in haploid case because length of 
population member string in diploid is $2\ell$. With increase in value of $\bm{\epsilon}$, 
oscillation in population diminishes, and oscillation completely ceases after certain threshold value for $\bm{\epsilon}$. 
Rate of dampening of oscillation in case of violation in $\bm{\chi}$ is slower than in case of violation in ${\bm{\mu}}$, 
and so we see some oscillation even for $\bm{\epsilon} = 0.5$ for haploid population in this case. 


\section{Summary}
In this chapter, we violated the condition \ref{OscCond} through violation in mutation and crossover distribution, and 
studied infinite and finite populations oscillation behavior with the violation through experiments. 
Infinite population ceases to oscillate when the condition \ref{OscCond} for convergence to 
periodic orbits is violated, but finite population continued to approximately oscillate for small values of $\bm{\epsilon}$. 
For smaller values of $\bm{\epsilon}$, finite population does not get aware of violation because the probability of using 
new masks created in the mutation distribution $\bm{\mu}$ and the crossover distribution $\bm{\mu}$ due to violation is very low, and 
finite population follows behavior of infinite population without violation in the condition for convergence to 
periodic orbits.





 
