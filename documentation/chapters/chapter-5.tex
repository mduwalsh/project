\chapter{Violation in Crossover Distribution} \label{ch:chiviolation}
The results from chapter \ref{ch:muviolation} show robustness of finite population oscillation 
demonstrating approximate oscillation can take place in finite populations 
when the mutation distribution $\bm{\mu}$ violates condition \ref{OscCond} .  
This chapter explores the robustness of finite population oscillation when condition \ref{OscCond} 
for the crossover distribution $\bm{\chi}$ is violated.
Violation of the condition, crossover-violation, as we call it, is expressed as:
\begin{equation}
\text{For all} \;g\;, \,g \neq 0, \quad \quad \quad
\label{chi-violation}
  1 \neq \sum \limits_{k \in \bar{g}\mathcal{R}} \bm{\chi}_{k+g} + \bm{\chi}_k  
\end{equation}
The question explored in this chapter is: Can finite populations exhibit approximate oscillation when there is violation in $\bm{\chi}$ 
and infinite population trajectories
have no periodic orbit?

Error $\bm{\epsilon}$ is introduced into the crossover distribution $\bm{\chi}$ so as to 
violate condition \ref{OscCond}; this guarantees that 
infinite population trajectories have no periodic orbit. Consequently, $\bm{p}^\ast \;=\; \bm{q}^\ast \;=\; \bm{z}^\ast$. 
Going forward, we use `limit $\bm{z}^\ast$' to denote evolutionary limit when crossover distribution 
$\bm{\chi}$ violates condition \ref{OscCond}, and 
`non-violation limits $\bm{p}^\ast$ and $\bm{q}^\ast$' to denote limits without violations (i.e., $\bm{\epsilon \,=\, 0}$).

\section{Violation}
The crossover distribution $\bm{\chi}$ was modified as 
\begin{equation*}
\bm{\chi}_i = (1-\bm{\epsilon}) \bm{\chi}_i \quad \quad
\text{so that } \quad \quad \quad \sum \bm{\chi}_i + \bm{\chi}_{i+g} = 1-\bm{\epsilon}  
\end{equation*}  
Then a single $j$ is chosen where $j \not\in \bar{g}\mathcal{R}$ and set $\bm{\chi}_j = \bm{\epsilon}$. 

Violation in crossover distribution $\bm{\chi}$ is different from violation in mutation distribution $\bm{\mu}$. 
The Markov chain formed by transition matrix $Q$ is regular under violation in $\bm{\mu}$ 
but that need not be the case under violation in  $\bm{\chi}$. 
The initial population is 
computed using the same procedure as described in section \ref{InitPopOsc}. To explore the effects of the degree  
of violation of condition \ref{OscCond} in $\bm{\chi}$, different values of $\bm{\epsilon} \in \{0.01, 0.1, 0.5\} $ are used in experiments. 
String length $\ell \;\in\; \{8, 10, 12, 14\}$ is considered for simulation.
The distances of both infinite and finite populations to limit $\bm{z}^\ast$ are plotted. 
The distances of both infinite and finite populations to 
non-violation limits $\bm{p}^\ast$ and $\bm{q}^\ast$ (i.e. $\bm{\epsilon} = 0$) are also plotted.

% \clearpage
% figures for chi violation
\subsection{Haploid Population $\mathtt{\sim}$ $\epsilon: 0.01$}

% l = 8

\begin{figure}[H]
\begin{center}
\subfloat{
\resizebox{8cm}{5cm}{\includegraphics{figures/eps/vio/chi/b8/e0.01/n00004096_fin_hap.eps}}} \hspace{-3em}%
\subfloat{
\resizebox{8cm}{5cm}{\includegraphics{figures/eps/vio/chi/b8/e0.01/n00004096_fin_hap_wovio.eps}}}\vspace{-1em} \hspace{-3em}%
\end{center}
\begin{center}
\subfloat{
\resizebox{8cm}{5cm}{\includegraphics{figures/eps/vio/chi/b8/e0.01/n00040960_fin_hap.eps}}} \hspace{-3em}%
\subfloat{
\resizebox{8cm}{5cm}{\includegraphics{figures/eps/vio/chi/b8/e0.01/n00040960_fin_hap_wovio.eps}}}\vspace{-1em} \hspace{-3em}%
\end{center}

\begin{center}
\subfloat{
\resizebox{8cm}{5cm}{\includegraphics{figures/eps/vio/chi/b8/e0.01/n00081920_fin_hap.eps}}} \hspace{-3em}%
\subfloat{
\resizebox{8cm}{5cm}{\includegraphics{figures/eps/vio/chi/b8/e0.01/n00081920_fin_hap_wovio.eps}}}\vspace{-1em} \hspace{-3em}%
\end{center}

\begin{center}
\subfloat{
\resizebox{8cm}{5cm}{\includegraphics{figures/eps/vio/chi/b8/e0.01/inf_hap.eps}}}\hspace{-3em}%
\subfloat{
\resizebox{8cm}{5cm}{\includegraphics{figures/eps/vio/chi/b8/e0.01/inf_hap_wovio.eps}}}\vspace{-0.5em} \hspace{-3em}%


\caption{\textbf{Infinite and finite haploid population oscillation behavior in case of violation in $\bm{\chi}$ for genome length $\ell = 8$ and $\bm{\epsilon} = 0.01$:} 
  In left column, $d'$ is distance of finite population of size $n$ or infinite population to limit $\bm{z}^\ast$ for $g$ generations. In right column, $d$ is distance of finite population of size $N$ or infinite population to limits without violation.}
\label{oscillation_8h_vio_chi_0.01}
\end{center}
\end{figure}


% l = 10

\begin{figure}[H]
\begin{center}
\subfloat{
\resizebox{8cm}{5cm}{\includegraphics{figures/eps/vio/chi/b10/e0.01/n00004096_fin_hap.eps}}} \hspace{-3em}%
\subfloat{
\resizebox{8cm}{5cm}{\includegraphics{figures/eps/vio/chi/b10/e0.01/n00004096_fin_hap_wovio.eps}}}\vspace{-1em} \hspace{-3em}%
\end{center}
\begin{center}
\subfloat{
\resizebox{8cm}{5cm}{\includegraphics{figures/eps/vio/chi/b10/e0.01/n00040960_fin_hap.eps}}} \hspace{-3em}%
\subfloat{
\resizebox{8cm}{5cm}{\includegraphics{figures/eps/vio/chi/b10/e0.01/n00040960_fin_hap_wovio.eps}}}\vspace{-1em} \hspace{-3em}%
\end{center}

\begin{center}
\subfloat{
\resizebox{8cm}{5cm}{\includegraphics{figures/eps/vio/chi/b10/e0.01/n00081920_fin_hap.eps}}} \hspace{-3em}%
\subfloat{
\resizebox{8cm}{5cm}{\includegraphics{figures/eps/vio/chi/b10/e0.01/n00081920_fin_hap_wovio.eps}}}\vspace{-1em} \hspace{-3em}%
\end{center}

\begin{center}
\subfloat{
\resizebox{8cm}{5cm}{\includegraphics{figures/eps/vio/chi/b10/e0.01/inf_hap.eps}}}\hspace{-3em}%
\subfloat{
\resizebox{8cm}{5cm}{\includegraphics{figures/eps/vio/chi/b10/e0.01/inf_hap_wovio.eps}}}\vspace{-0.5em} \hspace{-3em}%

\caption{\textbf{Infinite and finite haploid population oscillation behavior in case of violation in $\bm{\chi}$ for genome length $\ell = 10$ and $\bm{\epsilon} = 0.01$:} 
  In left column, $d'$ is distance of finite population of size $n$ or infinite population to limit $\bm{z}^\ast$ for $g$ generations. In right column, $d$ is distance of finite population of size $N$ or infinite population to limits without violation.}
\label{oscillation_10h_vio_chi_0.01}
\end{center}
\end{figure}


% l = 12
\begin{figure}[H]
\begin{center}
\subfloat{
\resizebox{8cm}{5cm}{\includegraphics{figures/eps/vio/chi/b12/e0.01/n00004096_fin_hap.eps}}} \hspace{-3em}%
\subfloat{
\resizebox{8cm}{5cm}{\includegraphics{figures/eps/vio/chi/b12/e0.01/n00004096_fin_hap_wovio.eps}}}\vspace{-1em} \hspace{-3em}%
\end{center}
\begin{center}
\subfloat{
\resizebox{8cm}{5cm}{\includegraphics{figures/eps/vio/chi/b12/e0.01/n00040960_fin_hap.eps}}} \hspace{-3em}%
\subfloat{
\resizebox{8cm}{5cm}{\includegraphics{figures/eps/vio/chi/b12/e0.01/n00040960_fin_hap_wovio.eps}}}\vspace{-1em} \hspace{-3em}%
\end{center}

\begin{center}
\subfloat{
\resizebox{8cm}{5cm}{\includegraphics{figures/eps/vio/chi/b12/e0.01/n00081920_fin_hap.eps}}} \hspace{-3em}%
\subfloat{
\resizebox{8cm}{5cm}{\includegraphics{figures/eps/vio/chi/b12/e0.01/n00081920_fin_hap_wovio.eps}}}\vspace{-1em} \hspace{-3em}%
\end{center}

\begin{center}
\subfloat{
\resizebox{8cm}{5cm}{\includegraphics{figures/eps/vio/chi/b12/e0.01/inf_hap.eps}}}\hspace{-3em}%
\subfloat{
\resizebox{8cm}{5cm}{\includegraphics{figures/eps/vio/chi/b12/e0.01/inf_hap_wovio.eps}}}\vspace{-0.5em} \hspace{-3em}%


\caption{\textbf{Infinite and finite haploid population oscillation behavior in case of violation in $\bm{\chi}$ for genome length $\ell = 12$ and $\bm{\epsilon} = 0.01$:} 
  In left column, $d'$ is distance of finite population of size $n$ or infinite population to limit $\bm{z}^\ast$ for $g$ generations. In right column, $d$ is distance of finite population of size $N$ or infinite population to limits without violation.}
\label{oscillation_12h_vio_chi_0.01}
\end{center}
\end{figure}

% l = 14

\begin{figure}[H]
\begin{center}
\subfloat{
\resizebox{8cm}{5cm}{\includegraphics{figures/eps/vio/chi/b14/e0.01/n00004096_fin_hap.eps}}} \hspace{-3em}%
\subfloat{
\resizebox{8cm}{5cm}{\includegraphics{figures/eps/vio/chi/b14/e0.01/n00004096_fin_hap_wovio.eps}}}\vspace{-1em} \hspace{-3em}%
\end{center}
\begin{center}
\subfloat{
\resizebox{8cm}{5cm}{\includegraphics{figures/eps/vio/chi/b14/e0.01/n00040960_fin_hap.eps}}} \hspace{-3em}%
\subfloat{
\resizebox{8cm}{5cm}{\includegraphics{figures/eps/vio/chi/b14/e0.01/n00040960_fin_hap_wovio.eps}}}\vspace{-1em} \hspace{-3em}%
\end{center}

\begin{center}
\subfloat{
\resizebox{8cm}{5cm}{\includegraphics{figures/eps/vio/chi/b14/e0.01/n00081920_fin_hap.eps}}} \hspace{-3em}%
\subfloat{
\resizebox{8cm}{5cm}{\includegraphics{figures/eps/vio/chi/b14/e0.01/n00081920_fin_hap_wovio.eps}}}\vspace{-1em} \hspace{-3em}%
\end{center}

\begin{center}
\subfloat{
\resizebox{8cm}{5cm}{\includegraphics{figures/eps/vio/chi/b14/e0.01/inf_hap.eps}}}\hspace{-3em}%
\subfloat{
\resizebox{8cm}{5cm}{\includegraphics{figures/eps/vio/chi/b14/e0.01/inf_hap_wovio.eps}}}\vspace{-0.5em} \hspace{-3em}%

\caption{\textbf{Infinite and finite haploid population oscillation behavior in case of violation in $\bm{\chi}$ for genome length $\ell = 14$ and $\bm{\epsilon} = 0.01$:} 
  In left column, $d'$ is distance of finite population of size $n$ or infinite population to limit $\bm{z}^\ast$ for $g$ generations. In right column, $d$ is distance of finite population of size $N$ or infinite population to limits without violation.}
\label{oscillation_14h_vio_chi_0.01}
\end{center}
\end{figure}

The right column in figures \ref{oscillation_8h_vio_chi_0.01} through \ref{oscillation_14h_vio_chi_0.01} 
shows distance of finite and infinite haploid populations to non-violation limits $\bm{p^\ast}$ and $\bm{q^\ast}$ with $\bm{\epsilon} \;=\; 0.01$. 
Graphs in the right column give picture of oscillating behavior of haploid population given violation. 
Both finite and infinite populations oscillate given violation. Since value of $\bm{\epsilon}$ 
is very small, damping of ripples is slow. New masks created in crossover distribution with $\bm{\epsilon} \;=\; 0.01$ have very small 
probability of being used during crossover. The probability of the new masks being used is so low for $\bm{\epsilon} \;=\; 0.01$ that even 
infinite population oscillation did not die out completely in 50 generations. 

The left column of figures \ref{oscillation_8h_vio_chi_0.01} through \ref{oscillation_14h_vio_chi_0.01} 
shows distance of finite and infinite haploid populations to limit $\bm{z^\ast}$ 
(limit with violation in crossover distribution $\bm{\chi}$) when $\bm{\epsilon} \;=\; 0.01$. 
The distance between finite population and limit $\bm{z}^\ast$ (limit with violation in $\bm{\chi}$ distribution) 
decreases as finite population size increases, 
and finite population shows behavior similar to infinite population behavior as finite population reach large number. 
The distance data for haploid population in case of violation in $\bm{\chi}$ distribution 
with $\bm{\epsilon} \;=\; 0.01$ for different finite population size $N$ are tabulated in table \ref{distanceChiHapEps0.01}.

\begin{table}[ht]
\caption{\textbf{Distance measured for violation in $\bm{\chi}$ with $\bm{\epsilon} \;=\; 0.01$  for haploids:} $\ell$ is genome length, 
and average distance between finite and 
infinite populations is tabulated in last three columns.}
\centering
\begin{tabularx}{0.75\textwidth}{ c *{3}{X}}
\toprule
$\ell$ & $N = 4096$ & $N = 40960$ & $N = 81920$  \\
\midrule
8 & 0.0186	&  0.0150 	& 0.0115 \\
10 & 0.0158	&  0.0062 	& 0.0051 \\ 
12 & 0.0158	&  0.0056	& 0.0045 \\
14 & 0.0156	&  0.0050	& 0.0036 \\ 
\bottomrule
\end{tabularx}
\label{distanceChiHapEps0.01}
\end{table} 

From table \ref{distanceChiHapEps0.01}, average distance calculated for finite population size $4096$ is $0.0164$, 
for size $40960$ is $0.0079$ and for size $81920$ is $0.0062$. These results show average distance 
between finite population and limit $\bm{z^\ast}$ closely follows expected single step distance 
between finite and infinite population given in \ref{tableExpectedDistance}. The distance decreased as $1/\sqrt{N}$. 
Also, the distance decreased as genome length $\ell$ increased for all sizes of finite haploid populations 
with $\bm{\epsilon} \;=\; 0.01$.

\subsection{Haploid Population $\mathtt{\sim}$ $\epsilon: 0.1$}

% l = 8

\begin{figure}[h]
\begin{center}
\subfloat{
\resizebox{8cm}{5cm}{\includegraphics{figures/eps/vio/chi/b8/e0.1/n00004096_fin_hap.eps}}} \hspace{-3em}%
\subfloat{
\resizebox{8cm}{5cm}{\includegraphics{figures/eps/vio/chi/b8/e0.1/n00004096_fin_hap_wovio.eps}}}\vspace{-1em} \hspace{-3em}%
\end{center}
\begin{center}
\subfloat{
\resizebox{8cm}{5cm}{\includegraphics{figures/eps/vio/chi/b8/e0.1/n00040960_fin_hap.eps}}} \hspace{-3em}%
\subfloat{
\resizebox{8cm}{5cm}{\includegraphics{figures/eps/vio/chi/b8/e0.1/n00040960_fin_hap_wovio.eps}}}\vspace{-1em} \hspace{-3em}%
\end{center}

\begin{center}
\subfloat{
\resizebox{8cm}{5cm}{\includegraphics{figures/eps/vio/chi/b8/e0.1/n00081920_fin_hap.eps}}} \hspace{-3em}%
\subfloat{
\resizebox{8cm}{5cm}{\includegraphics{figures/eps/vio/chi/b8/e0.1/n00081920_fin_hap_wovio.eps}}}\vspace{-1em} \hspace{-3em}%
\end{center}

\begin{center}
\subfloat{
\resizebox{8cm}{5cm}{\includegraphics{figures/eps/vio/chi/b8/e0.1/inf_hap.eps}}}\hspace{-3em}%
\subfloat{
\resizebox{8cm}{5cm}{\includegraphics{figures/eps/vio/chi/b8/e0.1/inf_hap_wovio.eps}}}\vspace{-0.5em} \hspace{-3em}%


\caption{\textbf{Infinite and finite haploid population oscillation behavior in case of violation in $\bm{\chi}$ for genome length $\ell = 8$ and $\bm{\epsilon} = 0.1$:} 
  In left column, $d'$ is distance of finite population of size $n$ or infinite population to limit $\bm{z}^\ast$ for $g$ generations. In right column, $d$ is distance of finite population or infinite population to limits $\bm{p}^\ast$ and $\bm{q}^\ast$ without violation.}
\label{oscillation_8h_vio_chi_0.1}
\end{center}
\end{figure}

% l = 10

\begin{figure}[h]
\begin{center}
\subfloat{
\resizebox{8cm}{5cm}{\includegraphics{figures/eps/vio/chi/b10/e0.1/n00004096_fin_hap.eps}}} \hspace{-3em}%
\subfloat{
\resizebox{8cm}{5cm}{\includegraphics{figures/eps/vio/chi/b10/e0.1/n00004096_fin_hap_wovio.eps}}}\vspace{-1em} \hspace{-3em}%
\end{center}
\begin{center}
\subfloat{
\resizebox{8cm}{5cm}{\includegraphics{figures/eps/vio/chi/b10/e0.1/n00040960_fin_hap.eps}}} \hspace{-3em}%
\subfloat{
\resizebox{8cm}{5cm}{\includegraphics{figures/eps/vio/chi/b10/e0.1/n00040960_fin_hap_wovio.eps}}}\vspace{-1em} \hspace{-3em}%
\end{center}

\begin{center}
\subfloat{
\resizebox{8cm}{5cm}{\includegraphics{figures/eps/vio/chi/b10/e0.1/n00081920_fin_hap.eps}}} \hspace{-3em}%
\subfloat{
\resizebox{8cm}{5cm}{\includegraphics{figures/eps/vio/chi/b10/e0.1/n00081920_fin_hap_wovio.eps}}}\vspace{-1em} \hspace{-3em}%
\end{center}

\begin{center}
\subfloat{
\resizebox{8cm}{5cm}{\includegraphics{figures/eps/vio/chi/b10/e0.1/inf_hap.eps}}}\hspace{-3em}%
\subfloat{
\resizebox{8cm}{5cm}{\includegraphics{figures/eps/vio/chi/b10/e0.1/inf_hap_wovio.eps}}}\vspace{-0.5em} \hspace{-3em}%

\caption{\textbf{Infinite and finite haploid population oscillation behavior in case of violation in $\bm{\chi}$ for genome length $\ell = 10$ and $\bm{\epsilon} = 0.1$:} 
  In left column, $d'$ is distance of finite population of size $n$ or infinite population to limit $\bm{z}^\ast$ for $g$ generations. In right column, $d$ is distance of finite population or infinite population to limits $\bm{p}^\ast$ and $\bm{q}^\ast$ without violation.}
\label{oscillation_10h_vio_chi_0.1}
\end{center}
\end{figure}


% l = 12
\begin{figure}[h]
\begin{center}
\subfloat{
\resizebox{8cm}{5cm}{\includegraphics{figures/eps/vio/chi/b12/e0.1/n00004096_fin_hap.eps}}} \hspace{-3em}%
\subfloat{
\resizebox{8cm}{5cm}{\includegraphics{figures/eps/vio/chi/b12/e0.1/n00004096_fin_hap_wovio.eps}}}\vspace{-1em} \hspace{-3em}%
\end{center}
\begin{center}
\subfloat{
\resizebox{8cm}{5cm}{\includegraphics{figures/eps/vio/chi/b12/e0.1/n00040960_fin_hap.eps}}} \hspace{-3em}%
\subfloat{
\resizebox{8cm}{5cm}{\includegraphics{figures/eps/vio/chi/b12/e0.1/n00040960_fin_hap_wovio.eps}}}\vspace{-1em} \hspace{-3em}%
\end{center}

\begin{center}
\subfloat{
\resizebox{8cm}{5cm}{\includegraphics{figures/eps/vio/chi/b12/e0.1/n00081920_fin_hap.eps}}} \hspace{-3em}%
\subfloat{
\resizebox{8cm}{5cm}{\includegraphics{figures/eps/vio/chi/b12/e0.1/n00081920_fin_hap_wovio.eps}}}\vspace{-1em} \hspace{-3em}%
\end{center}

\begin{center}
\subfloat{
\resizebox{8cm}{5cm}{\includegraphics{figures/eps/vio/chi/b12/e0.1/inf_hap.eps}}}\hspace{-3em}%
\subfloat{
\resizebox{8cm}{5cm}{\includegraphics{figures/eps/vio/chi/b12/e0.1/inf_hap_wovio.eps}}}\vspace{-0.5em} \hspace{-3em}%


\caption{\textbf{Infinite and finite haploid population oscillation behavior in case of violation in $\bm{\chi}$ for genome length $\ell = 12$ and $\bm{\epsilon} = 0.1$:} 
  In left column, $d'$ is distance of finite population of size $n$ or infinite population to limit $\bm{z}^\ast$ for $g$ generations. In right column, $d$ is distance of finite population or infinite population to limits $\bm{p}^\ast$ and $\bm{q}^\ast$ without violation.}
\label{oscillation_12h_vio_chi_0.1}
\end{center}
\end{figure}

% l = 14

\begin{figure}[h]
\begin{center}
\subfloat{
\resizebox{8cm}{5cm}{\includegraphics{figures/eps/vio/chi/b14/e0.1/n00004096_fin_hap.eps}}} \hspace{-3em}%
\subfloat{
\resizebox{8cm}{5cm}{\includegraphics{figures/eps/vio/chi/b14/e0.1/n00004096_fin_hap_wovio.eps}}}\vspace{-1em} \hspace{-3em}%
\end{center}
\begin{center}
\subfloat{
\resizebox{8cm}{5cm}{\includegraphics{figures/eps/vio/chi/b14/e0.1/n00040960_fin_hap.eps}}} \hspace{-3em}%
\subfloat{
\resizebox{8cm}{5cm}{\includegraphics{figures/eps/vio/chi/b14/e0.1/n00040960_fin_hap_wovio.eps}}}\vspace{-1em} \hspace{-3em}%
\end{center}

\begin{center}
\subfloat{
\resizebox{8cm}{5cm}{\includegraphics{figures/eps/vio/chi/b14/e0.1/n00081920_fin_hap.eps}}} \hspace{-3em}%
\subfloat{
\resizebox{8cm}{5cm}{\includegraphics{figures/eps/vio/chi/b14/e0.1/n00081920_fin_hap_wovio.eps}}}\vspace{-1em} \hspace{-3em}%
\end{center}

\begin{center}
\subfloat{
\resizebox{8cm}{5cm}{\includegraphics{figures/eps/vio/chi/b14/e0.1/inf_hap.eps}}}\hspace{-3em}%
\subfloat{
\resizebox{8cm}{5cm}{\includegraphics{figures/eps/vio/chi/b14/e0.1/inf_hap_wovio.eps}}}\vspace{-0.5em} \hspace{-3em}%

\caption{\textbf{Infinite and finite haploid population oscillation behavior in case of violation in $\bm{\chi}$ for genome length $\ell = 14$ and $\bm{\epsilon} = 0.1$:} 
  In left column, $d'$ is distance of finite population of size $n$ or infinite population to limit $\bm{z}^\ast$ for $g$ generations. In right column, $d$ is distance of finite population or infinite population to limits $\bm{p}^\ast$ and $\bm{q}^\ast$ without violation.}
\label{oscillation_14h_vio_chi_0.1}
\end{center}
\end{figure}

\clearpage

The right column in figures \ref{oscillation_8h_vio_chi_0.1} through \ref{oscillation_14h_vio_chi_0.1} 
shows distance of finite and infinite haploid populations to non-violation limits $\bm{p^\ast}$ and $\bm{q^\ast}$ with $\bm{\epsilon} \;=\; 0.1$. 
Those graphs indicate oscillating behavior of haploid population given violation. 
Both finite and infinite populations oscillate given violation, and oscillations amplitudes decreases with time. 
However, for $\bm{\epsilon} \;=\; 0.1$, oscillations in infinite populations die out quickly, 
but oscillations in finite populations does not die out completely. Rate of damping of ripples with $\bm{\epsilon} \;=\; 0.1$ is  
larger than with $\bm{\epsilon} \;=\; 0.01$. New masks created in crossover distribution with $\bm{\epsilon} \;=\; 0.1$ have small  
probability of being used during crossover for oscillation in finite populations to die out completely. 

The left column of figures \ref{oscillation_8h_vio_chi_0.1} through \ref{oscillation_14h_vio_chi_0.1} 
shows distance of finite and infinite haploid populations to limit $\bm{z^\ast}$ 
(limit with violation in crossover distribution $\bm{\chi}$) when $\bm{\epsilon} \;=\; 0.1$. 
The distance between finite population and limit $\bm{z}^\ast$ (limit with violation in $\bm{\chi}$ distribution) 
decreases as finite population size increases, 
and finite population shows behavior similar to infinite population behavior as finite population size grows. 
Average distance data for haploid population in case of violation in $\bm{\chi}$ distribution 
with $\bm{\epsilon} \;=\; 0.1$ for different finite population size $N$ are tabulated in table \ref{distanceChiHapEps0.1}.

\begin{table}[ht]
\caption{\textbf{Distance measured for violation in $\bm{\chi}$ with $\bm{\epsilon} \;=\; 0.1$  for haploids:} $\ell$ is genome length, 
average distance between finite and infinite population is tabulated in the last three columns, and last row is expected single step distance.}
\centering
\begin{tabularx}{0.75\textwidth}{ c *{3}{X}}
\toprule
$\ell$ & $N = 4096$ & $N = 40960$ & $N = 81920$  \\
\midrule
8 & 0.0163	& 0.0061 	& 0.0051 \\
10 & 0.0157	&  0.0051	& 0.0037 \\	
12 & 0.0157	&  0.0051	& 0.0037 \\	
14 & 0.0156	&  0.0049	& 0.0035 \\
\midrule
$1/\sqrt{N}$ & 0.0156 & 0.0049 & 0.0035 \\
\bottomrule
\end{tabularx}
\label{distanceChiHapEps0.1}
\end{table} 

From , average distance calculated for finite population size $4096$ is $0.0158$, 
for size $40960$ is $0.0053$ and for size $81920$ is $0.0038$. The results from Table \ref{distanceChiHapEps0.1} show average distance 
between finite population and limit $\bm{z^\ast}$ approaches the expected single step distance 
between finite and infinite population. The distance decreased as $1/\sqrt{N}$. 
Also, the distance decreased as genome length $\ell$ increased for all sizes of finite haploid populations 
with $\bm{\epsilon} \;=\; 0.1$.
 

\subsection{Haploid Population $\mathtt{\sim}$ $\epsilon: 0.5$}

% l = 8
\mbox{}\\[-0.75in]
\begin{figure}[!b]
\begin{center}
\subfloat{
\resizebox{8cm}{4.5cm}{\includegraphics{figures/eps/vio/chi/b8/e0.5/n00004096_fin_hap.eps}}} \hspace{-3em}%
\subfloat{
\resizebox{8cm}{4.5cm}{\includegraphics{figures/eps/vio/chi/b8/e0.5/n00004096_fin_hap_wovio.eps}}}\vspace{-1em} \hspace{-3em}%
\end{center}
\begin{center}
\subfloat{
\resizebox{8cm}{4.5cm}{\includegraphics{figures/eps/vio/chi/b8/e0.5/n00040960_fin_hap.eps}}} \hspace{-3em}%
\subfloat{
\resizebox{8cm}{4.5cm}{\includegraphics{figures/eps/vio/chi/b8/e0.5/n00040960_fin_hap_wovio.eps}}}\vspace{-1em} \hspace{-3em}%
\end{center}

\begin{center}
\subfloat{
\resizebox{8cm}{4.5cm}{\includegraphics{figures/eps/vio/chi/b8/e0.5/n00081920_fin_hap.eps}}} \hspace{-3em}%
\subfloat{
\resizebox{8cm}{4.5cm}{\includegraphics{figures/eps/vio/chi/b8/e0.5/n00081920_fin_hap_wovio.eps}}}\vspace{-1em} \hspace{-3em}%
\end{center}

\begin{center}
\subfloat{
\resizebox{8cm}{4.5cm}{\includegraphics{figures/eps/vio/chi/b8/e0.5/inf_hap.eps}}}\hspace{-3em}%
\subfloat{
\resizebox{8cm}{4.5cm}{\includegraphics{figures/eps/vio/chi/b8/e0.5/inf_hap_wovio.eps}}}\vspace{-0.5em} \hspace{-3em}%


\caption[\textbf{Infinite and finite haploid population behavior for $\bm{\chi}$ violation, $\ell = 8$ and $\bm{\epsilon} = 0.5$}]
{\textbf{Infinite and finite haploid population behavior for $\bm{\chi}$ violation, $\ell = 8$ and $\bm{\epsilon} = 0.5$:} 
  In left column, $d'$ is distance of finite or infinite population to limit $\bm{z}^\ast$ for $g$ generations. 
  In right column, $d$ is distance of finite or infinite population to limits $\bm{p}^\ast$ and $\bm{q}^\ast$.}
\label{oscillation_8h_vio_chi_0.5}
\end{center}
\end{figure}


% l = 10

\begin{figure}[h]
\begin{center}
\subfloat{
\resizebox{8cm}{4.5cm}{\includegraphics{figures/eps/vio/chi/b10/e0.5/n00004096_fin_hap.eps}}} \hspace{-3em}%
\subfloat{
\resizebox{8cm}{4.5cm}{\includegraphics{figures/eps/vio/chi/b10/e0.5/n00004096_fin_hap_wovio.eps}}}\vspace{-1em} \hspace{-3em}%
\end{center}
\begin{center}
\subfloat{
\resizebox{8cm}{4.5cm}{\includegraphics{figures/eps/vio/chi/b10/e0.5/n00040960_fin_hap.eps}}} \hspace{-3em}%
\subfloat{
\resizebox{8cm}{4.5cm}{\includegraphics{figures/eps/vio/chi/b10/e0.5/n00040960_fin_hap_wovio.eps}}}\vspace{-1em} \hspace{-3em}%
\end{center}

\begin{center}
\subfloat{
\resizebox{8cm}{4.5cm}{\includegraphics{figures/eps/vio/chi/b10/e0.5/n00081920_fin_hap.eps}}} \hspace{-3em}%
\subfloat{
\resizebox{8cm}{4.5cm}{\includegraphics{figures/eps/vio/chi/b10/e0.5/n00081920_fin_hap_wovio.eps}}}\vspace{-1em} \hspace{-3em}%
\end{center}

\begin{center}
\subfloat{
\resizebox{8cm}{4.5cm}{\includegraphics{figures/eps/vio/chi/b10/e0.5/inf_hap.eps}}}\hspace{-3em}%
\subfloat{
\resizebox{8cm}{4.5cm}{\includegraphics{figures/eps/vio/chi/b10/e0.5/inf_hap_wovio.eps}}}\vspace{-0.5em} \hspace{-3em}%

\caption[\textbf{Infinite and finite haploid population behavior for $\bm{\chi}$ violation, genome length $\ell = 10$ and $\bm{\epsilon} = 0.5$}]{\textbf{Infinite and finite haploid population behavior for $\bm{\chi}$ violation, genome length $\ell = 10$ and $\bm{\epsilon} = 0.5$:} 
  In left column, $d'$ is distance of finite or infinite population to limit $\bm{z}^\ast$ for $g$ generations. In right column, $d$ is distance of finite or infinite population to limits $\bm{p}^\ast$ and $\bm{q}^\ast$.}
\label{oscillation_10h_vio_chi_0.5}
\end{center}
\end{figure}

% l = 12
\begin{figure}[h]
\begin{center}
\subfloat{
\resizebox{8cm}{4.5cm}{\includegraphics{figures/eps/vio/chi/b12/e0.5/n00004096_fin_hap.eps}}} \hspace{-3em}%
\subfloat{
\resizebox{8cm}{4.5cm}{\includegraphics{figures/eps/vio/chi/b12/e0.5/n00004096_fin_hap_wovio.eps}}}\vspace{-1em} \hspace{-3em}%
\end{center}
\begin{center}
\subfloat{
\resizebox{8cm}{4.5cm}{\includegraphics{figures/eps/vio/chi/b12/e0.5/n00040960_fin_hap.eps}}} \hspace{-3em}%
\subfloat{
\resizebox{8cm}{4.5cm}{\includegraphics{figures/eps/vio/chi/b12/e0.5/n00040960_fin_hap_wovio.eps}}}\vspace{-1em} \hspace{-3em}%
\end{center}

\begin{center}
\subfloat{
\resizebox{8cm}{4.5cm}{\includegraphics{figures/eps/vio/chi/b12/e0.5/n00081920_fin_hap.eps}}} \hspace{-3em}%
\subfloat{
\resizebox{8cm}{4.5cm}{\includegraphics{figures/eps/vio/chi/b12/e0.5/n00081920_fin_hap_wovio.eps}}}\vspace{-1em} \hspace{-3em}%
\end{center}

\begin{center}
\subfloat{
\resizebox{8cm}{4.5cm}{\includegraphics{figures/eps/vio/chi/b12/e0.5/inf_hap.eps}}}\hspace{-3em}%
\subfloat{
\resizebox{8cm}{4.5cm}{\includegraphics{figures/eps/vio/chi/b12/e0.5/inf_hap_wovio.eps}}}\vspace{-0.5em} \hspace{-3em}%


\caption[\textbf{Infinite and finite haploid population behavior for $\bm{\chi}$ violation, genome length $\ell = 12$ and $\bm{\epsilon} = 0.5$}]{\textbf{Infinite and finite haploid population behavior for $\bm{\chi}$ violation, genome length $\ell = 12$ and $\bm{\epsilon} = 0.5$:} 
  In left column, $d'$ is distance of finite or infinite population to limit $\bm{z}^\ast$ for $g$ generations. In right column, $d$ is distance of finite or infinite population to limits $\bm{p}^\ast$ and $\bm{q}^\ast$.}
\label{oscillation_12h_vio_chi_0.5}
\end{center}
\end{figure}

% l = 14

\begin{figure}[h]
\begin{center}
\subfloat{
\resizebox{8cm}{4.5cm}{\includegraphics{figures/eps/vio/chi/b14/e0.5/n00004096_fin_hap.eps}}} \hspace{-3em}%
\subfloat{
\resizebox{8cm}{4.5cm}{\includegraphics{figures/eps/vio/chi/b14/e0.5/n00004096_fin_hap_wovio.eps}}}\vspace{-1em} \hspace{-3em}%
\end{center}
\begin{center}
\subfloat{
\resizebox{8cm}{4.5cm}{\includegraphics{figures/eps/vio/chi/b14/e0.5/n00040960_fin_hap.eps}}} \hspace{-3em}%
\subfloat{
\resizebox{8cm}{4.5cm}{\includegraphics{figures/eps/vio/chi/b14/e0.5/n00040960_fin_hap_wovio.eps}}}\vspace{-1em} \hspace{-3em}%
\end{center}

\begin{center}
\subfloat{
\resizebox{8cm}{4.5cm}{\includegraphics{figures/eps/vio/chi/b14/e0.5/n00081920_fin_hap.eps}}} \hspace{-3em}%
\subfloat{
\resizebox{8cm}{4.5cm}{\includegraphics{figures/eps/vio/chi/b14/e0.5/n00081920_fin_hap_wovio.eps}}}\vspace{-1em} \hspace{-3em}%
\end{center}

\begin{center}
\subfloat{
\resizebox{8cm}{4.5cm}{\includegraphics{figures/eps/vio/chi/b14/e0.5/inf_hap.eps}}}\hspace{-3em}%
\subfloat{
\resizebox{8cm}{4.5cm}{\includegraphics{figures/eps/vio/chi/b14/e0.5/inf_hap_wovio.eps}}}\vspace{-0.5em} \hspace{-3em}%

\caption[\textbf{Infinite and finite haploid population behavior for $\bm{\chi}$ violation, genome length $\ell = 14$ and $\bm{\epsilon} = 0.5$}]{\textbf{Infinite and finite haploid population behavior for $\bm{\chi}$ violation, genome length $\ell = 14$ and $\bm{\epsilon} = 0.5$:} 
  In left column, $d'$ is distance of finite or infinite population to limit $\bm{z}^\ast$ for $g$ generations. In right column, $d$ is distance of finite or infinite population to limits $\bm{p}^\ast$ and $\bm{q}^\ast$.}
\label{oscillation_14h_vio_chi_0.5}
\end{center}
\end{figure}

\clearpage

The right column in figures \ref{oscillation_8h_vio_chi_0.5} through \ref{oscillation_14h_vio_chi_0.5} 
shows distance of finite and infinite haploid populations to non-violation limits $\bm{p^\ast}$ and $\bm{q^\ast}$ with $\bm{\epsilon} \;=\; 0.5$. 
The graphs indicate oscillating behavior. 
Unlike mutation with violation $\bm{\epsilon} \;=\; 0.5$, oscillation is observed for longer length of generations. 
Finite populations still show some though not very clear oscillations, and then show randomness in behavior as generation progresses. 
Infinite population also oscillates but the oscillation dies out quickly. Randomness in finite population behavior increases 
more than for smaller values of $\bm{\epsilon}$, especially as $\ell$ increases.

The left column of figures \ref{oscillation_8h_vio_chi_0.5} through \ref{oscillation_14h_vio_chi_0.5} 
shows distance of finite and infinite haploid populations to limit $\bm{z^\ast}$ 
(limit with violation in crossover distribution $\bm{\chi}$) when $\bm{\epsilon} \;=\; 0.5$. 
The distance decreases as population size increases, 
and finite population shows behavior similar to infinite population behavior as finite population size grows. 
Average distance data for haploid population in case of violation in $\bm{\chi}$ distribution 
with $\bm{\epsilon} \;=\; 0.5$ for different finite population size $N$ are tabulated in table \ref{distanceChiHapEps0.5}.

\begin{table}[h]
\caption[\textbf{Distance measured for violation in $\bm{\chi}$ with $\bm{\epsilon} \;=\; 0.5$  for haploids}]{\textbf{Distance measured for violation in $\bm{\chi}$ with $\bm{\epsilon} \;=\; 0.5$  for haploids:} $\ell$ is genome length, 
average distance between finite and infinite population is tabulated in the last three columns, and last row is expected single step distance.}
\centering
\begin{tabularx}{0.75\textwidth}{ c *{3}{X}}
\toprule
$\ell$ & $N = 4096$ & $N = 40960$ & $N = 81920$  \\
\midrule
8 & 0.0156	&  0.0051	& 0.0036 \\
10 & 0.0155	&  0.0049	& 0.0035 \\
12 & 0.0157	&  0.0050	& 0.0035 \\
14 & 0.0156	&  0.0049	& 0.0035 \\      
\midrule
$1/\sqrt{N}$ & 0.0156 & 0.0049 & 0.0035 \\
\bottomrule
\end{tabularx}
\label{distanceChiHapEps0.5}
\end{table} 

Table \ref{distanceChiHapEps0.5} shows that the average distance 
between finite and infinite populations approaches the expected single step distance $1/\sqrt{N}$. 










\subsection{Diploid Population $\mathtt{\sim}$ $\epsilon: 0.01$}
% l = 8
\begin{figure}[h]
\begin{center}
\subfloat{
\resizebox{8cm}{5cm}{\includegraphics{figures/eps/vio/chi/b8/e0.01/n00004096_fin_dip.eps}}}\hspace{-3em}%
\subfloat{
\resizebox{8cm}{5cm}{\includegraphics{figures/eps/vio/chi/b8/e0.01/n00004096_fin_dip_wovio.eps}}}\vspace{-1em}  \hspace{-3em}%
\end{center}
\begin{center}
\subfloat{
\resizebox{8cm}{5cm}{\includegraphics{figures/eps/vio/chi/b8/e0.01/n00040960_fin_dip.eps}}}\hspace{-3em}%
\subfloat{
\resizebox{8cm}{5cm}{\includegraphics{figures/eps/vio/chi/b8/e0.01/n00040960_fin_dip_wovio.eps}}}\vspace{-1em}  \hspace{-3em}%
\end{center}


\begin{center}
\subfloat{
\resizebox{8cm}{5cm}{\includegraphics{figures/eps/vio/chi/b8/e0.01/n00081920_fin_dip.eps}}}\hspace{-3em}%
\subfloat{
\resizebox{8cm}{5cm}{\includegraphics{figures/eps/vio/chi/b8/e0.01/n00081920_fin_dip_wovio.eps}}}\vspace{-1em}  \hspace{-3em}%
\end{center}

\begin{center}
\subfloat{
\resizebox{8cm}{5cm}{\includegraphics{figures/eps/vio/chi/b8/e0.01/inf_dip.eps}}}\hspace{-3em}%
\subfloat{
\resizebox{8cm}{5cm}{\includegraphics{figures/eps/vio/chi/b8/e0.01/inf_dip_wovio.eps}}}\vspace{-0.5em}  \hspace{-3em}%


\caption[\textbf{Infinite and finite diploid population oscillation behavior in case of violation in $\bm{\chi}$ for genome length $\ell = 8$ and $\bm{\epsilon} = 0.01$}]{\textbf{Infinite and finite diploid population oscillation behavior in case of violation in $\bm{\chi}$ for genome length $\ell = 8$ and $\bm{\epsilon} = 0.01$:} 
  In left column, $d'$ is distance of finite population of size $n$ or infinite population to limit $\bm{z}^\ast$ for $g$ generations. In right column, $d$ is distance of finite population or infinite population to limits $\bm{p}^\ast$ and $\bm{q}^\ast$ without violation.}
\label{oscillation_8d_vio_chi_0.01}
\end{center}
\end{figure}

% l = 10

\begin{figure}[h]
\begin{center}
\subfloat{
\resizebox{8cm}{5cm}{\includegraphics{figures/eps/vio/chi/b10/e0.01/n00004096_fin_dip.eps}}}\hspace{-3em}%
\subfloat{
\resizebox{8cm}{5cm}{\includegraphics{figures/eps/vio/chi/b10/e0.01/n00004096_fin_dip_wovio.eps}}}\vspace{-1em}  \hspace{-3em}%
\end{center}
\begin{center}
\subfloat{
\resizebox{8cm}{5cm}{\includegraphics{figures/eps/vio/chi/b10/e0.01/n00040960_fin_dip.eps}}}\hspace{-3em}%
\subfloat{
\resizebox{8cm}{5cm}{\includegraphics{figures/eps/vio/chi/b10/e0.01/n00040960_fin_dip_wovio.eps}}}\vspace{-1em}  \hspace{-3em}%
\end{center}


\begin{center}
\subfloat{
\resizebox{8cm}{5cm}{\includegraphics{figures/eps/vio/chi/b10/e0.01/n00081920_fin_dip.eps}}}\hspace{-3em}%
\subfloat{
\resizebox{8cm}{5cm}{\includegraphics{figures/eps/vio/chi/b10/e0.01/n00081920_fin_dip_wovio.eps}}}\vspace{-1em}  \hspace{-3em}%
\end{center}

\begin{center}
\subfloat{
\resizebox{8cm}{5cm}{\includegraphics{figures/eps/vio/chi/b10/e0.01/inf_dip.eps}}}\hspace{-3em}%
\subfloat{
\resizebox{8cm}{5cm}{\includegraphics{figures/eps/vio/chi/b10/e0.01/inf_dip_wovio.eps}}}\vspace{-0.5em}  \hspace{-3em}%


\caption[\textbf{Infinite and finite diploid population oscillation behavior in case of violation in $\bm{\chi}$ for genome length $\ell = 10$ and $\bm{\epsilon} = 0.01$}]{\textbf{Infinite and finite diploid population oscillation behavior in case of violation in $\bm{\chi}$ for genome length $\ell = 10$ and $\bm{\epsilon} = 0.01$:} 
  In left column, $d'$ is distance of finite population of size $n$ or infinite population to limit $\bm{z}^\ast$ for $g$ generations. In right column, $d$ is distance of finite population or infinite population to limits $\bm{p}^\ast$ and $\bm{q}^\ast$ without violation.}
\label{oscillation_10d_vio_chi_0.01}
\end{center}
\end{figure}

% l = 12

\begin{figure}[h]
\begin{center}
\subfloat{
\resizebox{8cm}{5cm}{\includegraphics{figures/eps/vio/chi/b12/e0.01/n00004096_fin_dip.eps}}}\hspace{-3em}%
\subfloat{
\resizebox{8cm}{5cm}{\includegraphics{figures/eps/vio/chi/b12/e0.01/n00004096_fin_dip_wovio.eps}}}\vspace{-1em}  \hspace{-3em}%
\end{center}
\begin{center}
\subfloat{
\resizebox{8cm}{5cm}{\includegraphics{figures/eps/vio/chi/b12/e0.01/n00040960_fin_dip.eps}}}\hspace{-3em}%
\subfloat{
\resizebox{8cm}{5cm}{\includegraphics{figures/eps/vio/chi/b12/e0.01/n00040960_fin_dip_wovio.eps}}}\vspace{-1em}  \hspace{-3em}%
\end{center}


\begin{center}
\subfloat{
\resizebox{8cm}{5cm}{\includegraphics{figures/eps/vio/chi/b12/e0.01/n00081920_fin_dip.eps}}}\hspace{-3em}%
\subfloat{
\resizebox{8cm}{5cm}{\includegraphics{figures/eps/vio/chi/b12/e0.01/n00081920_fin_dip_wovio.eps}}}\vspace{-1em}  \hspace{-3em}%
\end{center}

\begin{center}
\subfloat{
\resizebox{8cm}{5cm}{\includegraphics{figures/eps/vio/chi/b12/e0.01/inf_dip.eps}}}\hspace{-3em}%
\subfloat{
\resizebox{8cm}{5cm}{\includegraphics{figures/eps/vio/chi/b12/e0.01/inf_dip_wovio.eps}}}\vspace{-0.5em}  \hspace{-3em}%


\caption[\textbf{Infinite and finite diploid population oscillation behavior in case of violation in $\bm{\chi}$ for genome length $\ell = 12$ and $\bm{\epsilon} = 0.01$}]{\textbf{Infinite and finite diploid population oscillation behavior in case of violation in $\bm{\chi}$ for genome length $\ell = 12$ and $\bm{\epsilon} = 0.01$:} 
  In left column, $d'$ is distance of finite population of size $n$ or infinite population to limit $\bm{z}^\ast$ for $g$ generations. In right column, $d$ is distance of finite population or infinite population to limits $\bm{p}^\ast$ and $\bm{q}^\ast$ without violation.}
\label{oscillation_12d_vio_chi_0.01}
\end{center}
\end{figure}

% l = 14

\begin{figure}[h]
\begin{center}
\subfloat{
\resizebox{8cm}{5cm}{\includegraphics{figures/eps/vio/chi/b14/e0.01/n00004096_fin_dip.eps}}}\hspace{-3em}%
\subfloat{
\resizebox{8cm}{5cm}{\includegraphics{figures/eps/vio/chi/b14/e0.01/n00004096_fin_dip_wovio.eps}}}\vspace{-1em}  \hspace{-3em}%
\end{center}
\begin{center}
\subfloat{
\resizebox{8cm}{5cm}{\includegraphics{figures/eps/vio/chi/b14/e0.01/n00040960_fin_dip.eps}}}\hspace{-3em}%
\subfloat{
\resizebox{8cm}{5cm}{\includegraphics{figures/eps/vio/chi/b14/e0.01/n00040960_fin_dip_wovio.eps}}}\vspace{-1em}  \hspace{-3em}%
\end{center}


\begin{center}
\subfloat{
\resizebox{8cm}{5cm}{\includegraphics{figures/eps/vio/chi/b14/e0.01/n00081920_fin_dip.eps}}}\hspace{-3em}%
\subfloat{
\resizebox{8cm}{5cm}{\includegraphics{figures/eps/vio/chi/b14/e0.01/n00081920_fin_dip_wovio.eps}}}\vspace{-1em}  \hspace{-3em}%
\end{center}

\begin{center}
\subfloat{
\resizebox{8cm}{5cm}{\includegraphics{figures/eps/vio/chi/b14/e0.01/inf_dip.eps}}}\hspace{-3em}%
\subfloat{
\resizebox{8cm}{5cm}{\includegraphics{figures/eps/vio/chi/b14/e0.01/inf_dip_wovio.eps}}}\vspace{-0.5em}  \hspace{-3em}%


\caption[\textbf{Infinite and finite diploid population oscillation behavior in case of violation in $\bm{\chi}$ for genome length $\ell = 14$ and $\bm{\epsilon} = 0.01$}]{\textbf{Infinite and finite diploid population oscillation behavior in case of violation in $\bm{\chi}$ for genome length $\ell = 14$ and $\bm{\epsilon} = 0.01$:} 
  In left column, $d'$ is distance of finite population of size $n$ or infinite population to limit $\bm{z}^\ast$ for $g$ generations. In right column, $d$ is distance of finite population or infinite population to limits $\bm{p}^\ast$ and $\bm{q}^\ast$ without violation.}
\label{oscillation_14d_vio_chi_0.01}
\end{center}
\end{figure}

\clearpage

The right column in figures \ref{oscillation_8d_vio_chi_0.01} through \ref{oscillation_14d_vio_chi_0.01} 
shows distance of finite and infinite diploid populations to non-violation limits $\bm{p^\ast}$ and $\bm{q^\ast}$ with $\bm{\epsilon} \;=\; 0.01$. 
Those graphs indicate oscillating behavior of diploid population given violation. 
Both finite and infinite populations oscillate given violation. Like in haploid case, oscillations are sharper. Since value of $\bm{\epsilon}$ 
is small, damping of ripples is slow. A new mask introduced in crossover distribution with $\bm{\epsilon} \;=\; 0.01$ has small 
probability of being used during crossover. Infinite population oscillation does not die out in 50 generations. As value of $\ell$ 
increases, random drifting and wiggling of finite population occurs for small population size, and oscillation gets better with increase in population size. 
These are noticed more clearly in figures \ref{oscillation_12d_vio_chi_0.01} and \ref{oscillation_14d_vio_chi_0.01}.

The left column of figures \ref{oscillation_8d_vio_chi_0.01} through \ref{oscillation_14d_vio_chi_0.01} 
shows distance of finite and infinite diploid populations to limit $\bm{z^\ast}$ 
(limit with violation in crossover distribution $\bm{\chi}$) when $\bm{\epsilon} \;=\; 0.01$. 
The distance between finite population and limit $\bm{z}^\ast$ (limit with violation in $\bm{\chi}$ distribution) 
decreases as finite population size increases. 
Average distance data for diploid population in case of violation in $\bm{\chi}$ distribution 
with $\bm{\epsilon} \;=\; 0.01$ for different finite population size $N$ are tabulated in table \ref{distanceChiDipEps0.01}.


\begin{table}[ht]
\caption[\textbf{Distance measured for violation in $\bm{\chi}$ with $\bm{\epsilon} \;=\; 0.01$ diploids}]{\textbf{Distance measured for violation in $\bm{\chi}$ with $\bm{\epsilon} \;=\; 0.01$ diploids:} $\ell$ is genome length, 
average distance between finite and infinite population is tabulated in the last three columns, and last row is expected single step distance.}
\centering
\begin{tabularx}{0.75\textwidth}{ c *{3}{X}}
\toprule
$\ell$ & $N = 4096$ & $N = 40960$ & $N = 81920$  \\
\midrule
8 & 0.0156	&  0.0051	& 0.0036 \\
10 & 0.0156	&  0.0049	& 0.0035 \\
12 & 0.0156	&  0.0049	& 0.0035 \\
14 & 0.0156	&  0.0049	& 0.0035 \\
\midrule
$1/\sqrt{N}$ & 0.0156 & 0.0049 & 0.0035 \\
\bottomrule
\end{tabularx}
\label{distanceChiDipEps0.01}
\end{table} 

The results from Table \ref{distanceChiDipEps0.01} show average distance 
between finite population and limit $\bm{z^\ast}$ approaches the expected single step distance 
between finite and infinite population. The distance decreased as $1/\sqrt{N}$. 
Also, the distance is smaller in diploid populations than in haploid populations with $\bm{\epsilon} \;=\; 0.01$.



\subsection{Diploid Population $\mathtt{\sim}$ $\epsilon: 0.1$}
% l = 8
\begin{figure}[H]
\begin{center}
\subfloat{
\resizebox{8cm}{5cm}{\includegraphics{figures/eps/vio/chi/b8/e0.1/n00004096_fin_dip.eps}}}\hspace{-3em}%
\subfloat{
\resizebox{8cm}{5cm}{\includegraphics{figures/eps/vio/chi/b8/e0.1/n00004096_fin_dip_wovio.eps}}}\vspace{-1em}  \hspace{-3em}%
\end{center}
\begin{center}
\subfloat{
\resizebox{8cm}{5cm}{\includegraphics{figures/eps/vio/chi/b8/e0.1/n00040960_fin_dip.eps}}}\hspace{-3em}%
\subfloat{
\resizebox{8cm}{5cm}{\includegraphics{figures/eps/vio/chi/b8/e0.1/n00040960_fin_dip_wovio.eps}}}\vspace{-1em}  \hspace{-3em}%
\end{center}


\begin{center}
\subfloat{
\resizebox{8cm}{5cm}{\includegraphics{figures/eps/vio/chi/b8/e0.1/n00081920_fin_dip.eps}}}\hspace{-3em}%
\subfloat{
\resizebox{8cm}{5cm}{\includegraphics{figures/eps/vio/chi/b8/e0.1/n00081920_fin_dip_wovio.eps}}}\vspace{-1em}  \hspace{-3em}%
\end{center}

\begin{center}
\subfloat{
\resizebox{8cm}{5cm}{\includegraphics{figures/eps/vio/chi/b8/e0.1/inf_dip.eps}}}\hspace{-3em}%
\subfloat{
\resizebox{8cm}{5cm}{\includegraphics{figures/eps/vio/chi/b8/e0.1/inf_dip_wovio.eps}}}\vspace{-0.5em}  \hspace{-3em}%


\caption{\textbf{Infinite and finite diploid population oscillation behavior in case of violation in $\bm{\chi}$ for genome length $\ell = 8$ and $\bm{\epsilon} = 0.1$:} 
  In left column, $d'$ is distance of finite population of size $n$ or infinite population to limit $\bm{z}^\ast$ for $g$ generations. In right column, $d$ is distance of finite population of size $N$ or infinite population to limits without violation.}
\label{oscillation_8d_vio_chi_0.1}
\end{center}
\end{figure}

% l = 10

\begin{figure}[H]
\begin{center}
\subfloat{
\resizebox{8cm}{5cm}{\includegraphics{figures/eps/vio/chi/b10/e0.1/n00004096_fin_dip.eps}}}\hspace{-3em}%
\subfloat{
\resizebox{8cm}{5cm}{\includegraphics{figures/eps/vio/chi/b10/e0.1/n00004096_fin_dip_wovio.eps}}}\vspace{-1em}  \hspace{-3em}%
\end{center}
\begin{center}
\subfloat{
\resizebox{8cm}{5cm}{\includegraphics{figures/eps/vio/chi/b10/e0.1/n00040960_fin_dip.eps}}}\hspace{-3em}%
\subfloat{
\resizebox{8cm}{5cm}{\includegraphics{figures/eps/vio/chi/b10/e0.1/n00040960_fin_dip_wovio.eps}}}\vspace{-1em}  \hspace{-3em}%
\end{center}


\begin{center}
\subfloat{
\resizebox{8cm}{5cm}{\includegraphics{figures/eps/vio/chi/b10/e0.1/n00081920_fin_dip.eps}}}\hspace{-3em}%
\subfloat{
\resizebox{8cm}{5cm}{\includegraphics{figures/eps/vio/chi/b10/e0.1/n00081920_fin_dip_wovio.eps}}}\vspace{-1em}  \hspace{-3em}%
\end{center}

\begin{center}
\subfloat{
\resizebox{8cm}{5cm}{\includegraphics{figures/eps/vio/chi/b10/e0.1/inf_dip.eps}}}\hspace{-3em}%
\subfloat{
\resizebox{8cm}{5cm}{\includegraphics{figures/eps/vio/chi/b10/e0.1/inf_dip_wovio.eps}}}\vspace{-0.5em}  \hspace{-3em}%


\caption{\textbf{Infinite and finite diploid population oscillation behavior in case of violation in $\bm{\chi}$ for genome length $\ell = 10$ and $\bm{\epsilon} = 0.1$:} 
  In left column, $d'$ is distance of finite population of size $n$ or infinite population to limit $\bm{z}^\ast$ for $g$ generations. In right column, $d$ is distance of finite population of size $N$ or infinite population to limits without violation.}
\label{oscillation_10d_vio_chi_0.1}
\end{center}
\end{figure}

% l = 12

\begin{figure}[H]
\begin{center}
\subfloat{
\resizebox{8cm}{5cm}{\includegraphics{figures/eps/vio/chi/b12/e0.1/n00004096_fin_dip.eps}}}\hspace{-3em}%
\subfloat{
\resizebox{8cm}{5cm}{\includegraphics{figures/eps/vio/chi/b12/e0.1/n00004096_fin_dip_wovio.eps}}}\vspace{-1em}  \hspace{-3em}%
\end{center}
\begin{center}
\subfloat{
\resizebox{8cm}{5cm}{\includegraphics{figures/eps/vio/chi/b12/e0.1/n00040960_fin_dip.eps}}}\hspace{-3em}%
\subfloat{
\resizebox{8cm}{5cm}{\includegraphics{figures/eps/vio/chi/b12/e0.1/n00040960_fin_dip_wovio.eps}}}\vspace{-1em}  \hspace{-3em}%
\end{center}


\begin{center}
\subfloat{
\resizebox{8cm}{5cm}{\includegraphics{figures/eps/vio/chi/b12/e0.1/n00081920_fin_dip.eps}}}\hspace{-3em}%
\subfloat{
\resizebox{8cm}{5cm}{\includegraphics{figures/eps/vio/chi/b12/e0.1/n00081920_fin_dip_wovio.eps}}}\vspace{-1em}  \hspace{-3em}%
\end{center}

\begin{center}
\subfloat{
\resizebox{8cm}{5cm}{\includegraphics{figures/eps/vio/chi/b12/e0.1/inf_dip.eps}}}\hspace{-3em}%
\subfloat{
\resizebox{8cm}{5cm}{\includegraphics{figures/eps/vio/chi/b12/e0.1/inf_dip_wovio.eps}}}\vspace{-0.5em}  \hspace{-3em}%


\caption{\textbf{Infinite and finite diploid population oscillation behavior in case of violation in $\bm{\chi}$ for genome length $\ell = 12$ and $\bm{\epsilon} = 0.1$:} 
  In left column, $d'$ is distance of finite population of size $n$ or infinite population to limit $\bm{z}^\ast$ for $g$ generations. In right column, $d$ is distance of finite population of size $N$ or infinite population to limits without violation.}
\label{oscillation_12d_vio_chi_0.1}
\end{center}
\end{figure}

% l = 14

\begin{figure}[H]
\begin{center}
\subfloat{
\resizebox{8cm}{5cm}{\includegraphics{figures/eps/vio/chi/b14/e0.1/n00004096_fin_dip.eps}}}\hspace{-3em}%
\subfloat{
\resizebox{8cm}{5cm}{\includegraphics{figures/eps/vio/chi/b14/e0.1/n00004096_fin_dip_wovio.eps}}}\vspace{-1em}  \hspace{-3em}%
\end{center}
\begin{center}
\subfloat{
\resizebox{8cm}{5cm}{\includegraphics{figures/eps/vio/chi/b14/e0.1/n00040960_fin_dip.eps}}}\hspace{-3em}%
\subfloat{
\resizebox{8cm}{5cm}{\includegraphics{figures/eps/vio/chi/b14/e0.1/n00040960_fin_dip_wovio.eps}}}\vspace{-1em}  \hspace{-3em}%
\end{center}


\begin{center}
\subfloat{
\resizebox{8cm}{5cm}{\includegraphics{figures/eps/vio/chi/b14/e0.1/n00081920_fin_dip.eps}}}\hspace{-3em}%
\subfloat{
\resizebox{8cm}{5cm}{\includegraphics{figures/eps/vio/chi/b14/e0.1/n00081920_fin_dip_wovio.eps}}}\vspace{-1em}  \hspace{-3em}%
\end{center}

\begin{center}
\subfloat{
\resizebox{8cm}{5cm}{\includegraphics{figures/eps/vio/chi/b14/e0.1/inf_dip.eps}}}\hspace{-3em}%
\subfloat{
\resizebox{8cm}{5cm}{\includegraphics{figures/eps/vio/chi/b14/e0.1/inf_dip_wovio.eps}}}\vspace{-0.5em}  \hspace{-3em}%


\caption{\textbf{Infinite and finite diploid population oscillation behavior in case of violation in $\bm{\chi}$ for genome length $\ell = 14$ and $\bm{\epsilon} = 0.1$:} 
  In left column, $d'$ is distance of finite population of size $n$ or infinite population to limit $\bm{z}^\ast$ for $g$ generations. In right column, $d$ is distance of finite population of size $N$ or infinite population to limits without violation.}
\label{oscillation_14d_vio_chi_0.1}
\end{center}
\end{figure}









\begin{table}[ht]
\caption{\textbf{Distance measured for violation in $\bm{\chi}$ with $\bm{\epsilon} \;=\; 0.1$ for diploids:} $\ell$ is genome length, 
and average distance between finite and 
infinite populations is tabulated in last three columns.}
\centering
\begin{tabularx}{0.75\textwidth}{ c *{3}{X}}
\toprule
$\ell$ & $N = 4096$ & $N = 40960$ & $N = 81920$  \\
\midrule
8 & 0.0156	&  0.0050	& 0.0035 \\
10 & 0.0156	&  0.0049	& 0.0035 \\
12 & 0.0156	&  0.0049	& 0.0035 \\
14 & 0.0156	&  0.0049	& 0.0035 \\
\bottomrule
\end{tabularx}
\label{distanceChiDipEps0.1}
\end{table} 



 

\subsection{Diploid Population $\mathtt{\sim}$ $\epsilon: 0.5$}
% l = 8
\begin{figure}[h]
\begin{center}
\subfloat{
\resizebox{8cm}{5cm}{\includegraphics{figures/eps/vio/chi/b8/e0.5/n00004096_fin_dip.eps}}}\hspace{-3em}%
\subfloat{
\resizebox{8cm}{5cm}{\includegraphics{figures/eps/vio/chi/b8/e0.5/n00004096_fin_dip_wovio.eps}}}\vspace{-1em}  \hspace{-3em}%
\end{center}
\begin{center}
\subfloat{
\resizebox{8cm}{5cm}{\includegraphics{figures/eps/vio/chi/b8/e0.5/n00040960_fin_dip.eps}}}\hspace{-3em}%
\subfloat{
\resizebox{8cm}{5cm}{\includegraphics{figures/eps/vio/chi/b8/e0.5/n00040960_fin_dip_wovio.eps}}}\vspace{-1em}  \hspace{-3em}%
\end{center}


\begin{center}
\subfloat{
\resizebox{8cm}{5cm}{\includegraphics{figures/eps/vio/chi/b8/e0.5/n00081920_fin_dip.eps}}}\hspace{-3em}%
\subfloat{
\resizebox{8cm}{5cm}{\includegraphics{figures/eps/vio/chi/b8/e0.5/n00081920_fin_dip_wovio.eps}}}\vspace{-1em}  \hspace{-3em}%
\end{center}

\begin{center}
\subfloat{
\resizebox{8cm}{5cm}{\includegraphics{figures/eps/vio/chi/b8/e0.5/inf_dip.eps}}}\hspace{-3em}%
\subfloat{
\resizebox{8cm}{5cm}{\includegraphics{figures/eps/vio/chi/b8/e0.5/inf_dip_wovio.eps}}}\vspace{-0.5em}  \hspace{-3em}%


\caption{\textbf{Infinite and finite diploid population oscillation behavior in case of violation in $\bm{\chi}$ for genome length $\ell = 8$ and $\bm{\epsilon} = 0.5$:} 
  In left column, $d'$ is distance of finite population of size $n$ or infinite population to limit $\bm{z}^\ast$ for $g$ generations. In right column, $d$ is distance of finite population or infinite population to limits $\bm{p}^\ast$ and $\bm{q}^\ast$ without violation.}
\label{oscillation_8d_vio_chi_0.5}
\end{center}
\end{figure}

% l = 10

\begin{figure}[h]
\begin{center}
\subfloat{
\resizebox{8cm}{5cm}{\includegraphics{figures/eps/vio/chi/b10/e0.5/n00004096_fin_dip.eps}}}\hspace{-3em}%
\subfloat{
\resizebox{8cm}{5cm}{\includegraphics{figures/eps/vio/chi/b10/e0.5/n00004096_fin_dip_wovio.eps}}}\vspace{-1em}  \hspace{-3em}%
\end{center}
\begin{center}
\subfloat{
\resizebox{8cm}{5cm}{\includegraphics{figures/eps/vio/chi/b10/e0.5/n00040960_fin_dip.eps}}}\hspace{-3em}%
\subfloat{
\resizebox{8cm}{5cm}{\includegraphics{figures/eps/vio/chi/b10/e0.5/n00040960_fin_dip_wovio.eps}}}\vspace{-1em}  \hspace{-3em}%
\end{center}


\begin{center}
\subfloat{
\resizebox{8cm}{5cm}{\includegraphics{figures/eps/vio/chi/b10/e0.5/n00081920_fin_dip.eps}}}\hspace{-3em}%
\subfloat{
\resizebox{8cm}{5cm}{\includegraphics{figures/eps/vio/chi/b10/e0.5/n00081920_fin_dip_wovio.eps}}}\vspace{-1em}  \hspace{-3em}%
\end{center}

\begin{center}
\subfloat{
\resizebox{8cm}{5cm}{\includegraphics{figures/eps/vio/chi/b10/e0.5/inf_dip.eps}}}\hspace{-3em}%
\subfloat{
\resizebox{8cm}{5cm}{\includegraphics{figures/eps/vio/chi/b10/e0.5/inf_dip_wovio.eps}}}\vspace{-0.5em}  \hspace{-3em}%


\caption{\textbf{Infinite and finite diploid population oscillation behavior in case of violation in $\bm{\chi}$ for genome length $\ell = 10$ and $\bm{\epsilon} = 0.5$:} 
  In left column, $d'$ is distance of finite population of size $n$ or infinite population to limit $\bm{z}^\ast$ for $g$ generations. In right column, $d$ is distance of finite population or infinite population to limits $\bm{p}^\ast$ and $\bm{q}^\ast$ without violation.}
\label{oscillation_10d_vio_chi_0.5}
\end{center}
\end{figure}

% l = 12

\begin{figure}[h]
\begin{center}
\subfloat{
\resizebox{8cm}{5cm}{\includegraphics{figures/eps/vio/chi/b12/e0.5/n00004096_fin_dip.eps}}}\hspace{-3em}%
\subfloat{
\resizebox{8cm}{5cm}{\includegraphics{figures/eps/vio/chi/b12/e0.5/n00004096_fin_dip_wovio.eps}}}\vspace{-1em}  \hspace{-3em}%
\end{center}
\begin{center}
\subfloat{
\resizebox{8cm}{5cm}{\includegraphics{figures/eps/vio/chi/b12/e0.5/n00040960_fin_dip.eps}}}\hspace{-3em}%
\subfloat{
\resizebox{8cm}{5cm}{\includegraphics{figures/eps/vio/chi/b12/e0.5/n00040960_fin_dip_wovio.eps}}}\vspace{-1em}  \hspace{-3em}%
\end{center}


\begin{center}
\subfloat{
\resizebox{8cm}{5cm}{\includegraphics{figures/eps/vio/chi/b12/e0.5/n00081920_fin_dip.eps}}}\hspace{-3em}%
\subfloat{
\resizebox{8cm}{5cm}{\includegraphics{figures/eps/vio/chi/b12/e0.5/n00081920_fin_dip_wovio.eps}}}\vspace{-1em}  \hspace{-3em}%
\end{center}

\begin{center}
\subfloat{
\resizebox{8cm}{5cm}{\includegraphics{figures/eps/vio/chi/b12/e0.5/inf_dip.eps}}}\hspace{-3em}%
\subfloat{
\resizebox{8cm}{5cm}{\includegraphics{figures/eps/vio/chi/b12/e0.5/inf_dip_wovio.eps}}}\vspace{-0.5em}  \hspace{-3em}%


\caption{\textbf{Infinite and finite diploid population oscillation behavior in case of violation in $\bm{\chi}$ for genome length $\ell = 12$ and $\bm{\epsilon} = 0.5$:} 
  In left column, $d'$ is distance of finite population of size $n$ or infinite population to limit $\bm{z}^\ast$ for $g$ generations. In right column, $d$ is distance of finite population or infinite population to limits $\bm{p}^\ast$ and $\bm{q}^\ast$ without violation.}
\label{oscillation_12d_vio_chi_0.5}
\end{center}
\end{figure}

% l = 14

\begin{figure}[h]
\begin{center}
\subfloat{
\resizebox{8cm}{5cm}{\includegraphics{figures/eps/vio/chi/b14/e0.5/n00004096_fin_dip.eps}}}\hspace{-3em}%
\subfloat{
\resizebox{8cm}{5cm}{\includegraphics{figures/eps/vio/chi/b14/e0.5/n00004096_fin_dip_wovio.eps}}}\vspace{-1em}  \hspace{-3em}%
\end{center}
\begin{center}
\subfloat{
\resizebox{8cm}{5cm}{\includegraphics{figures/eps/vio/chi/b14/e0.5/n00040960_fin_dip.eps}}}\hspace{-3em}%
\subfloat{
\resizebox{8cm}{5cm}{\includegraphics{figures/eps/vio/chi/b14/e0.5/n00040960_fin_dip_wovio.eps}}}\vspace{-1em}  \hspace{-3em}%
\end{center}


\begin{center}
\subfloat{
\resizebox{8cm}{5cm}{\includegraphics{figures/eps/vio/chi/b14/e0.5/n00081920_fin_dip.eps}}}\hspace{-3em}%
\subfloat{
\resizebox{8cm}{5cm}{\includegraphics{figures/eps/vio/chi/b14/e0.5/n00081920_fin_dip_wovio.eps}}}\vspace{-1em}  \hspace{-3em}%
\end{center}

\begin{center}
\subfloat{
\resizebox{8cm}{5cm}{\includegraphics{figures/eps/vio/chi/b14/e0.5/inf_dip.eps}}}\hspace{-3em}%
\subfloat{
\resizebox{8cm}{5cm}{\includegraphics{figures/eps/vio/chi/b14/e0.5/inf_dip_wovio.eps}}}\vspace{-0.5em}  \hspace{-3em}%


\caption{\textbf{Infinite and finite diploid population oscillation behavior in case of violation in $\bm{\chi}$ for genome length $\ell = 14$ and $\bm{\epsilon} = 0.5$:} 
  In left column, $d'$ is distance of finite population of size $n$ or infinite population to limit $\bm{z}^\ast$ for $g$ generations. In right column, $d$ is distance of finite population or infinite population to limits $\bm{p}^\ast$ and $\bm{q}^\ast$ without violation.}
\label{oscillation_14d_vio_chi_0.5}
\end{center}
\end{figure}

\clearpage

The right column in figures \ref{oscillation_8d_vio_chi_0.5} through \ref{oscillation_14d_vio_chi_0.5} 
shows distance of finite and infinite diploid populations to non-violation limits $\bm{p^\ast}$ and $\bm{q^\ast}$ with $\bm{\epsilon} \;=\; 0.5$. 
Those graphs indicate oscillating behavior of diploid population given violation. 
Infinite populations oscillate for some generations, and then cease to oscillate given violation. 
Finite populations show some oscillations when $\ell \;=\; 8$ for higher population size as in figure \ref{oscillation_8d_vio_chi_0.5}, but 
for larger $\ell$, finite populations show only random wiggles. 

The left column of figures \ref{oscillation_8d_vio_chi_0.5} through \ref{oscillation_14d_vio_chi_0.5} 
shows distance of finite and infinite diploid populations to limit $\bm{z^\ast}$ 
(limit with violation in crossover distribution $\bm{\chi}$) when $\bm{\epsilon} \;=\; 0.5$. 
The distance between finite population and limit $\bm{z}^\ast$ (limit with violation in $\bm{\chi}$ distribution) 
decreases as finite population size increases.
The distance data for diploid population in case of violation in $\bm{\chi}$ distribution 
with $\bm{\epsilon} \;=\; 0.5$ for different finite population size $N$ are tabulated in table \ref{distanceChiDipEps0.5}.


\begin{table}[ht]
\caption{\textbf{Distance measured for violation in $\bm{\chi}$ with $\bm{\epsilon} \;=\; 0.5$ for diploids:} $\ell$ is genome length, 
average distance between finite and infinite population is tabulated in the last three columns, and last row is expected single step distance.}
\centering
\begin{tabularx}{0.75\textwidth}{ c *{3}{X}}
\toprule
$\ell$ & $N = 4096$ & $N = 40960$ & $N = 81920$  \\
\midrule
8 & 0.0156	&  0.0049	& 0.0035 \\	
10 & 0.0156	&  0.0049	& 0.0035 \\
12 & 0.0156	&  0.0049	& 0.0035 \\
14 & 0.0156	&  0.0049	& 0.0035 \\
\midrule
$1/\sqrt{N}$ & 0.0156 & 0.0049 & 0.0035 \\
\bottomrule
\end{tabularx}
\label{distanceChiDipEps0.5}
\end{table} 

From table \ref{distanceChiDipEps0.5}, average distance calculated for finite population size $4096$ is $0.0156$, 
for size $40960$ is $0.0049$ and for size $81920$ is $0.0035$. These results show average distance 
between finite population and limit $\bm{z^\ast}$ closely follows expected single step distance 
between finite and infinite population. The distance decreased as $1/\sqrt{N}$. 
Also, the distance is smaller in diploid populations than in haploid populations with $\bm{\epsilon} \;=\; 0.5$.
 


\section{Discussion}
In the presence of violation in $\bm{\mu}$, the amplitude of oscillation decreases 
as string length $\ell$ increases. 
Larger population sizes show better oscillation. 
Since diploid populations have effective string length twice the size of haploid populations, 
diploid populations need larger population size than haploid population to exhibit good oscillation. 
As in the case of violation in $\bm{\mu}$, increasing string length $\ell$ 
degrades convergence (as finite population size increases) to infinite population behavior for diploid populations. 
That behavior is noticeable in figures 
\ref{oscillation_8d_vio_chi_0.01} through \ref{oscillation_14d_vio_chi_0.5} for violation in $\bm{\chi}$. 
That behavior is less noticeable in haploid populations.

With increase in the value of $\bm{\epsilon}$, 
oscillation in population diminishes and dampening of oscillation increases. 
Randomness increases with increasing $\bm{\epsilon}$.
Comparing oscillation with violation in $\bm{\mu}$ and $\bm{\chi}$, rate of dampening of oscillation with violation 
in $\bm{\chi}$ seems to be slower than with violation in ${\bm{\mu}}$. 
Diploid populations jumping to other levels 
were observed for string lengths 12 and 14 and population size 4096  ( 
figures \ref{oscillation_12d_vio_chi_0.01}, \ref{oscillation_14d_vio_chi_0.01}, \ref{oscillation_12d_vio_chi_0.1}, 
\ref{oscillation_14d_vio_chi_0.1}, \ref{oscillation_12d_vio_chi_0.5} and \ref{oscillation_14d_vio_chi_0.5}), 
but unlike the case of violation in $\bm{\mu}$, 
the behavior is noticeable when the population size is larger (figure \ref{oscillation_12d_vio_chi_0.1}). 

% As in $\bm{\mu}$ violation, we noticed similar unexpected behavior also in presence of violation in $\bm{\chi}$: 
% figures \ref{oscillation_12d_vio_chi_0.5} and \ref{oscillation_14d_vio_chi_0.5} 
% show that the lines representing distance of population to limits $\bm{p}^\ast$ and $\bm{q}^\ast$ overlap. 
% The graphs suggest that finite population evolution is exploring a plane equidistant from $\bm{p}^\ast$ and $\bm{q}^\ast$.

% Moreover, in figures \ref{oscillation_12d_vio_chi_0.1} and \ref{oscillation_14d_vio_chi_0.1}, infinite population oscillation dies out to 
% give graph of single line. But infinite population is converging to limit $\bm{z}^\ast$. 
% This suggests $\bm{z}^\ast$ may be somewhere in equidistant plane from $\bm{p}^\ast$ and $\bm{q}^\ast$.

\begin{figure}[!ht]
\begin{center}
\subfloat{
\resizebox{16cm}{10cm}{\includegraphics{figures/eps/vio/dist_chi.eps}}}\hspace{-3em}%
\caption[\textbf{Distance between finite and infinite population in case of violation in $\bm{\chi}$}]{\textbf{Distance between finite and infinite population in case of violation in $\bm{\chi}$:}  
  $d$ is distance; $N$ is finite population size; $\bm{\epsilon}$ is level of violation; 
  red line represents distance for $\ell = 8$, green line for $\ell = 10$, blue line for $\ell = 12$, pink line for $\ell = 14$ 
  and black dotted line for expected single step distance.}
\label{vio_chi_dist}
\end{center}
\end{figure}

Figure \ref{vio_chi_dist} summarizes the distance data from tables \ref{distanceChiHapEps0.01} through 
\ref{distanceChiDipEps0.5}. Distance between infinite and finite populations 
for population sizes {4096, 40960, 81920} are plotted for different $\ell$. 
Plots for different violation levels $\bm{\epsilon}$ are arranged in columns. 
Plots for haploid and diploid populations are arranged in two rows. With increase in $\ell$, 
distance between finite and infinite population moves closer to the single step distance. So, since diploid effective population 
string length is twice that of haploid population, 
distance in diploid case moves closer to the single step distance for the same value of $\ell$ than in haploid case. 
Like in the case of $\bm{\mu}$ violation, it is more noticeable in haploid population case that as $\bm{\epsilon}$ increases, 
the distance moves closer to the single step distance.

\section{Summary}
In this chapter, we violated condition \ref{OscCond} for the crossover distribution, 
so that infinite population trajectories have no periodic orbit. 
We explored infinite and finite population oscillation behavior with the violation through experiments. 
We don't know if the Markov chain is regular in this case but we suspect it is not.  
Like in case of $\bm{\mu}$ violation, infinite population oscillation dies out when condition \ref{OscCond} for convergence to 
periodic orbits is violated, but finite populations approximately oscillate for small values of $\bm{\epsilon}$ 
because the probability of using the new mask is low, and not used, 
finite population follows behavior of infinite population without violation in the condition for convergence to 
periodic orbits. However, rate of dampening of oscillation with violation 
in $\bm{\chi}$ is observed to be slower than with violation in ${\bm{\mu}}$. Also more randomness in oscillations are observed 
in this case than in violation in mutation, especially for diploid population.




