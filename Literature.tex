
Schema theorem/ Holland
% genetic algorithm short intro
Genetic algorithm is inspired by population genetics, and is population based, and proceeds over a number of generations to obtain optimal solution. The algorithm was developed by Holland (1975). Holland hoped to develop powerful, broadly applicable techniques to solve problems unsusceptible to other known methods.Basic mechanism of a GA, although there are many variants, consists of:
1. Evaluation of individual fitness and formation of a gene pool
2. Recombination and mutation.

The members of next generation population is formed by individuals produced from these operations. Criteria for fitness evaluation exerts evolutionary force for populations to produce more fit individuals. The process is repeated until system stops to improve or threshold is met.

The members of population are typically fixed length binary strings (also called genome length in later chapters). These members contribute to gene pool according to their relative fitness which is calculated using some objective funtion. They are mutated and recombined by crossover. Mutation corresponds to flipping the bits of an individual with small probability, mutation rate. During crossover, two parents are selected from the pool, a random but same position is chosen within each parent string and segments are exchanged. Small probability, crossover rate, is used perform crossover and otherwise parents are cloned. New offsprings produced after mutation and crossover form next generation.

% The schema theorem ( see \cite{Holland1975}) suggests prerequisite features which a representation should exhibit in order to utilize a 
% GA processing. There is a high probability that above average, short, low-order schemata combine to form a higher order above average schemata. The schema theorem shows that above average schemata will multiply at a given expected minimum rate but does not specify if this increase will occur at optimal rate.




Vose and infinite population model:
Vose has proved infinite population models are equally valid in explaining finite population behavior. 
Given a finite population with proportional representation vector $\bm{p^n}$ at generation $n$ with 
component $p_i^n$ to describe proportion of string $i$ in finite population, infinite population model 
can be used to predict proportion $p_i^{n+1}$ of string $i$ as result of \textit{selection} and 
\textit{mixing} in next generation finite population $p^{n+1}$. Vose and Liepins (see \cite{VoseLiepins1991}) 
modeled simple GA by computing expected population trajectories through time based on infinite population. 
Vose and Liepins, if $r_{i,j}(k)$ is probability of recombination $i$ and $j$ to produce $k$, and $s_i^t$ and $s_j^t$ 
are probability of selection of $i$ and $j$ as parents, gives expected proporiton of $k$ in next generation as
\[
\mathcal{E}(p_k^{t+1}) = \sum_{i,j} s_i^t s_j^t r_{i,j}(k) ; \hspace{1pt} \mahtcal{E} denotes expectation
\]
If $M$ is recombination matrix with elements $m_{i,j} = r_{i,j}(0)$ and permutation $\sigma_j$ defined as 
\[
\sigma_j{\langle S_0,..,S_{2^\ell - 1} \rangle}^{T} = {\langle S_{0+j},..,S_{(2^\ell - 1)+j} \rangle}^{T}
\]
where $T$ is transpose and $\ell$ is bit length of binary string, Vose and Liepins represented expected proportion 
of $k$ in next generation in using recombination or mixing matrix $M$ as
\[
\mathcal{E}(p_k^{t+1}) = (\sigma_k s)^T M (\sigma_k s)
\]
Vose and Liepins apply Walsh transform to mixing to analyse and get results. 

There had been previous applications of Walsh transform in field of GA. Bethke, in his dissertation 
"Genetic Algorithms as Function Optimizers" (see \cite{Bethke1981}), first introduced 
idea of using Walsh transforms to analyse process of GA in case of binary-coded strings. The idea of using Walsh transforms 
were given greater incitement in papers by Goldberg (see \cite{Goldberg1989a}, \cite{Goldberg1989b}). But usage of 
Walsh transforms does not involve the direct application of the Walsh transform to crossover and mutation, or to any of their 
associated mathematical objects. Vose and Liepins use Walsh transform directly to mutation and recombination, and show that the 
twist (denoted by $M_*$) of the mixing matrix ($M$) is triangularized by the Walsh transform and used $M_*$ in study of fixed points where $(M_*)_{i,j} = M_{i+j, i}$. In a related paper, Koehler (see \cite{Koehler1995}) gives a congruence 
transformation defined by lower triangular matrix that diagonalizes the mixing matrix for 1-point crossover and mutation given by a rate and mathecally proved conjecture provided by Vose and Liepins on eigenvalues of matrix $M_*$. Koehler, Bhattacharyya and Vose (see \cite{KoehlerBhatta1998}) applied Fourier transform to mixing in generalizing results concerning simple genetic algorithm which were previously established for binary case (in binary case, Fourier transform is Walsh transform) extending analysis to strings over an alphbet of cardinality $c$. Vose and Wright (see \cite{VoseWright1998}) applied Walsh transform to mixing matrix and simplified the matrix from dense to sparse in Walsh basis giving advantage of computational efficiency from $O(n^3)$ to $O(n^{log_2 3})$. The cost of conversion of standard coordinates to Walsh basis need not be sustained since fast Walsh transform (see \cite{Shanks1969}) can do that in $O{n log_2 n}$ time.


With Markov chain modeling of simple GA, Nix and Vose (see \cite{Nix1992}) investigated asymptotic behavior of steady 
state distributions as population size increases and showed that if finite population is sufficiently large, 
convergence behavior of a real GA can be predicted. As population size increases, the correspondence improves 
between expected population predicted using infinite population model and the actual population observed in 
finite population genetic algorithm.  Vose calculated variance of expected next generation population and 
actual next generation population in his book Simple Genetic Algorithm (\cite{Vose1999}) under topic 
Random Heuristic Search which is discussed later in this chapter. 

