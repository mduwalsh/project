\documentclass[aspectratio=169]{beamer}
\usepackage{enumitem}
\usepackage{bm}
\usepackage{float}
\usepackage{subfig}
\usepackage{caption}
\usepackage{amsfonts}
\usepackage{url}
% \usepackage[config, labelfont={bf}]{subfloat, subfig} % nice sub figures
\setbeamertemplate{footline}[frame number]

\newcommand{\nudge}{\hspace{0.01in}}

\title{Efficient Simulation Of A Simple Evolutionary System}
\author{Mahendra Duwal Shrestha}
\institute{The University Of Tenessee}
\date{\today}


\begin{document}

  \begin{frame}
    \titlepage
  \end{frame}

  \begin{frame}
    \frametitle{Outline}
    \begin{itemize}
      \item{Background}
      \item{Question 1: Distance between finite and infinite population}
      \item{Question 2: Oscillation in finite population}
      \item{Question 3: Oscillation in finite population under violation in mutation}
      \item{Question 4: Oscillation in finite population under violation in crossover}
      \item{Conclusion}
    \end{itemize}
  \end{frame}


  \begin{frame}
    \frametitle{}
    \begin{itemize}
      \item{}      
    \end{itemize}
  \end{frame}
  
    
  \begin{frame}
    \frametitle{History}
    \begin{itemize}
      \item{Haldane, in 1932, summarized basic population genetics  models : Wright, Fisher and Haldane}
      \item{Several people working with evolution-inspired algorithms in the 1950s and the 1960s –  
      Box (1957), Friedman(1959), Bledsoe (1961), Bremermann (1962), and Reed, Toombs and Baricelli (1967) }
      \item{In 1960s and 1970s, Holland and colleagues formalized  and promoted population based algorithms with crossover and mutation }
      \item{Vose (1999) presented efficient methods for computing with a haploid model using mask-based operators introduced by Geiringer (1944)}
    \end{itemize}
  \end{frame}
  
\end{document}