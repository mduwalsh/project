\chapter{Violation} \label{ch:evolutionary limits}
% \section{Violation}
% Previous results show that oscillation occurs when the crossover distribution $\bm{\chi}$ ,and the mutation distribution $\bm{\mu}$ 
% satisfy (\ref{OscCond}). Error $\bm{\epsilon}$ was introduced to $\bm{\mu}$ and $\bm{\chi}$ distributions to 
% violate condition (\ref{OscCond}). Consequently, $\bm{p}^\ast \;=\; \bm{q}^\ast \;=\; \bm{z}^\ast$. 
% Going forward, we use 'limit $\bm{z}^\ast$' to denote evolutionary limit when crossover distribution 
% $\bm{\chi}$ or mutation distribution $\bm{\mu}$ is violated, and 
% 'non-violation limits $\bm{p}^\ast$ and $\bm{q}^\ast$' to denote limits without violation.
% 
% \subsection{Violation in Mutation Distribution}
% The mutation distribution $\bm{\mu}$ was modified as follows
% \[
% \bm{\mu}_i = (1-\bm{\epsilon}) \bm{\mu}_i \nudge; \tabspace i = \{0, 1, 2,.., 2^{\ell}-1\}.
% \]
% So that sum of $\bm{\mu}$ distribution becomes, 
% \[
% 1-\bm{\epsilon} = \sum \limits_{i=0}^{2^{\ell}-1} \bm{\mu}_i
% \]
% Then set
% \[
% \bm{\mu}_0 = \bm{\epsilon}
% \]
% % \[
% % \bm{\mu}_0 = (1-\bm{\epsilon})\bm{\epsilon}
% % \]
% % $c$ is total number components in $\bm{\mu}$ satisfying condition $\bm{\mu}_i = 0$ and set those components value as
% % \[
% % \bm{\mu}_i = \bm{\epsilon}^2/c \; ; \; where \; \bm{\mu}_i = 0
% % \]
% The mutation distribution $\bm{\mu}$ is normalized such that  $\sum\limits_{i=0}^{2^{\ell}-1}\bm{\mu}_i \;=\; 1$.
% This modification makes possible for any population member to mutate to any other possible population member via mutation. 
% Let us exlore for two cases of $g$ in \ref{OscCond}:
% 
% 1. When $g$ is all $1$s:\newline
% Any mask $j$ with only one $1$ at any position $k$ ($0 \leq k < \ell$) can mutate $k$th bit in string, and since mask 
% all $0$s in has positive probability, strings have option to not mutate. This gives possibility for any string to mutate to 
% any possible string in population. Let us take an example with $\ell \;=\; 8$. Let $g \;=\; 11111111$. Then, mask 
% $i \;=\; 00000100$ will have positive probability according to condition \ref{OscCond} implementation. 
% Mask $i$ can be used to mutate bit of a population string at position $k \;=\; 5$ from left.
% 
% 2. When $g$ has atleast one $0$:\newline
% Any mask $j$ with only one $1$ at any position $k$ ($0 \leq k < \ell$) where $k$th position in $g$ also has $1$ in it 
% will have positive probability. So any string can mutate that particular $k$th bit using those masks. And any mask 
% with $1$ in any one of the position where $g$ has one in it and $1$ in any position where $g$ has $0$s in it can be used to 
% mutate bits in string where $g$ has $0$s in it. Let us take an example with $\ell \;=\; 8$. Let $g \;=\; 11001111$. Then, 
% mask $i \;=\; 00000100$ will have positive probability according to condition \ref{OscCond} implementation. Then, also mask 
% $j \;=\; 00010100$ will have positive probability. Mask $i$ can be used to mutate string at position $k \;=\; 5$. 
% To mutate bit of any string at position $k \;=\; 3$, string can be mutated using mask $i$ and mask $j$ in sequence.
% 
% Population in state $P$ can mutate to another state $P\prime$ in next generation. Then it can mutate back to original state $P$ or 
% stay in its state $P\prime$ or move to another state $P\prime\prime$ in next generation. Since population state can mutate to 
% any state, and any state can be returned to in any period of time, the Markov chain formed by GA is irreducible and aperiodic and hence, 
% ergodic (see \cite{MarkovChain}). Because of ergodic Markov chain, steady state distribution with positive value components exists for the GA.   
% 
% Simulations were repeated with the violations in (\ref{OscCond}) described above for mutation distribution.
% The distances of both infinite and finite populations to limit $\bm{z}^\ast$ were plotted. 
% The distances of both infinite and finite populations to non-violation limits $\bm{p}^\ast$ and $\bm{q}^\ast$ were also plotted.
% 
% % figures for mu violation
% \input{chapters/violation-mu-hap.tex}
% The left column of figures \ref{oscillation_8h_vio_mu_0.01} through \ref{oscillation_14h_vio_mu_0.5} 
% shows distance of finite and infinite haploid populations to limit $\bm{z^\ast}$ (limit with violation in mutation distribution $\bm{\mu}$). The 
% right column shows distance of finite and infinite haploid populations to non-violation limits $\bm{p^\ast}$ and $\bm{q^\ast}$. 
% Graphs in right column give picture of oscillating behavior of haploid population given violation. 
% Graphs in left column in figures \ref{oscillation_8h_vio_mu_0.01} through \ref{oscillation_14h_vio_mu_0.5} show distance 
% between finite population and limit $\bm{z}^\ast$ (limit with violation in $\bm{\mu}$ distribution) for haploid case.
% 
% Both finite and infinite populations oscillate given violation. However, ripples die out quickly, 
% and oscillation ceases for infinite populations.
% In the finite population case, although ripple amplitudes decrease, oscillation didn't die out completely. 
% 
% Change in oscillating behavior of population with change in $\bm{\epsilon}$ values $\{0.01, 0.1, 0.5\}$ were also studied. 
% Results show ripples damp out faster with increase $\bm{\epsilon}$.
% With smaller values of $\bm{\epsilon}$, oscillations were sharper and as value of $\bm{\epsilon}$ increased, 
% rate of damping increased. With $\bm{\epsilon} = 0.01$, oscillation is clearly visible and ripples are sharper; with $\bm{\epsilon} = 0.1$ oscillation was visible 
% but ripples were damping out quickly;  with $\bm{\epsilon} = 0.5$, oscillation was very minimal or not visible at all. 
% 
% The distance between finite population and limit $\bm{z}^\ast$ (limit with violation in $\bm{\mu}$ distribution) 
% decreases as finite population size increases, 
% and finite population shows behavior similar to infinite population behavior as finite population reach large number. 
% Simulation results show infinite population converges 
% to limit $\bm{z^\ast}$ quicker with increase in $\bm{\epsilon}$. 
% 
% The distance data for haploid population in case of violation in $\bm{\mu}$ distribution 
% with different values of $\bm{\epsilon}$ for different finite population size $N$ are tabulated in table \ref{distanceMuHap}.
% % figures for mu violation
% % \begin{table}[ht]
% \caption{\textbf{Experimental distance measured for violation in $\bm{\mu}$:} $\ell$ is genome length, 
% $\epsilon$ is error introduced to $\bm{\mu}$ for violation, and average distance between finite and 
% infinite populations is tabulated in last three columns.}
% \centering
% \begin{tabularx}{0.75\textwidth}{ c *{5}{X}}
% \toprule
% $\epsilon$ & case & $\ell$ & $N = 4096$ & $N = 40960$ & $N = 81920$ \\
% \midrule
% \multirow{8}{*}{0.01} & \multirow{4}{*}{haploid} 	& 8 & 0.0176	& 0.0094	& 0.0093 \\
% 		      &   				& 10 & 0.0168	& 0.0088 	& 0.0077 \\ 
% 		      & 			 	& 12 & 0.0161	& 0.0064 	& 0.0053 \\
% 		      &  	 			& 14 & 0.0157	& 0.0051 	& 0.0038 \\ 
% 		      \cmidrule(l){2-6}
% 		      & \multirow{4}{*}{diploid} 	& 8 & 0.0156	&  0.0050	& 0.0035 \\
% 		      &  	 			& 10 & 0.0156	&  0.0049	& 0.0035 \\
% 		      & 			 	& 12 & 0.0156	&  0.0049	& 0.0035 \\
% 		      &  	 			& 14 & 0.0156	&  0.0049	& 0.0035 \\
% 		      \midrule
% \multirow{8}{*}{0.1} & \multirow{4}{*}{haploid} 	& 8 & 0.0158	& 0.0054 	& 0.0041 \\
% 		      &   				& 10 & 0.0158	& 0.0053 	& 0.0039 \\	
% 		      & 				& 12 & 0.0157	& 0.0051 	& 0.0036 \\	
% 		      &  	 			& 14 & 0.0156	& 0.0050 	& 0.0035 \\
% 		      \cmidrule(l){2-6}
% 		      & \multirow{4}{*}{diploid} 	& 8 & 0.0156	&  0.0049	& 0.0035 \\
% 		      &  	 			& 10 & 0.0156	&  0.0049	& 0.0035 \\
% 		      & 				& 12 & 0.0156	&  0.0049	& 0.0035 \\
% 		      &  	 			& 14 & 0.0156	&  0.0049	& 0.0035 \\
% 		      \midrule
% \multirow{8}{*}{0.5} & \multirow{4}{*}{haploid} 	& 8 & 0.0161	&  0.0056	& 0.0042 \\
% 		      &   				& 10 & 0.0161	&  0.0055	& 0.0040 \\
% 		      & 				& 12 & 0.0157	&  0.0051	& 0.0036 \\
% 		      &  	 			& 14 & 0.0157	&  0.0051	& 0.0037 \\
% 		      \cmidrule(l){2-6}
% 		      & \multirow{4}{*}{diploid} 	& 8 & 0.0156	&  0.0049	& 0.0035 \\	
% 		      &  	 			& 10 & 0.0156	&  0.0049 	& 0.0035 \\
% 		      &  				& 12 & 0.0156	&  0.0049	& 0.0035 \\
% 		      &  	 			& 14 & 0.0156	&  0.0049	& 0.0035 \\
% \bottomrule
% \end{tabularx}
% \label{distanceMu}
% \end{table}

\begin{table}[ht]
\caption{\textbf{Experimental distance measured for violation in $\bm{\mu}$ for haploids:} $\ell$ is genome length, 
$\epsilon$ is error introduced to $\bm{\mu}$ for violation, and average distance between finite and 
infinite populations is tabulated in last three columns.}
\centering
\begin{tabularx}{0.75\textwidth}{ c *{5}{X}}
\toprule
$\epsilon$ & case & $\ell$ & $N = 4096$ & $N = 40960$ & $N = 81920$ \\
\midrule
\multirow{4}{*}{0.01} & \multirow{4}{*}{haploid} 	& 8 & 0.0176	& 0.0094	& 0.0093 \\
		      &   				& 10 & 0.0168	& 0.0088 	& 0.0077 \\ 
		      & 			 	& 12 & 0.0161	& 0.0064 	& 0.0053 \\
		      &  	 			& 14 & 0.0157	& 0.0051 	& 0.0038 \\ 
		      \midrule
\multirow{4}{*}{0.1} & \multirow{4}{*}{haploid} 	& 8 & 0.0158	& 0.0054 	& 0.0041 \\
		      &   				& 10 & 0.0158	& 0.0053 	& 0.0039 \\	
		      & 				& 12 & 0.0157	& 0.0051 	& 0.0036 \\	
		      &  	 			& 14 & 0.0156	& 0.0050 	& 0.0035 \\
		      \midrule
\multirow{4}{*}{0.5} & \multirow{4}{*}{haploid} 	& 8 & 0.0161	&  0.0056	& 0.0042 \\
		      &   				& 10 & 0.0161	&  0.0055	& 0.0040 \\
		      & 				& 12 & 0.0157	&  0.0051	& 0.0036 \\
		      &  	 			& 14 & 0.0157	&  0.0051	& 0.0037 \\
\bottomrule
\end{tabularx}
\label{distanceMuHap}
\end{table}


% From table \ref{distanceMuHap}, average distance calculated for finite population size $4096$ is $0.0158$, 
% for size $40960$ is $0.0055$ and for size $81920$ is $0.0041$. These results show average distance 
% between finite population and limit $\bm{z^\ast}$ closely follows expected single step distance 
% between finite and infinite population given in \ref{tableExpectedDistance}. The distance decreased as $1/\sqrt{N}$.
% 
% \input{chapters/violation-mu-dip.tex}
% The left column of figures \ref{oscillation_8d_vio_mu_0.01} through \ref{oscillation_14d_vio_mu_0.5} 
% shows distance of finite and infinite diploid populations to limit $\bm{z^\ast}$ (limit with violation in $\bm{\mu}$ distribution). The 
% right column shows distance of finite and infinite diploid populations to non-violation limits $\bm{p^\ast}$ and $\bm{q^\ast}$. 
% Graphs in right column give picture of oscillating behavior of diploid population given violation. 
% Graphs in left column in figures \ref{oscillation_8d_vio_mu_0.01} through \ref{oscillation_14d_vio_mu_0.5} show distance 
% between finite population and limit $\bm{z}^\ast$ (limit with violation in $\bm{\mu}$ distribution) for diploid case.
% 
% Like in haploid case, both finite and infinite populations oscillate given violation. Ripples die out quickly, 
% and oscillation ceases for infinite populations, and in the finite population case, although ripple amplitudes decrease, 
% oscillation didn't die out completely. 
% 
% Change in oscillating behavior of population with change in $\bm{\epsilon}$ values $\{0.01, 0.1, 0.5\}$ were also studied. 
% Results show ripples damp out faster with increase $\bm{\epsilon}$.
% With smaller values of $\bm{\epsilon}$, oscillations were sharper and as value of $\bm{\epsilon}$ increased, 
% rate of damping increased. With $\bm{\epsilon} = 0.01$, oscillation is clearly visible and ripples are sharper; with $\bm{\epsilon} = 0.1$ oscillation was visible 
% but ripples were damping out quickly;  with $\bm{\epsilon} = 0.5$, oscillation was very minimal or not visible at all. 
% 
% Like just in haploid case, the distance between finite population and limit $\bm{z}^\ast$ (limit with violation in $\bm{\mu}$ distribution) 
% decreases as finite population size increases, 
% and finite population shows behavior similar to infinite population behavior as finite population reach large number. 
% Simulation results show infinite population converges 
% to limit $\bm{z^\ast}$ quicker with increase in $\bm{\epsilon}$. 
% 
% The distance data for diploid population in case of violation in $\bm{\mu}$ distribution 
% with different values of $\bm{\epsilon}$ for different finite population size $N$ are tabulated in table \ref{distanceMuDip}.
% 
% % figures for mu violation
% 
\begin{table}[ht]
\caption{\textbf{Experimental distance measured for violation in $\bm{\mu}$ for diploids:} $\ell$ is genome length, 
$\epsilon$ is error introduced to $\bm{\mu}$ for violation, and average distance between finite and 
infinite populations is tabulated in last three columns.}
\centering
\begin{tabularx}{0.75\textwidth}{ c *{5}{X}}
\toprule
$\epsilon$ & case & $\ell$ & $N = 4096$ & $N = 40960$ & $N = 81920$ \\
\midrule
\multirow{4}{*}{0.01} & \multirow{4}{*}{diploid} 	& 8 & 0.0156	&  0.0050	& 0.0035 \\
		      &  	 			& 10 & 0.0156	&  0.0049	& 0.0035 \\
		      & 			 	& 12 & 0.0156	&  0.0049	& 0.0035 \\
		      &  	 			& 14 & 0.0156	&  0.0049	& 0.0035 \\
		      \midrule
\multirow{4}{*}{0.1}  & \multirow{4}{*}{diploid} 	& 8 & 0.0156	&  0.0049	& 0.0035 \\
		      &  	 			& 10 & 0.0156	&  0.0049	& 0.0035 \\
		      & 				& 12 & 0.0156	&  0.0049	& 0.0035 \\
		      &  	 			& 14 & 0.0156	&  0.0049	& 0.0035 \\
		      \midrule
\multirow{4}{*}{0.5}  & \multirow{4}{*}{diploid} 	& 8 & 0.0156	&  0.0049	& 0.0035 \\	
		      &  	 			& 10 & 0.0156	&  0.0049 	& 0.0035 \\
		      &  				& 12 & 0.0156	&  0.0049	& 0.0035 \\
		      &  	 			& 14 & 0.0156	&  0.0049	& 0.0035 \\
\bottomrule
\end{tabularx}
\label{distanceMuDip}
\end{table}
% 
% From table \ref{distanceMuDip}, average distance calculated for finite population size $4096$ is $0.0158$, 
% for size $40960$ is $0.0055$ and for size $81920$ is $0.0041$. These results show average distance 
% between finite population and limit $\bm{z^\ast}$ closely follows expected single step distance 
% between finite and infinite population given in \ref{tableExpectedDistance}. The distance decreased as $1/\sqrt{N}$.
% 
% Violation introduced to the mutation distribution $\bm{\mu}$ creates new masks (different than in case of without violation) to be used in mutation 
% during transmission. With small $\bm{\epsilon}$, the probablility of using the new masks available due to violation is very small and 
% those masks might not be used at all during mutation in finite population, but with higher values of $\bm{\epsilon}$, 
% those new masks have higher chance of usage during mutation which cause oscillation to damp out quickly or cause no oscillation at all.
% 
% From graphs on the right in 
% figures \ref{oscillation_8h_vio_mu_0.01} through \ref{oscillation_14h_vio_mu_0.5}, 
% and \ref{oscillation_8d_vio_mu_0.01} through \ref{oscillation_14d_vio_mu_0.5}, haploid population shows sharper and 
% more distinct oscillations for same size of population size than diploid population; diploid population 
% requires larger population size to exhibit sharper oscillations. 
% And from graphs on the left in figures \ref{oscillation_8h_vio_mu_0.01} through \ref{oscillation_14h_vio_mu_0.5}, and \ref{oscillation_8d_vio_mu_0.01} through \ref{oscillation_14d_vio_mu_0.5}, diploid population converges to limit $\bm{z}^\ast$ more quickly than haploid population. 
% 
% \subsection{Violation in Crossover Distribution}
% The crossover distribution $\bm{\chi}$ was modified as
% \[
% \bm{\chi}_i = (1-\bm{\epsilon}) \bm{\chi} ; \tabspace i = \{1, 2,.., 2^{\ell}-1\} 
% \]
% So that 
% \[
% \bm{\chi}_i + \bm{\chi}_{i+g} = 1-\bm{\epsilon} ; \tabspace g \text{ is defined in  section } \ref{Limits}
% \]
% 
% Then $j$ is chosen where $\bm{\chi}_j = 0$ and set $\bm{\chi}_j = \bm{\epsilon}$. 
% 
% Simulations were repeated with the violations in (\ref{OscCond}) described above for crossover distribution.
% The distances of both infinite and finite populations to limit $\bm{z}^\ast$ were plotted. 
% The distances of both infinite and finite populations to non-violation limits $\bm{p}^\ast$ and $\bm{q}^\ast$ were also plotted.
% 
% % figures for chi violation
% 
% l = 8

\begin{figure}[H]
\begin{center}
\subfloat{
\resizebox{8cm}{5cm}{\includegraphics{figures/eps/vio/chi/b8/e0.01/n00004096_fin_hap.eps}}} \hspace{-3em}%
\subfloat{
\resizebox{8cm}{5cm}{\includegraphics{figures/eps/vio/chi/b8/e0.01/n00004096_fin_hap_wovio.eps}}}\vspace{-1em} \hspace{-3em}%
\end{center}
\begin{center}
\subfloat{
\resizebox{8cm}{5cm}{\includegraphics{figures/eps/vio/chi/b8/e0.01/n00040960_fin_hap.eps}}} \hspace{-3em}%
\subfloat{
\resizebox{8cm}{5cm}{\includegraphics{figures/eps/vio/chi/b8/e0.01/n00040960_fin_hap_wovio.eps}}}\vspace{-1em} \hspace{-3em}%
\end{center}

\begin{center}
\subfloat{
\resizebox{8cm}{5cm}{\includegraphics{figures/eps/vio/chi/b8/e0.01/n00081920_fin_hap.eps}}} \hspace{-3em}%
\subfloat{
\resizebox{8cm}{5cm}{\includegraphics{figures/eps/vio/chi/b8/e0.01/n00081920_fin_hap_wovio.eps}}}\vspace{-1em} \hspace{-3em}%
\end{center}

\begin{center}
\subfloat{
\resizebox{8cm}{5cm}{\includegraphics{figures/eps/vio/chi/b8/e0.01/inf_hap.eps}}}\hspace{-3em}%
\subfloat{
\resizebox{8cm}{5cm}{\includegraphics{figures/eps/vio/chi/b8/e0.01/inf_hap_wovio.eps}}}\vspace{-0.5em} \hspace{-3em}%


\caption{\textbf{Infinite and finite haploid population oscillation behavior in case of violation in $\bm{\chi}$ for genome length $\ell = 8$ and $\epsilon = 0.01$:} 
  In left column, $d$ is distance of finite population of size $n$ or infinite population to limit for $g$ generations. In right column, $d$ is distance of finite population of size $N$ or infinite population to limits without violation.}
\label{oscillation_8h_vio_chi_0.01}
\end{center}
\end{figure}


\begin{figure}[H]
\begin{center}
\subfloat{
\resizebox{8cm}{5cm}{\includegraphics{figures/eps/vio/chi/b8/e0.1/n00004096_fin_hap.eps}}}  \hspace{-3em}%
\subfloat{
\resizebox{8cm}{5cm}{\includegraphics{figures/eps/vio/chi/b8/e0.1/n00004096_fin_hap_wovio.eps}}}\vspace{-1em}  \hspace{-3em}%
\end{center}
\begin{center}
\subfloat{
\resizebox{8cm}{5cm}{\includegraphics{figures/eps/vio/chi/b8/e0.1/n00040960_fin_hap.eps}}}  \hspace{-3em}%
\subfloat{
\resizebox{8cm}{5cm}{\includegraphics{figures/eps/vio/chi/b8/e0.1/n00040960_fin_hap_wovio.eps}}}\vspace{-1em}  \hspace{-3em}%
\end{center}

\begin{center}
\subfloat{
\resizebox{8cm}{5cm}{\includegraphics{figures/eps/vio/chi/b8/e0.1/n00081920_fin_hap.eps}}}  \hspace{-3em}%
\subfloat{
\resizebox{8cm}{5cm}{\includegraphics{figures/eps/vio/chi/b8/e0.1/n00081920_fin_hap_wovio.eps}}}\vspace{-1em}  \hspace{-3em}%
\end{center}

\begin{center}
\subfloat{
\resizebox{8cm}{5cm}{\includegraphics{figures/eps/vio/chi/b8/e0.1/inf_hap.eps}}} \hspace{-3em}%
\subfloat{
\resizebox{8cm}{5cm}{\includegraphics{figures/eps/vio/chi/b8/e0.1/inf_hap_wovio.eps}}}\vspace{-0.5em}  \hspace{-3em}%


\caption{\textbf{Infinite and finite haploid population oscillation behavior in case of violation in $\bm{\chi}$ for genome length $\ell = 8$ and $\epsilon = 0.1$:} 
  In left column, $d$ is distance of finite population of size $n$ or infinite population to limit for $g$ generations. In right column, $d$ is distance of finite population of size $N$ or infinite population to limits without violation.}
\label{oscillation_8h_vio_chi_0.1}
\end{center}
\end{figure}

\begin{figure}[H]

\begin{center}
\subfloat{
\resizebox{8cm}{5cm}{\includegraphics{figures/eps/vio/chi/b8/e0.5/n00004096_fin_hap.eps}}}  \hspace{-3em}%
\subfloat{
\resizebox{8cm}{5cm}{\includegraphics{figures/eps/vio/chi/b8/e0.5/n00004096_fin_hap_wovio.eps}}}\vspace{-1em}  \hspace{-3em}%
\end{center}
\begin{center}
\subfloat{
\resizebox{8cm}{5cm}{\includegraphics{figures/eps/vio/chi/b8/e0.5/n00040960_fin_hap.eps}}}  \hspace{-3em}%
\subfloat{
\resizebox{8cm}{5cm}{\includegraphics{figures/eps/vio/chi/b8/e0.5/n00040960_fin_hap_wovio.eps}}}\vspace{-1em}  \hspace{-3em}%
\end{center}

\begin{center}
\subfloat{
\resizebox{8cm}{5cm}{\includegraphics{figures/eps/vio/chi/b8/e0.5/n00081920_fin_hap.eps}}}  \hspace{-3em}%
\subfloat{
\resizebox{8cm}{5cm}{\includegraphics{figures/eps/vio/chi/b8/e0.5/n00081920_fin_hap_wovio.eps}}}\vspace{-1em}  \hspace{-3em}%
\end{center}

\begin{center}
\subfloat{
\resizebox{8cm}{5cm}{\includegraphics{figures/eps/vio/chi/b8/e0.5/inf_hap.eps}}} \hspace{-3em}%
\subfloat{
\resizebox{8cm}{5cm}{\includegraphics{figures/eps/vio/chi/b8/e0.5/inf_hap_wovio.eps}}}\vspace{-0.5em}  \hspace{-3em}%


\caption{\textbf{Infinite and finite haploid population oscillation behavior in case of violation in $\bm{\chi}$ for 
  genome length $\ell = 8$ and $\epsilon = 0.5$:} In left column, $d$ is distance of finite population of size $n$ or infinite population to limit for $g$ generations. In right column, $d$ is distance of finite population of size $N$ or infinite population to limits without violation.}
\label{oscillation_8h_vio_chi_0.5}
\end{center}
\end{figure}


% l = 10

\begin{figure}[H]
\begin{center}
\subfloat{
\resizebox{8cm}{5cm}{\includegraphics{figures/eps/vio/chi/b10/e0.01/n00004096_fin_hap.eps}}} \hspace{-3em}%
\subfloat{
\resizebox{8cm}{5cm}{\includegraphics{figures/eps/vio/chi/b10/e0.01/n00004096_fin_hap_wovio.eps}}}\vspace{-1em} \hspace{-3em}%
\end{center}
\begin{center}
\subfloat{
\resizebox{8cm}{5cm}{\includegraphics{figures/eps/vio/chi/b10/e0.01/n00040960_fin_hap.eps}}} \hspace{-3em}%
\subfloat{
\resizebox{8cm}{5cm}{\includegraphics{figures/eps/vio/chi/b10/e0.01/n00040960_fin_hap_wovio.eps}}}\vspace{-1em} \hspace{-3em}%
\end{center}

\begin{center}
\subfloat{
\resizebox{8cm}{5cm}{\includegraphics{figures/eps/vio/chi/b10/e0.01/n00081920_fin_hap.eps}}} \hspace{-3em}%
\subfloat{
\resizebox{8cm}{5cm}{\includegraphics{figures/eps/vio/chi/b10/e0.01/n00081920_fin_hap_wovio.eps}}}\vspace{-1em} \hspace{-3em}%
\end{center}

\begin{center}
\subfloat{
\resizebox{8cm}{5cm}{\includegraphics{figures/eps/vio/chi/b10/e0.01/inf_hap.eps}}}\hspace{-3em}%
\subfloat{
\resizebox{8cm}{5cm}{\includegraphics{figures/eps/vio/chi/b10/e0.01/inf_hap_wovio.eps}}}\vspace{-0.5em} \hspace{-3em}%


\caption{\textbf{Infinite and finite haploid population oscillation behavior in case of violation in $\bm{\chi}$ for genome length $\ell = 10$ and $\epsilon = 0.01$:} 
  In left column, $d$ is distance of finite population of size $n$ or infinite population to limit for $g$ generations. In right column, $d$ is distance of finite population of size $N$ or infinite population to limits without violation.}
\label{oscillation_10h_vio_chi_0.01}
\end{center}
\end{figure}


\begin{figure}[H]
\begin{center}
\subfloat{
\resizebox{8cm}{5cm}{\includegraphics{figures/eps/vio/chi/b10/e0.1/n00004096_fin_hap.eps}}}  \hspace{-3em}%
\subfloat{
\resizebox{8cm}{5cm}{\includegraphics{figures/eps/vio/chi/b10/e0.1/n00004096_fin_hap_wovio.eps}}}\vspace{-1em}  \hspace{-3em}%
\end{center}
\begin{center}
\subfloat{
\resizebox{8cm}{5cm}{\includegraphics{figures/eps/vio/chi/b10/e0.1/n00040960_fin_hap.eps}}}  \hspace{-3em}%
\subfloat{
\resizebox{8cm}{5cm}{\includegraphics{figures/eps/vio/chi/b10/e0.1/n00040960_fin_hap_wovio.eps}}}\vspace{-1em}  \hspace{-3em}%
\end{center}

\begin{center}
\subfloat{
\resizebox{8cm}{5cm}{\includegraphics{figures/eps/vio/chi/b10/e0.1/n00081920_fin_hap.eps}}}  \hspace{-3em}%
\subfloat{
\resizebox{8cm}{5cm}{\includegraphics{figures/eps/vio/chi/b10/e0.1/n00081920_fin_hap_wovio.eps}}}\vspace{-1em}  \hspace{-3em}%
\end{center}

\begin{center}
\subfloat{
\resizebox{8cm}{5cm}{\includegraphics{figures/eps/vio/chi/b10/e0.1/inf_hap.eps}}} \hspace{-3em}%
\subfloat{
\resizebox{8cm}{5cm}{\includegraphics{figures/eps/vio/chi/b10/e0.1/inf_hap_wovio.eps}}}\vspace{-0.5em}  \hspace{-3em}%


\caption{\textbf{Infinite and finite haploid population oscillation behavior in case of violation in $\bm{\chi}$ for genome length $\ell = 10$ and $\epsilon = 0.1$:} 
  In left column, $d$ is distance of finite population of size $n$ or infinite population to limit for $g$ generations. In right column, $d$ is distance of finite population of size $N$ or infinite population to limits without violation.}
\label{oscillation_10h_vio_chi_0.1}
\end{center}
\end{figure}


\begin{figure}[H]

\begin{center}
\subfloat{
\resizebox{8cm}{5cm}{\includegraphics{figures/eps/vio/chi/b10/e0.5/n00004096_fin_hap.eps}}}  \hspace{-3em}%
\subfloat{
\resizebox{8cm}{5cm}{\includegraphics{figures/eps/vio/chi/b10/e0.5/n00004096_fin_hap_wovio.eps}}}\vspace{-1em}  \hspace{-3em}%
\end{center}
\begin{center}
\subfloat{
\resizebox{8cm}{5cm}{\includegraphics{figures/eps/vio/chi/b10/e0.5/n00040960_fin_hap.eps}}}  \hspace{-3em}%
\subfloat{
\resizebox{8cm}{5cm}{\includegraphics{figures/eps/vio/chi/b10/e0.5/n00040960_fin_hap_wovio.eps}}}\vspace{-1em}  \hspace{-3em}%
\end{center}

\begin{center}
\subfloat{
\resizebox{8cm}{5cm}{\includegraphics{figures/eps/vio/chi/b10/e0.5/n00081920_fin_hap.eps}}}  \hspace{-3em}%
\subfloat{
\resizebox{8cm}{5cm}{\includegraphics{figures/eps/vio/chi/b10/e0.5/n00081920_fin_hap_wovio.eps}}}\vspace{-1em}  \hspace{-3em}%
\end{center}

\begin{center}
\subfloat{
\resizebox{8cm}{5cm}{\includegraphics{figures/eps/vio/chi/b10/e0.5/inf_hap.eps}}} \hspace{-3em}%
\subfloat{
\resizebox{8cm}{5cm}{\includegraphics{figures/eps/vio/chi/b10/e0.5/inf_hap_wovio.eps}}}\vspace{-0.5em}  \hspace{-3em}%


\caption{\textbf{Infinite and finite haploid population oscillation behavior in case of violation in $\bm{\chi}$ for 
  genome length $\ell = 10$ and $\epsilon = 0.5$:} In left column, $d$ is distance of finite population of size $n$ or infinite population to limit for $g$ generations. In right column, $d$ is distance of finite population of size $N$ or infinite population to limits without violation.}
\label{oscillation_10h_vio_chi_0.5}
\end{center}
\end{figure}


\begin{figure}[H]
\begin{center}
\subfloat{
\resizebox{8cm}{5cm}{\includegraphics{figures/eps/vio/chi/b12/e0.01/n00004096_fin_hap.eps}}} \hspace{-3em}%
\subfloat{
\resizebox{8cm}{5cm}{\includegraphics{figures/eps/vio/chi/b12/e0.01/n00004096_fin_hap_wovio.eps}}}\vspace{-1em} \hspace{-3em}%
\end{center}
\begin{center}
\subfloat{
\resizebox{8cm}{5cm}{\includegraphics{figures/eps/vio/chi/b12/e0.01/n00040960_fin_hap.eps}}} \hspace{-3em}%
\subfloat{
\resizebox{8cm}{5cm}{\includegraphics{figures/eps/vio/chi/b12/e0.01/n00040960_fin_hap_wovio.eps}}}\vspace{-1em} \hspace{-3em}%
\end{center}

\begin{center}
\subfloat{
\resizebox{8cm}{5cm}{\includegraphics{figures/eps/vio/chi/b12/e0.01/n00081920_fin_hap.eps}}} \hspace{-3em}%
\subfloat{
\resizebox{8cm}{5cm}{\includegraphics{figures/eps/vio/chi/b12/e0.01/n00081920_fin_hap_wovio.eps}}}\vspace{-1em} \hspace{-3em}%
\end{center}

\begin{center}
\subfloat{
\resizebox{8cm}{5cm}{\includegraphics{figures/eps/vio/chi/b12/e0.01/inf_hap.eps}}}\hspace{-3em}%
\subfloat{
\resizebox{8cm}{5cm}{\includegraphics{figures/eps/vio/chi/b12/e0.01/inf_hap_wovio.eps}}}\vspace{-0.5em} \hspace{-3em}%


\caption{\textbf{Infinite and finite haploid population oscillation behavior in case of violation in $\bm{\chi}$ for genome length $\ell = 12$ and $\epsilon = 0.01$:} 
  In left column, $d$ is distance of finite population of size $n$ or infinite population to limit for $g$ generations. In right column, $d$ is distance of finite population of size $N$ or infinite population to limits without violation.}
\label{oscillation_12h_vio_chi_0.01}
\end{center}
\end{figure}


\begin{figure}[H]
\begin{center}
\subfloat{
\resizebox{8cm}{5cm}{\includegraphics{figures/eps/vio/chi/b12/e0.1/n00004096_fin_hap.eps}}}  \hspace{-3em}%
\subfloat{
\resizebox{8cm}{5cm}{\includegraphics{figures/eps/vio/chi/b12/e0.1/n00004096_fin_hap_wovio.eps}}}\vspace{-1em}  \hspace{-3em}%
\end{center}
\begin{center}
\subfloat{
\resizebox{8cm}{5cm}{\includegraphics{figures/eps/vio/chi/b12/e0.1/n00040960_fin_hap.eps}}}  \hspace{-3em}%
\subfloat{
\resizebox{8cm}{5cm}{\includegraphics{figures/eps/vio/chi/b12/e0.1/n00040960_fin_hap_wovio.eps}}}\vspace{-1em}  \hspace{-3em}%
\end{center}

\begin{center}
\subfloat{
\resizebox{8cm}{5cm}{\includegraphics{figures/eps/vio/chi/b12/e0.1/n00081920_fin_hap.eps}}}  \hspace{-3em}%
\subfloat{
\resizebox{8cm}{5cm}{\includegraphics{figures/eps/vio/chi/b12/e0.1/n00081920_fin_hap_wovio.eps}}}\vspace{-1em}  \hspace{-3em}%
\end{center}

\begin{center}
\subfloat{
\resizebox{8cm}{5cm}{\includegraphics{figures/eps/vio/chi/b12/e0.1/inf_hap.eps}}} \hspace{-3em}%
\subfloat{
\resizebox{8cm}{5cm}{\includegraphics{figures/eps/vio/chi/b12/e0.1/inf_hap_wovio.eps}}}\vspace{-0.5em}  \hspace{-3em}%


\caption{\textbf{Infinite and finite haploid population oscillation behavior in case of violation in $\bm{\chi}$ for genome length $\ell = 12$ and $\epsilon = 0.1$:} 
  In left column, $d$ is distance of finite population of size $n$ or infinite population to limit for $g$ generations. In right column, $d$ is distance of finite population of size $N$ or infinite population to limits without violation.}
\label{oscillation_12h_vio_chi_0.1}
\end{center}
\end{figure}

\begin{figure}[H]

\begin{center}
\subfloat{
\resizebox{8cm}{5cm}{\includegraphics{figures/eps/vio/chi/b12/e0.5/n00004096_fin_hap.eps}}}  \hspace{-3em}%
\subfloat{
\resizebox{8cm}{5cm}{\includegraphics{figures/eps/vio/chi/b12/e0.5/n00004096_fin_hap_wovio.eps}}}\vspace{-1em}  \hspace{-3em}%
\end{center}
\begin{center}
\subfloat{
\resizebox{8cm}{5cm}{\includegraphics{figures/eps/vio/chi/b12/e0.5/n00040960_fin_hap.eps}}}  \hspace{-3em}%
\subfloat{
\resizebox{8cm}{5cm}{\includegraphics{figures/eps/vio/chi/b12/e0.5/n00040960_fin_hap_wovio.eps}}}\vspace{-1em}  \hspace{-3em}%
\end{center}

\begin{center}
\subfloat{
\resizebox{8cm}{5cm}{\includegraphics{figures/eps/vio/chi/b12/e0.5/n00081920_fin_hap.eps}}}  \hspace{-3em}%
\subfloat{
\resizebox{8cm}{5cm}{\includegraphics{figures/eps/vio/chi/b12/e0.5/n00081920_fin_hap_wovio.eps}}}\vspace{-1em}  \hspace{-3em}%
\end{center}

\begin{center}
\subfloat{
\resizebox{8cm}{5cm}{\includegraphics{figures/eps/vio/chi/b12/e0.5/inf_hap.eps}}} \hspace{-3em}%
\subfloat{
\resizebox{8cm}{5cm}{\includegraphics{figures/eps/vio/chi/b12/e0.5/inf_hap_wovio.eps}}}\vspace{-0.5em}  \hspace{-3em}%


\caption{\textbf{Infinite and finite haploid population oscillation behavior in case of violation in $\bm{\chi}$ for 
  genome length $\ell = 12$ and $\epsilon = 0.5$:} In left column, $d$ is distance of finite population of size $n$ or infinite population to limit for $g$ generations. In right column, $d$ is distance of finite population of size $N$ or infinite population to limits without violation.}
\label{oscillation_12h_vio_chi_0.5}
\end{center}
\end{figure}


\begin{figure}[H]
\begin{center}
\subfloat{
\resizebox{8cm}{5cm}{\includegraphics{figures/eps/vio/chi/b14/e0.01/n00004096_fin_hap.eps}}} \hspace{-3em}%
\subfloat{
\resizebox{8cm}{5cm}{\includegraphics{figures/eps/vio/chi/b14/e0.01/n00004096_fin_hap_wovio.eps}}}\vspace{-1em} \hspace{-3em}%
\end{center}
\begin{center}
\subfloat{
\resizebox{8cm}{5cm}{\includegraphics{figures/eps/vio/chi/b14/e0.01/n00040960_fin_hap.eps}}} \hspace{-3em}%
\subfloat{
\resizebox{8cm}{5cm}{\includegraphics{figures/eps/vio/chi/b14/e0.01/n00040960_fin_hap_wovio.eps}}}\vspace{-1em} \hspace{-3em}%
\end{center}

\begin{center}
\subfloat{
\resizebox{8cm}{5cm}{\includegraphics{figures/eps/vio/chi/b14/e0.01/n00081920_fin_hap.eps}}} \hspace{-3em}%
\subfloat{
\resizebox{8cm}{5cm}{\includegraphics{figures/eps/vio/chi/b14/e0.01/n00081920_fin_hap_wovio.eps}}}\vspace{-1em} \hspace{-3em}%
\end{center}

\begin{center}
\subfloat{
\resizebox{8cm}{5cm}{\includegraphics{figures/eps/vio/chi/b14/e0.01/inf_hap.eps}}}\hspace{-3em}%
\subfloat{
\resizebox{8cm}{5cm}{\includegraphics{figures/eps/vio/chi/b14/e0.01/inf_hap_wovio.eps}}}\vspace{-0.5em} \hspace{-3em}%


\caption{\textbf{Infinite and finite haploid population oscillation behavior in case of violation in $\bm{\chi}$ for genome length $\ell = 14$ and $\epsilon = 0.01$:} 
  In left column, $d$ is distance of finite population of size $n$ or infinite population to limit for $g$ generations. In right column, $d$ is distance of finite population of size $N$ or infinite population to limits without violation.}
\label{oscillation_14h_vio_chi_0.01}
\end{center}
\end{figure}


\begin{figure}[H]
\begin{center}
\subfloat{
\resizebox{8cm}{5cm}{\includegraphics{figures/eps/vio/chi/b14/e0.1/n00004096_fin_hap.eps}}}  \hspace{-3em}%
\subfloat{
\resizebox{8cm}{5cm}{\includegraphics{figures/eps/vio/chi/b14/e0.1/n00004096_fin_hap_wovio.eps}}}\vspace{-1em}  \hspace{-3em}%
\end{center}
\begin{center}
\subfloat{
\resizebox{8cm}{5cm}{\includegraphics{figures/eps/vio/chi/b14/e0.1/n00040960_fin_hap.eps}}}  \hspace{-3em}%
\subfloat{
\resizebox{8cm}{5cm}{\includegraphics{figures/eps/vio/chi/b14/e0.1/n00040960_fin_hap_wovio.eps}}}\vspace{-1em}  \hspace{-3em}%
\end{center}

\begin{center}
\subfloat{
\resizebox{8cm}{5cm}{\includegraphics{figures/eps/vio/chi/b14/e0.1/n00081920_fin_hap.eps}}}  \hspace{-3em}%
\subfloat{
\resizebox{8cm}{5cm}{\includegraphics{figures/eps/vio/chi/b14/e0.1/n00081920_fin_hap_wovio.eps}}}\vspace{-1em}  \hspace{-3em}%
\end{center}

\begin{center}
\subfloat{
\resizebox{8cm}{5cm}{\includegraphics{figures/eps/vio/chi/b14/e0.1/inf_hap.eps}}} \hspace{-3em}%
\subfloat{
\resizebox{8cm}{5cm}{\includegraphics{figures/eps/vio/chi/b14/e0.1/inf_hap_wovio.eps}}}\vspace{-0.5em}  \hspace{-3em}%


\caption{\textbf{Infinite and finite haploid population oscillation behavior in case of violation in $\bm{\chi}$ for genome length $\ell = 14$ and $\epsilon = 0.1$:} 
  In left column, $d$ is distance of finite population of size $n$ or infinite population to limit for $g$ generations. In right column, $d$ is distance of finite population of size $N$ or infinite population to limits without violation.}
\label{oscillation_14h_vio_chi_0.1}
\end{center}
\end{figure}


\begin{figure}[H]

\begin{center}
\subfloat{
\resizebox{8cm}{5cm}{\includegraphics{figures/eps/vio/chi/b14/e0.5/n00004096_fin_hap.eps}}}  \hspace{-3em}%
\subfloat{
\resizebox{8cm}{5cm}{\includegraphics{figures/eps/vio/chi/b14/e0.5/n00004096_fin_hap_wovio.eps}}}\vspace{-1em}  \hspace{-3em}%
\end{center}
\begin{center}
\subfloat{
\resizebox{8cm}{5cm}{\includegraphics{figures/eps/vio/chi/b14/e0.5/n00040960_fin_hap.eps}}}  \hspace{-3em}%
\subfloat{
\resizebox{8cm}{5cm}{\includegraphics{figures/eps/vio/chi/b14/e0.5/n00040960_fin_hap_wovio.eps}}}\vspace{-1em}  \hspace{-3em}%
\end{center}

\begin{center}
\subfloat{
\resizebox{8cm}{5cm}{\includegraphics{figures/eps/vio/chi/b14/e0.5/n00081920_fin_hap.eps}}}  \hspace{-3em}%
\subfloat{
\resizebox{8cm}{5cm}{\includegraphics{figures/eps/vio/chi/b14/e0.5/n00081920_fin_hap_wovio.eps}}}\vspace{-1em}  \hspace{-3em}%
\end{center}

\begin{center}
\subfloat{
\resizebox{8cm}{5cm}{\includegraphics{figures/eps/vio/chi/b14/e0.5/inf_hap.eps}}} \hspace{-3em}%
\subfloat{
\resizebox{8cm}{5cm}{\includegraphics{figures/eps/vio/chi/b14/e0.5/inf_hap_wovio.eps}}}\vspace{-0.5em}  \hspace{-3em}%


\caption{\textbf{Infinite and finite haploid population oscillation behavior in case of violation in $\bm{\chi}$ for 
  genome length $\ell = 14$ and $\epsilon = 0.5$:} In left column, $d$ is distance of finite population of size $n$ or infinite population to limit for $g$ generations. In right column, $d$ is distance of finite population of size $N$ or infinite population to limits without violation.}
\label{oscillation_14h_vio_chi_0.5}
\end{center}
\end{figure}


% The left column of figures \ref{oscillation_8h_vio_chi_0.01} through \ref{oscillation_14h_vio_chi_0.5} 
% shows distance of finite and infinite haploid populations to limit $\bm{z^\ast}$ (limit with violation in crossover distribution $\bm{\chi}$). The 
% right column shows distance of finite and infinite haploid populations to non-violation limits $\bm{p^\ast}$ and $\bm{q^\ast}$. 
% Graphs in right column give picture of oscillating behavior of haploid population given violation. 
% Graphs in left column in figures \ref{oscillation_8h_vio_chi_0.01} through \ref{oscillation_14h_vio_chi_0.5} show distance 
% between finite population and limit $\bm{z}^\ast$ (limit with violation in $\bm{\chi}$ distribution) for haploid case.
% 
% Both finite and infinite populations oscillate given violation. However, ripples die out quickly, 
% and oscillation ceases for infinite populations.
% In the finite population case, although ripple amplitudes decrease, oscillation didn't die out completely. 
% 
% Change in oscillating behavior of population with change in $\bm{\epsilon}$ values $\{0.01, 0.1, 0.5\}$ were also studied. 
% Results show ripples damp out faster with increase $\bm{\epsilon}$.
% With smaller values of $\bm{\epsilon}$, oscillations were sharper and as value of $\bm{\epsilon}$ increased, 
% rate of damping increased. With $\bm{\epsilon} = 0.01$, oscillation is clearly visible and ripples are sharper; with $\bm{\epsilon} = 0.1$ oscillation was visible 
% but ripples were damping out quickly;  with $\bm{\epsilon} = 0.5$, oscillation was very minimal or not visible at all. 
% 
% The distance between finite population and limit $\bm{z}^\ast$ (limit with violation in $\bm{\chi}$ distribution) decreases as finite population size increases,
% and finite population shows behavior similar to infinite population behavior as finite population reach large number. Simulation results show infinite population converges 
% to limit $\bm{z^\ast}$ quicker with increase in $\bm{\epsilon}$. 
% 
% The distance data for haploid population in case of violation in $\bm{\chi}$ distribution 
% with different values of $\bm{\epsilon}$ for different finite population size $N$ are tabulated in table \ref{distanceChiHap}.
% 
% % figures for chi violation
% \begin{table}[ht]
\caption{\textbf{Distance measured for violation in $\bm{\chi}$ with $\epsilon \;=\; 0.01$  for haploids:} $\ell$ is genome length, 
and average distance between finite and 
infinite populations is tabulated in last three columns.}
\centering
\begin{tabularx}{0.75\textwidth}{ c *{3}{X}}
\toprule
$\ell$ & $N = 4096$ & $N = 40960$ & $N = 81920$  \\
\midrule
8 & 0.0186	&  0.0150 	& 0.0115 \\
10 & 0.0158	&  0.0062 	& 0.0051 \\ 
12 & 0.0158	&  0.0056	& 0.0045 \\
14 & 0.0156	&  0.0050	& 0.0036 \\ 
\bottomrule
\end{tabularx}
\label{distanceChiHapEps0.01}
\end{table} 

\begin{table}[ht]
\caption{\textbf{Distance measured for violation in $\bm{\chi}$ with $\epsilon \;=\; 0.0$  for haploids:} $\ell$ is genome length, 
and average distance between finite and 
infinite populations is tabulated in last three columns.}
\centering
\begin{tabularx}{0.75\textwidth}{ c *{3}{X}}
\toprule
$\ell$ & $N = 4096$ & $N = 40960$ & $N = 81920$  \\
\midrule
8 & 0.0163	& 0.0061 	& 0.0051 \\
10 & 0.0157	&  0.0051	& 0.0037 \\	
12 & 0.0157	&  0.0051	& 0.0037 \\	
14 & 0.0156	&  0.0049	& 0.0035 \\
\bottomrule
\end{tabularx}
\label{distanceChiHapEps0.1}
\end{table} 

\begin{table}[ht]
\caption{\textbf{Distance measured for violation in $\bm{\chi}$ with $\epsilon \;=\; 0.5$  for haploids:} $\ell$ is genome length, 
and average distance between finite and 
infinite populations is tabulated in last three columns.}
\centering
\begin{tabularx}{0.75\textwidth}{ c *{3}{X}}
\toprule
$\ell$ & $N = 4096$ & $N = 40960$ & $N = 81920$  \\
\midrule
8 & 0.0156	&  0.0051	& 0.0036 \\
10 & 0.0155	&  0.0049	& 0.0035 \\
12 & 0.0157	&  0.0050	& 0.0035 \\
14 & 0.0156	&  0.0049	& 0.0035 \\      
\bottomrule
\end{tabularx}
\label{distanceChiHapEps0.5}
\end{table} 




% From table \ref{distanceChiHap}, 
% average distance calculated for finite population size $4096$ is $0.0158$, 
% for size $40960$ is $0.0055$ and for size $81920$ is $0.0040$.
% These results show average distance 
% between finite population and limit $\bm{z^\ast}$ closely follows expected single step distance 
% between finite and infinite population given in \ref{tableExpectedDistance}. The distance decreased as $1/\sqrt{N}$.
% 
% 
% l = 8
\begin{figure}[H]
\begin{center}
\subfloat{
\resizebox{8cm}{5cm}{\includegraphics{figures/eps/vio/chi/b8/e0.01/n00004096_fin_dip.eps}}}\hspace{-3em}%
\subfloat{
\resizebox{8cm}{5cm}{\includegraphics{figures/eps/vio/chi/b8/e0.01/n00004096_fin_dip_wovio.eps}}}\vspace{-1em}  \hspace{-3em}%
\end{center}
\begin{center}
\subfloat{
\resizebox{8cm}{5cm}{\includegraphics{figures/eps/vio/chi/b8/e0.01/n00040960_fin_dip.eps}}}\hspace{-3em}%
\subfloat{
\resizebox{8cm}{5cm}{\includegraphics{figures/eps/vio/chi/b8/e0.01/n00040960_fin_dip_wovio.eps}}}\vspace{-1em}  \hspace{-3em}%
\end{center}


\begin{center}
\subfloat{
\resizebox{8cm}{5cm}{\includegraphics{figures/eps/vio/chi/b8/e0.01/n00081920_fin_dip.eps}}}\hspace{-3em}%
\subfloat{
\resizebox{8cm}{5cm}{\includegraphics{figures/eps/vio/chi/b8/e0.01/n00081920_fin_dip_wovio.eps}}}\vspace{-1em}  \hspace{-3em}%
\end{center}

\begin{center}
\subfloat{
\resizebox{8cm}{5cm}{\includegraphics{figures/eps/vio/chi/b8/e0.01/inf_dip.eps}}}\hspace{-3em}%
\subfloat{
\resizebox{8cm}{5cm}{\includegraphics{figures/eps/vio/chi/b8/e0.01/inf_dip_wovio.eps}}}\vspace{-0.5em}  \hspace{-3em}%


\caption{\textbf{Infinite and finite diploid population oscillation behavior in case of violation in $\bm{\chi}$ for genome length $\ell = 8$ and $\epsilon = 0.01$:} 
  In left column, $d$ is distance of finite population of size $n$ or infinite population to limit for $g$ generations. In right column, $d$ is distance of finite population of size $N$ or infinite population to limits without violation.}
\label{oscillation_8d_vio_chi_0.01}
\end{center}
\end{figure}


\begin{figure}[H]
\begin{center}
\subfloat{
\resizebox{8cm}{5cm}{\includegraphics{figures/eps/vio/chi/b8/e0.1/n00004096_fin_dip.eps}}}\hspace{-3em}%
\subfloat{
\resizebox{8cm}{5cm}{\includegraphics{figures/eps/vio/chi/b8/e0.1/n00004096_fin_dip_wovio.eps}}}\vspace{-1em}  \hspace{-3em}%
\end{center}
\begin{center}
\subfloat{
\resizebox{8cm}{5cm}{\includegraphics{figures/eps/vio/chi/b8/e0.1/n00040960_fin_dip.eps}}}\hspace{-3em}%
\subfloat{
\resizebox{8cm}{5cm}{\includegraphics{figures/eps/vio/chi/b8/e0.1/n00040960_fin_dip_wovio.eps}}}\vspace{-1em}  \hspace{-3em}%
\end{center}

\begin{center}
\subfloat{
\resizebox{8cm}{5cm}{\includegraphics{figures/eps/vio/chi/b8/e0.1/n00081920_fin_dip.eps}}}\hspace{-3em}%
\subfloat{
\resizebox{8cm}{5cm}{\includegraphics{figures/eps/vio/chi/b8/e0.1/n00081920_fin_dip_wovio.eps}}}\vspace{-1em}  \hspace{-3em}%
\end{center}

\begin{center}
\subfloat{
\resizebox{8cm}{5cm}{\includegraphics{figures/eps/vio/chi/b8/e0.1/inf_dip.eps}}}\hspace{-3em}%
\subfloat{
\resizebox{8cm}{5cm}{\includegraphics{figures/eps/vio/chi/b8/e0.1/inf_dip_wovio.eps}}}\vspace{-0.5em}  \hspace{-3em}%


\caption{\textbf{Infinite and finite diploid population oscillation behavior in case of violation in $\bm{\chi}$ for genome length $\ell = 8$ and $\epsilon = 0.1$:} 
  In left column, $d$ is distance of finite population of size $n$ or infinite population to limit for $g$ generations. In right column, $d$ is distance of finite population of size $N$ or infinite population to limits without violation.}
\label{oscillation_8d_vio_chi_0.1}
\end{center}
\end{figure}


\begin{figure}[H]
\begin{center}
\subfloat{
\resizebox{8cm}{5cm}{\includegraphics{figures/eps/vio/chi/b8/e0.5/n00004096_fin_dip.eps}}}\hspace{-3em}%
\subfloat{
\resizebox{8cm}{5cm}{\includegraphics{figures/eps/vio/chi/b8/e0.5/n00004096_fin_dip_wovio.eps}}}\vspace{-1em}  \hspace{-3em}%
\end{center}
\begin{center}
\subfloat{
\resizebox{8cm}{5cm}{\includegraphics{figures/eps/vio/chi/b8/e0.5/n00040960_fin_dip.eps}}}\hspace{-3em}%
\subfloat{
\resizebox{8cm}{5cm}{\includegraphics{figures/eps/vio/chi/b8/e0.5/n00040960_fin_dip_wovio.eps}}}\vspace{-1em}  \hspace{-3em}%
\end{center}

\begin{center}
\subfloat{
\resizebox{8cm}{5cm}{\includegraphics{figures/eps/vio/chi/b8/e0.5/n00081920_fin_dip.eps}}}\hspace{-3em}%
\subfloat{
\resizebox{8cm}{5cm}{\includegraphics{figures/eps/vio/chi/b8/e0.5/n00081920_fin_dip_wovio.eps}}}\vspace{-1em}  \hspace{-3em}%
\end{center}

\begin{center}
\subfloat{
\resizebox{8cm}{5cm}{\includegraphics{figures/eps/vio/chi/b8/e0.5/inf_dip.eps}}}\hspace{-3em}%
\subfloat{
\resizebox{8cm}{5cm}{\includegraphics{figures/eps/vio/chi/b8/e0.5/inf_dip_wovio.eps}}}\vspace{-0.5em}  \hspace{-3em}%


\caption{\textbf{Infinite and finite diploid population oscillation behavior in case of violation in $\bm{\chi}$ for genome length $\ell = 8$ and $\epsilon = 0.5$:} 
  In left column, $d$ is distance of finite population of size $n$ or infinite population to limit for $g$ generations. In right column, $d$ is distance of finite population of size $N$ or infinite population to limits without violation.}
\label{oscillation_8d_vio_chi_0.5}
\end{center}
\end{figure}


% l = 10

\begin{figure}[H]
\begin{center}
\subfloat{
\resizebox{8cm}{5cm}{\includegraphics{figures/eps/vio/chi/b10/e0.01/n00004096_fin_dip.eps}}}\hspace{-3em}%
\subfloat{
\resizebox{8cm}{5cm}{\includegraphics{figures/eps/vio/chi/b10/e0.01/n00004096_fin_dip_wovio.eps}}}\vspace{-1em}  \hspace{-3em}%
\end{center}
\begin{center}
\subfloat{
\resizebox{8cm}{5cm}{\includegraphics{figures/eps/vio/chi/b10/e0.01/n00040960_fin_dip.eps}}}\hspace{-3em}%
\subfloat{
\resizebox{8cm}{5cm}{\includegraphics{figures/eps/vio/chi/b10/e0.01/n00040960_fin_dip_wovio.eps}}}\vspace{-1em}  \hspace{-3em}%
\end{center}


\begin{center}
\subfloat{
\resizebox{8cm}{5cm}{\includegraphics{figures/eps/vio/chi/b10/e0.01/n00081920_fin_dip.eps}}}\hspace{-3em}%
\subfloat{
\resizebox{8cm}{5cm}{\includegraphics{figures/eps/vio/chi/b10/e0.01/n00081920_fin_dip_wovio.eps}}}\vspace{-1em}  \hspace{-3em}%
\end{center}

\begin{center}
\subfloat{
\resizebox{8cm}{5cm}{\includegraphics{figures/eps/vio/chi/b10/e0.01/inf_dip.eps}}}\hspace{-3em}%
\subfloat{
\resizebox{8cm}{5cm}{\includegraphics{figures/eps/vio/chi/b10/e0.01/inf_dip_wovio.eps}}}\vspace{-0.5em}  \hspace{-3em}%


\caption{\textbf{Infinite and finite diploid population oscillation behavior in case of violation in $\bm{\chi}$ for genome length $\ell = 10$ and $\epsilon = 0.01$:} 
  In left column, $d$ is distance of finite population of size $n$ or infinite population to limit for $g$ generations. In right column, $d$ is distance of finite population of size $N$ or infinite population to limits without violation.}
\label{oscillation_10d_vio_chi_0.01}
\end{center}
\end{figure}


\begin{figure}[H]
\begin{center}
\subfloat{
\resizebox{8cm}{5cm}{\includegraphics{figures/eps/vio/chi/b10/e0.1/n00004096_fin_dip.eps}}}\hspace{-3em}%
\subfloat{
\resizebox{8cm}{5cm}{\includegraphics{figures/eps/vio/chi/b10/e0.1/n00004096_fin_dip_wovio.eps}}}\vspace{-1em}  \hspace{-3em}%
\end{center}
\begin{center}
\subfloat{
\resizebox{8cm}{5cm}{\includegraphics{figures/eps/vio/chi/b10/e0.1/n00040960_fin_dip.eps}}}\hspace{-3em}%
\subfloat{
\resizebox{8cm}{5cm}{\includegraphics{figures/eps/vio/chi/b10/e0.1/n00040960_fin_dip_wovio.eps}}}\vspace{-1em}  \hspace{-3em}%
\end{center}

\begin{center}
\subfloat{
\resizebox{8cm}{5cm}{\includegraphics{figures/eps/vio/chi/b10/e0.1/n00081920_fin_dip.eps}}}\hspace{-3em}%
\subfloat{
\resizebox{8cm}{5cm}{\includegraphics{figures/eps/vio/chi/b10/e0.1/n00081920_fin_dip_wovio.eps}}}\vspace{-1em}  \hspace{-3em}%
\end{center}

\begin{center}
\subfloat{
\resizebox{8cm}{5cm}{\includegraphics{figures/eps/vio/chi/b10/e0.1/inf_dip.eps}}}\hspace{-3em}%
\subfloat{
\resizebox{8cm}{5cm}{\includegraphics{figures/eps/vio/chi/b10/e0.1/inf_dip_wovio.eps}}}\vspace{-0.5em}  \hspace{-3em}%


\caption{\textbf{Infinite and finite diploid population oscillation behavior in case of violation in $\bm{\chi}$ for genome length $\ell = 10$ and $\epsilon = 0.1$:} 
  In left column, $d$ is distance of finite population of size $n$ or infinite population to limit for $g$ generations. In right column, $d$ is distance of finite population of size $N$ or infinite population to limits without violation.}
\label{oscillation_10d_vio_chi_0.1}
\end{center}
\end{figure}

\begin{figure}[H]
\begin{center}
\subfloat{
\resizebox{8cm}{5cm}{\includegraphics{figures/eps/vio/chi/b10/e0.5/n00004096_fin_dip.eps}}}\hspace{-3em}%
\subfloat{
\resizebox{8cm}{5cm}{\includegraphics{figures/eps/vio/chi/b10/e0.5/n00004096_fin_dip_wovio.eps}}}\vspace{-1em}  \hspace{-3em}%
\end{center}
\begin{center}
\subfloat{
\resizebox{8cm}{5cm}{\includegraphics{figures/eps/vio/chi/b10/e0.5/n00040960_fin_dip.eps}}}\hspace{-3em}%
\subfloat{
\resizebox{8cm}{5cm}{\includegraphics{figures/eps/vio/chi/b10/e0.5/n00040960_fin_dip_wovio.eps}}}\vspace{-1em}  \hspace{-3em}%
\end{center}

\begin{center}
\subfloat{
\resizebox{8cm}{5cm}{\includegraphics{figures/eps/vio/chi/b10/e0.5/n00081920_fin_dip.eps}}}\hspace{-3em}%
\subfloat{
\resizebox{8cm}{5cm}{\includegraphics{figures/eps/vio/chi/b10/e0.5/n00081920_fin_dip_wovio.eps}}}\vspace{-1em}  \hspace{-3em}%
\end{center}

\begin{center}
\subfloat{
\resizebox{8cm}{5cm}{\includegraphics{figures/eps/vio/chi/b10/e0.5/inf_dip.eps}}}\hspace{-3em}%
\subfloat{
\resizebox{8cm}{5cm}{\includegraphics{figures/eps/vio/chi/b10/e0.5/inf_dip_wovio.eps}}}\vspace{-0.5em}  \hspace{-3em}%


\caption{\textbf{Infinite and finite diploid population oscillation behavior in case of violation in $\bm{\chi}$ for genome length $\ell = 10$ and $\epsilon = 0.5$:} 
  In left column, $d$ is distance of finite population of size $n$ or infinite population to limit for $g$ generations. In right column, $d$ is distance of finite population of size $N$ or infinite population to limits without violation.}
\label{oscillation_10d_vio_chi_0.5}
\end{center}
\end{figure}



\begin{figure}[H]
\begin{center}
\subfloat{
\resizebox{8cm}{5cm}{\includegraphics{figures/eps/vio/chi/b12/e0.01/n00004096_fin_dip.eps}}}\hspace{-3em}%
\subfloat{
\resizebox{8cm}{5cm}{\includegraphics{figures/eps/vio/chi/b12/e0.01/n00004096_fin_dip_wovio.eps}}}\vspace{-1em}  \hspace{-3em}%
\end{center}
\begin{center}
\subfloat{
\resizebox{8cm}{5cm}{\includegraphics{figures/eps/vio/chi/b12/e0.01/n00040960_fin_dip.eps}}}\hspace{-3em}%
\subfloat{
\resizebox{8cm}{5cm}{\includegraphics{figures/eps/vio/chi/b12/e0.01/n00040960_fin_dip_wovio.eps}}}\vspace{-1em}  \hspace{-3em}%
\end{center}


\begin{center}
\subfloat{
\resizebox{8cm}{5cm}{\includegraphics{figures/eps/vio/chi/b12/e0.01/n00081920_fin_dip.eps}}}\hspace{-3em}%
\subfloat{
\resizebox{8cm}{5cm}{\includegraphics{figures/eps/vio/chi/b12/e0.01/n00081920_fin_dip_wovio.eps}}}\vspace{-1em}  \hspace{-3em}%
\end{center}

\begin{center}
\subfloat{
\resizebox{8cm}{5cm}{\includegraphics{figures/eps/vio/chi/b12/e0.01/inf_dip.eps}}}\hspace{-3em}%
\subfloat{
\resizebox{8cm}{5cm}{\includegraphics{figures/eps/vio/chi/b12/e0.01/inf_dip_wovio.eps}}}\vspace{-0.5em}  \hspace{-3em}%


\caption{\textbf{Infinite and finite diploid population oscillation behavior in case of violation in $\bm{\chi}$ for genome length $\ell = 12$ and $\epsilon = 0.01$:} 
  In left column, $d$ is distance of finite population of size $n$ or infinite population to limit for $g$ generations. In right column, $d$ is distance of finite population of size $N$ or infinite population to limits without violation.}
\label{oscillation_12d_vio_chi_0.01}
\end{center}
\end{figure}


\begin{figure}[H]
\begin{center}
\subfloat{
\resizebox{8cm}{5cm}{\includegraphics{figures/eps/vio/chi/b12/e0.1/n00004096_fin_dip.eps}}}\hspace{-3em}%
\subfloat{
\resizebox{8cm}{5cm}{\includegraphics{figures/eps/vio/chi/b12/e0.1/n00004096_fin_dip_wovio.eps}}}\vspace{-1em}  \hspace{-3em}%
\end{center}
\begin{center}
\subfloat{
\resizebox{8cm}{5cm}{\includegraphics{figures/eps/vio/chi/b12/e0.1/n00040960_fin_dip.eps}}}\hspace{-3em}%
\subfloat{
\resizebox{8cm}{5cm}{\includegraphics{figures/eps/vio/chi/b12/e0.1/n00040960_fin_dip_wovio.eps}}}\vspace{-1em}  \hspace{-3em}%
\end{center}

\begin{center}
\subfloat{
\resizebox{8cm}{5cm}{\includegraphics{figures/eps/vio/chi/b12/e0.1/n00081920_fin_dip.eps}}}\hspace{-3em}%
\subfloat{
\resizebox{8cm}{5cm}{\includegraphics{figures/eps/vio/chi/b12/e0.1/n00081920_fin_dip_wovio.eps}}}\vspace{-1em}  \hspace{-3em}%
\end{center}

\begin{center}
\subfloat{
\resizebox{8cm}{5cm}{\includegraphics{figures/eps/vio/chi/b12/e0.1/inf_dip.eps}}}\hspace{-3em}%
\subfloat{
\resizebox{8cm}{5cm}{\includegraphics{figures/eps/vio/chi/b12/e0.1/inf_dip_wovio.eps}}}\vspace{-0.5em}  \hspace{-3em}%


\caption{\textbf{Infinite and finite diploid population oscillation behavior in case of violation in $\bm{\chi}$ for genome length $\ell = 12$ and $\epsilon = 0.1$:} 
  In left column, $d$ is distance of finite population of size $n$ or infinite population to limit for $g$ generations. In right column, $d$ is distance of finite population of size $N$ or infinite population to limits without violation.}
\label{oscillation_12d_vio_chi_0.1}
\end{center}
\end{figure}


\begin{figure}[H]
\begin{center}
\subfloat{
\resizebox{8cm}{5cm}{\includegraphics{figures/eps/vio/chi/b12/e0.5/n00004096_fin_dip.eps}}}\hspace{-3em}%
\subfloat{
\resizebox{8cm}{5cm}{\includegraphics{figures/eps/vio/chi/b12/e0.5/n00004096_fin_dip_wovio.eps}}}\vspace{-1em}  \hspace{-3em}%
\end{center}
\begin{center}
\subfloat{
\resizebox{8cm}{5cm}{\includegraphics{figures/eps/vio/chi/b12/e0.5/n00040960_fin_dip.eps}}}\hspace{-3em}%
\subfloat{
\resizebox{8cm}{5cm}{\includegraphics{figures/eps/vio/chi/b12/e0.5/n00040960_fin_dip_wovio.eps}}}\vspace{-1em}  \hspace{-3em}%
\end{center}

\begin{center}
\subfloat{
\resizebox{8cm}{5cm}{\includegraphics{figures/eps/vio/chi/b12/e0.5/n00081920_fin_dip.eps}}}\hspace{-3em}%
\subfloat{
\resizebox{8cm}{5cm}{\includegraphics{figures/eps/vio/chi/b12/e0.5/n00081920_fin_dip_wovio.eps}}}\vspace{-1em}  \hspace{-3em}%
\end{center}

\begin{center}
\subfloat{
\resizebox{8cm}{5cm}{\includegraphics{figures/eps/vio/chi/b12/e0.5/inf_dip.eps}}}\hspace{-3em}%
\subfloat{
\resizebox{8cm}{5cm}{\includegraphics{figures/eps/vio/chi/b12/e0.5/inf_dip_wovio.eps}}}\vspace{-0.5em}  \hspace{-3em}%


\caption{\textbf{Infinite and finite diploid population oscillation behavior in case of violation in $\bm{\chi}$ for genome length $\ell = 12$ and $\epsilon = 0.5$:} 
  In left column, $d$ is distance of finite population of size $n$ or infinite population to limit for $g$ generations. In right column, $d$ is distance of finite population of size $N$ or infinite population to limits without violation.}
\label{oscillation_12d_vio_chi_0.5}
\end{center}
\end{figure}


\begin{figure}[H]
\begin{center}
\subfloat{
\resizebox{8cm}{5cm}{\includegraphics{figures/eps/vio/chi/b14/e0.01/n00004096_fin_dip.eps}}}\hspace{-3em}%
\subfloat{
\resizebox{8cm}{5cm}{\includegraphics{figures/eps/vio/chi/b14/e0.01/n00004096_fin_dip_wovio.eps}}}\vspace{-1em}  \hspace{-3em}%
\end{center}
\begin{center}
\subfloat{
\resizebox{8cm}{5cm}{\includegraphics{figures/eps/vio/chi/b14/e0.01/n00040960_fin_dip.eps}}}\hspace{-3em}%
\subfloat{
\resizebox{8cm}{5cm}{\includegraphics{figures/eps/vio/chi/b14/e0.01/n00040960_fin_dip_wovio.eps}}}\vspace{-1em}  \hspace{-3em}%
\end{center}


\begin{center}
\subfloat{
\resizebox{8cm}{5cm}{\includegraphics{figures/eps/vio/chi/b14/e0.01/n00081920_fin_dip.eps}}}\hspace{-3em}%
\subfloat{
\resizebox{8cm}{5cm}{\includegraphics{figures/eps/vio/chi/b14/e0.01/n00081920_fin_dip_wovio.eps}}}\vspace{-1em}  \hspace{-3em}%
\end{center}

\begin{center}
\subfloat{
\resizebox{8cm}{5cm}{\includegraphics{figures/eps/vio/chi/b14/e0.01/inf_dip.eps}}}\hspace{-3em}%
\subfloat{
\resizebox{8cm}{5cm}{\includegraphics{figures/eps/vio/chi/b14/e0.01/inf_dip_wovio.eps}}}\vspace{-0.5em}  \hspace{-3em}%


\caption{\textbf{Infinite and finite diploid population oscillation behavior in case of violation in $\bm{\chi}$ for genome length $\ell = 14$ and $\epsilon = 0.01$:} 
  In left column, $d$ is distance of finite population of size $n$ or infinite population to limit for $g$ generations. In right column, $d$ is distance of finite population of size $N$ or infinite population to limits without violation.}
\label{oscillation_14d_vio_chi_0.01}
\end{center}
\end{figure}


\begin{figure}[H]
\begin{center}
\subfloat{
\resizebox{8cm}{5cm}{\includegraphics{figures/eps/vio/chi/b14/e0.1/n00004096_fin_dip.eps}}}\hspace{-3em}%
\subfloat{
\resizebox{8cm}{5cm}{\includegraphics{figures/eps/vio/chi/b14/e0.1/n00004096_fin_dip_wovio.eps}}}\vspace{-1em}  \hspace{-3em}%
\end{center}
\begin{center}
\subfloat{
\resizebox{8cm}{5cm}{\includegraphics{figures/eps/vio/chi/b14/e0.1/n00040960_fin_dip.eps}}}\hspace{-3em}%
\subfloat{
\resizebox{8cm}{5cm}{\includegraphics{figures/eps/vio/chi/b14/e0.1/n00040960_fin_dip_wovio.eps}}}\vspace{-1em}  \hspace{-3em}%
\end{center}

\begin{center}
\subfloat{
\resizebox{8cm}{5cm}{\includegraphics{figures/eps/vio/chi/b14/e0.1/n00081920_fin_dip.eps}}}\hspace{-3em}%
\subfloat{
\resizebox{8cm}{5cm}{\includegraphics{figures/eps/vio/chi/b14/e0.1/n00081920_fin_dip_wovio.eps}}}\vspace{-1em}  \hspace{-3em}%
\end{center}

\begin{center}
\subfloat{
\resizebox{8cm}{5cm}{\includegraphics{figures/eps/vio/chi/b14/e0.1/inf_dip.eps}}}\hspace{-3em}%
\subfloat{
\resizebox{8cm}{5cm}{\includegraphics{figures/eps/vio/chi/b14/e0.1/inf_dip_wovio.eps}}}\vspace{-0.5em}  \hspace{-3em}%


\caption{\textbf{Infinite and finite diploid population oscillation behavior in case of violation in $\bm{\chi}$ for genome length $\ell = 14$ and $\epsilon = 0.1$:} 
  In left column, $d$ is distance of finite population of size $n$ or infinite population to limit for $g$ generations. In right column, $d$ is distance of finite population of size $N$ or infinite population to limits without violation.}
\label{oscillation_14d_vio_chi_0.1}
\end{center}
\end{figure}


\begin{figure}[H]

\begin{center}
\subfloat{
\resizebox{8cm}{5cm}{\includegraphics{figures/eps/vio/chi/b14/e0.5/n00004096_fin_dip.eps}}}\hspace{-3em}%
\subfloat{
\resizebox{8cm}{5cm}{\includegraphics{figures/eps/vio/chi/b14/e0.5/n00004096_fin_dip_wovio.eps}}}\vspace{-1em}  \hspace{-3em}%
\end{center}
\begin{center}
\subfloat{
\resizebox{8cm}{5cm}{\includegraphics{figures/eps/vio/chi/b14/e0.5/n00040960_fin_dip.eps}}}\hspace{-3em}%
\subfloat{
\resizebox{8cm}{5cm}{\includegraphics{figures/eps/vio/chi/b14/e0.5/n00040960_fin_dip_wovio.eps}}}\vspace{-1em}  \hspace{-3em}%
\end{center}

\begin{center}
\subfloat{
\resizebox{8cm}{5cm}{\includegraphics{figures/eps/vio/chi/b14/e0.5/n00081920_fin_dip.eps}}}\hspace{-3em}%
\subfloat{
\resizebox{8cm}{5cm}{\includegraphics{figures/eps/vio/chi/b14/e0.5/n00081920_fin_dip_wovio.eps}}}\vspace{-1em}  \hspace{-3em}%
\end{center}

\begin{center}
\subfloat{
\resizebox{8cm}{5cm}{\includegraphics{figures/eps/vio/chi/b14/e0.5/inf_dip.eps}}}\hspace{-3em}%
\subfloat{
\resizebox{8cm}{5cm}{\includegraphics{figures/eps/vio/chi/b14/e0.5/inf_dip_wovio.eps}}}\vspace{-0.5em}  \hspace{-3em}%


\caption{\textbf{Infinite and finite diploid population oscillation behavior in case of violation in $\bm{\chi}$ for genome length $\ell = 14$ and $\epsilon = 0.5$:} 
  In left column, $d$ is distance of finite population of size $n$ or infinite population to limit for $g$ generations. In right column, $d$ is distance of finite population of size $N$ or infinite population to limits without violation.}
\label{oscillation_14d_vio_chi_0.5}
\end{center}
\end{figure}



% The left column of figures \ref{oscillation_8d_vio_chi_0.01} through \ref{oscillation_14d_vio_chi_0.5} 
% shows distance of finite and infinite diploid populations to limit $\bm{z^\ast}$ (limit with violation in $\bm{chi}$). The 
% right column shows distance of finite and infinite diploid populations to non-violation limits $\bm{p^\ast}$ and $\bm{q^\ast}$. 
% Graphs in right column give picture of oscillating behavior of diploid population given violation. 
% Graphs in left column in figures \ref{oscillation_8d_vio_chi_0.01} through \ref{oscillation_14d_vio_chi_0.5} show distance 
% between finite population and limit $\bm{z}^\ast$ (limit with violation in $\bm{\chi}$ distribution) for diploid case.
% 
% Likewise in hapliod population, both finite and infinite diploid populations oscillate given violation. However, ripples die out quickly, 
% and oscillation ceases for infinite populations.
% In the finite population case, although ripple amplitudes decrease, oscillation didn't die out completely. 
% 
% Change in oscillating behavior of population with change in $\bm{\epsilon}$ values $\{0.01, 0.1, 0.5\}$ were also studied. 
% Results show ripples damp out faster with increase $\bm{\epsilon}$.
% With smaller values of $\bm{\epsilon}$, oscillations were sharper and as value of $\bm{\epsilon}$ increased, 
% rate of damping increased. With $\bm{\epsilon} = 0.01$, oscillation is clearly visible and ripples are sharper; with $\bm{\epsilon} = 0.1$ oscillation was visible 
% but ripples were damping out quickly;  with $\bm{\epsilon} = 0.5$, oscillation was very minimal or not visible at all. 
% 
% The distance between finite population and limit $\bm{z}^\ast$ (limit with violation in $\bm{\chi}$ distribution) decreases as finite population size increases, 
% and finite population shows behavior similar to infinite population behavior as finite population reach large number. Simulation results show infinite population converges 
% to limit $\bm{z^\ast}$ quicker with increase in $\bm{\epsilon}$. 
% 
% The distance data for diploid population in case of violation in $\bm{\chi}$ distribution 
% with different values of $\bm{\epsilon}$ for different finite population size $N$ are tabulated in table \ref{distanceChiDip}.
% 
% % figures for chi violation
% 
\begin{table}[ht]
\caption{\textbf{Distance measured for violation in $\bm{\chi}$ with $\epsilon \;=\; 0.01$ diploids:} $\ell$ is genome length, 
and average distance between finite and 
infinite populations is tabulated in last three columns.}
\centering
\begin{tabularx}{0.75\textwidth}{ c *{3}{X}}
\toprule
$\ell$ & $N = 4096$ & $N = 40960$ & $N = 81920$  \\
\midrule
8 & 0.0156	&  0.0051	& 0.0036 \\
10 & 0.0156	&  0.0049	& 0.0035 \\
12 & 0.0156	&  0.0049	& 0.0035 \\
14 & 0.0156	&  0.0049	& 0.0035 \\
\bottomrule
\end{tabularx}
\label{distanceChiDipEps0.01}
\end{table} 



\begin{table}[ht]
\caption{\textbf{Distance measured for violation in $\bm{\chi}$ with $\epsilon \;=\; 0.1$ for diploids:} $\ell$ is genome length, 
and average distance between finite and 
infinite populations is tabulated in last three columns.}
\centering
\begin{tabularx}{0.75\textwidth}{ c *{3}{X}}
\toprule
$\ell$ & $N = 4096$ & $N = 40960$ & $N = 81920$  \\
\midrule
8 & 0.0156	&  0.0050	& 0.0035 \\
10 & 0.0156	&  0.0049	& 0.0035 \\
12 & 0.0156	&  0.0049	& 0.0035 \\
14 & 0.0156	&  0.0049	& 0.0035 \\
\bottomrule
\end{tabularx}
\label{distanceChiDipEps0.1}
\end{table} 


\begin{table}[ht]
\caption{\textbf{Distance measured for violation in $\bm{\chi}$ with $\epsilon \;=\; 0.5$ for diploids:} $\ell$ is genome length, 
and average distance between finite and 
infinite populations is tabulated in last three columns.}
\centering
\begin{tabularx}{0.75\textwidth}{ c *{3}{X}}
\toprule
$\ell$ & $N = 4096$ & $N = 40960$ & $N = 81920$  \\
\midrule
8 & 0.0156	&  0.0049	& 0.0035 \\	
10 & 0.0156	&  0.0049	& 0.0035 \\
12 & 0.0156	&  0.0049	& 0.0035 \\
14 & 0.0156	&  0.0049	& 0.0035 \\
\bottomrule
\end{tabularx}
\label{distanceChiDipEps0.5}
\end{table} 


% 
% From table \ref{distanceChiDip}, 
% average distance calculated for finite population size $4096$ is $0.0158$, 
% for size $40960$ is $0.0055$ and for size $81920$ is $0.0040$.
% These results show average distance 
% between finite population and limit $\bm{z^\ast}$ closely follows expected single step distance 
% between finite and infinite population given in \ref{tableExpectedDistance}. The distance decreased as $1/\sqrt{N}$.
% 
% Violation introduced to the crossover distribution $\bm{\chi}$ creates new masks (different than in case of without violation) to be used in crossover 
% during transmission. With small $\bm{\epsilon}$, the probablility of using the new masks available due to violation is very small and 
% those masks might not be used at all during crossover in finite population, but with higher values of $\bm{\epsilon}$, 
% those new masks have higher chance of usage during crossover which cause oscillation to damp out quickly or cause no oscillation at all.
% 
% From graphs on the right in 
% figures \ref{oscillation_8h_vio_chi_0.01} through \ref{oscillation_14h_vio_chi_0.5}, 
% and \ref{oscillation_8d_vio_chi_0.01} through \ref{oscillation_14d_vio_chi_0.5}, haploid population shows sharper and 
% more distinct oscillations for same size of population size than diploid population; diploid population 
% requires larger population size to exhibit sharper oscillations. 
% And from graphs on the left in figures \ref{oscillation_8h_vio_chi_0.01} through \ref{oscillation_14h_vio_chi_0.5}, and \ref{oscillation_8d_vio_chi_0.01} through \ref{oscillation_14d_vio_chi_0.5}, diploid population converges to limit $\bm{z}^\ast$ more quickly than haploid population. 
% 
% \section{Summary}
% In this chapter, we described limits predicted by Vose for infinite population, and necessary and sufficient condition 
% for convergence to periodic orbits. Through experiment, we showed finite population also oscillate, 
% and converge to infinite population behavior as population size increases. 
% Then we studied through simulations the effect on infinite and finite population behavior of violating the condition 
% for convergence to periodic orbits. Infinite population ceases to oscillate when the condition for convergence to 
% periodic orbits is violated, but finite population continued to approximately oscillate for small values of $\bm{\epsilon}$. 
% For smaller values of $\bm{\epsilon}$, finite population does not get aware of violation because the probability of using 
% new masks created in the mutation distribution $\bm{\mu}$ and the crossover distribution $\bm{\mu}$ due to violation is very low, and 
% finite population follows behavior of infinite population without violation in the condition for convergence to 
% periodic orbits.
% 




 
