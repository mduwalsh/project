\documentclass{article}
\renewcommand{\baselinestretch}{1.5}     % line Spacing
%%%%%%%%%%%%%%%%%%%%%%%%%%%%%%%%%%%%%%%%%%%%%%%%%%%%%%%%%%%%%%%%%%%%%%%%%%%%%%%%%%%%%%%%%%%%%%%%%%%%%
% LOAD SOME USEFUL PACKAGES
%%%%%%%%%%%%%%%%%%%%%%%%%%%%%%%%%%%%%%%%%%%%%%%%%%%%%%%%%%%%%%%%%%%%%%%%%%%%%%%%%%%%%%%%%%%%%%%%%%%%%
\usepackage{nomencl}                    % produces a nomenclature
\usepackage{float}                      % figure floats
\usepackage{natbib}                     % this package allows you to link your references
\usepackage{graphicx}                                   % graphics package
\graphicspath{ {figures/}{figures/eps/}{figures/pdf/} }% specify the path where figures are located
\usepackage{fancyhdr}                   % fancy headers and footers
\usepackage{url}                        % nicely format url breaks
\usepackage[inactive]{srcltx}                   % necessary to use forward and inverse searching in DVI
\usepackage{relsize}                    % font sizing hierarchy
\usepackage{booktabs}                   % professional looking tables
%\usepackage{subfigure}% Support for small, `sub' figures and tables
\usepackage[config, labelfont={bf}]{caption,subfig} % nice sub figures
\usepackage{mathrsfs}                   % additional math scripts
\usepackage{bm}
\usepackage{amsfonts}
\usepackage{multirow}
\usepackage{tabularx, booktabs}

\newcommand{\nudge}{\hspace{0.01in}}
%%% PACKAGES THAT ARE PRELOADED WITH THE CLASS ARE: amsmath,amsthm,amssymb,setspace,geometry,hyperref,and color
%%%%%%%%%%%%%%%%%%%%%%%%%%%%%%%%%%%%%%%%%%%%%%%%%%%%%%%%%%%%%%%%%%%%%%%%%%%%%%%%%%%%%%%%%%%%%%%%%%%%%
\begin{document}
\pagenumbering{arabic}
\setcounter{page}{1}    
\newlength{\mywidth}
\setlength{\mywidth}{0.9\linewidth}
\newlength{\myheight}
\setlength{\myheight}{0.5in}
    
\begin{enumerate}
\item
  \begin{itemize}
  \item I will talk about my thesis: \hfill\mbox{ }\linebreak
      ``Simulation of Simple Evolutionary System'' 
  \end{itemize}  
    
\item
  \begin{itemize}
  \item      
      I will give background, and computations involving our model in the simulations we did. 
  \item
      Then I will address these four questions, and make
      concluding remarks.    
  \end{itemize}
  
\item Part I
  
\item
  \begin{itemize}
  \item Population is a collection of length $\ell$ binary strings 
  \item Population can be represented by a vector p; the $j$th component
      is the proportion of string j in the population.
  \item For example, if P is this, 3rd component of vector p is equal to 1/3.
    \item $\mathcal{R}$ is the set of length $\ell$ binary strings and operations under $\mathcal{R}$ are bitwise modulo 2 operations. $Z$ is cardinality of $\mathcal{R}$.
    \item Here are examples of some operations under $\mathcal{R}$
  \end{itemize}
    
\item
  \begin{itemize}
  \item Crossover and mutation operators are defined using the
    bitwise operations in $\mathcal{R}$.
  \item Crossover exchanges bits in parents u and v using crossover mask
  to produce children u' and v' using this
  \item Here's an example. These parents crossover to produce these offspings.
  \item Results depend on crossover mask m. 
  \item $\bm{\chi}_m$ gives probability of using crossover mask m.
  \item
  \item Mutation flips bits in $x$ using mutation mask m.
  \item In this example, this mutates to this using this mask.
  \item $\bm{\mu}_m$ gives probability of using mutation mask m
  \item The use of masks dates back to Geiringer, 1944
  \end{itemize}
    
\item
  \begin{itemize}
  \item This flowchart illustrates finite population genetic algorithm
  \item We start from random initial population p.
  \item Then randomly select parents u and v, 
  \item And crossover to produce u' and v' according to randomly chosen crossover mask
  \item We keep one of u', v' and mutate using randomly chosen mutation mask to produce gamete g
  \item We repeat above procedures to form next generation $\tau(p)$
  \end{itemize}
  
\item
  \begin{itemize}
  \item Random Heuristic Search generalizes simple genetic algorithm
  \item Given population p, $\tau$ is transition rule that maps p to
    the next generation p' where p and $p^\prime$ both belong to population space $\Lambda_N$.
    N is population size. 
    \item However, $\tau$ is stochastic function of crossover and
    mutation, and can not be predicted with
    certainty 
  \item Finite population evolution sequence forms
    Markov chain
  
  \end{itemize}
    
\item
  \begin{itemize}
  \item In infinite population model, population is modeled by vector p where p belongs to population space $\Lambda$.    
  \item $\mathcal{G}$ is a function that maps $\bm{p}$ to the next
    generation $\bm{p}^\prime$ where $j$th component is proportion of string $j$
    occurs in the next generation.
  \item The sequence shows evolution of p under infinite population
    model.
    \item $\mathcal{G}$ gives the expectation of $\tau(\bm{p})$
  \item And the variance is of finite population in next generation is given by this.
  \end{itemize}
    
\item
  \begin{itemize}
  \item This is our model setup for diploid population.
  \item We consider diploid genome $\alpha$ with genome length l.
  \item Population is modeled by vector $q$
  \item $q_\alpha$ is prevalence of diploid $\alpha$
  \item $t_{\alpha}(g)$ is transmission function which is probability
    of gamete g being produced from parent $\alpha$
  \item $q^\prime$ is next generation
  \item This is the standard evolutionary model from population genetics. 
  \item This assumes panmixia i.e., parents mate randomly without bias.
%   \item $q_\gamma^{n+1} \; = \; \sum_{\alpha} \, q_\alpha^n \,
%     t_\alpha(\gamma_0) \sum_{\beta} \,q_\beta^n \,
%     t_\beta(\gamma_1)\\[-.05in]$ gives us next generation population.
  \end{itemize}
    
\item
  \begin{itemize}
  \item Diploids can be determined in terms of haploid
    distributions.
  \item Haploids can be determined in terms of diploid
    distributions.
  \item And evolution equation in terms of haploid
    distributions can be expressed as this
    \[\bm{p}_{\gamma_0}^{\prime} \,=\, \sum_{\alpha_0, \, \alpha_1} \, \bm{p}_{\alpha_0} \, \bm{p}_{\alpha_1} \,
	  t_{\langle \alpha_0, \,\alpha_1 \rangle}(\gamma_0) \]	  
  \item Square brackets here are Iverson brackets which returns 1 if expression inside is true and returns 0 when expression inside is false.  
  
  \item In the matrix form, evolution equation can be expressed as $\bm{p}_g^\prime \; = \; \bm{p}^T M_g \, \bm{p}$
  \item $\bm{p}^\prime$ is next generation.
  \item Where matrix M(g) describes transmission funciton. If g = 0, we call it mixing matrix.
  \end{itemize}
    
\item
  \begin{itemize}
  \item $\bm{\mu}$ of i gives probability that i is used as mutation mask. Mu is mutation rate.
  \item $\bm{\chi}$ of i gives probability that i is used as crossover mask. Chi is crossover rate.
  \item t of g is transmission function. This is expensive to compute because it has three sum terms in it.  
  \end{itemize}
    
\item
  \begin{itemize}
  \item Component of walsh matrix W is given by this, Z = $2^\ell$
  \item Walsh transform of matrix A is given by this. 
    and Walsh transform of vector w is given by this.
  \item Here's the algorithm of computing walsh transform.
  \item Using this algorithm, walsh transform can be computed in O(ZlogZ)
  \end{itemize}
  
\item
  \begin{itemize}
  \item Walsh transform of Mixing matrix $M$ is given by the expression
    \[
      \widehat{M}_{u,v} \; = \; 2^{\,\ell-1} \,[\nudge u \nudge v = {\bf
      0}\nudge]\, \widehat{\bm{\mu}}_u \nudge \widehat{\bm{\mu}}_v \!  \sum_{k
    \nudge \in \nudge \overline{u+v} \nudge \mathcal{R}} \bm{\chi}_{k + u} +
    \bm{\chi}_{k + v}
    \]
  \item Evolution eqn in Walsh basis takes this form
  \[
    \widehat{\bm{p}}_g^{\,\,\prime} \; = \; 2^{\,\ell/2} \sum_{i \nudge \in \nudge g \mathcal{R}}
    \widehat{\bm{p}}_i \, \nudge \widehat{\bm{p}}_{i+g} \,\widehat{M}_{i,i+g}
  \]  
  where $\hat{p}^\prime$ is next generation
  \end{itemize}
    
\item
  \begin{itemize}
  \item Now if we compare evolution eqn in Walsh basis to what we had before,
  \item we have single sum in computing p' compared to double sum from matrix multiplication before. 
  \item We need only one mixing matrix compared to $2^\ell$ matrices for different g s.
  \item Also calculating each mixing matrix in previous eqn required 3 sum terms, which is reduced to 1 sum term in walsh basis.   
  \end{itemize}
    
\item
  \begin{itemize}
  
  \item So in summary, Reduction to haploid model and Walsh basis simplifiy computation, which otherwise for diploid case would have been impractical
  \item Only one mixing matrix as opposed to $2^l$ is needed to compute next generation
  \item Consider l = 14, $2^{14}$ matrices would require 32 TB of
    memory, while one mixing matrix would require only 2 GB of memory
  
  \end{itemize}
    
\item
  \begin{itemize}
  \item In the 2nd part of this presentation, I talk about distance between finite population and infinite population.
  \item If f is finite population and q is infinite population,
  \item With naive computation as shown, there would
    be $2^l \cdot 2^l$ multiplication terms to be computed because $\alpha$ is diploid and have two components  
  \item We simplify it as shown in eqn (2) where it needs only $2^l$ +
    cardinal value of $S_f$ terms
  \item $S_f$ is set of diploids that are in finite populations and is byproduct of finite GA.
  
  \item Writing code to compute in the Walsh basis, and running simulations using the Walsh basis 
  is a significant part of my thesis. It represents the "Efficient" part in the thesis title.
  \end{itemize}
  
\item {Part II}
  
\item
  \begin{itemize}
  \item Our first question concerns about convergence of finite population to infinite population.
  \item Chebyshev's inequality, Jensen's inequality and geometric point of view, all, suggest the distance might decrease
    as $1/\sqrt{ N}$
  \item But all of 3 implications are from inequalities. The distance
    might decrease much smaller than $1/\sqrt{ N}$.
  \item We investigate if the distance infact can decrease like $1/\sqrt{N}$ in practice. 
  \end{itemize}
  
\item
  \begin{itemize}
  \item We ran some simulations with $\chi$ = 0.1 and $\mu$ = 0.001 for
    different values of l
  \item TThese graphs are in logarithmic scale.
  \item  This axis is the distance, (point to axis) This axis is population size, (point to axis) this axis is number of generations
  \item The thing to notice about these graphs is how flat and planar they are. 
  \end{itemize}
    
\item
  \begin{itemize}
  \item So we model the data by linear equation
  \item Regression shows slope m is nearly equal to negative half (-1/2).
  \item This proves distance does decrease as $1/\sqrt{N}$  
  \end{itemize}

\item
  \begin{itemize}
  \item $1/\sqrt{N}$ was theoretical upper bound 
  but our simulaion shows it is really very good bound 
  and the distance decreases as $1/\sqrt{N}$. 
  \end{itemize}
    
\item 
  \begin{itemize}
  \item Our 2nd question concerns about oscillation in
    finite populations.
  \end{itemize}
    
\item
  \begin{itemize}
%   \item First I want to introduce term limit; we also call it fixed
%     point.
  \item The infinite population sequence  $\bm{p}, \, \mathcal{G}(\bm{p}), \, {\mathcal{G}}^2(\bm{p}), \, \cdots$ may
    converge to a point, and we call that point a fixed point.
  \item But under certain conditions, the sequence converges to a
    periodic orbit between two fixed points, say p* and q*
\end{itemize}

\item
  \begin{itemize}
  \item And the conditions are: For some $g$
  \begin{eqnarray*}
      -1 &=& \sum \limits_{j} (-1)^{g^T j} \bm{\mu}_j \\
      1 &=& \sum \limits_{k \in \bar{g}\mathcal{R}} \bm{\chi}_{k+g} + \bm{\chi}_k 
      \end{eqnarray*}
  \item for mutation distribution to meet this condition 1 and for crossover  distribution to meet this condition 2.
  \item We try to answer our 2nd question through simulations; that is can finite populations exhibit
     oscillations from random intial populations when infinite populations oscillate?
  \end{itemize}
    
\item
  \begin{itemize}
    \item Few people have studied oscillations in the past: Akin 1981, Hasting 1982, Wright and Bidwell 1997, and Wright and Agapie 2001.
%   \item Other people in the past also have been interested in studying
%     oscillations in populations.
%    \item Akin (1982) proved existence of cycling for continuous-time
%      2-bit diploid model
%   \item Hasting (1981) studied cycling in populations with infinite
%     2-bit diploid population model
%   \item Wright and Bidwell (1997) provided examples when cycles in an
%     infinite haploid population model occur with crossover and
%     mutation for 3 bit and 4 bit populations
%   \item Wright and Agapie (2001) described cyclings in infinite
%     populations for up to 4 bits, and also presented data for cyclings
%     in finite populations
  \end{itemize}

    
\item
  \begin{itemize}
  \item  However,
  \item Akin considers - continuous time model, we consider -
    discrete time model
   \item Hastings' study is limited to two bits length, includes only
     crossover, but no mutation; we consider both crossover and mutation
  \item Wright and Bidwell compute specific set of parameter values; there is particular relation between crossover, mutation and population values; in contrast we use random crossover, mutation and population.
  \item Wright and Agapie use dynamic mutation that depends upon
    where population is in the population space while we use static
    mutation
  \end{itemize}

\item
  \begin{itemize}
  \item Simulations were run for both haploid and diploid populations
%   \item for different string lengths l and population sizes N
  \item To visualize oscillations, distance to fixed points (p* , q*) are plotted
  \end{itemize}
    
\item
  \begin{itemize}
  \item These are the results for haploid and diploid population of
    length 8
   \item top row shows results for haploid, bottom row for diploids
  \item Population size is in increasing order in columns
  \item In the figure, Green line is distance of population to $p^\ast$, red line is distance to $q^\ast$
  \item As population size increases, oscillation approaches the
    behavior exhibited by infinite population
  \end{itemize}
    
    
\item
  \begin{itemize}
  \item So our answer to the 2nd question is
   \item Yes, finite populations can exhibit approximate oscillation when infinite populations oscillate
  \end{itemize}
  
\item
  \begin{itemize}
  \item Question 3 concerns the robustness of finite population oscillation under mutation-violation
  \item By mutation-violation we mean, mutation distribuiton holding this condition.
  \item This means no periodic orbits for infinite population
  \end{itemize}

\item
  \begin{itemize}
  \item We introduce violation $\epsilon$ as following
  \item The modification in $\bm{\mu}$ makes the Markov chain regular 
  \item This means no periodic orbits for finite population  
  \item We study if finite population can exhibit approximate oscillations in that case 
  \end{itemize}
  
\item
  \begin{itemize}
  \item Simulations were run for different values of $\epsilon$.
  \item Distance of population to limits p∗ and q∗ without violation are plotted  
  \end{itemize}
  
\item
  \begin{itemize}
  \item These figures show results for haploid population behavior of string length 8
  \item As population increases, finite population oscillation converges to infinite population oscillation
  \item As value of $\epsilon$ grows, oscillation diminishes
  \item Oscillation dies out for infinite population with time.
  \item Even though finite population also appears to be dying out, since Markov chain is regular, population must visit every population state infinitely. 
  So these population will reoccur and oscillate infinitely often (for = 0.1).  
  \end{itemize}
  
  
\item
  \begin{itemize}
  \item Our conclusions from this simulation are
  \item Finite populations exhibit approximate oscillation even if Markov chain is regular when violation is small
  \item If violation becomes larger, finite population oscillation decreases
  \item As string length increases, oscillation degrades
  \end{itemize}
  
  %% violation in crossover
\item
  \begin{itemize}
  \item Our previous question concerns with the robustness of finite population oscillation under mutation-violation, 
  our next question also concerns with the robustness of finite population but under crossover-violation 
  \item By crossover-violation, we mean this condition holds.
  \item Crossover-violation means no periodic orbit exists for infinite population
  \end{itemize}

\item
  \begin{itemize}
  \item We introduce violation $\epsilon$ in crossover distribution
  
  \item And investigate if finite population can exhibit approximate oscillations in this case     
  \end{itemize}
  
\item
  \begin{itemize}
  \item Simulations were run for different values of $\epsilon$
  \item Distances of population to limits p∗ and q∗ without violation are plotted  
  \end{itemize}
  
\item
  \begin{itemize}
  \item Graphs show results for haploid population behavior of string length 8
  \item Results show similar behavior to that of violation in mutation case
  \item However, we noticed that rate of damping of amplitudes of oscillation is slower than in mutation violation 
  
  \end{itemize}  

  
\item
  \begin{itemize}
  \item Conclusions from experiment on violation in crossover are 
  \item Finite populations can exhibit approximate oscillation if violation is small
  \item If violation becomes larger, finite population oscillation decreases
  \end{itemize}
  
\item
  \begin{itemize}
  \item Overall conclusions from this research are
  \item Vose's haploid model makes computation efficient in diploid case by reducing to haploid case 
  \item Distance between finite population and infinite population can decrease like $1/\sqrt{N}$
  \item When infinite populations oscillate, finite populations exhibit approximate oscillation
  \item Finite populations exhibit approximate oscillation for small mutation-violation
  \item Finite populations exhibit approximate oscillation for small crossover-violation        
  \end{itemize}
  
\item
  \begin{itemize}
  \item We noticed some unexpected behavior with oscillation in violation case. 
  \item Distance of infinite population to $\bm{p}^\ast$ and $\bm{q}^\ast$ appears to die out evenly to give a single line graph. And since infinite population is converging to $\bm{z}^\ast$, it implies $z^\ast$ is between $\bm{p}^\ast$ and $\bm{q}^\ast$ and equidistant. So we ran some simulations to test and it confirmed $z^\ast$ is infact between and equidistant from $\bm{p}^\ast$ and $\bm{q}^\ast$.
  \end{itemize}
  
%   \item
%   \begin{itemize}
%   \item Black dots represent finite populations;\hfill\mbox{
%   }\linebreak arrows connect one generation to the next.
%   \item White dots represent infinite populations;\hfill\mbox{ }\linebreak
%       arrows connect one generation to the next.
%     \item We are concerned with the distance between the finite and infinite populations.
%   \end{itemize}
% 
% \item
%   \begin{itemize}
%   \item Dots represent populations;\hfill\mbox{ }\linebreak
%     arrows connect one generation to the next.
%   \item We are interested in oscillation between a pair of populations.
%   \end{itemize}
%   
% \item
%   \begin{itemize}
%   \item Lot of people have contributed in evolution models. Here I
%     have listed few.
%   \end{itemize}

  % \item
%   \begin{itemize}
%   \item Our first question concerns about distance between finite
%     population and infinite population
%   \end{itemize}
% 
% \item
%   \begin{itemize}
%   \item Chebyshev's inequality suggests distance might decrease as
%     $1/\sqrt{ r}$
%   \end{itemize}
% 
% \item
%   \begin{itemize}
%   \item This tetrahedron represents population space.
%   \item Dots represent finite population points; larger dot, less
%     likey point; smaller dot, more likely point
%   \item But infinite population can be anywhere in the space
%   \item The distance between finite population point and infinite
%     population is $O(1/\sqrt{r})$
%   \end{itemize}
%     
% \item
%   \begin{itemize}
%   \item Jensen's inequality also suggests the distance might decrease
%     as $1/\sqrt{ r}$
%   \item All of 3 implications are from inequalities. So the distance
%     might decrease much smaller than $1/\sqrt{ r}$.
%   \item We investigate what happens in practice. For this we set up
%     simple diploid model.
%   \end{itemize}
\end{enumerate}

\end{document}


