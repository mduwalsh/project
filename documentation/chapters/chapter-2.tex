\chapter{Extending A Genetic Algorithm Model To The Diploid Case} \label{ch:GA model Diploid}

This chapter describes a simple Markov model for evolution under the
influence of crossing over and mutation; it is a non-overlapping,
generational, infinite population model under the assumption of {\em complete panmixia} (random mating) and no
selective pressure. This chapter contributes to the elegance and
simplicity of the abstract development and manifests diploid evolution equations can be represented by haploid equations.

A basic syntactic model for haploid and diploid genomes is considered in the beginning and commented on its expressive power. Then the mechanics of how the $(n+1)$th generation is obtained from the $n$th generation are
defined abstractly in procedural terms, which serves to motivate the equations governing evolution.

Next evolution equations are developed corresponding to the
procedural description defining evolution for a population of
diploid genomes. Observations concerning the form and symmetry of
those equations directly lead to decoupling from the diploid case a
haploid model sufficient to determine evolutionary trajectories for
the diploid case.    

\section{Model} \label{Model}
A haploid genome $g$ is defined syntactically as a length $\ell$
binary string.  A collection of $h$ chromosomes may be modeled by
partitioning $g$ into $h$ segments (of arbitrary lengths $\ell_1,
\ldots , \ell_h$; thus $\ell = \ell_1 + \cdots + \ell_h$).
Partitioning may be extended to chromosomes so as to interpret each as
a collection of genes.  If continued to the granularity of pairs of
bits, partitioning allows, for example, representing the four
possibilities Adenine, Guanine, Cytosine, and Thymine.

A diploid genome $\alpha = \langle \alpha_0, \alpha_1 \rangle$ is
likewise defined syntactically as a pair of length $\ell$ binary
strings.  Although simple, that syntax is flexible and possesses
significant modeling power by means of tailoring partitioning to
application.  We concentrate on the abstract level, considering the
evolution of a non-overlapping, generational, infinite population
model assuming panmixia and no selective pressure. We are not concerned with 
whether and how partitioning is defined as it is irrelevant 
to the development.

Following Hardy \cite{Hardy1908}, the model $q^{n}$ at generation $n$
is a vector having for component $q_\alpha^n$ the prevalence of
diploid $\alpha\,$ (the probability of selecting $\alpha$ \nudge at
generation $n$, assuming unbiased selection).\footnote{The
  representation here is the conceptual equivalent of Hardy's model.}
Ordered diploid $\gamma = \langle \gamma_0, \gamma_1 \rangle$ is
produced for generation $n+1$ according to following procedural
description.

  Assuming independent selection events:
\begin{itemize}
\item From parent $\alpha$ --- selected with probability
  $q_\alpha^n$ --- obtain gamete $\gamma_0$
\item From parent $\beta$ --- selected with probability $q_\beta^n$
  --- obtain gamete $\gamma_1$
\end{itemize}
Following Gieringer \cite{Geiringer1944}, let the transmission
function $t_\alpha(g)$ be the probability that gamete $g$ is produced
from parental genome $\alpha$.  It follows from the above that the
equation determining the next generation $q^{n+1}$ is
\begin{equation}
\label{model0}
q_\gamma^{n+1} \; = \;
\sum_{\alpha} \, q_\alpha^n \, t_\alpha(\gamma_0) 
\sum_{\beta} \,q_\beta^n \, t_\beta(\gamma_1)\\[-.05in]
\end{equation}

It should be appreciated that the Mendelian \cite{Mendel1866} laws of
segregation\footnote{Alleles of a given locus segregate into separate
  gametes.} and independent assortment\footnote{Alleles of one gene
  sort into gametes independently of the alleles of another gene.}
need not be respected by the transmission function.


The right hand side of (\ref{model0}) is invariant under interchange
of the summation variables $\alpha$ and $\beta$, which is equivalent
to interchanging $\gamma_0$ and $\gamma_1$.  This symmetry reflects
the fact that which haploid of $\gamma$ is designated as $\gamma_0$ is
arbitrary,\\[-.05in]
\[
q_{\langle \gamma_0, \gamma_1 \rangle}^{n+1} \; = \;
q_{\langle \gamma_1, \gamma_0 \rangle}^{n+1}\\[.1in]
\]
The model corresponding to (\ref{model0}) is low-level in the sense
that it regards $\langle \gamma_0, \gamma_1 \rangle$ and $\langle
\gamma_1, \gamma_0 \rangle$ as distinct when $\gamma_1 \neq \gamma_0$.
A higher-level model based on sets is easily obtained,
\[
q_{\{\gamma_0, \nudge \gamma_1\}} \; = \; \left\{
\begin{array}{ll}
2 \nudge q_{\langle \gamma_0, \gamma_1 \rangle} & \mbox{ if $\gamma_0 \neq \gamma_1$}\\
\phantom{2 \nudge }q_{\langle \gamma_0, \gamma_1 \rangle} & \mbox{ otherwise}
\end{array}
\right.\\[.05in]
\]
which is in agreement with Hardy\cite{Hardy1908} (issues he considered
and results he obtained relating to invariant distributions for a
particular sort of transmission function are not here mentioned
because they are irrelevant to the purpose of this section).

\section{Reduction}

Evolution equation (\ref{model0}) may be reduced to the haploid case.
Its right hand side is the product of two summations; denote the first
by $p_{\gamma_0}^{n+1}$ and the second by $p_{\gamma_1}^{n+1}$ so that
\begin{equation}
\label{model1}
q_{\langle \gamma_0, \gamma_1 \rangle}^{n+1} \; = \;
p_{\gamma_0}^{n+1} \, p_{\gamma_1}^{n+1}\\[.05in]
\end{equation}
where for any haploid $\gamma_0$,
\begin{equation}
\label{model00}
p_{\gamma_0}^{n+1} \; = \;
\sum_{\alpha} \,q_\alpha^n \, t_\alpha(\gamma_0)
\end{equation}
It suffices to determine the evolution of the distributions $p^{n}$.
Uncoupling \nudge $p$ \nudge from \nudge $q$ \nudge using
(\ref{model00}), and equation (\ref{model1}) with superscript $n$ ---
instantiate the $n$ in (\ref{model1}) with $n-1$ --- yields the
evolution equation
\begin{eqnarray}
\label{model2}
p_{\gamma_0}^{n+1} & = &
\sum_{\alpha_0, \, \alpha_1} \, q_{\langle \alpha_0, \,\alpha_1 \rangle}^n \,
t_{\langle \alpha_0, \,\alpha_1 \rangle}(\gamma_0) \nonumber \\
& = &
\sum_{\alpha_0, \, \alpha_1} \, p_{\alpha_0}^n \, p_{\alpha_1}^n \,
t_{\langle \alpha_0, \,\alpha_1 \rangle}(\gamma_0) 
\end{eqnarray}
The $p^n$ are in fact distributions; summing equation
(\ref{model1}) with superscript $n$ yields
\[
1 \; = \; \sum_\alpha \, q_\alpha^n \; = \;
\sum_{\alpha_0, \, \alpha_1} \, p_{\alpha_0}^n \, p_{\alpha_1}^n \; = \;
\Big( \sum_{\alpha_0} \, p_{\alpha_0}^n \Big)^2
\]
Let $[\mbox{\em expression\/} ]$ denote $1$ if {\em expression\/} is
true, and $0$ otherwise.\footnote{$[ \cdots ]$ is sometimes referred to
  as an {\em Iverson bracket}.}  The weighted count of haploid
$g$ in generation $n$ is
\begin{eqnarray}
\label{project}
  & &
  \sum_{\alpha_0, \, \alpha_1} \, q_{\langle \alpha_0, \alpha_1 \rangle}^n
([g = \alpha_0] + [g = \alpha_1]) \\ & = &
\sum_{\alpha_0, \, \alpha_1} \, p_{\alpha_0}^n \, p_{\alpha_1}^n [g = \alpha_0] + 
\sum_{\alpha_0, \, \alpha_1} \, p_{\alpha_0}^n \, p_{\alpha_1}^n [g = \alpha_1] \\[0.05in]
& = & 2 \nudge p_g^n
\end{eqnarray}

Hence the (normalized) prevalence of haploid $g$ in generation $n$ is
the $g\,$th component of the distribution $p^n$. \linebreak
Moreover, (\ref{project}) and (\ref{model1}) show (for $n >
0$) invertibility of the map
\[
  \pi \nudge : \nudge {\bm q}^{n} \; \longmapsto \; {\bm p}^{n}
\]

Evolution equation (\ref{model2}) in matrix form is
\begin{equation}
\label{model3}
p_g^\prime \; = \; p^T M_g \,\nudge p
\end{equation}
where current state $p$ (generation $n$) and next state $p^\prime$
(generation $n+1$) are column vectors, and the $g\,$th transmission
matrix is
\begin{equation} \label{Mg}
\Big(M_g \Big)_{u,v} \; = \; t_{\langle u, v \rangle}(g)
\end{equation}
(vectors and matrices are indexed by haploids --- length $\ell$ binary
strings).







