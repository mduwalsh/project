\documentclass{article}
\renewcommand{\baselinestretch}{1.5}     % line Spacing
%%%%%%%%%%%%%%%%%%%%%%%%%%%%%%%%%%%%%%%%%%%%%%%%%%%%%%%%%%%%%%%%%%%%%%%%%%%%%%%%%%%%%%%%%%%%%%%%%%%%%
% LOAD SOME USEFUL PACKAGES
%%%%%%%%%%%%%%%%%%%%%%%%%%%%%%%%%%%%%%%%%%%%%%%%%%%%%%%%%%%%%%%%%%%%%%%%%%%%%%%%%%%%%%%%%%%%%%%%%%%%%
\usepackage{nomencl}                    % produces a nomenclature
\usepackage{float}                      % figure floats
\usepackage{natbib}                     % this package allows you to link your references
\usepackage{graphicx}                                   % graphics package
\graphicspath{ {figures/}{figures/eps/}{figures/pdf/} }% specify the path where figures are located
\usepackage{fancyhdr}                   % fancy headers and footers
\usepackage{url}                        % nicely format url breaks
\usepackage[inactive]{srcltx}                   % necessary to use forward and inverse searching in DVI
\usepackage{relsize}                    % font sizing hierarchy
\usepackage{booktabs}                   % professional looking tables
%\usepackage{subfigure}% Support for small, `sub' figures and tables
\usepackage[config, labelfont={bf}]{caption,subfig} % nice sub figures
\usepackage{mathrsfs}                   % additional math scripts
\usepackage{bm}
\usepackage{amsfonts}
\usepackage{multirow}
\usepackage{tabularx, booktabs}

\newcommand{\nudge}{\hspace{0.01in}}
%%% PACKAGES THAT ARE PRELOADED WITH THE CLASS ARE: amsmath,amsthm,amssymb,setspace,geometry,hyperref,and color
%%%%%%%%%%%%%%%%%%%%%%%%%%%%%%%%%%%%%%%%%%%%%%%%%%%%%%%%%%%%%%%%%%%%%%%%%%%%%%%%%%%%%%%%%%%%%%%%%%%%%
\begin{document}
\pagenumbering{arabic}
\setcounter{page}{1}    
\newlength{\mywidth}
\setlength{\mywidth}{0.9\linewidth}
\newlength{\myheight}
\setlength{\myheight}{0.5in}
    
\begin{enumerate}
\item
  \begin{itemize}
  \item I will talk about my thesis: \hfill\mbox{ }\linebreak
      ``Simulation of Simple Evolutionary System'' 
  \end{itemize}
  
  \item Part I
    
\item
  \begin{itemize}
  \item      
      I will give background, and computations involving our model in the simulations we did. 
  \item
      Then I will address these four questions, and make
      concluding remarks.    
  \end{itemize}

\item
  \begin{itemize}
  \item Population is a collection of length $\ell$ binary strings 
  \item Population can be represented by a vector; the $j$th component
      is the proportion of string j in the population.
    \item $\mathcal{R}$ is the set of length $\ell$ binary strings and operations under $\mathcal{R}$ are bitwise modulo 2 operations.
  \end{itemize}
    
\item
  \begin{itemize}
  \item Crossover and mutation operators are defined using the
    bitwise operations in $\mathcal{R}$.
  \item Crossover exchanges bits in parents $u$ and $v$ using crossover mask to produce children $u'$ and $v'$ 
  using the rule $u^\prime \,=\, um + v\bar{m} , v^\prime \,=\, u\bar{m} + vm$
  \item $\bm{\chi}_m$ is probability of using crossover mask m
  \item
  \item Mutation flips bits in $x$ using mutation mask m.
  \item $\bm{\mu}_m$ is probability of using mutation mask m
  \end{itemize}
    
\item
  \begin{itemize}
  \item This flowchart illustrates finite population genetic algorithm
  \item Start from random initial population p.
  \item Randomly select parents u and v 
  \item Crossover u and v to produce u' and v' according to randomly chosen crossover mask
  \item Keep one of u', v' and mutate using randomly chosen mutation mask to produce gamete g
  \item Repeat to form next generation $\tau(p)$
  \end{itemize}
  
\item
  \begin{itemize}
  \item In Random Heuristic Search,
  \item Given population p, $\tau$ is transition rule that maps p to
    the next generation p' where p and $p^\prime$ both belong to population space $\Lambda_N$.
    N is population size. 
  \item Finite population evolution forms
    Markov chain
  \item However, $\tau$ is stochastic function of crossover and
    mutation and $\tau(p)$ can not be predicted with
    certainty 
  \end{itemize}
    
\item
  \begin{itemize}
  \item In infinite population model, population is modeled by vector p where p belongs to population space $\Lambda$.    
  \item $\mathcal{G}$ is a function that maps $\bm{p}$ to the next
    generation $\bm{p}^\prime$ where $j$th component is proportion of string $j$
    occurs in the next generation.
  \item The sequence shows evolution of p under infinite population
    model.
  \item The variance is expectation of square of distance between finite population and infinite population.
  \end{itemize}
    
\item
  \begin{itemize}
  \item This is our model setup for diploid population.
  \item We consider diploid genome $\alpha$ with genome length l.
  \item Population is modeled by vector $q$
  \item $q_\alpha$ is prevalence of diploid $\alpha$
  \item $t_{\alpha}(g)$ is transmission function which is probability
    of gamete g being produced from parent $\alpha$
  \item $q^\prime$ is next generation
%   \item $q_\gamma^{n+1} \; = \; \sum_{\alpha} \, q_\alpha^n \,
%     t_\alpha(\gamma_0) \sum_{\beta} \,q_\beta^n \,
%     t_\beta(\gamma_1)\\[-.05in]$ gives us next generation population.
  \end{itemize}
    
\item
  \begin{itemize}
  \item Diploids can be determined in terms of haploid
    distributions.
  \item Haploids can be determined in terms of diploid
    distributions.
  \item And evolution equation can be expressed in terms of haploid
    distributions as 
    \[\bm{p}_{\gamma_0}^{\prime} \,=\, \sum_{\alpha_0, \, \alpha_1} \, \bm{p}_{\alpha_0} \, \bm{p}_{\alpha_1} \,
	  t_{\langle \alpha_0, \,\alpha_1 \rangle}(\gamma_0) \]
	  
  \item In the matrix form, evolution equation can be expressed as $\bm{p}_g^\prime \; = \; \bm{p}^T M_g \, \bm{p}$
  \item Where M(g) is mixing matrix
  \end{itemize}
    
\item
  \begin{itemize}
  \item This slide shows computation of crossover and mutation
    distributions, and transmission function  
  \item Transmission function is computed with this expression
  \item There are 3 sum terms in it, which is expensive to compute
  \end{itemize}
    
\item
  \begin{itemize}
  \item W is walsh matrix
  \item $\hat{A}$ is walsh transform of matrix A, 
    and $\hat{w}$ is walsh transform of column vector $w$.
  \item Mixing matrix is given by the expression
    \[
      \widehat{M}_{u,v} \; = \; 2^{\,\ell-1} \,[\nudge u \nudge v = {\bf
      0}\nudge]\, \widehat{\bm{\mu}}_u \nudge \widehat{\bm{\mu}}_v \!  \sum_{k
    \nudge \in \nudge \overline{u+v} \nudge \mathcal{R}} \bm{\chi}_{k + u} +
    \bm{\chi}_{k + v}
    \]
  \item This is evolution eqn in Walsh basis 
  \[
    \widehat{\bm{p}}_g^{\,\,\prime} \; = \; 2^{\,\ell/2} \sum_{i \nudge \in \nudge g \mathcal{R}}
    \widehat{\bm{p}}_i \, \nudge \widehat{\bm{p}}_{i+g} \,\widehat{M}_{i,i+g}
  \]  
  \end{itemize}
    
\item
  \begin{itemize}
  \item Now if we compare evolution eqn in Walsh basis to what we had before, we got rid of matrix multiplication to compute next generation. 
  \item We don't need to refer to $2^\ell$ mixing matrices either, only one mixing matrix is required in walsh basis computation.
  \item Calculating each mixing matrix in previous eqn required 3 sum terms, which is reduced to 1 sum term in walsh basis.
  \item Consider l = 14, $2^{14}$ mixing matrices would require 32 TB of
    memory, while one mixing matrix would require only 2 GB of memory
  \item We think that is great optimization in resource
  \end{itemize}
    
\item
  \begin{itemize}
  \item With naive computation as shown in equation (1), there would
    be $2^l \cdot 2^l$ multiplication terms to be computed
  \item where f is finite population and q is infinite population
  \item We simplify it as shown in eqn (2) where it needs only $2^l$ +
    cardinal value of $S_f$ terms
  \item $S_f$ is set of diploids that are in finite populations
  \end{itemize}
  \item Writing code to compute in the Walsh basis, and running simulations using the Walsh basis 
  is a significant part of my thesis. It represents the "Efficient" in the thesis title "Efficient Simulation of a Simple Evolutionary System".
  
\item {}
  
\item
  \begin{itemize}
  \item Our first question concerns about convergence of finite population to infinite population.
  \item Chebyshev's inequality, Jensen's inequality and geometric point of view, all, suggest the distance might decrease
    as $1/\sqrt{ N}$
  \item But all of 3 implications are from inequalities. The distance
    might decrease much smaller than $1/\sqrt{ N}$.
  \item We investigate if the distance infact decreases like $1/\sqrt{N}$ in practice. 
  \end{itemize}
  
\item
  \begin{itemize}
  \item We ran some simulations with $\chi$ = 0.1 and $\mu$ = 0.001 for
    different values of l
  \item These graphs show the results in logarithmic scale.
  \item  (point to axis) d is the distance, (point to axis) N is population size, (point to axis) n is number of
    generations
  \item Data shows as population size increases, distance decreases and 
    converge to infinite population
  \item Graphs show flat surface so we model the data by linear equation  
  \end{itemize}
    
\item
  \begin{itemize}
  \item Regression shows slope m is nearly equal to -0.5
  \item This proves distance does decrease as $1/\sqrt{N}$  
  \end{itemize}

\item
  \begin{itemize}
  \item $1/\sqrt{N}$ was theoretical upper bound 
  but our simulaion shows it is really very good bound 
  and the distance decreases as $1/\sqrt{N}$. 
  \end{itemize}
    
\item 
  \begin{itemize}
  \item Our 2nd question concerns about oscillation in
    finite populations.
  \end{itemize}
    
\item
  \begin{itemize}
%   \item First I want to introduce term limit; we also call it fixed
%     point.
  \item The infinite population sequence  $\bm{p}, \, \mathcal{G}(\bm{p}), \, {\mathcal{G}}^2(\bm{p}), \, \cdots$ may
    converge to a point, and we call that point a fixed point, we also call it limit.
  \item But under certain conditions, the sequence converges to a
    periodic orbit between two fixed points, say p* and q*
\end{itemize}

\item
  \begin{itemize}
  \item And the conditions are: For some $g$
  \begin{eqnarray*}
      -1 &=& \sum \limits_{j} (-1)^{g^T j} \bm{\mu}_j \\
      1 &=& \sum \limits_{k \in \bar{g}\mathcal{R}} \bm{\chi}_{k+g} + \bm{\chi}_k 
      \end{eqnarray*}
  \item 1st one is for mutation distribution and 2nd one is for crossover distribution
  \item We try to answer our 2nd question through simulations; that is do finite populations exhibit
     oscillations from random intial populations when infinite populations oscillate?
  \end{itemize}
    
\item
  \begin{itemize}
    \item Akin, Hasting, Wright, Bidwell and Agapie have studied oscillations in the past.
%   \item Other people in the past also have been interested in studying
%     oscillations in populations.
%    \item Akin (1982) proved existence of cycling for continuous-time
%      2-bit diploid model
%   \item Hasting (1981) studied cycling in populations with infinite
%     2-bit diploid population model
%   \item Wright and Bidwell (1997) provided examples when cycles in an
%     infinite haploid population model occur with crossover and
%     mutation for 3 bit and 4 bit populations
%   \item Wright and Agapie (2001) described cyclings in infinite
%     populations for up to 4 bits, and also presented data for cyclings
%     in finite populations
  \end{itemize}

    
\item
  \begin{itemize}
  \item Akin considers - continuous time model, we consider -
    discrete time model
   \item Hastings' study is limited to two bits length, includes only
     crossover, but no mutation
  \item Wright and Bidwell compute a specific fitness function and a
    specific initial population for randomly generated mutation and
    crossover distributions in an attempt to find cyclic behavior
  \item Wright and Agapie use dynamic mutation that depends upon
    where population is in the population space while we use static
    mutation
  \end{itemize}

\item
  \begin{itemize}
  \item Simulations were run for both haploid and diploid populations
%   \item for different string lengths l and population sizes N
  \item To visualize oscillations, distance to fixed points (p* , q*) are plotted
  \end{itemize}
    
\item
  \begin{itemize}
  \item These are the results for haploid and diploid population of
    length 8
   \item top row shows results for haploid, bottom row for diploids
  \item Population size is in increasing order in columns
  \item As population size increases, oscillation approaches the
    behavior exhibited by infinite population
  \item We observe more randomness in diploids than in haploids for
    same string length and population size.
  \end{itemize}
    
\item
  \begin{itemize}
  \item Graphs show the results for haploid and diploid population of
    length 12
   \item We observe that increase in l degrades oscillation (particularly in diploids)
  \end{itemize}
    
\item
  \begin{itemize}
  \item So our answer to the 2nd question is
   \item Finite populations exhibit approximate oscillation when
     infinite populations oscillate
  \end{itemize}
  
\item
  \begin{itemize}
  \item Question 3 concerns the robustness of finite population oscillation under mutation-violation
  
  \end{itemize}

\item
  \begin{itemize}
  \item We introduce violation $\epsilon$ as following
  \item This means no periodic orbits for finite population
  \item The modification in $\bm{\mu}$ makes the Markov chain regular
  \item This means no periodic orbits for infinite population
  \item We study if finite population can exhibit approximate oscillations in that case 
  \end{itemize}
  
\item
  \begin{itemize}
  \item Simulations were run for different values of $\epsilon$.
  \item Distance of population to limits p∗ and q∗ without violation are plotted  
  \end{itemize}
  
\item
  \begin{itemize}
  \item These figures show results for haploid population behavior of string length 8
  \item Graphs are arranged in rows for increasing $\epsilon$ and columns for increasing population size
  \item Graphs show convergence of finite population behavior to infinite population behavior as population increases
  \item Results show oscillating behavior of population for smaller values of epsilon that diminishes with time.
  \item As value of $\epsilon$ grows, oscillation diminishes
  \item Oscillation dies out for infinite population.
  \item Even though finite population also appears to be dying out, since Markov chain is regular, 
  finite population will oscillate infinitely often (for $\epsilon$ = 0.1).    
  \end{itemize}
  
\item
  \begin{itemize}
  \item Graphs show results for diploid population behavior of string length 12
  \item We observed oscillation also degrades as string length increases. 
  It is seen in both haploid and diploid cases, but particularly noticeable in diploids.  
  \end{itemize}
  
\item
  \begin{itemize}
  \item Our conclusions from this simulation are
  \item Finite populations exhibit approximate oscillation even if Markov chain is regular when violation is small
  \item If violation becomes larger, finite population oscillation decreases
  \item As string length increases, oscillation degrades
  \end{itemize}
  
  %% violation in crossover
\item
  \begin{itemize}
  \item Our previous question concerns with the robustness of finite population oscillation under mutation-violation, 
  our next question also concerns with the robustness of finite population but under crossover-violation  
  \end{itemize}

\item
  \begin{itemize}
  \item We introduce violation $\epsilon$ in crossover distribution
  \item Crossover-violation means no periodic orbit exists for infinite population
  \item We investigate if finite population can exhibit approximate oscillations in this case     
  \end{itemize}
  
\item
  \begin{itemize}
  \item Simulations were run for different values of $\epsilon$
  \item Distances of population to limits p∗ and q∗ without violation are plotted  
  \end{itemize}
  
\item
  \begin{itemize}
  \item Graphs show results for haploid population behavior of string length 8
  \item Results show similar behavior to that of violation in mutation case
  \end{itemize}
  
\item 
  \begin{itemize}
  \item Graphs show results for diploid population behavior of string length 12
  \item However, we noticed that rate of damping of amplitudes of oscillation is slower than in mutation violation  
  \item And, we see more randomness in population behavior than in mutation violation, especially for diploid case
  \end{itemize}
  
\item
  \begin{itemize}
  \item Conclusions from experiment on violation in crossover are 
  \item Finite populations exhibit approximate oscillation if violation is small
  \item If violation becomes larger, finite population oscillation decreases
  \end{itemize}
  
\item
  \begin{itemize}
  \item Overall conclusions from this research are
  \item Vose's haploid model makes computation efficient in diploid case by reducing to haploid case 
  \item Distance between finite population and infinite population can decrease like $1/\sqrt{N}$
  \item When infinite populations oscillate, finite populations exhibit approximate oscillation
  \item Finite populations exhibit approximate oscillation for small mutation-violation
  \item Finite populations exhibit approximate oscillation for small crossover-violation      
  
  \end{itemize}
  
%   \item
%   \begin{itemize}
%   \item Black dots represent finite populations;\hfill\mbox{
%   }\linebreak arrows connect one generation to the next.
%   \item White dots represent infinite populations;\hfill\mbox{ }\linebreak
%       arrows connect one generation to the next.
%     \item We are concerned with the distance between the finite and infinite populations.
%   \end{itemize}
% 
% \item
%   \begin{itemize}
%   \item Dots represent populations;\hfill\mbox{ }\linebreak
%     arrows connect one generation to the next.
%   \item We are interested in oscillation between a pair of populations.
%   \end{itemize}
%   
% \item
%   \begin{itemize}
%   \item Lot of people have contributed in evolution models. Here I
%     have listed few.
%   \end{itemize}

  % \item
%   \begin{itemize}
%   \item Our first question concerns about distance between finite
%     population and infinite population
%   \end{itemize}
% 
% \item
%   \begin{itemize}
%   \item Chebyshev's inequality suggests distance might decrease as
%     $1/\sqrt{ r}$
%   \end{itemize}
% 
% \item
%   \begin{itemize}
%   \item This tetrahedron represents population space.
%   \item Dots represent finite population points; larger dot, less
%     likey point; smaller dot, more likely point
%   \item But infinite population can be anywhere in the space
%   \item The distance between finite population point and infinite
%     population is $O(1/\sqrt{r})$
%   \end{itemize}
%     
% \item
%   \begin{itemize}
%   \item Jensen's inequality also suggests the distance might decrease
%     as $1/\sqrt{ r}$
%   \item All of 3 implications are from inequalities. So the distance
%     might decrease much smaller than $1/\sqrt{ r}$.
%   \item We investigate what happens in practice. For this we set up
%     simple diploid model.
%   \end{itemize}
\end{enumerate}

\end{document}


