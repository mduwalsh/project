
\chapter*{Efficient Simulation Of A Simple Evolutionary System}\label{ch:abstract}
\section*{Abstract}
An infinite population model is considered for diploid evolution under the influence of crossing over
and mutation. The evolution equations show how Vose's 
haploid model for Genetic Algorithms extends to the diploid case, thereby making feasible simulations
which otherwise would require excessive resources. This is illustrated through computations confirming
the convergence of finite diploid population short-term behaviour to the behaviour predicted by the
infinite diploid model. The results show the distance between finite and infinite population evolutionary trajectories can 
decrease in practice like the reciprocal of the square root of population size. 

Under necessary and sufficient conditions (NS) concerning mutation and crossover, 
infinite populations show oscillating behavior. 
We explore whether finite populations can also exhibit oscillation or approximate oscillation. 
Simulation results confirm that approximate finite population oscillation is possible when NS are satisfied. 

We also investigate the robustness of finite population oscillation.  
We show that when the part of NS concerning mutation is violated, 
the Markov chain which models finite population evolution is regular, 
and perfect oscillation  should not occur.
However, our simulation results show finite population 
approximate oscillation can occur even though the Markov chain is regular. 
Finite populations can also exhibit approximate oscillating behavior when the part of NS concerning crossover is violated. 


% Genetic algorithm has been used to evolve solutions to problems not yielding to other known methods. 
% It has been analyzed and developed over time. Relationship between finite population and infinite population models 
% has been explored. Vose's infinite haploid population model in Random Heuristic Search suggests the distance between 
% finite population and infinite population might decrease by square root of finite population size. 

% This research investigates, as our first question, through experiment if this rate of decrease of distance is exhibited in practice.
% This is explored as our second research question. Conditions in crossing over and mutation projected 
% by Vose for infinite population to oscillate in periodic orbits were implemented for finite 
% population, and 