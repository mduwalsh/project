\chapter{Experimental Simulations and Measurements} \label{ch:distance}
This chapter describes how distance between finite diploid population and infinite population is calculated.
Then it discusses simplifications in computations made by our evolutionary equations and in distance computation. 
And it goes on to discuss results from simulations and convergence of finite diploid population short-term behavior 
to evolutionary behavior predicted by infinite population model.

\section{Distance}
Let vector ${\bm f}$ represent a finite diploid population; component
${\bm f}_\alpha$ is the prevalence of diploid $\alpha$.  Let the
support $S_{\bm f}$ of ${\bm f}$ be the set of diploids occurring in
the population represented by ${\bm f}$,\\[-0.03in]
\[
S_{\bm f} \; = \; \{ \alpha \, | \, {\bm f}_\alpha > 0 \}
\]
Let ${\bm q}$ similarly represent an infinite diploid population (see
section \ref{Model}).  As points in $\mathbb{R}^{2^\ell
  \times 2^\ell}$, the Euclidean distance between ${\bm f}$ and ${\bm
  q}$ is \\[-0.175in]
\[
\|{\bm f} - {\bm q} \hspace{0.005in} \| \; = \;
  {\sum_{\alpha}}^{\frac{1}{2}} ({\bm f}_\alpha-{\bm q}_\alpha)^2\\[-.02in]
\]
Whereas a naive computation of this distance involves ${2^\ell \cdot
  2^\ell}$ terms, leveraging equation (\ref{model1}) can significantly
reduce the number of terms involved.  Note that
\begin{equation} \label{d1}
\|{\bm f} - {\bm q} \hspace{0.005in} \|^2 \; = \;
\sum_{\alpha \notin {S_{\bm f}}} ({\bm f}_\alpha-{\bm q}_\alpha)^2 +
\sum_{\alpha \in {S_{\bm f}}} ({\bm f}_\alpha-{\bm q}_\alpha)^2 \\[-.02in]
\end{equation}
Using equation (\ref{model1}) --- ${\bm q}_\alpha = {\bm p}_{\alpha_0}
\nudge {\bm p}_{\alpha_1}$ (suppressing superscripts to streamline
notation) --- together with the fact that ${\bm f}_\alpha = 0$ in
every term of the first sum above, the first sum reduces to
\begin{eqnarray}
  \sum_{\langle \alpha_0, \nudge \alpha_1 \rangle \nudge \notin
    {S_{\bm f}}} ({\bm p}_{\alpha_0} \nudge {\bm p}_{\alpha_1})^2 & =
  & \sum_{\langle \alpha_0, \nudge \alpha_1 \rangle} ({\bm
    p}_{\alpha_0})^2 \nudge ({\bm p}_{\alpha_1})^2 \, - \sum_{\langle
    \alpha_0, \nudge \alpha_1 \rangle\in {S_{\bm f}}} \big( {\bm
    p}_{\alpha_0}\nudge {\bm p}_{\alpha_1} \big)^2 \nonumber
  \\[0.05in] & = & {\sum_{g}}^2 ({\bm p}_{g})^2 \, - \sum_{\alpha \in
    {S_{\bm f}}} ( {\bm q}_{\alpha})^2
      \label{d2}
\end{eqnarray}
\mbox{ }\\[-.15in]
It follows from (\ref{d1}) and (\ref{d2}) that
\begin{eqnarray}
  \|{\bm f} - {\bm q} \hspace{0.005in} \|^2
  & = & 
      {\sum_{g}}^2 ({\bm p}_{g})^2 \, +
      \sum_{\alpha \in {S_{\bm f}}} ({\bm f}_\alpha-{\bm q}_\alpha)^2 -
      \sum_{\alpha \in {S_{\bm f}}} ( {\bm q}_{\alpha})^2
      \nonumber \\[0.05in]
      & = &
      {\sum_{g}}^2 ({\bm p}_{g})^2 \, +
      \sum_{\alpha \in {S_{\bm f}}} {\bm f}_\alpha ({\bm f}_\alpha- 2 {\bm q}_\alpha) \label{d3}
\end{eqnarray}
\mbox{ }\\[-.1in] which involves $2^\ell + |S_{\bm f}|$ terms,
assuming that  $S_{\bm f}$ is known as a byproduct of computing ${\bm f}$.

(\ref{d3}) computes distance between finite and infinite population efficiently.


\section{Simplification} 
The haploid case simplified by equations (\ref{Mhat}) and (\ref{model5})
are the consequence of specializing to Vose's infinite population model and computing in the Walsh basis. Time switching between the standard basis and the Walsh basis is negligible; the fast Walsh transform (in dimension $n$) has complexity $n \nudge \log n$ \cite{Shanks1969}.

Only one mixing matrix as opposed to $2^\ell$ matrices is needed to compute the next generation; evolution equation (\ref{model5}) references the same matrix for every $g$, whereas evolution equation (\ref{model3}) depends upon a different matrix $M_g$ for each choice of $g$. The matrix is computed by a single sum as opposed to a triple sum; compare equation (\ref{Mhat}) with equation (\ref{transmission}).  Also, the relevant quadratic form is computed with a single sum as opposed to a double sum; computing via (\ref{model5}) is linear time in the size of $g \mathcal{R}$ (for each $g$) as opposed to the quadratic time computation (for each $g$) represented by equation (\ref{model3}).

From a computational standpoint, the best-case scenario is where
recomputation of the matrices mentioned in the previous paragraph is
obviated by sufficient memory.  The reduction from $2^\ell$ matrices
to one matrix helps significantly in that regard. To demonstrate this
advantage in concrete terms, consider genomes of length $\ell = 14$.
Using $2^{14}$ matrices each of which contains $2^{14} \times \nudge
2^{14}$ entries of type \verb@double@ requires $32$ terabytes, whereas
the mixing matrix at $2$ gigabytes fits easily within the memory of a
laptop.  Moreover, for a population size of $N \le 2^{20}$, the
distance computation described in the previous section reduces the
number of terms involved by a factor of
$2^{28}/(2^{14} + 2^{N}) \; > \; 252$.


